\section{Luther’s Literary Activity}

Among the writings which Luther composed during the years in
which he concluded his translation of the Bible, are many sermons
and scattered prefaces written for publications of his friends and
adherents. Popular religious treatises gradually become less frequent.
One of the more noteworthy of the latter bears the title, “A Simple Method
of Prayer, for a Dear Friend, Master Peter, Barber”
(1535).\footnote{Weimar ed., Vol. XXXVIII, pp. 358 sqq.; Erl. ed., Vol. XXXIII, pp. 214 sqq.}
 This booklet, written at the solicitation of his barber, is an
appeal to all Christians who desire to lead a devout life, recommending
to them particularly meditation on the “Our Father” and the Ten
Commandments. Luther relates that he himself used passages from the
Pater Noster and sections of the Decalogue as fuel with which to
kindle a little fire within his heart. He was unfortunate in the selection
of his addressee; for Barber Peter stabbed his son-in-law in the
following year. Owing to Luther’s intercession, the sole punishment
meted out to him was exile.

A more voluminous and important treatise is that on the morals of
princes and courtiers, published in 1534, under the title, “Psalm 101,
with a Commentary by M. Luther.”\footnote
{Weimar ed., Vol. LI, pp. 200 sqq. and LIII, pp. 679 sqq.; Erl. ed., Vol. XXXIX, pp.
265 sqq.}
In this work Luther vigorously reminds
the upper classes of their duties, bewails the decline
of morality in Germany, and points to intemperance as a hereditary
evil. If every nation has its own devil, he says, the German devil must
be a good wine-skin. At that time it was a prevalent custom at the
Electoral court to indulge in riotous drinking bouts. The new Elector,
John Frederick, was not the last one who might profit from Luther’s
admonition. In one of his Table Talks Luther expressed himself about
his protector thus: “He is possessed of every virtue; but just fancy
him swilling like that!”\footnote
{\textit{Tischreden}, Weimar ed., Vol. IV, n. 4933; cfr. Grisar, Luther, Vol. IV, p. 203.
Steinhausen (\textit{Kulturgeschichte der Deutschen}, p. 508) calls John Frederick quite simply a
drunkard.}
Duke Henry of Braunschweig-Wolfenbüttel
openly denounced the Elector as an inebriate, whereat the latter
wrote a letter admitting his inebriety, ``after the German custom,''
but said the Duke of Braunschweig was not the man to find to fault,
for he was an even harder drinker. Luther on one occasion refers to the
appearance of the Electoral courtiers on the morning after a nightly
carousal, and says that their heads appeared to have been immersed in
salt brine during the night.

A work to which Luther was sincerely devoted was his exposition of
the Epistle to the Galatians, which he composed in Latin in 1535. It
was well known that he deemed this Pauline Epistle to be the grand
citadel of his doctrine.

In 1533 he produced his controversial work, “On the Corner-Mass
and the Ordination of Clerics,”\footnote
{Weimar ed., Vol. XXXVIII, pp. 183 sqq.; Erl. ed., Vol. XXXI, pp. 307 sqq. Cfr. Grisar,
Luther, Vol. IV, pp. 518 sqq. “Corner-Mass” (Winckelmesse) is a contemptuous term applied
to the Mass because, according to Catholic doctrine, it is equally valid whether celebrated
by the priest alone in a lonely chapel or amid a concourse of faithful who unite their
prayers with his and communicate with him. (Grisar, \textit{op. cit.}, Vol. IV, pp. 519 sq.).}
which his friend Jonas characterized
as an effective battering-ram against the papacy. The opposition which
his neighbors offered to the innovations he had introduced into the
divine service, was the occasion which once more induced him to attack
the Mass, so dear to all good Catholics. The fiction of his nocturnal
disputation with the devil, which he elaborated with great
skill in this book, has become famous. The devil tries to make Luther
despond because he was guilty of saying many Masses whilst still
a monk, and therefore, was guilty of sacrilege. The abomination of the
Mass is cleverly explained by the devil, and Luther’s excuses are repudiated;
he is told that he must regard himself as lost in the sight
of God. He says it required his utmost efforts to retain hope for his
own salvation whilst repenting of his former blindness. Some Catholic
apologists have erroneously assumed that this clever bit of fiction was
a confession on Luther’s part that his objections to the Mass had been
inspired by the devil. To the attentive reader, the objections are discernible
as typical of Luther, but the disputing devil is a literary
device and the overwhelming fear by which Luther pretends to have
been suffocated, is pure fiction. Luther expresses the wish that all who
celebrate Mass be afflicted by a similar mental agony. Such is the
\textit{punctum saliens} of the much discussed disputation.

In his treatise “On the Corner-Mass” Luther repudiated both the
sacrificial character of the Mass and the Real Presence of Christ on
the altar without communion. This was explained by some of his
friends in the sense that, like the Sacramentarians, he denied that
Christ became really present in the Eucharist by virtue of the words
of institution. In the beginning of 1534 he explained this point
in a published letter,\footnote{Weimar ed., Vol. XXXVIII, p. 262; Erl. ed., Vol. XXXI, p. 377.}
 in which he contended that Christ was not
present in private Masses, when the priest was the sole celebrant, because
there was no communion of the faithful; and that the Eucharist,
because it was essentially a food, was to be distinguished from the
Mass, just as God is to be differentiated from the devil. “May God”--
this is his prayer--“may God bestow upon all pious Christians courage,
so that, when they hear the word Mass, they may bless themselves:
as if they were in the presence of a Satanic abomination!”\footnote{\textit{Briefwechsel}, Vol. X, pp. 8 sqq.; about March 11, 1534.}


Luther’s work, “On the Servile Will,” which he launched against
Erasmus, was answered by the latter in a vigorous and triumphant
pamphlet, entitled “Hyperaspistes,” published in 1526. It was an ingenious
exposure of the heresies and distortions of Luther.

For a long time Luther was silent, without, however, recovering from
the blow which had been administered to him. He was very much pained at
the secession of the Erasmian humanists from his party, maintaining that he
was justified in being angry at Erasmus because the latter, although professedly
a member of the Church, minimized, nay, destroyed the essence of
the Christian religion by his strictures, couched in playful and polished form.
In the course of an embittered conversation, in 1532, he called Erasmus “a
rogue by nature,” who regarded the Blessed Trinity as ridiculous, and added:
“Erasmus is as certain that there is no God, as I am certain that I see.”
Although that charge was a product of his hateful imagination, Luther
continued to indulge in similar declarations until he finally believed
them himself.

A letter from his old friend, Nicholas Amsdorf of Magdeburg, caused him
to vent his pent-up wrath. On January 28, 1534, Amsdorf wrote him a
letter, composed with his customary fervor, in which he said he could
observe the “intervention and the miracles of God” in favor of the Gospel
all around him. God, he said, produced the faith, just as He had wrought
the Resurrection of Christ. George Witzel, Luther’s enemy, who attacked
the gospel of salvation, he said, was dependent upon Erasmus, from whom
he borrowed all his weapons. Erasmus would have to be “thoroughly unmasked”
by Luther “on account of his ignorance and malice.” He (Amsdorf)
advised Luther to perform this task in a book on the Church, since
the attitude of the Erasmian party towards the Church constituted their
vulnerable spot.

Luther was immensely pleased with Amsdorf’s letter, but, being
occupied with other matters, deferred writing the suggested treatise
on the Church until 1539. However, he forthwith printed Amsdorf’s
letter and accompanied it by a furious attack upon Erasmus in the
form of a reply to his friend at Magdeburg.\footnote{Grisar, \textit{Luther}, Vol. IV, pp. 181 sq.}
 Resorting to the worst
kind of distortions and disparagements, he tries to demonstrate. that
the sole purpose of Erasmus was to bring all Christian doctrine into
disrepute, that he was another cynical Democritus, a second Epicurus.
Melanchthon, in a letter to Erasmus, pronounced the rash publication
of Luther’s reply a lamentable blunder. Another friend of Erasmus,
Boniface Amerbach, characterized Luther’s pamphlet as “the product
of a diseased brain” and asserted that Luther had been suffering
from paresis (\textit{cephalea}) for more than a year.\footnote{\textit{Ibid.}, pp. 182 sq.}


The calumnies which Luther had heaped upon the aged scholar of
Rotterdam were too monstrous for him to leave unanswered. He gave
vent to his indignation in a sarcastic Latin rejoinder: “\textit{Purgatio adversus
Epistolam non Sobriam M. Lutheri.}”\footnote
{Amerbach had judged Luther’s attack “insane”; Erasmus, on his part, addressed his
biting reply to “one not sober.” (Grisar, \textit{Luther}, Vol. IV, p, 184.)}
He treats Luther as one
who is drunk or mentally unbalanced. He convicts him of a long
series of bare-faced lies. In his indignation, the maltreated scholar
asserts that “no text is safe against his violent distortions, based upon
premeditated calumny.” Rather than admit the justice of Luther’s
malevolent charge that Erasmus was bent upon fostering infidelity,
“the world will believe that Martinus has become demented through
hatred, or that he suffers from some other mental disorder, or is dominated
by an evil spirit.”

Thereafter Luther preferred to observe silence in public; but in the company
of his intimates, he expressed himself harshly about Erasmus. “This
man,” he said, “simply insists on believing what the pope believes”; but,
“the Italians are hypocrites.” “I fear,” he also said, “that he [Erasmus] will
die a wretched death.” Not long afterwards, in 1536, Erasmus died in the
city of Basle, where, due to the prevalence of the new religion, he was unable
to receive the last Sacraments. He passed away, loyal to his religion and
filled with sincere piety, as even the reports to Wittenberg announced, adding
that his last words were: “I will bless the mercy of the Lord and His
judgments.” Luther did not wish to believe this, and in his Table Talks
represents Erasmus’s death as that of an unbelieving Epicurean. “He lived
in 2 presumptuous security,” he said, “and thus died (\textit{securissime vixit,
sicut etiam morixit}).” As late as 1544, Luther derisively says of Erasmus:
“He passed away \textit{sine crux et sine lux}”--without the Cross and without
light.\footnote
{Grisar, \textit{Luther}, Vol. IV, p. 185; \textit{Tischreden}, Weimar ed., Vol. IV, n. 3963; 4028; 4899;
5670.}

The great and decisive convention of the Schmalkaldic League assembled
February 9, 1537, a year after the death of Erasmus.
