\section{The Schmalkaldic League After 1531}

The military League of Schmalkalden was initiated as early as
February 27, 1531, by John of Saxony and Duke Ernest of Braunschweig at
the instigation of the jurists. It was completed at Schmalkalden on March
29, when the remaining members affixed their seal
to the document. Besides the rulers of electoral Saxony and Braunschweig-Lüneburg,
there were affiliated with it Landgrave Philip
of Hesse, Prince Wolfgang of Anhalt, Counts Gebhard and Albrecht
of Mansfeld, and the cities of Strasburg, Ulm, Constance, Reutlingen,
Memmingen, Lindau, Biberach, Isny, Liibeck, Magdeburg, and
Bremen. Others joined the league later. The members bound themselves by
an oath to come to the relief of any member attacked on
account of the Gospeél or on any other pretext.\footnote{Grisar, \textit{Luther}, Vol. I, p. 64.}

Thus a wedge was driven into the unity of the German nation at
the expense of its internal strength and external development.
The Protestants were now a united political power. What Landgrave Philip
and Elector John had commenced when they formed
their military alliance at Gotha, in 1526, was now completed. The
defense of the interests of the religious innovation passed from Luther
and his theologians to the secular authorities; and the latter knew well
how to pursue their selfish interests with advantage.

Luther may have been glad to remain somewhat in the background when the
League was formed. The jurists and rulers were
promoting his cause. If the League should prove disastrous for Germany
--which actually happened--his gospel would be less exposed
to criticism. Nevertheless, he had to reckon with the decline of his
popularity due to the existence of the League. In no small measure he
forfeited the direction of his work because of the action of the political
rulers.

Schmalkalden was a small city in Hesse, situated south of Eisenach
in the administrative area of Prussia. Inclosed in a pleasant valley,
where the Stille flows into the Schmalkalde (a tributary of the
Werra), the town presents a peaceful scene, in strong contrast with
the recollections of the religious struggles in which it played a rédle.
Verdant hills surround the city, the Rotberg and the Giefelsberg to
the north, the Wolfsberg and the Grasberg to the south. To the east
rises a hill, on which was enthroned the ancient castle of Walrab,
which is now reconstructed into the stately Wilhelmsschloss. The city
was encircled by a wall which has almost completely disappeared.
Within its bosom the parochial church of St. George raised aloft its
two spires; they are of a late Gothic design, and are still well preserved.
The venerable church testifies to the pious and vigorous
Catholic life that prevailed before the Protestant Reformation. The
sessions of the Schmalkaldic League were held in the old-fashioned
town-hall, which is still partially preserved. Luther resided in a Patrician
house at the foot of the castle-hill, which still exists. Here he
participated in the convention of the estates and theologians in 1537.
The residence of Melanchthon, known as “Rosenapotheke,” impresses
the eye of present-day visitors with its antique style of architecture.

A Zwinglian element had entered this city, dominated by Lutheranism at
the time the widowed sister of Philip of Hesse, Elizabeth von
Sachsen-Rochlitz, took up her abode in it. Her name is association
with the change of the city to the religion of the so-called Swiss
reformers. In the parish church, which was at one time richly decorated
with statues, the empty pedestals bear mute evidence to the work of
destruction wrought by the Zwinglian vandals. After these iconoclasts
had vented their fury on the altars, statues, reliquaries and
other works of sacred art, a train of wagons bearing religious objects
wended its way up the hill behind the Wilhelmsschloss, where
the fanatical mob consigned them to the flames.
