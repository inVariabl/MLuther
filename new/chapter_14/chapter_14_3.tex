\section{The Religious Peace of Nuremberg (1532) and Subsequent Events}

While the Protestant party in Germany prudently set about to
circumvent the consequences of the diet of Augsburg by procuring
a temporary religious peace through intimidation, an open religious
war broke out in Switzerland, in consequence of the conditions
obtaining in that country. By the harsh methods under which the
new religion made its appearance under Zwingli’s leadership in
Zurich, that city had gotten into a war with the five Catholic
cantons of the Swiss Confederacy.

In the battle of Kappel, which was won by the Catholics on the
eleventh of October, 1531, Zwingli was wounded and killed. This
accident removed an inexorable enemy of Luther, and bettered the
prospects of a more intimate union of the German Zwinglians with
Lutheranism. Luther candidly expressed his satisfaction with the
judgment which God had visited upon Zwingli for denying the Eucharist,
just as He had once singled out Münzer for punishment. “I
have been a prophet when I said that God would not tolerate the
violent blasphemies of which his party was full.” Thus he wrote to
Link in Nuremberg.\footnote{On January 3, 1532; \textit{Briefwechsel}, IX, p. 139.}
 Without displaying the least compassion, he
consigned Zwingli to hell. Not long after Zwingli’s death, on November
24, Oecolampadius died at Basle, deeply humiliated. Both men,
said Luther, were to be proclaimed “damned,” even though this led to
“violence being offered them,” because this was the best way to
make people shrink from their false doctrines.\footnote{Grisar, \textit{Luther}, Vol. IV, p. 87. \textit{Tischreden}, Weim. ed., Vol. I, No. 1045; II, No. 2845.}
 H. Barge, a Protestant
author, justly says in his life of Karlstadt that Luther “particularly
availed himself in his systematic, most experienced and malicious
manner, of the forceful language he had at his command, in order
to decry Zwingli as a heretic after the latter’s death.”\footnote{Grisar, \textit{op. cit.}, IV, p. 90.}
 In a letter
to Henry Bullinger, who inherited the leadership of the Zurich
faction from Zwingli, and with whom he desired to be at peace,
Luther declared that he had learned to esteem Zwingli as a very
good man after he had made his acquaintance at Marburg, and that
he mourned his death.\footnote{May 14, 1538; \textit{Briefwechsel}, XI, pp. 368 sq.}


On December 25, 1530, the Protestants at Schmalkalden protested
against the election of Ferdinand as king of the Germans, which
was desired by Charles V. This protest was a hostile act against the
Hapsburgs. In spite of it, Ferdinand was elected at Cologne on
January 5, 1531.

When the Schmalkaldians pleaded with the Emperor for the
abolition of the proceedings before the “Kammergericht,” they received
an evasive reply, but one effect of their alliance was that
the term of April 15, which the recess of the diet of Augsburg had
fixed against them, was nullified. The Emperor was prevented from
interfering. Even the Catholic dukes of Bavaria, instigated by their
ancient hostility to the Hapsburgs, entered upon a formal alliance
with the Schmalkaldians against Ferdinand at Saalfeld. Alliances
with France, England, Denmark and Zapolya of Hungary were
sought partly by Bavaria and partly by Philip of Hesse, and were
all directed against the Emperor. The invasion of Hungary by the
Turks, however, and the menace of the latter to Germany, constituted
the greatest obstacle to the Emperor’s plans. It was political
and not dogmatic reasons that compelled Charles V to yield to the
schismatics.

There were prospects of a preliminary peace. The Protestants,
however, demanded that the peace should include all who in future
would declare their adherence to the Augsburg Confession. Even
Luther, writing to the Elector John, characterized this demand as
unacceptable to the Catholic party, because it would inflict too great
an injury upon them. He recommends that the opposition to Ferdinand
be abandoned for the sake of avoiding war. “Who would wish to
be guilty of so much bloodshed for the sake of such a cause?”\footnote
{Prior to May 16, 1532; Erl. ed., Vol. LIV, p. 301 (\textit{Briefwechsel}, IX, p. 186).}
Finally, a treaty of peace, or, more correctly, an armistice, was
signed at Nuremberg on July 23, 1532. According to its terms,
nothing was to be done contrary to the existing religious status
until the assembling of a general council. The Protestants were even
assured in a secret agreement that the legal proceedings concerning
the confiscation of church property would be terminated. The terms
of this peace, however, were to apply only to present, not to future,
adherents of the Augsburg Confession.

In the same year, 1532, Charles V proceeded to Italy, where, in
February, 1533, he concluded an agreement with the Pope. Thence
he repaired to Spain. It was to the great disadvantage of the Catholic
cause in Germany that he did not return to the latter country until
nine years later.

The state of public affairs was not essentially changed by the demise
of two individuals. John, Elector of Saxony, who died in August,
1532, was succeeded in the government of his country by his son,
John Frederick, who was as devoted to the cause of Lutheranism as
his father had been. Pope Clement VII passed away in September,
1534, and was succeeded by Cardinal Alexander Farnese, who assumed
the name of Paul III. The spirit of Catholic reform animated the latter
more intensely than it had Pope Clement, and he elevated determined
men to the cardinalate, such as Contarini, Pole, Sadoleto, and Caraffa.
His nepotistic inclinations, however, induced him to raise two relatives
to the dignity of the cardinalate, of whom one was only fourteen
and the other sixteen years of age. At the very beginning of his
pontificate, Paul III expressed himself in favor of convoking an
ecumenical council; but political complications prevented the execution
of this plan for a long time. Shortly he sent Pietro Paolo
Vergerio as nuncio to Germany, in order to influence the Protestants
in favor of a council. The Elector John Frederick refused to make
a definite reply. It was decided to assemble the council in Mantua,
in May, 1537; but the prospects of a propitious meeting were meager
because of the attitude of France and the threat of a Turkish invasion.
The Protestants everywhere urged excuses for their nonparticipation
in the council, notwithstanding that they had always
clamored for one. Consequently in his circuit of the country Vergerio
experienced very little cooperation, but was constrained to combat
the idea of holding a German national synod instead of an ecumenical
council.

On the day after Vergerio had entered Wittenberg, November 6,
1535, he invited Luther and Bugenhagen to breakfast with him in
the Elector’s castle.\footnote{For the following, see Grisar, \textit{Luther}, Vol. III, pp. 424 sqq.}
He assuredly would not have invited Luther,
had he been better acquainted with him, or had he examined his
recently published work, “Certain Aphorisms against the Council of
Constance,” in which Luther indulged in the most disgraceful language
about the Romish Church, “the mad, blood-thirsty, red harlot,”
and “the dragon’s heads which peep out from the posterior of the
pope-ass.” The author of this offensive pamphlet availed himself of
the opportunity extended by the incautious papal legate, during
whose visit he delivered himself of insults to the Pope and the
Catholics, and boasted of his security in the possession of his new
doctrine.

In discussing the proposed council, Luther said to the papal legate: “I am
willing to lose my head, if I do not defend my teachings against the world.
This anger of my mouth is not my anger, but God’s anger.” He averred
that, though he and his followers were in no need of a council, he would
nevertheless attend it, in order to give testimony to the truth. He adverted
to “foolish and childish matters,” of which ecumenical councils treat in lieu
of matters of faith. He spoke of his “priests,” whom “Bishop” Bugenhagen,
who was present at this interview, was ordaining according to the command
of the Apostle Paul, and of a dozen other hateful things, among them that
“reverend nun,” his wife, who had borne him five children, of whom the
eldest, Hans, was going to be a great preacher of the gospel. He evidently
wished to irritate the nuncio and to confront Rome with an air of superiority.

Vergerio strangely attached great importance to Luther’s readiness to
attend the council, and reported it with satisfaction to Rome. In this report
he also described the exterior appearance of Luther. He found him possessed
of a powerful frame, with exceptionally large features. Although he was past
fifty, Luther appeared to be but forty. His deportment displayed “arrogance,
malevolence, and lack of consideration”; he was “a man devoid of depth,
without judgment, a simpleton.” He wore a heavy golden chain around his
neck and several rings on his fingers. He was dressed in a doublet of dark
camelot, the sleeves of “which were trimmed with satin, over which he wore
a coat of serge lined with fox-fur. Luther himself informs us that he had
himself carefully shaved before the visit of the nuncio; it being necessary,
he says, to appear youthful to the legate, so that the latter might report to
his master that Luther was yet able to accomplish many things. In order to
create an impression, the ex-monk solemnly rode to the castle in a coach,--
“the German pope,” as he said to Bugenhagen, while they were riding, “and
Cardinal Pomeranus, instruments of God.” Vergerio closely observed him
during the interview, especially his eyes, and writes in his report that, the
longer he watched his uncanny eyes, the more he was reminded of certain
persons who were regarded as possessed by the devil. He also claims to have
heard from former intimate friends of Luther certain discreditable facts
about the latter’s youth, but refrains from mentioning their nature.

The nuncio was no model of a circumspect and reliable ecclesiastical
\textit{chargé d’affaires}. Upon his return to Italy he succeeded in obtaining
a bishopric. Subsequently, in 1548, he seceded from the
Catholic faith and embraced the new theology. After wandering
about restlessly, agitating against the papacy, he died at Tübingen
in 1565, unreconciled with the Church.

In the year in which Vergerio visited Germany in the discharge
of his legatine duties, the Schmalkaldic allies received a communication
from Henry VIII of England, who had dragged his kingdom
into the schism. He wrote that he was not disinclined “to be admitted
into the Christian league of electors and princes.” It was a move
which was all the more gratifying to the Schmalkaldic League, since
that body had, on a former occasion, sought the friendship of this
powerful monarch. But the subsequent negotiations proved fruitless,
because there was no indication that the King could be induced to
embrace the Lutheran dogmas.

Relative to the divorce of Henry VIII, which constituted the cause and
occasion of the schism, Luther had previously proposed to the king a
surprising, nay, offensive solution. In an opinion on the permissibility of
divorcing Catherine of Aragon, the King’s legitimate wife, which Luther
delivered on September 3, 1531, he openly and candidly pronounced the marriage
of the King to be indissoluble, but, in order to satisfy the King, pointed
out that, with the permission of the Queen, he might “marry an additional
queen, in conformity with the example of the ancients, who had many
wives.”\footnote
{\textit{Briefwechsel}, IX, p. 88; Grisar, \textit{Luther}, Vol. IV, pp. 3 sqq.}
 Owing to his narrow-minded pre-occupation with the Old Testament,
Luther had gradually accustomed himself to regard bigamy as something
to be permitted by way of exception also in the Christian dispensation.\footnote
{\textit{Op. cit.}, Vol. III, pp. 259 sqq.}

But it is not known that he granted this exception in a single
instance at any time prior to this embarrassing memorandum. Later on, however,
he agreed to the bigamous marriage of Philip of Hesse, who in support
of his own cause expressly referred to Luther’s opinion in the case of
Henry VIII.

Melanchthon, on August 23, also declared in favor of the bigamous marriage
of the King, saying: “The King may, with a good conscience (\textit{tutissimum
est regi}), take a second wife, while retaining the first.”\footnote{\textit{Op. cit.}, Vol. IV, p. 5.}

Blinded by passion, Henry VIII insisted upon divorcing Catherine
and, in spite of the adverse decision of Rome, which refused to countenance
bigamy, married Anne Boleyn as his sole queen, cut
loose from the papacy, and, by means of his well-known brutal
measures, compelled the English clergy to submit in all spiritual
matters to his usurped ecclesiastical sovereignty.\footnote{On the attitude of Pope Clement VII, see Grisar, \textit{op. cit.}, Vol. IV, pp. 6 sq.}


As a result of fresh advances made to the Wittenberg theologians
through Robert Barnes, they were now induced to expect the King
to embrace the new doctrine. Luther eagerly hugged the delusion. He
wrote to Chancellor Brück that the King was “ready to accept the
gospel”; that it was necessary to avail themselves of this opportunity
of forming an alliance with him, since such a move would “throw
the papists into confusion.” Melanchthon received 500 gold pieces
from Henry VIII for a work which he dedicated to him. “We have
at least received fifty,” Catherine von Bora said at that time with a
tinge of envy.\footnote{\textit{Tischreden}, Weim. ed., Vol. II, n. 4957.}
 The execution in the year 1535 of those noble and
pious scholars, Thomas More and John Fisher, ordered by the ruthless tyrant
who could not break their opposition, was sanctioned
by the Wittenberg theologians. Melanchthon asserted that the use
of violence against godless fanatics was a divine command.\footnote{\textit{Corp. Ref.}, II, p. 928.}
Luther wrote to Melanchthon in the beginning of December, 1535: “One
is apt to fly into a passion, when one realizes what traitors, thieves,
murderers, yea, veritable devils the cardinals, popes, and their legates
are. Would they had several kings of England to execute them.”\footnote{\textit{Briefwechsel}, Vol. X, p. 275: “\textit{Utinam baberent plures reges Angliae, qui eos occiderent}.”}

About this time, envoys of Henry VIII arrived at Wittenberg
and were gratified to learn that the theologians of that town, including
Luther, had abandoned their view of the validity of the
former marriage of the King, which they now regarded as contrary
to the natural law. In the first months of 1536, articles were drawn
up, designed to effect an agreement with England in matters of
faith, which had been desired by the Protestants. In the judgment of
their Protestant discoverer, these articles reveal “a surprisingly great
accommodation” even in most important questions, such as that of
good works.\footnote{Words of G. Mentz; cfr. Grisar, \textit{Luther}, Vol. IV, pp. 9 sqq.}
 Luther describes them as “the extreme limit of what
could be granted.” Nevertheless, they were not accepted by the
English King. The prospects of an alliance with the Schmalkaldians
began to vanish. An additional reason was because the demands of
Henry VIII to have a commanding influence in the affairs of the
League appeared excessive to the others. The ambitious and agitated
members of the League, as well as Luther himself, believed that the
King intended to usurp the place of the Elector of Saxony in the
leadership of the anti-papal party in Germany.

Henceforth, the Wittenberg theologians were very indignant at
Henry.

Luther, in 1540, referred to him as a worthless wretch (\textit{nebulo}).\footnote{\textit{Tischreden}, Weim. ed., Vol. IV, n. 5139.}

To Luther’s sorrow, his friend Robert Barnes was afterwards burnt at the stake
because he defended the Protestant doctrine on justification. Barnes incurred
the displeasure of the English tyrant also for the reason that he and Thomas
Cromwell had procured a fourth wife for him in the person of Anne of
Cleve; for the author of the English schism, who had divorced successively
Anne Boleyn and Jane Seymour, also became tired of Anne. In the year in
which Barnes was executed, Melanchthon wrote a letter to Vitus Dietrich,
in which he said, respecting Henry VIII: “How very true it is that there is
no sacrifice more acceptable to God than the killing of a tyrant. Would
that God might inspire some courageous man with this idea!”\footnote
{Corp. Ref., HI, p. 1076: “\textit{Quam vere dixit ille in tragoedia, non gratiorem victimam
Deo mactari posse quam tyrannum. Utinam alicui forti viro Deus hanc mentem inserat!}”
On Luther and Cromwell, who was likewise executed, cfr. Grisar, Luther, Vol. IV, pp.
11 sq.}

But though the hopes of the Protestants regarding England were
shattered, their position was strengthened when Philip of Hesse
conquered Württemburg. Philip wrested this country from Ferdinand
of Austria in 1534, by force of arms, in order to reinstate Duke
Ulrich, a follower of the Reformation, who had legitimately forfeited
the crown in 1519. Previous to Philip’s adventure, Luther, according
to his own oral report, had declared that the breach of the public
peace and the spoliation of Ferdinand were “contrary to the Gospel”
and “a stain upon our doctrine.”\footnote{\textit{Tischreden}, Weim. ed., IV, n. 5038.}
 After the Landgrave had subjugated
that country, and, in view of the fact that the terms of the
treaty with its equivocal religious article offered the best prospects
for the introduction of the new religion by Duke Ulrich, Luther
expressed his delight and congratulations to the Hessian court through
the preacher Justus Menius. “We rejoice,” he said, “that the Landgrave
has returned in safety and with the coveted peace. God is
manifestly with this cause. Contrary to our common expectation,
He has transformed fear into peace! He who began this work will
also accomplish it. Amen.”\footnote{July 14, 1534; \textit{Briefwechsel}, Vol. X, p. 63.}
 Luther also informed his friends that
the Landgrave; previous to his attack upon Württemburg, had visited
the King of France and obtained from him a loan of 200,000 crowns
in support of the war.\footnote{\textit{Tischreden}, Weim. ed., Vol. IV, n. 5038, pp. 628 and 630.}


To the best of his ability, Ulrich complied with the expectations of
his friends and began to Protestantize his country.

In the year 1534, the Anabaptists obtained the upper hand at
Münster by the well-known methods so characteristic of them. Their
triumph proved that Luther was not wrong when he suspected that
this furtive sect was capable of anything. Indeed, since the beginning
of the twenties, he might have learned a lesson from the sharp
criticism to which the Anabaptists subjected him. In many respects
this criticism was justified. It was largely based on religious grounds.
But the horrors which the capital of Westphalia was compelled to
suffer, the alleged divine revelations, the cruelties and the polygamy
of the Anabaptist sect, produced an outbreak of terrible fanaticism,
of which the new religion and Luther’s proclaimed Christian freedom
were not guiltless. The sectaries of Münster now write--so Luther
indignantly exclaims--that “there are two false prophets, the Pope
and Luther, but of the two Luther is the worse.”\footnote{Grisar, \textit{Luther}, Vol. III, p. 419.}
 In his preface to
Urban Rhegius’ work against the Anabaptists of Münster, Luther
pronounced a characteristic verdict upon them: “It is perfectly
evident that the devil reigns there in person, yea, one devil sits on
the back of another like the toads do.”\footnote{\textit{Ibid.}}


When Münster, after a siege, had fallen on June 25, and the reign
of terror had been ended by the execution of the ring-leader, John
of Leyden, and his associates, Catholicism found its position in the
north somewhat strengthened.

To offset the growing popular sympathy in favor of the ancient
Church, the Protestants endeavored to fortify themselves by a more
intimate union of Lutheranism with the people of Upper Germany,
who in the eyes of the Lutherans, were still too much inclined towards
Zwinglianism. A complete understanding, especially on the question
of the Eucharist, was all the more urgent, since the religious peace
of Nuremberg was only temporary. The Landgrave of Hesse and
Martin Bucer of Strasburg were active in trying to conclude a more
intimate union with Lutheranism, without, however, wishing to abandon
the Zwinglian denial of the Real Presence. Bucer endeavored
to deceive the others by resorting to ambiguous formulas. Both he
and Philip flattered themselves with the thought that they could succeed
in inducing the people of Upper Germany and even those of
Switzerland to join a league of the followers of the new gospel.

“A union between us and the Sacramentarians is being attempted
with great expectations and longing,” Luther wrote in August, 1535.
He on his part was quite sincere in his intentions. On May 22 of the
following year, he had the satisfaction of seeing the representatives
of Strasburg, Augsburg, Memmingen, Ulm, Esslingen, Reutlingen,
Frankfort and Constance assemble in Wittenberg. They were accompanied
by two Lutheran leaders, Menius of Eisenach and Myconius
of Gotha. Not one of the expected Swiss delegates appeared. All
present adopted the so-called “Wittenberger Concordie,” a product
of Melanchthon’s subtle pen.\footnote{\textit{Op. cit.}, Vol. III, pp. 421 sq.}
 The articles followed Luther in
recognizing the practice of infant baptism and confession. The article
on the Eucharist affirmed that the body and blood of Christ were
“really and substantially” present in the Sacrament, so that even
the “unworthy” verily receive the body and blood of Christ. But
the interpretation which they placed upon the words showed that
the Upper Germans still clung to the view that Christ is present only
by that faith which even the “unworthy” may have and that He bestows
on the communicant, not His flesh and blood, but merely His
grace. Even Melanchthon secretly adopted this interpretation in opposition
to Luther.

The issue now depended upon Luther. For the nonce he was
contented with the closer union which the Upper Germans had
achieved by means of the so-called “Wittenberg Concord.” By
various friendly letters to the Swiss he tried “to calm down, smoothe,
and further matters for the best.”\footnote{\textit{Ibid.}, p. 422.}
 For the time he did not wish to
mention even to Bullinger of Zurich the doctrinal points in which
they differed. His attitude, otherwise so abrupt, waxed strangely
latitudinarian. It was similar to his conduct at the time of his
negotiations with the King of England. Undoubtedly, he thought
to himself--which might excuse him--that by considerate treatment the
Zwinglians as a whole would gradually come over to his
side, in doctrinal matters, especially since the cities of Upper Germany
were in need of assistance. In this case, the wish was father to
the thought. Nevertheless, even Protestant biographers have found
Luther’s way of ignoring the differences inherent in the Concord to
be very peculiar,\footnote{Köstlin-Kawerau, \textit{M. Luther}, Vol. II, p. 348.}
 especially in view of the fact that he had looked
with suspicion upon Bucer’s artful endeavors (\textit{admonui enim ne
simularet}).\footnote{Grisar, Luther, Vol. III, p. 421, note 1.}


Distrustful of the sincerity of the Swiss, he at first distrusted the so-called
Helvetian Confession, drafted in the beginning of 1536 by Bullinger, the
leader of the Zurich faction, with the aid of Bucer. But in May, 1538, filled
with happy expectations, he wrote to Duke Albert of Prussia: “Things have
been set going with the Swiss \dots I hope God will put an end to this
scandal, not for our sake, for we have not deserved it, but for His name’s sake,
and in order to vex the abomination at Rome; for they are greatly affrighted
and apprehensive at the new tidings.”\footnote{\textit{Ibid.}, p. 423.}
 Meanwhile, the Swiss “scandal”
continued and disillusioned him painfully. Bullinger and Leo Judae,
his associate at Zurich, persevered in their sharp opposition to Luther. Their letters
contain bitter denunciations of his doctrine and character.

Leo Judae continued to write in the tone which he had adopted in 1534,
when in a letter to Bucer he complained about Luther’s wanton distortion
of the teachings of Christ; the Apostle Paul, he said, would not have
tolerated such a bishop. It was not sufficient for anyone to preach merely that
Christ is our salvation. Luther was guilty of disgraceful mistakes, and
ignored and execrated everybody else. “Since the Apostolic age,” Leo continues,
“no one has discussed the most sacred things in a manner so disgraceful,
ridiculous, and irreligious as Luther.” And, whilst indulging in such conduct,
he (Luther) set himself up for a pope. Was not a teacher to be judged
in the light of his writings? What would be left of Luther if he were to be
judged by his fruits, according to the saying of Christ? Do his aspersions
reflect moral grandeur? “I cannot imagine that his writings will meet with
the approval of anyone whose mind is not entirely perverted.” “I implore
the Lord Jesus that he make Luther mild and modest. May He bestow His
spirit and His love upon him, that he may discontinue his repugnant agitation;
or--take him from our midst.”\footnote
{Kolde, \textit{Analecta}, p. 229; to which should be added, for purposes of consultation, the
passages in \textit{Histor. Jahrbuch}, 1919 (\textit{Lutheranalekten}, IV), pp. 510 sqq., which have been
supplemented by me from Baum’s Collected Letters in the Strasburg library.}

Luther’s subsequent provocative words against the Swiss and those
who shared their beliefs, such as Schwenckfeld, resulted in deepening
the rancorous sentiments of the Swiss. They were vexed, for instance,
when he averred that Oecolampadius, a Zwinglian, was suddenly removed
from this life by the devil. Bullinger discloses his
mind to Bucer on the “cynical, scurrilous language” of Luther. He
laments his insistence upon his own doctrines and the infallibility
of his own German version of the Bible, “which, after all, was prepared
with too little freedom from prejudice,” etc.\footnote{Grisar, \textit{Luther}, Vol. V, p. 409.}
Later on Bullinger indulged
in even more violent diatribes in his “True
Confession.” In view of the declarations of such leaders among the Swiss
theologians, it is unintelligible how, in the nineteenth century, the
rulers of Prussia could urge the union of Lutheranism with the Calvinists
(Zwinglians) under the name of the Reformed Evangelical
Church.
