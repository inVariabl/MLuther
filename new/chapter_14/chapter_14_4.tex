\section{Luther’s Translation of the Bible (Completed in 1534)}

Luther’s perseverance in working on his German translation of
the Bible enabled him to publish a complete translation in 1534. It
was printed by Hans Lufft at Wittenberg, and bore the title: Biblia,
that is, the Entire Sacred Scriptures, Done into German by Martin
Luther, Wittenberg.

The New Testament had been edited by him in September, 1522,
twelve years previously, after the hasty labors begun during his
sojourn at the Wartburg. It was repeatedly revised in subsequent
editions. Meanwhile, the several parts of the Old Testament were
edited from time to time by the translator. When the entire Bible
finally appeared, it contained Luther’s marginal notes, composed in
short sentences which were intended to explain the text or to render
its religious meaning intelligible to the reader. These notes expressed
the peculiar views of the editor. The different books were accompanied
by prefaces which embody remarkable explanatory and controversial
passages. The entire book was richly illustrated with woodcuts.

In his preface to the book of Job, Luther discusses the pains it
required to make a German version of the Bible. “We all,” he says,
“Master Philip [Melanchthon], Aurogallus, and I, labored with such
care on Job that we were sometimes barely able to get through three
verses in four days. \dots The reader “does not perceive what hindrances
and stumbling-blocks lay in the path he now glides along as
easily as down a greasy pole. To us, however, it cost much toil and
sweat to remove all the hindrances and stumbling-blocks.”\footnote
{\textit{Op. cit.}, Vol. V, p. 497.}

The linguistic excellence of this German version, in contrast with
former translations, which were rather clumsy because too literal, is
so undisputed, even on the part of Catholic critics, that we need not
print any words of appreciation of it here. The immense influence
which this work exercised on the development of the German language is
universally acknowledged. In his larger work on Luther, the
present author has treated of these excellent features extensively, and
discussed in detail the various phases of Luther’s Bible and its relation
to the medieval translations.\footnote
{Grisar, \textit{Luther}, Vol. V, pp. 494--546. See, however, J. M. Lenhart, O. M. Cap.,
``Luther’s Indebtedness to the Catholic Bible'' in \textit{Fortnightly Review}, St. Louis, May, 1930,
pp. 103 ff.}

The linguistic advance was achieved in a twofold manner. In the
first place, Luther adopted the style of the Saxon curia, a purified
form of modern High German, which had developed since the middle
of the fourteenth century. He did not create this language, but made
the greatest contributions toward its propagation in virtue of the
popularity of his literary productions, and particularly his translation
of the Bible. In the second place, he infused into the language which
he had inherited, his own animated spirit, and popularized it by
listening to the genuinely colloquial expressions current among the
common people and introducing these into the literary language. He
himself tells us that he found it of the utmost service to “look into
the jaw of the man in the street,” \textit{i.e.}, to observe closely the speech
of the common people. In rejecting an inadequate expression, he was
apt to say: “No German speaks thus,” or “The German language
does not tolerate that.”\footnote
{Grisar, Luther, Vol. V, p. 503. Cfr. Janssen-Pastor, \textit{Geschichte des deutschen Volkes},
Vol. VII, 14th ed., 1904, p. 648. (English tr., Vol. XIV, PP. 401 sqq.)}

Besides these linguistic characteristics, the method he employed had
two excellent features. He always takes pains to go back to the
original text of Sacred Scripture, whereas former translators had
invariably followed the Latin Vulgate. In this respect Luther followed
the humanistic tendency of his age. Credit is due to Erasmus
for having furnished the first great impulse to the study of the
original text of Scripture by his edition of the Greek New Testament.
The other excellence of Luther’s version consists in the clearness he
imparts by means of circumlocutions to such expressions as would
otherwise be difficult of comprehension. Thus he tried to make the
meaning clear to all. In doing so, however, he proceeded in an altogether
too arbitrary manner, but he succeeded in making the Bible a
readable and popular book.

His translation had a remarkable sale, to which the celebrity of
the author and strong partisan interest naturally contributed no small
share. The latest bibliographer of Luther’s German Bible, Paul Pietsch,
notes 34 Wittenberg impressions of the Weimar Edition alone, and
72 reprints for the rest of Germany during the decade between 1530
and 1540; and again, from 1541 to 1546, the year of Luther’s death,
18 additional Wittenberg impressions and 26 reprints. It is believed,
according to rather reliable investigations, that the press of Lotther
at Wittenberg published no less than 100,000 complete Bibles from
1534 to 1584, to which must be added particularly the Bibles published
by the press of Lufft.\footnote{Grisar, \textit{Luther}, Vol. V, p. 498.}
One might almost say that Germany
at this time was deluged with Bibles. What a powerful influence
Luther’s German Bible must have exercised in enlivening disputations
about religion, is obvious; it is equally obvious that it served
very much to fortify the existing prejudice that the ancient Church
had withheld the Bible from the people and that it was now necessary
to purify it by interpreting it in the light of Lutheran opinions.

To the end of his life, Luther devoted himself assiduously to the improvement
of each subsequent edition of his German Bible. After 1539 special
meetings of scholars were held at Wittenberg to assist him with their
linguistic or theological knowledge in polishing his translation. In enumerating
the names of the regular or occasional members of this “Sanhedrin of the best
people,” Mathesius\footnote{\textit{Ibid.}, pp. 499 sq.}
 mentions Melanchthon, Bugenhagen, Jonas, Cruciger,
Matthew Aurogallus, teacher of Hebrew at the University of Wittenberg,
Bernard Ziegler, a learned Hebraist of Leipsic, and Dr. Forstemius of Tübingen.
Luther’s experienced amanuensis, Rorer, usually kept the minutes of these
meetings. The Weimar edition of Luther’s works prints what is left of the
minutes of the sessions of this Bible-revision committee which met from 1539
to 1541. As a consequence of its labors, the Wittenberg Bibles published by
Lufft from 1540 to 1541 show a decided improvement.

The most celebrated edition of Luther’s German Bible was the so-called
“Normalbibel,” which appeared in 1545. It was the last to
appear during his lifetime. A facsimile of Luther’s handwriting
with parts of the Old Testament, which is reproduced in the section devoted
to the “German Bible” in the new Weimar edition,
enables one to see with what diligence he filed and polished the text,
and how he often struggled to find the best expression for the thought.

Passing from this more exterior appreciation of his work to its
intrinsic value as a translation, the so-called revised Lutheran Bible,
published in 1883, has shown what a large amount of textual corrections
was necessary before Luther’s version satisfied the requirements
of modern critical scholarship. This edition not only eliminated many
errors, but also altered expressions no longer intelligible at the present
time. The revised editions which have appeared since 1883 were an
attempt at improving Luther’s work still more.\footnote
{A severe criticism of the official edition of 1913, by which the edition of 1892 was to
be improved, in the \textit{Christliche Welt}, Marburg, 1913, p. 1010.}
Christian Josias
Bunsen (d. 1860), who was the author of a Protestant “Bibelwerk,”
said there were ``3,000 passages in Luther’s Bible which call for revision.”\footnote{Grisar, \textit{Luther}, Vol. V, p. 511.}

Discussing the defects of scholarship which remained even
after the revision of 1883, the learned Protestant philologist and
Bible expert, E. Nestle, said: “A comparison with the English or
Swiss work of revision shows how much farther we might and ought
to have gone.”\footnote{\textit{Ibid.}, pp. 511 sq.}
 In 1885, the Protestant theologian and orientalist,
Paul de Lagarde, vigorously criticized the Lutheran text as well as
its first jejune official revision of 1883. He prints a long list of passages
which he holds to be manifestly mistranslations of the original,
some of which are arbitrary and evidently made for theological purposes.\footnote{\textit{Ibid.}, p. 512.}

Undoubtedly these theological variations constitute a serious defect
of Luther’s Bible, which diminish its value as a religious work. In the
interests of his new doctrine Luther took the liberty to alter the
sacred text without warrant. Döllinger’s numerous exposures relative
to this point met with the approval of Paul de Lagarde. Janssen has
once more called attention to them in his History of the German
People.\footnote{Döllinger, \textit{Die Reformation}, Vol. III, pp. 140 sqq.; Janssen-Pastor, op. cit., Vol. VII,
14 ed., pp. 654 sqq.}
The Protestant Paulsen in his \textit{Geschichte des Gelehrten Unterrichts}
criticizes Luther’s arbitrary alterations of the Biblical text.
Notwithstanding these criticisms, the mistranslations are retained in
the most recent popular editions of Luther’s Bible.

Thus, in the texts which treat of justification and the significance of the
law, Luther retouches the wording to suit his own doctrine. The law, according
to his translation, “worketh \textit{only} wrath”; “by the law only cometh
the knowledge of sin.” In Rom. 4:15 and 3:20 he interpolates the word
“alone.”

Again, Luther’s reproduction of Rom. 3:28 would have it that man is
justified by faith alone. In this passage, Luther arbitrarily inserted the word
“alone” and tried to justify this insertion as follows: “The text and the
meaning of St. Paul demand, nay, compel this amplification.” This, however,
is merely a requisite of his false theory, which he imputes to the Apostle in
order to square it with his own doctrinal system. The word “alone” in
the Pauline text is an obtrusive recommendation of Luther’s principal
heresy and a subjective falsification. The context makes it quite evident,
however, that, objectively speaking, the real thought of St. Paul could have
been expressed by the word ``alone.''\footnote
{For Luther’s hectoring justification of the “alone” see \textit{supra}.}

In Rom. 3:25 sq., Luther again fortifies his doctrine by adding the
word “alone” and twice inserts the clause: “an offering of justice which
availeth before God,” which is not found in the original text. Luther falsely
translates Rom. 10:4: “For the end of the law is Christ; he that believeth in
him is righteous.” The same is true of Rom. 8:3, where the Greek text
is incompatible with the German rendition.

The illustrations above cited are all taken from St. Paul’s Epistle to the
Romans.

The word “pious” is persistently and designedly substituted for “just.”
Noe, Job, Zacharias, Elizabeth, Simeon, and Joseph, the foster-father of
Christ, are all pious, but not just, as in the original text. To be pious,
according to Luther, is to have faith, and, through faith, imputed justice. He
does not admit real, personal justification. In like manner, Luther everywhere
uses the word “congregation” instead of “church,” in conformity
with the tendency of his doctrine. In reproducing Baruch 6:30, he ridicules
the “priests sitting in their temples with their voluminous copes, with shaven
faces and wearing tonsures.”\footnote{Grisar, \textit{Luther}, Vol. V, pp. 514 sq.}

In addition there are interspersed glosses and prefaces to the several
sacred books which give him a fine chance for indulging in polemics.
He displays a truly marvelous dexterity in interpreting the text in
favor of his new doctrine. This is particularly true of his preface to
the Epistle to the Romans.\footnote{\textit{Op. cit.}, Vol. V, p. 526.}
 In his commentary on the passage which
records the divine foundation of the primacy (“Thou art Peter,” etc.;
Matt. 16:18), he declares that in this passage “Peter” means “all
Christians with Peter,” and the creed of the congregation is the rock
upon which the Church is built. The story of the anointment of
Christ by Mary Magdalen elicits the following comment from him:
“Thus one sees that faith alone makes the work good.”\footnote{\textit{Op. cit.}, Vol., V, p. 518.}
 And so on.
After the appearance of Luther’s New Testament, the Catholic
ducal court of Saxony conceived the idea of publishing Luther’s work
without its distorted reflections on Catholic doctrine. Commissioned
by Duke George, Jerome Emser undertook this task, in 1527. The
conditions then prevalent in the publishing trade sanctioned such a
measure. Emser did not claim that the publication was a new translation.
Luther’s grounds of complaint, both as a matter of fact and
in law, were unfounded, although the procedure is contrary to our
modern ideas. Moreover, the title merely announced that the New
Testament was “restored to its original sense” in this edition.\footnote{On Emser’s German New Testament, cf. Grisar, \textit{Luther}, Vol. V, pp. 518 sqq.}

Later on, after Emser’s death, Augustine Alfeld published a reprint, which
bore the inaccurate title: “The New Testament, translated into German
by the late Emser.” The Catholics, however, were determined to
counteract Luther’s great success by means of other translations. In
1534, John Dietenberger, a Mayence Dominican, published a complete
translation of the Bible, in which he availed himself to a great
extent of Luther’s work. Dr. John Eck proceeded more independently in
his German Bible of 1537, which, however, because of its
stilted style, found but few readers.

Cochlaeus complained that Luther’s work was highly regarded,
even among the common people; that cobblers and old women poured
over it and debated its arbitrary interpretations as if they were the
word of God.\footnote{Grisar, \textit{op. cit.}, Vol. V, p. 529.}

The liberties which Luther took in his appraisal of entire books of
the Bible were fundamentally even more reprehensible than the defects
which have been censured above. It is known that he did not feel
bound by the “canon” in force since the early days of the Church,
which tradition and the teaching magisterium had sanctioned, and
which decided what books constituted the Bible.\footnote{\textit{Ibid.}, p. 521.}

In addition to the illustrations already given, the following examples may
be adduced. The second book of the Maccabees and the book of Esther were
rejected the former because it is “too much inclined to Judaize,” the latter
on account of its “heathen naughtiness.” The Epistle to the Hebrews was
set aside as “a made-up epistle consisting of fragments amongst which,
there is wood, hay and chaff.” The Epistle of St. Jude the Apostle is
ranked “below the chief books [of the Bible].” The Apocalypse he regarded
as “neither apostolic nor prophetic” and said: “Let each one judge of it as he
thinks fit.”

He asserts that “the Epistle of St. James,” which has previously been discussed,
“justifies [good] works,” and compared with other books of the
Bible, which (he maintains) clearly proclaim the doctrine of justification
by faith alone, is “but an epistle of straw,” which has “nothing
evangelical about it.” Of this verdict, one of the most celebrated modern
Protestant Bible scholars, Theodor von Zahn, says that it is “an act of injustice
as incomprehensible as it is regrettable.”\footnote{\textit{Ibid.}, p. 523}
 It is quite comprehensible,
however, when one takes into consideration the stupendous levity with which
Luther regarded his doctrine as an infallible criterion.

In rejecting the canon of the Bible, Luther destroyed the basis on
which the authority of the Sacred Book had been founded. It was
tragic that no Christian writer ever inflicted so much damage upon
the Book of Books as the man who boasted of having favored it in so
high a degree and represented it as the great, nay, the sole source of
faith.

It is psychologically interesting to follow the motives and the
general ideas which guided Luther in the course of his protracted
work of translating Sacred Scripture, expressed in his own words.
He intends to show the “papists” that they are not competent to
translate the Bible properly, because they do not possess the “mind of
Christ,” and hence they ought to leave it “in peace” and undisturbed.\footnote{Thus at the head of the edition of 1545.}

“It is not an easy matter,” he says, to find “others as sincerely devoted
to the Bible as we are here at Wittenberg, who have been the first to receive
the grace to reveal once more the Word of God, unadulterated and purified.”
“None of them knows how to translate correctly.”\footnote{Grisar, \textit{Luther}, Vol. V, pp. 526 sq.}
 The new Bible, therefore,
was to put the papists thoroughly to shame. Above all else, it was to
show how mistaken they were when they reduced it to a “code” for the
performance of good works, since it “condemns such works and demands
faith in Christ.”\footnote{\textit{Ibid.}, p. 528.}

Hence, Luther was preeminently swayed by a polemical purpose. He
wished to see the Bible read by everybody. Rich and poor alike should be
enabled to judge that his doctrine was the only true doctrine. Hausrath,
a Protestant biographer of Luther, writing in complete accord with the
sentiments of his hero, says: “Only now could the burghers feel that they
had attained to manhood in the matter of religion, and that the universal
priesthood had become a reality. The head of each household now had the
wellspring of all religious truth brought to his very door \dots For a while
this might lead to strange excesses, as the theology of the New Prophets
showed.” Still, “the advent of the German Bible was the dawn of freedom.”\footnote{\textit{Luther’s Leben}, I, p. 136; quoted by Grisar, \textit{ibid.}, p. 529.}

It is easy to understand the conscious pride which filled Luther upon the
completion of his work. “St. Jerome and many others have made more mistakes in
translating than we.” “I know that I am more learned than all the universities,
those sophists by the grace of God.” He invited those who censured him to “do
even the twentieth part” of what he had done. “Since the Church has existed, we
have never had a Bible like this one.” Incidentally it is a real pleasure to
him that he was able to rouse the fury (\textit{furias concitare}) of the
papists by means of such a great work.\footnote{\textit{Ibid.}, p. 331} He
realized withal that he had to overcome many temptations whilst occupied with
this work; thus he tells us that, whilst engaged in translating the story of
Jonas, he “looked into the belly of the whale, where everything seemed given
over to despair.”

Besides his habitual interior struggles he was oppressed with the idea that his
laborious task “would not be duly appreciated even by his own
followers.”\footnote{\textit{Ibid.}, p. 532.}

He felt an interior satisfaction, however, when he recalled that his text
counteracted the Jewish commentators, “who cannot know or understand what is
said by Moses, the Prophets and the Psalms.”\footnote{\textit{Ibid.}, p. 533.}

The refutation of Jewish errors embodied in Luther’s Bible is a
satisfactory feature of his work.
Julius Köstlin declares that in his translation of the Bible, Luther
“bestowed on his German people the greatest possible gift” by making of
the Book of Books “an heirloom of the whole German nation.”

The whole German nation? From what has been said above it can be
seen that Luther’s much-lauded translation was rather a piece of subjective
propaganda put forth in the interests of his own party. And
as regards “his German people,” it was precisely while he was engaged
in this work that he applied to the German nation opprobrious
epithets which place “the greatest gift” in a peculiar light. Thus the
reproaches of the Prophets inspired him to use the following language:
“I have begun the translation of the Prophets--a work that is quite
in keeping with the gratitude I have hitherto met with from this
heathenish, nay, utterly bestial nation.”\footnote{\textit{Ibid.}, p. 534.}

A word in reference to the illustrations which accompanied Luther’s
German Bible. The Catholics, who at that time still constituted a
majority of the German nation, were aggrieved to see how their
Church was ridiculed by the pictures contained in the complete Lutheran
Bible of 1534, just as had been the case in the previously published New
Testament. The polemical illustrations contained in the
New Testament were reproduced in the complete Bible. The Babylonian
harlot and the dragon once more appear crowned with the
papal tiara. On the title-page of the Wittenberg edition of Luther’s
complete works, issued by himself in 1541, there is a picture, presumably
drawn by the elder Cranach, depicting the ancient Church
in the act of driving men into hell; whereas the new Gospel leads
them to heaven.\footnote
{The frontispiece of 1541 is reproduced in its original size in Grisar and Heege,
\textit{Luthers Kampfbilder}, Heft III, plate 9; cfr. the same work, pp. 19 sqq.}
A furious looking devil in the shape of a beast,
with a cardinal’s hat on its head, and death in the form of a skeleton,
are depicted as driving a man, clad only in a loin-cloth, into the yawning
abyss of hell, where the pope, crowned with the tiara, and two
other forms are burning. The obverse side of the picture glorifies the
new Church. John the Baptist leads a nude penitent to the Crucified
Saviour, from whose side a stream of blood gushes over a sinner who
has been saved by faith. Verily, the Catholics could not regard Luther’s
Bible as “the greatest gift to his German nation.”

When the Catholic spokesmen took into consideration Luther’s
principles relative to the rank and use of the Bible, they regarded
themselves as justified in assuming an attitude of severe condemnation.\footnote
{Cf., \textit{e.g.}, Johann Fabri, quoted in Grisar, Luther, Vol. V, p. 529.}
These principles contained within themselves the seeds of religious
chaos. Luther held that beside, and in addition to, the words
of the Bible, the “spirit” was to be the true touchstone of orthodoxy.
The spirit would teach everyone to understand the sacred text. In case
of doubt, the interior “feeling,” which comes from above, must
assume the direction of reason. Everyone stands “for himself.” To the
pious, according to Luther, the Word of God is perfectly clear; but
he frequently emphasizes its obscurity. He holds that, owing to the
absurd interpretations of many obscure passages, the Bible must almost
be styled “a heretical book.” Hence, in order to avoid theological
anarchy, he makes the self-contradictory demand that the interpretations
of the Wittenberg school, \textit{i.e.}, his own tribunal, should always
be followed. The “external Word,” upon which he insists in opposition
to the spirit of arbitrariness, is equivalent to his own word. Yea,
penalties are to be inflicted upon those who contemn the magisterium
exercised by himself.\footnote
{Grisar, \textit{Luther}, Vol. IV, pp. 387 sqq., 420 sqq.; Vol. VI, pp. 237 sqq., 279 sqq.}

It should not be overlooked, however, that Luther’s interpretation
of the Bible has the undisputed merit that, in contrast with the older
allegorical method of exposition, he always tries to establish the literal
sense, and for this purpose lays under contribution the study of languages.
Not infrequently he employs to advantage the exegetical
works produced before his time, for instance, those of the celebrated
Nicholas of Lyra. A well-known saying of his opponents was: “\textit{Si
Lyra non lyrasset, Lutherus non saltasset}.” If Lyra (the lyre) had
not played, Luther would not have danced.\footnote{Grisar, \textit{op. cit.}, Vol. V, p. 535.}
 Generally speaking,
however, Luther treated the Biblical lore of the past with such supreme
disregard that his neglect of the older commentators redounded to the
very great disadvantage of his own work. In addition to this scientific
defect of Luther’s Bible, the Catholic is bound above all else to
take into account the fundamental detriment to exegesis resulting from
Luther’s abandonment of ecclesiastical tradition. Catholics believe
that the Church has been constituted by God the official interpreter
of Sacred Scripture. Her voice, resounding down the ages, is sufficient
guaranty that her children will not go astray in their study of the
Bible. Luther repudiated her guidance, to his own detriment as well
as to that of his followers, even of those who were sincere in their intentions
and inclined to positive religion. He saw this fact with his
own eyes and bitterly rued it. He speaks with horror of the “rubbish
in Scripture.”

It is easy to refute Luther’s assertion that he “pulled the Bible from
underneath the bench,” where it lay buried, owing to its complete
misunderstanding under the papacy, and because it had been denied
to the laity and the clergy.\footnote{\textit{Ibid.}, pp. 536 sq.}

Catholic exegetes are accused of having ignored the fact “that
Christ forms the true content of Scripture.” This accusation has been
refuted by a mass of quotations from writers like Augustine and
Thomas Aquinas, and also from Luther’s older contemporaries, such
as J. Perez of Valencia.\footnote{\textit{Ibid.}, p. 541.}
 Still more striking is the assertion that the
Bible was not read during the Middle Ages, nay, that its reading was
prohibited by the Church.

Frederick Kropatschek, a Protestant, states in his scholarly work,
Das Schriftprinzip der lutherischen Kirche (1904): “If everything
be taken into account, it will no longer be possible to say, as the old
polemics did, that the Bible was a sealed book to both theologians and
laity. The more we study the Middle Ages, the more this fable tends
to dissolve into thin air \dots The Middle Ages concerned themselves
with Bible translation much more than was formerly supposed.”\footnote{\textit{Ibid.}, p. 536.}
Similar admissions could be cited from other Protestant scholars
such as Walther Köhler, Ch. Nestle, J. Geffcken, W. L. Krafft, E.
v. Dobschütz, O. Reichert, G. W. Meyer, A. Risch, etc.\footnote{\textit{Ibid.}, pp. 545 sq.}
Above all
it is now agreed that the laity were never prohibited from reading the
Bible, as has often been alleged. It was only when there was danger
that the Bible would be abused during menacing heretical movements,
that the Church authorities from time to time adopted measures forbidding
the laity to read the Bible. Historical researches show that the
Bible was widely circulated both in the original languages and in translations,
in manuscripts and printed editions, in the age that preceded
Luther. On account of the importance of the subject, a synopsis of
these researches is hereby offered.

Wilhelm Walther of Rostock and Franz Falk, a scholarly clergyman of
Mayence, have devoted themselves with distinction to a study of this matter.
In recent times, a German-American scholar, W. Kurrelmeyer, has edited
in installments \textit{Die erste deutsche Bibel} (The First German Bible), in the
Library of the Literary Society of Tübingen; these installments are supplied
with critical contributions from all German translations prior to Luther.\footnote
{See Ch. Nestle in the Protestant \textit{Enzyklopädie für Theologie}, Ergänzungsband XXIII,
p. 317.}

The oldest complete printed German Bible appeared in 1466 and was published
by Mentel of Strasburg. It was followed by thirteen other editions,
some of which deviated to some extent from the first, but all appeared
before the publication of Luther’s Bible. The name of the editor of the Bible
published by Mentel cannot be established with certainty. In consequence
of the republication of this version, its text had become a kind of German
Vulgate.

Dr. Falk has established the fact that no less than 156 different editions
of the Latin Bible were printed in the period between 1450 and 1520.\footnote{Grisar, \textit{Luther}, Vol. V, p. 536.}

In addition, there are the many extant manuscripts, which have been classified
by Walther, as well as numerous prints and manuscripts of separate
parts of the Bible, such as the Psalter and the Gospels and the Epistles of the
ecclesiastical year.

The latter, being lessons taken from the Old and the New Testament,
had been translated and collected in so-called ``\textit{plenaria}'' or postils, which
were to be found everywhere in the hands of the faithful. In lieu of the
complete Bible, which was expensive and only partially intelligible to many,
these postils supplied the people with reading material which was adapted
to their needs and was explained to them during divine service. In virtue of
these “\textit{plenaria},” all classes of the people were well grounded in the most
essential and instructive portions of the Bible.

Relative to the partial and complete translations of the Bible, Sebastian
Brant could truly say of the pre-Lutheran epoch: “Every
country is now filled with Sacred Scripture.”\footnote{\textit{Ibid.}, pp. 536, 540.}

The extant German translations of the Bible, particularly those contained
in the “\textit{plenaria},” were by no means unknown to Luther. It
may also be assumed--in fact it has been specifically demonstrated in
many instances--that he made use of them more or less in his translation,
as any scholar would have done when engaged in a similar
work. His merit of having proceeded in an independent manner remains undiminished,
even if it is to be assumed that his translation
was influenced to some extent by a fixed German vocabulary expressive
of Biblical words and phrases.\footnote
{\textit{Ibid.}, p. 460. On “Luther’s Indebtedness to the Catholic Bible” see J. M. Lenhart in
the \textit{Fortnightly Review}, St. Louis, Mo., XXXVII, 5 (May, 1930), pp. 103 ff.}

A remark remains to be made on the practical use of the Bible during the
later Middle Ages. It was an abuse that the Bible was excessively allegorized,
and that it was not sufficiently used in scholastic
disputations in comparison with philosophy; but it always retained its
prestige in the pulpit and religious literature. Popular devotional literature
furnishes a striking refutation of the claim that the Reformation rescued
the Bible from oblivion, even if the statistics and facts
which have been adduced above were unknown. The literature composed for
the instruction and religious edification of the faithful is
replete both with Biblical passages and the spirit of the Bible; at times,
it is even excessive in its application of Holy Writ.

The legend of the chained Bible, which is occasionally encountered even
at the present time, is little less than grotesque. In Protestant popular tracts
young Luther’s discovery of the Bible at Erfurt is associated with the queer
notion that the copy which he happened upon while a student was
fastened by a chain in the library. Copies of the Bible and other books intended
for the common use of the public were frequently chained, to safeguard
them from unjustifiable appropriation or removal from the room in
which they were intended to be used. This very practical custom, still in
vogue at present in the parlors of Italian convents, has given rise to the
false notion that the Bible was kept chained in the Middle Ages.
