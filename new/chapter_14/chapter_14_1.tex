\section{Luther’s Change of Opinion Relative to Armed Resistance}

After the diet of Augsburg, a striking change took place in Luther’s attitude
toward the question of armed resistance to the Emperor.
The stringent measures of the “Reichskammergericht” announced in the
“Abschied” of the diet against the secularization of
Catholic church property and the rigorous steps which were generally adopted
against the new religion, produced a definite attitude on
the part of Protestants. The jurists of Electoral Saxony expressed
themselves to Chancellor Brück increasingly in favor of preparedness
and forceful resistance to the imperial mandates. Philip of Hesse, who
had formed ambitious projects against the Emperor, was prepared to
open hostilities at the first favorable opportunity and counted on the
support of all those who shared his ideas.

Luther personally would have preferred a policy of watchful
waiting without the use of violence. He held that the execution of
the resolutions adopted at Augsburg should be demonstrated as impossible
by permitting the innovations to progress in a peaceful way.
He would have been pleased if things had been left as they were and
time thus gained for the further propagation of the new gospel.
However, circumstances forced him to change his attitude—a change
which led to self-contradiction and open sanction of that armed resistance
which he had previously condemned.

His former teaching had been that it was not permissible to meet
violence with violence, especially against the Emperor; that, according
to the gospel, unjust persecution was to be suffered with Christian
resignation and in the expectation of final assistance from above.
Despite his blustering, various reasons determined him to issue such
declarations repeatedly. In the first place, he was influenced by the
after-effects of the mystical idealism which he had developed in the
monastery, and according to which the kingdom of God knew only
a yielding disposition, humility, and submission; every true Christian
must allow himself to be “oppressed and disgraced,” but the defense
of rights was the business of the secular authorities. In addition,
he had firmly persuaded himself that God would and must prosper his
cause. Luther was convinced that he was right and, consequently,
enjoyed the protection of Heaven, while, on the other hand he
“knew” that “the Emperor is not and cannot be sure of his cause.”\footnote{Grisar, \textit{Luther}, Vol. III, pp. 47 sq.}

Secondly, he was influenced by the consideration that, if military
force were to be invoked in support of his gospel, the prospects of
success on the part of the princes who favored him were unfavorable
and that the frightful misfortune of war had better be averted for
humanitarian reasons. “May God preserve us from such a horror!”\footnote{\textit{Ibid.}, p. 49.}

He exclaims, since, as he puts it, a breach and disturbance of the
public peace would be “a stain on our doctrine.”\footnote{\textit{Ibid.}, p. 45.}
His religious
innovations would more readily recommend themselves to the princes
if they made their way as peaceably as possible, without any disturbance
and conflict.

Hence his assertion, repeated particularly during the first years
after his apostasy, that the Word alone must accomplish all
things, and his appeal to “the breath of Christ,” which, according to
the Bible, is to destroy Antichrist. Even as late as 1530, Elector John
of Saxony was in complete accord with Luther’s idea that armed resistance
to the Emperor was unlawful.\footnote{Köstlin-Kawerau, \textit{M. Luther}, Vol. II, p. 249.}


Nevertheless, it is well known that, from the beginning, Luther
allowed himself again and again to clamor for war. It was a demand
born of his agitated temperament and his ardent zeal for his gospel.
Thus, in 1522, he declared that “every power must yield to the Gospel.”\footnote{Grisar, Luther, Vol. 111, p. 76.}

“Not only the spiritual, but also the secular power, must yield
to the Gospel whether cheerfully or otherwise.” “Not a hair’s breadth
will I yield to the opponents”; and: “If war will ensue, let there be
war” (1530). “If Germany will perish, if it will go to rack and
ruin, how can I help it? I cannot save it.” In 1523 he had already
conceded to the Elector Frederick the right to bear arms in defense
of the new doctrine, provided he did this “at the call of a singular
spirit and faith,” not as a Christian prince engaged in his own affairs,
but as a stranger who comes to the rescue. In an opinion which he rendered
for the successor of Frederick, in 1 529, he said: “There must be
no resistance unless actual violence is done or dire necessity compels.”\footnote{The cited passages \textit{ibid.}, pp. 45--50.}


These utterances laid the foundation for the change of mind which
came over him in 1531.

He was still wavering when, just prior to the assembly of the diet of
Augsburg, he explained to his Elector in a rather lengthy memorandum
that military resistance “can in no wise be reconciled with Scripture.”
“In the confusion and tumult which would ensue,” he says,
“everyone would want to be emperor, and what horrible bloodshed and
misery would that not cause!” “A Christian ought to be ready to suffer
violence and injustice, more particularly from his own ruler.” It were
preferable to sacrifice life and limb, \textit{i.e.}, endure martyrdom.”\footnote
{Erl. ed., Vol. LIV, pp. 138 sq. (\textit{Briefwechsel}, VII, p. 239); Grisar, \textit{Luther}, Vol. III,
p. 52.}
It seems
that he was at that time very much frightened at the thought of the
“disgrace” which would attach to his doctrine if it stirred up a religious
war. This memorandum was formulated after Luther had conferred
with his three advisers, Jonas, Bugenhagen, and Melanchthon. It was,
however, kept secret, perhaps in order to avoid any friction with the
electoral jurists, who were rather inclined to war. An abstract was sent
only to Spengler at Nuremberg, which city was likewise disposed to disapprove of resistance.

This memorandum caused the adherents of the new religion great
embarrassment later on, after Luther had changed his mind and the
Protestant Estates, appealing to his authority had entered the Schmalkaldic
War. Cochlaeus obtained its text, and published it with glosses
directed against its vacillating author. The courageous abbot, Paul
Bachmann of Altenzelle, appended a reply to Luther, in which he says
that Luther had ever raved against the Emperor and the Pope, as
though they were worse than the Turks; but in this memorandum,
“being apprehensive of resistance, the old serpent turns round and
faces its tail, simulating a false humility, patience, and reverence for
the authorities, and says: A Christian must be ready to endure violence
from his ruler.”\footnote{Grisar, \textit{op. cit.}, Vol. III, p. 63.}
Driven into a corner, Luther’s advisers, who
had approved the memorandum, shortly after his demise tried to impugn
its authenticity. Melanchthon did so incidentally, Bugenhagen
of set purpose, for which he was justly reproved in public by Ratzeberger,
a well-informed friend of Luther.\footnote{\textit{Ibid.}, p. 74}

The rapidity with which Luther changed his mind after formulating the
above-mentioned memorandum was chiefly owing to the
decision of the diet of Augsburg which was so unfavorable to him.
His inclination to offer resistance manifests itself at once in his
“Warning” to his dear Germans, which he composed shortly after
the close of the diet and which has been discussed above, as well as
in the tract which he penned against “the assassin of Dresden.”

At the end of October, 1530, he was obliged to repair to Torgau
with Melanchthon and Jonas for a conference concerning the question
of resistance. There he met the legal advisers of his territorial
lord and, perhaps, those of other princes. He was unable to resist
their influence. At first he refused to declare himself, claiming that
the question did not concern him, since it was his sole duty as a
theologian to teach Christ. The laws of the Empire ought to be
obeyed; what these were, he neither knew nor cared to know. But
the jurists insisted that he express an opinion on a lengthy document
which they had drawn up in justification of war.
After enumerating alleged juridical and theological reasons, this document
asserted that the “proceedings and acts” of the Emperor “were in
contravention of law”; that, so far as the decision of this matter is
concerned, he was “but a private individual.” Luther did not contradict
them, but placed the responsibility upon the jurists, leaving
them to proceed as they pleased. Conjointly with Melanchthon and
Jonas he declared that, up to now, he and the theologians, precisely
as theologians, had taught that it was “not right to offer outright
resistance to the authorities,” but they did not know that, as the
jurists pointed out, the authorities themselves conferred the right of
armed resistance in cases such as that under consideration. Hence,
they could “not quote the Bible against such resistance, when necessary
for defense, even if it were against the Emperor in person.”
Taking these things into consideration, all three declared that the
warlike preparations were justified.\footnote{\textit{Ibid.}}

In writing about this affair to Link, Luther said: “In no wise
have we counseled the use of force. But if the Emperor by virtue of
his laws concedes the right of resistance in such a case, then let
him bear the consequences.” In that event, he says, the princes,
\textit{qua} princes, and in this capacity only, may offer resistance. “To
a Christian, nothing [of that sort] is lawful, for he is dead to the
world.”\footnote{\textit{Briefwechsel}, VIII, p. 344; January 15, 1531; Grisar, Luther, Vol. 11, p. 60.}


These vexatious and threadbare explanations did not, however,
satisfy Link and his followers at Nuremberg, who continued to
side with Spengler in his stand against resistance. Neither would the
people of the margravate of Brandenburg listen to this over-refined
casuistry, but persisted in their refusal to offer resistance. Luther
relied all the more on his pretext, that this question should be decided
by the politicians and jurists; that he, as a theologian, was
obliged to refrain from offering advice; and said he abstained from
offering counsel for reasons of a more lofty piety; and that he would
have the entire matter rest not on “the power of man,” but on that
of God; for, then only “it would turn out well, even if it be a
downright error and sin.”\footnote
{Letter to a citizen of Nuremberg, March 18, 1531; Erl. ed, Vol. LIV, p. 221
(\textit{Briefwechsel}, VIII, p. 378); Grisar, \textit{op. cit.}, III, p. 62.}
Despite these extenuating phrases, the
jurists, as was to be expected, made use of his declaration given at
Torgau as a simple and complete acquiescence in their endeavor to
bring about the formation of the military League of Schmalkalden.
Spengler, who had a copy of Luther’s opinion of March, 1530, in
which he strictly declined to approve of resistance, wrote from
Nuremberg that he was amazed “that Doctor Martin should so
contradict himself.”\footnote
{See Enders in \textit{Briefwechsel}, VIII, p. 298; Grisar, \textit{op. cit.}, p. 59.}
 Besides the pressure of the jurists, the following
circumstances may have contributed to change Luther’s attitude: First,
the prospect of successful resistance as a result of the
increased opposition to Rome, especially in consequence of the defection
of England initiated at that time, and the prospective
Protestantization of Württemburg; secondly, the Emperor’s preoccupation
with the hostile king of France; and, finally, the weakness and
indecision of some of the Catholic estates which manifested itself at
the diet of Augsburg.

As a matter of fact, Luther waxed ever more positive in his demand
for armed resistance after 1531.

“We may not deviate a hair’s breadth on the plea of disturbing the
public peace,” he wrote. “We must trust in God, who has thus far
protected His Church during the most terrible wars.”\footnote
{Grisar, \textit{Luther}, Vol. III, pp. 78 sq.}
In 1536,
subscribing to a document which emphasized the duty to offer
armed resistance for the protection of the Gospel, he said: “I, Martin
Luther, will do my best by prayer and, if needs be, with the fist.”\footnote{\textit{Ibid.}, p. 433.}

In his excitability and tempestuous nature he even demanded the
infliction of the death penalty upon the pope and his rabble, \textit{i.e.},
his defenders. In 1540, he gave notice that there was no other choice
“but to take up arms in common against all the monks and shavelings;
I too shall join in, for it is right to slay the miscreants like mad
dogs.”\footnote{\textit{Op cit.}, Vol. VI, p. 247.}
 In this same year he also said: “We shall not prevail against
the Turks unless we slay them in time, together with the priests, and
even hurl them to death.”\footnote
{\textit{Op. cit.}, Vol. III, p. 69. Cfr. Grisar and Heege, \textit{Luthers Kampfbilder}, n. 4 (\textit{Lutherstudien}),
pp. 138 sq., where more pertinent passages as well as Luther’s cartoon: “The Pope
and the Cardinals on the Gallows,” may be found (p. 32); cfr. also \textit{ibid.}, n. 31.}
Neither in this nor in similar passages
is there any question of defense against force, but rather advocacy
of bloody aggression. Yet, even if these ravings are outburst of impetuous
anger, and not the expressions of calm deliberation, they
indicate a deplorable state of mind and served upon occasion to
justify the bloody crimes which resulted from the conflict between
the advancing party of religious reform and the defenders of the
old order.
