\section{Luther’s First Biblical Lectures—His Mysticism}

The first lectures of the new professor of Biblical science were
delivered in the years 1513 to 1515 and dealt with the Psalms.
Those of his pupils who were monks and had to recite the Divine
Office in choir, were particularly interested in the Psalms. The interpretation
offered them by Luther has been preserved in his works. It is, however, not an explanation made in accordance with
our modern ideas, but rather a collection of allegorical and moral
sentences based upon the text, as was the custom in those days. Luther
justly abandoned this allegorical manner of interpretation in later
life. Non-Catholics have endeavored, without justification, to discover
in these lectures the germs of his later teaching. His manner
of expression is often indefinite and elastic and generally more rhetorical
than theologically correct. His teaching on justification,
grace, and free will, is, like his other doctrines, still fundamentally
Catholic, or at least can be so interpreted if the dogmatic teaching
of the Church is properly understood. Still there are a few indications
of the coming change. Take, for instance, his emphatic
assertion that Christ died for all men and his exaggerated opposition
to the doctrine of justification by means of good works.\footnote
{Cf. H. Boechmer in \textit{Allg. Evang.-Luth. Kirchenzeitung}, 1924.}
In general
these lectures reveal talent, religious zeal, and fertile imagination--
qualities which must have charmed his auditors to an unusual degree.

Luther was very amiable and communicative towards his pupils.
His entire personality, the very gleam of his eye, exerted a certain
fascination over those who associated with him.

The young professor of Sacred Scripture displayed a pronounced
inclination towards mysticism. Mysticism had always been cultivated
to a certain extent in the religious orders of the Catholic Church.
The reading of Bonaventure had pointed Luther, even as a young
monk, to the pious union with God at which Mysticism aims. Toward
the close of his lectures on the Psalms, he became acquainted with
certain works on Mysticism which he imbibed with great avidity.
They were the sermons of Tauler and the tract “\textit{Theologia deutsch}.”
They dominate his thoughts in 1515. Although these works were not
designed to do so, they helped to develop his unecclesiastical ideas.
His lively experience of the weakness of the human will induced him
to hearken readily to the mystical voices which spoke of the complete
relinquishment of man to God, even though he did not understand
them perfectly. His opposition to good works opened his mind to a
fallacious conception of the doctrines of those books of the mystical
life. It appeared to him that, by following such leaders, his internal
fears could be dispelled by a calm immersion in the Godhead.

John Tauler, an ornament of the Dominican Order (died in
1361), was a famous preacher in the pulpits of Strasburg. His
writings and sermons are filled with profound thoughts and have
a strong popular appeal. They abound in attractive imagery and
are replete with devotion. Tauler stands four-square on the basis of
Catholic teaching and the best scholastic theology. Two points in
his mystical admonitions found a special echo in Luther’s soul, namely,
the interior calmness with which God’s operations are to be received,
and the darkness which fills the souls of pious persons, of whom
he speaks consolingly. Luther, however, introduced his own erroneous
ideas into the teaching of Tauler. His demand that the soul be calmly
absorbed in God, Luther interpreted as complete passivity, yea,
self-annihilation. And what Tauler says concerning trials arising from
the withdrawal of all religious joy, of all emotions of grace in the
dark night of the soul, he referred directly to his own morbid
attacks of fear, to which he endeavored to oppose a misconceived
quietism, a certain repose generated by despair. In brief, he tried to
transform all theology into what he called a theology of the Cross.
Misconstruing Tauler’s doctrine of perfection he would recognize
only the highest motives, namely, reasons of the greatest perfection
for himself as well as for others. Fear of divine punishment and hope
of divine reward were to be excluded.

These were extravagances which could not aid him, but, on the
contrary, involved great danger to his orthodoxy; in fact, constituted
a serious aberration. But he trusted his new lights with the utmost
self-confidence. Writing of Tauler to his friend Lang at Erfurt, who
was also fascinated by the works of that mystic, Luther compares
him with contemporary and older theologians and says that while
Tauler was unknown to the Schoolmen, he offered more real theology
than the combined theological professors of all the universities.\footnote
{Grisar, \textit{Luther}, Vol. I, p. 87.}

The other mystical writer who interested him, was discovered by
Luther in a manuscript. He lived in the fourteenth century and was
the author of the “\textit{Theologia deutsch}.” His name is unknown to us.
He was a priest at Frankfort on the Main. His work, which is a
didactic treatise on perfection, is Catholic, although not exempt from
obscurities. Luther esteemed it as a book of gold, particularly in
view of its praise of the sole domination of God in the soul that
suffers for Him. He edited this book, at first incompletely, in 1516,
then in its entirety, in 1518. It is remarkable that a book on Mysticism
was his first publication. Soon he occupied himself with the
mystical writings of the so-called Dionysius the Areopagite, the father
of Mysticism, and with those of Gerard Groote, a more modern
author.

His style in those days, as also later on, reveals how profoundly
he was animated by the devout tone of these mystics. Thus, in
writing to George Leiffer, a fellow-monk at Erfurt, who was afflicted
by persecutions and interior sufferings, he says (1516): “Do
not cast away thy little fragment of the Cross of Christ, but deposit
it as a sacrosanct relic in a golden shrine, namely, in a heart filled
with gentle charity. For even the hateful things which we experience,
are priceless relics. True, they are not, like the wood of the
Cross, hallowed by contact with the body of the Lord, yet, in as far
as we embrace them out of love for His most loving heart and His
divine will, they are kissed and blessed beyond measure.”\footnote
{\textit{Briefwechsel}, I, p. 68 (April 15, 1516); Grisar, \textit{Luther}, I, 88.}
In discussing the idea of self-annihilation under the guidance of
God, which was his favorite thought in these days, he shows that he
has gone astray. He says that man should not choose among good
works, but abandon himself to God’s inspiration, as the steed is
governed by the reins. In an address delivered in 1516 he declares:
“The man of God goeth, whithersoever God directs him as a rider.
He never knows whither he is headed; he is passive rather than active.
He journeys ever onward, no matter what the condition of the
road, through water, mud, rain, snow, wind, etc. Thus are the men
of God who are led by the divine spirit.”\footnote{Grisar, \textit{Luther}, VI}
Such are the doctrines
which he opposed to those who became distasteful to him on account
or their insistence on good works and what he called their Pharisaical
observance of external practices.

On May 1, 1515, a chapter of the Augustinian congregation was
held at Gotha under the presidency of Staupitz. Luther preached the
sermon at the opening assembly. The theme which he selected
treated of the contrasts which must have developed in the monasteries
of the congregation, namely, the “little saints” and their
calumnies against the monastic brethren who disagreed with them in
matters of discipline. With extreme acerbity, and employing the
crudest and most repulsive figures of speech, he scourged their criticism
of others as inspired by love of scandal and malevolent detraction.\footnote{\textit{Ibid.}, I, pp. 69 sq.}
Apparently the majority of the brethren of his Order sided
with him, for they elected him to the office of rural vicar, \textit{i.e.},
special superior of a number of monasteries as the representative of
Staupitz.

At stated times he visited the monasteries thus entrusted to him.
There were eleven of them, including Erfurt and Wittenberg. After
the middle of April, 1516, he made a visitation of the congregations
of the Order at Dresden, Neustadt on the Orla, Erfurt, Gotha,
Langensalza, and Nordhausen. The letters written by him during
his term of office as rural vicar, which normally lasted three years,
contain practical directions and admonitions concerning monastic
discipline and are, in part, quite edifying. Some of his visitations,
however, were conducted with such astonishing rapidity that no
fruitful results could be expected of them. Thus the visitation of
the monastery at Gotha occupied but one hour, that at Langensalza
two hours. “In these places,” he wrote to Lang, “the Lord will work
without us and direct the spiritual and temporal affairs in spite of
the devil.”\footnote{\textit{Ibid.}, I, 65.}
At Neustadt he deposed the prior, Michael Dressel, without a hearing,
because the brethren could not get along with him.
“I did this,” he informed Lang in confidence, “because I hoped to
rule there myself for the half-year.”\footnote{\textit{Ibid.}, I, 266.}

In a letter to the same friend he writes as follows about the engagements
with which he was overwhelmed at that time: ``I really
ought to have two secretaries or chancellors. I do hardly anything
all day but write letters \dots I am at the same time preacher to
the monastery, have to preach in the refectory, and am even expected
to preach daily in the parish church. I am regent of the
\textit{studium} [\textit{i.e.}, of the younger monks] and vicar, that is to say prior
eleven times over; I have to provide for the delivery of the fish from
the Leitzkau pond and to manage the litigation of the Herzberg
fellows [monks] at Torgau; I am lecturing on Paul, compiling
an exposition of the Psalter, and, as I said before, writing letters
most of the time \dots It is seldom that I have time for the recitation
of the Divine Office or to celebrate Mass, and then, too,
I have my peculiar temptations from the flesh, the world, and the
devil.”\footnote{\textit{Ibid.}, I, p. 275, October 26, 1516.}
