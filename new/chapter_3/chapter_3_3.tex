\section{Opposition to Self-Righteousness and Religious Observance}

Contradictory conceptions of monastic life continued to be harbored
in the Augustinian congregation even after the settlement of
the contention with regard to Staupitz’s plans of union.

Those brethren who treasured the ancient monastic discipline,
protected by papal privileges and exemptions, were accused of self-righteous
Pharisaism and of disobedience towards the General of the
order by Luther and his party. They were the “Little Saints” against
whom he had inveighed in his impetuous address at Gotha. In his
lectures and sermons he reproached, though often only in allusions,
their “observantine” practices, their adherence to the doctrine of
good works, and their want of charity. His invectives, however,
were launched with a bitterness which those concerned assuredly
did not merit, even though there might have been reasons for complaint.
It may be said that the ancient and the modern wings opposed
each other in the Wittenberg monastery. Probably there was friction
also in the monastery at Erfurt, where Luther’s friend Lang was
prior, as well as in other monasteries of the congregation. Luther’s
monastery, however, was the center of the contention. The young
students of the Order brought with them their divergent views out
of the cloisters whence they came and they carried the new atmosphere
of Wittenberg along with them when they left. Luther’s partisans
at Wittenberg boasted that they were more closely attached to the
General of the Order at Rome than their opponents. The General,
they contended, was not in favor of the singularities of the Observantines.

At the commencement of his first series of lectures on the Psalms, Luther
delivered a sarcastic address on the obedience due to religious superiors.
“How many do we not find,” he says, “who believe they are very religious,
and yet they are, if I may so express it, only men of an extremely
sanguine temperament (\textit{sanguinissimi}) and true Idumaeans [\textit{i.e.}, pagan-minded].
There are people who so revere and praise their monastic
state, their order, their Saints, and their institutions, that they cast a
shadow upon all others, not wishing to grant them their proper place.
In a very unspiritual manner they are humble followers (\textit{observantes}) of their
fathers and boast of them. Oh, the frenzy that prevails in this day! It has
almost come to this that every monastery repudiates the customs of all
others and is imbued with such pride as to preclude taking over or learning
anything at all from another. That is the pride of Jews and heretics,
with which we, unfortunate ones, are also encompassed,” etc.\footnote
{Grisar, \textit{Luther}, Vol. VI, pp, 497 sqq. \textit{Sanguinosissimi} must probably be read in lieu of
\textit{sanguinicissinii}.}
In the addresses which he delivered in the monastery church he frequently
alludes to the obstinate pride of the Jews and heretics, in condemnation of
those members of his order or of other orders who adhered to the strict
observance. These “Observantines, exempted and privileged characters”--thus he
fulminates in another’ lecture, are devoid of obedience, which is the very
soul of good works. It will be seen--he continues--how detrimental to the
Church they are; in the interests of the rule, they were determined to
insist upon exceptions; “but that is a light that comes from the devil.”\footnote
{\textit{\textit{Op. cit.}}, Vol. VI, p. 458.}

This was the contest which led the fiery monk to enter upon
doubtful ways. His opposition to the so-called doctrine of self-righteousness
caused him to form a false conception of righteousness;
instead of attacking an heretical error, he combated the true worth
of good works and the perfections of the monastic life.

Voluntary poverty, as practiced by the mendicants, was one of the
foundations of his Order. The inmates of monastic houses were to
live on alms according to the practice introduced by the great Saint
Francis of Assisi and for the benefactions received were to devote
themselves gratis to the spiritual needs of their fellowmen. Many
abuses, it is true, had attached themselves to the mendicant system:
self-interest, avarice, and worldly-mindedness infected the itinerant
mendicants. But in his explanation of the Psalms Luther attacks the
life of poverty \textit{per se}: “O mendicants! O mendicants! O mendicants!”
he pathetically exclaims, “who can excuse you? \dots Look to it
yourselves,” etc. He places the practice of poverty in an unfavorable
light.\footnote{\textit{Op. cit.}, Vol. VI, p. 500.}
In his criticism of the “self-righteousness” of his irksome
enemies, he confronts them with the righteousness of the spirit that
cometh from Christ. These people, whom he believed it his duty to
expose, were guilty, in his opinion, of a Pharisaical denial of the true
righteousness of Christ. His righteousness, and not our good works,
effect our salvation; works generate a fleshly sense and boastfulness.
These thought-processes evince how false mysticism, unclear theological
notions, a darkening of the monastic spirit, and passionate
obstinacy conspired in Luther’s mind.

In the years 1515 and 1516, the phalanx of the self-righteous, the \textit{justitiarii},
as he styles them, again constitute the object of his attacks. There
is Christ, the hen with its protecting wings, which he must defend against
the vultures that pounce upon us in their self-righteousness. These enemies
of the sweet righteousness imputed to us by God are “a pestilence in the
Church; intractable, nay, rebellious against their superiors, they decry others
and clothe themselves with the lamb-skins of their good works.”
\footnote{\textit{Op. cit.}, Vol. VI, pp. 502 sq.}

An Augustinian friend of his, George Spenlein, having become weary
of certain persecutions, had had himself transferred from Wittenberg to
the monastery at Memmingen. Luther sent him a peculiar letter of conseem that
dolence on April 8, 1516. According to this missive, it would seem that
the self-righteous Spenlein had been for a long time ``in opposition to the
self-righteousness of God, which had been bestowed most lavishly and
gratuitously upon him by Christ''; whereas he (Spenlein) desired to stand
before God with his own works and merits, which, of course, is impossible.
He (Luther), too, had harbored this notion, and says he still wrestles with
this error. “Learn, therefore, my sweet brother,” thus he addresses Spenlein
in the vocabulary of mysticism, “learn to sing to the Lord Jesus and, distrusting
yourself, say to Him: Thou, O Lord Jesus, art my righteousness, but
I am Thy sin. Thou hast accepted what was mine and hast given to me what
was Thine. Oh, that thou wouldst boldly appear thyself as a sinner, yea,
be a sinner in reality; for Christ abides only in sinners.” “But, if you are
a lily and a rose of Christ, then learn to bear persecution with patience,
lest your secret pride convert you into a thorn.”\footnote
{Enders, \textit{Luthers Briefwechsel}, I, p. 29.}

The germ of Luther’s reformatory doctrine is plainly contained
in this species of Mysticism. Step by step he had arrived at his new
dogma in the above described manner. The system which attacked
the basic truths of the Catholic Church, was complete in outline.
Before giving a fuller exposition of it, we must consider the individual
factors which cooperated in its development in Luther’s mind.

Confession and penance were a source of torturing offense to the
young monk. Can one obtain peace with God by the performance
of penitential works? He discussed this question with Staupitz on an
occasion when he sought consolation. Staupitz pointed out to him that
all penance must begin and end with love; that all treasures are
hidden in Christ, in whom we must trust and whom we must love.\footnote
{\textit{Tischreden}, Weimar, ed., II, Nr. 2654.}
These words contain nothing new; but the exhortation to combine
love with penance entered the inflammable soul of Luther as a voice
from heaven. According to his own expression, it
“clung to his soul as the sharp arrows of the mighty” (Ps. 120:4);
% modernized bible citation format
% (Ps. CXIX, 4) is 119, but 120:4 mentions sharp arrows, so increment +1
henceforward, he says, he would execrate the hypocrisy by means of which he had
formerly sought to express a “fabricated and forced” penitential
spirit during the tortures of confession. Now that the merits
of Christ covered everything, penance appeared easy and sweet
to him. He expresses himself on this point in a grateful letter to
Staupitz, written in 1518.\footnote
{Enders, \textit{Luthers Briefwechsel}, I, p. 196 (May 30, 1518.)}

On the occasion referred to, it is probable that Staupitz, as was
his custom, expressed himself in a vague and sentimental manner,
rather than in clear theological terms. His writings are susceptible of
improvement in many respects. The influence which he exerted on
Luther was not a wholesome one. He was too fond of him to penetrate
his character. He perceived in him a rising star of his congregation,
a very promising ornament of his Order. Even in the most
critical period anterior to Luther’s apostasy, he eulogized his courage
and said: Christ speaks out of your mouth,--so well it pleased him
that Luther, in the matter of righteousness and good works, ascribed
everything to Christ, to whom alone glory should be given.\footnote{Weimar ed., XL, I, p. 131.}
Certain of a favorable response on the part of his superior, Luther wrote thus
in the above letter to him: “My sweet Saviour and Pardoner, to whom I
shall sing as long as I live (Ps. 104:33),
% (Ps. CIII, 33) -> (Ps. 104:33)
is sufficient for me. If there
be anyone who will not sing with me, what is that to me? Let him
howl if it please him.” The shortsighted Staupitz sided with Luther
even after he had been condemned by the Church.

Nor was Staupitz the man who could thoroughly free Luther from
his doubts about predestination, although Luther says he helped
him. His general references to the wounds of Christ could not permanently
set the troubled monk aright. He should have placed
definitely before him the Catholic dogma, based on Sacred Scripture,
that God sincerely desires the salvation of all men, and should have
made clear to the doubter that voluntary sin is the sole cause
of damnation. But he himself seems not to have grasped these truths,
for in certain critical passages of his writings he allows them to retreat
before a certain mysterious predestination. Luther’s fear of predestination
constituted the obscure substratum of his evolving new
religious system. Recalling Staupitz’s exhortations, he says, in 1532:
We must stop at the wounds of Christ, and may not ponder over
the awful mystery. The only remedy consists in dismissing from our
minds the possibility of a verdict of damnation. “When I attend
to these ideas, I forget what Christ and God are, and sometimes arrive
at the conclusion that God is a scoundrel \dots The idea of
predestination causes us to forget God, and the \textit{Laudate} ceases and
the \textit{Blasphemate} begins.”\footnote{\textit{Tischreden}, Weimar ed., II, Nr. 2654.}
The part which these struggles had in the
origin of his new doctrine, is to be sought in Luther’s violent efforts
to attain to a certain repose in the face of his presumptive predestination.

It is also remarkable that the last-quoted utterance is followed by
one concerning his “great spiritual temptations.” In contrast with
the struggles of despair which he underwent, he is not deeply impressed
by ordinary temptations. “No one,” he writes, “can really
write or say anything about grace, unless he has been disciplined by
spiritual temptations.”\footnote{Grisar, \textit{Luther}, Vol. 1, pp. 204 sqq.}
His opponents, he says elsewhere, not
having had such experiences, it behooved them to observe silence.
When his doctrine encountered opposition in Rome, he wrote to
Staupitz that Roman citations and other matters made no impression
on him. “My sufferings, as you know, are incomparably greater,
and these force me to regard such temporal flashes as extremely
trivial.”\footnote{\textit{Op. cit.}, Vol. VI, pp. 100 sqq.; cfr. I, 14 sqq.}
He meant “doubtlessly, personal, inward sufferings and
attacks which were connected with bodily ailment \dots, whereby,
as formerly, he was always seized with fear for his personal salvation
when he pondered on the hidden depths of the divine will.”\footnote{Julius Köstlin.}

In his interpretation of the Epistle of St. Paul to the Romans,
given during the years 1515 and 1516, Luther completely unfolded
his new doctrine.
