\section{Interior State}

The last sentence quoted above contains a remarkable declaration
about his spiritual condition and his compliance with his monastic
duties at that time. He seldom found time to recite the Divine Office
and to say Mass. It was his duty so to arrange his affairs as to
be able to comply with these obligations. The canonical hours were
strictly prescribed. Saying Mass is the central obligation of every
priest, especially if he is a member of a religious order. If Luther did not
know how to observe due moderation in his labors; if he was derelict
in the principal duties of the spiritual life; it was to be feared that
he would gradually drift away from the religious state, particularly
in view of the fact that he had adopted a false Mysticism which favored
the relaxation of the rule. As rural vicar, it is probable that he
did not sustain among the brethren the good old spirit which the
zealous Proles had introduced into the society. Of the “temptations
of the flesh” which he mentions we learn nothing definite. He was
not yet in conflict with his vows. His wrestlings with the devil may
signify the fears and terrors to which he was subject.

He continued to be on good terms with his friend Staupitz, who
was interested in the young monk’s manifold activities. Staupitz also
posed as a mystic, and favored the spiritual tendency which Luther
followed. This talented and sociable man was very popular as a useful
adviser in the homes of the rich and as an entertainer at table.
Whilst Luther could not accompany him on such errands, he enjoyed
his company on monastic visitations. In July, 1515, he accompanied
Staupitz to Eisleben, when the latter opened the new
Augustinian monastery at that place. As he walked in his sacerdotal
vestments in the procession through the city of his birth at the side
of his vicar, who carried the Blessed Sacrament, Luther was suddenly
seized with unspeakable fright at the thought of the proximity of
Christ. On mentioning the incident to his superior afterwards, the
latter comforted him by saying: “Your thought is not Christ,” and
assuring him that Christ did not desire this fear.\footnote{\textit{Tischreden}, Weimar ed., Vol. I, Nr. 137.}
At times, in consequence
either of a disordered affection of the heart or of overwork,
he was so distressed that he could not eat or drink for a long
time. One day he was found seemingly dead in his cell, so completely
was he exhausted as a result of agitation and lack of food. His friend
Ratzeberger, a physician, mentions this incident, without, however,
indicating the exact time of its occurrence. Luther was relieved of
this pitiable condition by recourse to music, which always stimulated
him. After he had regained his strength, he was able once more
to prosecute his labors. As a result of his suffering and worry he became
very much emaciated.

Did Luther subject himself to extraordinary deeds of penance at
any period of his monastic life, as he frequently affirmed in his subsequent
conflict with the papacy and monasticism, when he was impelled
by polemical reasons to describe himself as the type of a
holy and mortified monk, one who could not find peace of mind
during his whole monastic career? Holding then that peace of mind
was simply impossible in the Catholic Church, he arbitrarily misrepresents
monasticism, in order to exhibit in a most glaring manner
the alleged inherent impossibility of “papistic” ethics to produce the
assurance of God’s mercy. “I tormented my body by fasting, vigils,
and cold \dots In the observance of these matters I was so precise
and superstitious, that I imposed more burdens upon my body than
it could bear without danger to health.” “If ever a monk got to
heaven by monkery, then I should have got there.” “I almost died
a-fasting, for often I took neither a drop of water nor a morsel
of food for three days.”\footnote
{See the passages quoted by me in \textit{Luther}, Vol. VI, pp. 191 sqq. A special chapter in that
volume (pp. 187 sqq.) discusses “Luther’s Later Embellishment of His Early Life” in the
various phases of its development.}

Such exaggerated penitential exercises were prohibited by the
statutes of the congregation, which were distinguished for great discretion,
and insisted upon proper moderation as a matter of strict
duty.

The above picture of singular holiness is produced not by early
witnesses, but by assertions which Luther made little by little at a
later period of life. The established facts contradict the legend. Perhaps
his description is based partly on reminiscences of his distracted
days in the monastery, or on eccentric efforts to overcome his sombre
moods by means of a false piety. His greatest error, and the one
which most betrays him, is that he ascribes his fictitious asceticism
to all serious-minded members of his monastery, yea, of all monasteries.
He would have it that all monks consumed themselves in
wailing and grief, wrestling for the peace of God, until he supplied
the remedy.\footnote{Cf. Grisar, \textit{Luther}, Vol. II, pp. 157 sqq.}
It is a rule of the most elementary criticism finally to
cut loose from the distorted presentation of the matter which has
maintained itself so tenaciously in Protestant biographies of Luther.
It may be admitted that, on the whole, Luther was a dutiful monk
for the greatest part of his monastic life. “When I was in the monastery,”
he stated on one occasion, in 1535, “I was not like the rest
of men, the robbers, the unjust, the adulterous; but I observed chastity,
obedience, and poverty.”\footnote{\textit{Op. cit.}, Vol. VI, pp. 233 sqq.}

Yet, after his transfer to Wittenberg, and in consequence of the
applause which was accorded to him there, the unpleasant traits of
his character, especially his positive insistence on always being in the
right, began to manifest themselves more and more disagreeably.
In his opinion, the Scholastic theologians, even the greatest among
them, were sophists. They were a herd of “swine theologians,” while
he was the enlightened pupil of St. Paul and St. Augustine.\footnote{\textit{Op. cit,}, Vol. I, pp. 130 sqq.}
The finer achievements of Scholasticism, especially those of its intellectual
giant, Thomas of Aquin, were scarcely known to him. Could
his confused mysticism perhaps supplement his deficient knowledge
of Scholasticism? No, it only made him more self-conscious and
arbitrary in the sphere of theology. He gave free vent to his criticism
of highly respected ascetical writers. An example of his egotistical
excess in this respect is furnished by his glosses for the year 1515,
which he indited on the Psalter of Mary, a work of Mark of
Weida.\footnote{\textit{Theol. Studien u. Kritiken}, 1917, pp. 81 sqq.}

In addition to these characteristics, there was his peculiar
irritability, which is strikingly exhibited in his correspondence during
1514. The theologians of Erfurt, led by Nathin, had reproved him for
taking the doctorate at Erfurt instead of at Wittenberg, since the
Erfurt school had claims on him as one of its own pupils. It is possible
that some harsh words were exchanged in regard to this matter. The
young professor in a letter addressed to the monastery at Erfurt says
that he had well nigh resolved to “pour out the entire vial of his
wrath and indignation upon Nathin and the whole monastery”
on account of their lies and mockery. They had received two shocking
letters (\textit{litterae stupidae}) from him, for which he now wants to
excuse himself, though his indignation “was only too well founded,”
especially since he now heard even worse things about Nathin and
his complaints against his (Luther’s) person. In the meantime, God
had willed his separation from the Erfurt monastery, etc.

The ill-feeling between Nathin and his Erfurt colleagues, on the
one hand, and Luther and his monastic partisans on the other, arose
from the controversy concerning the stricter observance of the rule
within the Order.
