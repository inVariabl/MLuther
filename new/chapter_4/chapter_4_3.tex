\section{Effects of Luther’s First Appearance}

Protests were immediately voiced. In September, 1516, Luther
arranged for a disputation at the University of Wittenberg, to be
conducted by Bartholomew Bernhardi, a native of Feldkirch in
Vorarlberg, for which he himself supplied the theses. They were
chiefly concerned with the utter moral incompetence of man in the
state of nature. The solemn disputation, according to his own testimony,
was intended to silence the “yelping curs” which might rise up against him.
The theses stated that man sins even when doing the best he can;
for he is absolutely unable by his own unaided efforts to keep the commandments of God.\footnote
{\textit{Op. cit.}, Vol. 1, pp. 310 sq.}

In October, 1516, he began his celebrated shorter lectures on the
Epistle to the Galatians. In the printed edition with which we are
familiar, as well as in the condensed form in which they have been
recently published, they constitute a still more vigorous expression of
his views. These lectures were followed, in 1517, by his prelections on
the Epistle to the Hebrews, which have not yet been printed. The
Wittenberg disputation and the lectures delivered by Luther at that
time were subsequently represented by him and his friends as “the
commencement of the gospel business.”\footnote{\textit{Op. cit.}, Vol. I, pp. 303 sqq.}

Whilst the reports concerning Luther caused the monks at Erfurt
and other cloisters to be startled, and whilst the opposition to him
at Wittenberg increased, influential professors at the University
gradually espoused his cause. Among these were the following:
Andrew Karlstadt, who up to this time had labored energetically in
the interests of Scholasticism; Nicholas Amsdorf, who was destined
to distinguish himself as an ardent enthusiast, more Lutheran even
than Luther himself; and Peter Lupinus, a former professor and
originally an antagonist of the new thought. Of special importance
for Luther was the support and cooperation of the Augustinian,
Wenceslaus Link, prior of the Wittenberg monastery and since 1511
a doctor at the University, a confidant of Luther and one of the
supporters of Staupitz. After having accompanied the latter on several
visitation tours, he arrived in 1517 at Nuremberg, where he pleaded
the cause of Lutheranism in eloquent sermons. At Nuremberg, the
humanist Christopher Scheurl became a friend of the Lutherans.
At Erfurt Luther had an active supporter in his old friend, John
Lang, the prior of that place. To him Luther wrote, in 1517, that he
contemned the reproaches of the Augustinians of Erfurt for his alleged
presumption, since God’s work would be realized in him. “Pray
fervently for me,” he wrote, “as I pray for you, that our Lord Jesus
may assist us and help us bear our temptations, which are known
to no one but ourselves.”\footnote{Enders, \textit{Luthers Briefwechsel}, I, p. 124 (November II.)}

The law faculty of Wittenberg entertained diverse opinions of
Luther. Among the first to defend him was the jurist Jerome Schurf,
who subsequently became the patriarch of jurisprudence at Wittenberg. The
renowned Martin Pollich, who, with Staupitz, was one of
the founders of the University, freely acknowledged Luther’s extraordinary
talents.\footnote{Grisar, \textit{Luther}, Vol. IV, pp. 248 sq.}
Most of the other professors, however, were either
against him or noncomittal, because they disapproved of his opposition
to canon law. Already at that time Luther was not fond
of the representatives of jurisprudence, to whose legal demands, on
the basis of his peculiar mysticism, he opposed an exaggerated “passive” complaisance.

The reformer’s most agitated opponents at Wittenberg quite naturally
were those monks and clergymen whom Luther in his lectures
and in the pulpit was wont to castigate as self-righteous. We know
from what has been said above that the high esteem in which the
Observance was held by many Augustinians was severely criticized
by Luther. Among the younger student monks were some who brought
with them from the cloister sound traditions and a spirit of enthusiasm
for the external exercises of piety which Luther and his supporters
disliked. They became the kernel of an active party who attacked the
new theology. Luther assailed the self-righteous and Pharisaical Observantines
all the more hotly. At the time of his lectures on the
Epistle to the Romans (1515) he delivered a Christmas sermon, in
which he raised his voice against them.

As the ancient prophets, philosophers, and scribes who proclaimed the
truth were persecuted, so, Luther assures us, he in turn is being persecuted.
Men exclaimed that he erred when he called Christ a hen, who, as it were,
gathers us under His wings, in virtue of His merits and righteousness, in
order to make us righteous. The opponents of the hen should know that
their righteousness is sin. Since man is unable to fulfill the law, it behooves
him to exclaim: “O sweet hen!” However, one may not seek virtues and
gifts according to one’s own opinion. Thus, with a lavish hand, Luther
casts about him the products of his mysticism in the course of a
sermon.\footnote{Grisar, \textit{Luther}, Vol. III, p. 970 (German ed.)}

Meanwhile, his adversaries, who were solicitous about the Church and
their Order, did not remain inactive. Luther charged them in a sermon in
midsummer, 1516, with shooting arrows at those who are pure of heart.
They, in common with countless other contemporaries, he exclaims, are
becoming obdurate in their “carnal righteousness and wisdom.” They are
the greatest evil in the Church; their shibboleth, that one is obliged to
do what is good, is a pestilence whereby they antagonize the goodness of
God. He enumerates no less than seven transgressions of which they were
guilty.\footnote
{\textit{Op. cit.}, Vol. III, p. 971 (original German edition; omitted in the English translation
which we always quote in this volume, except where the German edition is expressly
mentioned.)}

Luther’s enemies were cowed by his vehement attacks. They could not
compete with him as an orator. In Wittenberg especially there was no one
to challenge him. It was a tragical advantage in his favor that his talents
enabled him to stand head and shoulders above all the brethren of his Order.
Opposition merely increased his audacity.

On September 4, 1517, he arranged another sensational disputation
by his disciple Franz Günther of Nordhausen on ninety-seven
theses composed by him against Scholasticism and Aristotle. One
of these theses was that man can “desire and do only evil.” Another,
that “his will is not free.” Another, that “the sole disposition for
grace is predestination, eternal election by God.” Another that
“neither the Jewish ceremonial code, nor the decalogue, nor whatever
may be externally taught or commanded, is a good law; the
only good law is the love of God,” etc. At the end of these paradoxical
theses we find the assurance that nothing in their contents
contradicted Catholic dogma and the ecclesiastical writers. These
ninety-seven propositions of 1517 enjoyed a certain circulation prior
to the publication, in the same year, of Luther’s famous theses on
indulgences, which are not nearly so far-reaching. In consequence
of this disputation several distinguished men, like Scheurl of Nuremberg,
expressed the belief that a “restoration of the theology of
Christ” was under way. Some time previously Luther triumphantly
wrote to Lang: “Our theology and St. Augustine are making good
progress and thanks be to God they prevail at our university \dots The
lectures on the Sentences [delivered by the Scholastic teachers] are
completely ignored, and no one can assure himself of an audience
who does not profess this [i. e., our] theology.”\footnote{\textit{Op. cit.}, Vol. IV, pp. (May 18, 1517.)}

The assertion that Augustine was coming into his own in virtue
of the new doctrine was as little in accord with the truth as the
statement that that doctrine was genuinely Pauline. On the contrary,
Luther’s errors may be refuted point for point from the writings
of Augustine, who, in his contest with the Pelagians, forcefully points
out the role of grace, yet neither denies that man has a part in
every good deed, nor, much less, that he has a free will. The great
Doctor of the Church acknowledges the devastating results of original
sin, but he demands the fulfillment of the entire law with divine
aid. In the theology of St. Augustine the grace of divine son-ship
is not merely an imputation of alien righteousness, but a supernatural
state of the soul which has been truly cleansed of sin. He
admits that the distribution of grace is an inscrutable mystery, but
rejects the theory of absolute predestination to hell as an abomination.
He frequently argues in favor of the freedom of the human
spirit and the dominion of love. But he does not hold man to be
independent of the laws of God or the Church; or that love must
be the sole motive of action to the exclusion of fear of God’s punishments
so natural to man. Above all, Augustine is a defender of
ecclesiastical authority and tradition, who inexorably combated
arbitrariness, the spirit of innovation and subjectivism in doctrinal
matters. In spite of Luther’s boastfulness, he and Augustine are
poles apart. All that may be said in extenuation of Luther is that
Augustine’s thought is frequently profound, and that, as a rule,
he does not propound his doctrines methodically, but either in the
service of controversy or adapted to the changing requirements of
souls and the Kingdom of Heaven. As a consequence, his teaching
in some respects is more easily misunderstood than that of other
ecclesiastical writers. Luther repeatedly read his own ideas into Augustine
with a rashness that was but little removed from conscious
falsification.

For the rest, to understand Luther, we must bear in mind that
the theological tradition since the days of Scholasticism was not yet
fully clarified. Some of the theological doctrines which he criticized
had not been treated adequately in the schools. The subsequent
labors of the theologians at the Council of Trent indicate how much
still remained to be done in the matter of clearing up such questions
as original sin, grace, and justification.

It is highly amazing, however, to see Luther proposing new theories
without interrogating more closely the well-established doctrinal
tradition of the Church and questioning the latter’s prerogative to
teach, which had been instituted by Christ for the defense of dogma.
We marvel at the fact that this sacred authority is almost completely
set aside by him, as though it were non-existent. Luther acts
as if his fight were merely a fight against the schools, against Scholastic
sophists, against the Aristotelians, against the new writers and
their friends, such as the followers of Gabriel Biel, whom he calls
“Gabrielists,” or against the defenders of such older authors as the
“\textit{sententiarii}.” The mighty authority of the Church, which every
Catholic theologian must consult at every step, did not impress
him sufficiently during his formative period. This deficiency is attributable,
in part, to the schools where he studied and disputed,
for the later Scholastics, while indulging in hair-splitting investigation,
neglected the important doctrine on the Church, or at least did
not pay enough attention to it. Too much preference was given to
speculation in contrast with authority and the question of its binding
power, and, in general, with the positive study of the treasures
of tradition entrusted to the Church and her supreme government.

True Luther was not entirely unfamiliar with the doctrinal authority of
the Church. On the contrary, it is very remarkable that
in his exposition of the Epistle to the Romans, despite his deviations
from the faith, he strongly and in vehement language condemns
the activities of heretics who separated themselves from the
Church. Evidently he did not at that time entertain any idea of
revolt against the hierarchy, though his complete defection from
the common doctrine of the past, if insisted upon, was bound to lead
to a separation from the hierarchy as his next step.
In his resolution to continue his criticism of good works--a
criticism dictated by false mysticism--it is notable that he still retains
the idea of the monastic state for a number of years. He devotes some beautiful
passages to the excellence of the monastic vows
in his lectures on the Epistle to the Romans. Hence, he arrived at his
new doctrine not as a result of his eagerness to break the sacred bonds,
but by following entirely different and most intricate routes, especially
that of a morbid mysticism.
