\section{The Principal Propositions of Luther’s Doctrine}

The ancient Church, above all things, upheld the freedom of the
human will to do good. She steadfastly maintained that God wills
the salvation of all men without exception and to this end offers
them the necessary grace, with which men should freely cobperate.
Baptism makes a man a child of God by virtue of sanctifying grace;
but the inclination to evil remains through no fault of his own, provided
he does not consent to sin. Concupiscence is diminished by the
means of grace offered by the Church. If any one has been seduced
into committing mortal sin, he may confidently hope to regain the
state of grace through the merits of the death of Christ, provided he
submits to penance and resolves to amend his life. Merefaith in the
application of the merits of Christ is not sufficient. Actual grace
assists man to be converted and to persevere in doing good.

In these few propositions we have described a splendid system
of doctrine, which accords with the free, rational nature of man
as well as with the infinite goodness of God. This is not the place to
demonstrate its truth from the sources of divine revelation, Scripture
and tradition.

Luther, with the Epistle to the Romans in his hand, proclaimed
that man was not free to do good; that all his efforts were sinful,
because evil concupiscence dwelled in his soul; that God did everything
in him, governing him as the rider governs his steed. He did
not differentiate between natural and supernatural good. Christ,
he said, has fulfilled the law for me and atoned for every weakness
and sin. Through His righteousness the believing and trustful sinner
is covered, apart from his own works and his own righteousness. He
remains a sinner as before, but is justified by the imputation of the
justice of Christ and necessarily brings forth good works through
the infinite causality of God, just as trust in God is imparted to the
hesitating only through the divine omnipotence.

According to Luther’s teaching, not all men are thus favored by
God, since His inscrutable decree consigns many to eternal damnation.
Resignation to hell is the highest virtue because it connotes
complete submission to the will of God. But this very resignation
reconciles the despondent soul with the thought of a merciful God.
Perfect humility and submission (\textit{perfecta humilitas}) must serve us
as a kind of anchor. The doctrine of the absolute certainty of salvation,
or rather, the certitude of justification through mere belief
in Christ, had not yet been discovered by Luther. Instead, he still
upheld the Catholic teaching on merit, similarly as when, in connection
with justification, he employed Catholic expressions, albeit
obscurely and hesitatingly, to set forth his conception of the renovation
of the inward man. The renovation, however, which Luther
indicates, is far removed from that which the ancient Church teaches
on the basis of Sacred Scripture, namely, that the spirit of God,
poured forth into the soul, abides in man. Luther ridicules the outpouring
of grace in man in virtue of the so-called habits and the implanted
supernatural virtues. He had nothing but scorn for the
“sophists” who entertained such “silly notions.”

His exposition of the new theology, which we have condensed
above, is accompanied by a haughty and repellent treatment of the
traditional dogmas and theology. He selects some actual deficiencies
of the older theologians, in order to stigmatize the entire past, especially
the Scholastic system, with which he was but inadequately acquainted.\footnote
{On the doctrinal content of Luther's Commentary on the Epistle to the Romans, see
Grisar, \textit{Grisar}, Vol. I, pp. 184 sqq., 374 sqq.}
