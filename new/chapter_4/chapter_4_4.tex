\section{Historical Background of Luther’s Doctrine}

The publication of Luther’s commentaries on the Epistle to the
Romans (1515--1516) and the study of contemporary sources, has
clearly demonstrated, among other things, the falsity of the assumption
that Luther’s innovation originated in the intention of effecting
a thorough and universal reform of the Church.

Ideas of powerful external and internal “reforms” have been ascribed
to the “reformer” as the starting-point of his public appearance;
in reality, however, such ideas did not influence him at all.
What he primarily intended was to get his peculiar doctrines accepted
and introduced in his monastery and Order, in the University
of Wittenberg, and, finally, in the whole realm of contemporary
scholarship.

It is true, as the sources mentioned above indicate, he combined
with his efforts to impose his doctrines a loud demand for a reformation
of ecclesiastical conditions. He ascribed the chief cause
of the prevailing abuses in the religious sphere directly to the neglect
of the truths newly discovered by him. Blinded by his imagination
and by his reckless habit of fault-finding, he exaggerated beyond
all measure, both in his exposition of the Epistle to the Romans and
in his other addresses, the shortcomings of his age in faith and morals.
In matter of fact the young monk of thirty-two years, living within
the narrow confines of a monastery, knew but little of the actual
conditions existing out in the world! Thus far his contacts with
the world had been but few. Nevertheless, he believed he saw a
veritable “deluge” of errors and abuses everywhere, because mankind
was completely estranged from the “Word,” namely, the requirements
of the Bible and “righteousness,” as he understood them.
“The pope and the ecclesiastical superiors”--thus he expresses himself
with blind audacity in these lectures--``have rendered themselves execrable
in their endeavors\footnote{“\textit{Corrupti sunt et abominabiles facti sunt in studiis suis.}” (Ps. XIII, 1).};
they are now unmasked as seducers of the Christian populace.'' These superiors, he hotly avers,
fill the whole world with their sodomitical and other vices; the faithful
have completely forgotten the significance of good works, faith
and humility. Even the better class are idolaters rather than true
Christians, because they are self-righteous.\footnote{Grisar, \textit{Luther}, Vol. 1, p. 227.}
After such a wholesale condemnation of Christians, he is able, in turn, to describe
minutely, in grotesque outlines and with humorous incisiveness, the
errors of the narrower world by which he was surrounded, \textit{e.g.},
the monks in choir impatiently hastening to the end of the services.
He criticizes those of high degree with a mystical zeal, because they
insisted upon their rights in preference to suffering injustice, as
true Christians should do. It was his opinion that Pope Julius II
should have relinquished the rights of the Church in his conflict
with the republic of Venice. Duke George of Saxony, who was engaged in
warfare with the rebels in Frisia, would have done better
had he and his people patiently suffered chastisement for their sins
at the hands of their enemies for God’s sake.\footnote{\textit{Op. cit.}, Vol. I, p. 228.}
“I fear,” he says,
“that we shall all perish on account of our worldliness.” These are
utterances of a disturbed mind.

The memory of the abominations which he had witnessed, or
which he claims to have heard of in Rome, ever rises up before
him and fills him with terror. “Should we not regard as sinful the
shocking corruption of the entire curia and the mountain of revolting immorality
, pomp, covetousness, vanity, and sacrilege prevalent
there?” Germany at that time was in an ugly mood, not only
because of the avarice of the Roman collectors, but also on account
of the reports of lax morality incident to the Renaissance, since and
even before the days of the unworthy Borgia Pope, Alexander VI
Staupitz, too, was embittered against Rome and freely vented his feelings
in the presence of Luther, who stated subsequently that Staupitz
had incited him against Rome.\footnote{\textit{Tischreden}, Weimar ed., IV, Nr. 4707.}

Did this disaffection, which cropped out in spots already before
Luther, lead to any opposition in dogma or practice, similar to his?
Did Lutheranism have any precursors? Did it arise from a soil
prepared by others?

Many prominent men had raised their voices in protest against
the corruption and mistakes of the ecclesiastical authorities, but
despite their criticisms, they generally remained loyal to the teachings
of the Church, and simply demanded a reform on the basis
of ancient dogmas. Such, for example, was that powerful preacher
Geiler von Kaysersberg. Only a few before Luther dared to go as
far as did John of Wesel (+1481) and Wessel Gansfort (+1489).
The former who was a pastor of the cathedral at Mayence,
was cited before the Inquisition and sentenced to spend the balance
of his days in an Augustinian monastery, after he had recanted
his erroneous propositions, which, among other things, attacked indulgences
and approached the Hussite heresy concerning the Church
and predestination. While still an orthodox theologian, Wesel had
taught at the University of Erfurt, but he did not influence Luther’s
development. Wessel Gansfort, who is often confused with John
of Wesel, was celebrated as a great scholar by his admirers, but
obscured the doctrine of the Church by many heretical propositions.
Thus he affirmed the fallibility of ecumenical councils. The
righteous, he taught, have the power of the keys in a certain sense.
He asserted that satisfaction for sins committed was superfluous
after their remission and there was no need for indulgences, etc.
Still he conforms as little with Luther in the principal points of
the latter’s teaching, as did John of Wesel. He holds that man
has a free will, that only faith animated by charity can effect justification,
and that justification is not merely a declaration of righteousness,
but an actual process of making man just. Neither the one
nor the other of these scholars agrees with Luther in his reformatory
demands; and hence they are incorrectly hailed as precursors of the
Lutheran movement.

It has been asserted that long before Luther there existed a so-called Augustinian
school of theology which propagated Lutheran
ideas on liberty, grace and justification down to the days of the
Protestant Reformation. In reality, however, no such school existed,
either during or at the close of the Middle Ages. Isolated writers,
especially during the early period of Scholasticism, did advance risqué
propositions that smacked of Lutheranism, but they were not in
any true sense precursors of Luther, particularly since they did not
create a tradition. At the same time it is difficult, yea impossible,
to ascertain to what extent Luther knew and used these earlier
writers or appreciated their teaching. There is no reason to challenge
his independent discovery of his heresies, hence we may readily concede
their originality.\footnote{On so-called precursors of Luther in the “School of Augustine” see Grisar, \textit{Luther}
(German original, Vol. III, pp. 1011 sqq.; this appendix is omitted in the English translation).
Grabmann in the \textit{Katholik}, 1913, Nr. 3, pp. 157 sqq. The connection of Fidatus
of Cascia (died 1348) with Luther is rejected by N. Paulus in the Innsbruck \textit{Zeitschrift
für kath}. \textit{Theologie}, 1922, pp. 169 sqq.; cf. the same writer in the \textit{Theol. Revue}, 1922,
pp. 18 sq. and \textit{Histor. Jahrb.}, 1922, p. 323. Some Protestant authors also reject the theory,
\textit{e.g.}, R. Seeberg in \textit{Die Lehre Luthers} (1917), p. 118, and, relative to Fidatus, in \textit{Theol.
Literaturblatt}, 1923, pp. 197 sqq.; also Scheel in the \textit{Zeitschrift für Kirchengesch}. 1922,
pp. 258 sq. Cf. W. Köhler in the \textit{Histor. Zeitschrift,} Vol. CXI, Nr. 1, p. 153, and W.
Braun in the \textit{Evangel. Kirchenzeitung,} 113, pp. 181 sqq.
}

The great Luther monument at Worms, which was unveiled in
1868, embraces quite a number of statues of so-called heralds of
the Reformation. The central figure of Luther is encircled by
statues of Savonarola, Hus, Wiclif, Reuchlin, and Peter Waldus.
Do they belong in this constellation? As precursors of Luther’s principal
doctrines, certainly not; at most they may pass as opponents
of the papacy in virtue of other doctrines or because of some particular
controversies.

The most advanced of these opponents of the papacy was Hus,
whose unfortunate end at the Council of Constance was the result
of heretical doctrines subversive of both Church and State. Though
Luther agreed with him in some things, and afterwards glorified him
exceedingly, he was not a disciple of Hus. When, in his early monastic
years, he chanced upon a volume of Hus, he refused to read
it, though he noticed some good therein, because of his aversion
for the author’s name.\footnote{Grisar, \textit{Luther}, Vol. I, p. 25.}
Soon after his change of front, however,
he exploited in the interest of his own cause the unhistorical legend
that Hus, when he faced the stake, said: Now they are roasting a
goose [Hus in Bohemian signifies goose], but a swan will come which
they will not master. Luther, with a power of illusion which considerably
exceeded that of the dreaming and meditative figure of
Hus on the Worms monument, applied this alleged prophecy to
himself.

Nor was there any greater affinity between Luther’s teaching
and that of Hus’s precursor, John Wiclif, or that of Peter Waldus.
Savonarola, the eccentric Dominican of Florence, who lost his life
because of his unfortunate political activities and his schismatic attitude
towards Pope Alexander VI, to some extent shared the stormy
temperament of Luther, but he kept aloof from heresy. It has been
aptly said of his peculiar posture on the monument of Worms
that it appeared as if he wished to run away because he felt he did
not fit in properly with Luther’s company. Finally, there is Reuchlin,
the scholarly founder of Hebrew philology, who remained a loyal
Catholic. After a lengthy conflict concerning his theories of the
Talmud, his book, “Augenspiegel,” was prohibited by Leo X, chiefly
on account of the undue use the young German humanists and incipient
Lutherans made of his name. It was only the desire of
throwing Luther into greater relief which procured for this learned
writer an unmerited place on the monument at Worms.

The demand for so-called forerunners of the Reformation originated in a
tendency of the nineteenth century, which has now been
more or less overcome. Scholars admit the disparity of the ways which
led away from Rome and regard it as superfluous to posit any precursors
for the great and original Luther.

It must be admitted, however, that in the theological schools of
Luther’s day there were certain preparations for his doctrine. The
evidence for this statement is supplied by a glance at Nominalism,
particularly in the form in which it was taught by Ockham. True,
at the close of the Middle Ages philosophical and theological Nominalism
prevailed in many universities, without any particular injury
or separatism. The eminent nominalist Gabriel Biel was quite
orthodox in his teaching. But here and there dangerous errors crept
in with the Nominalism inspired by the singular mind of William
of Ockham. Young Luther absorbed some of these with his reading.
“I am a member of Ockham’s school” (\textit{factionis Occamicae}),
he says and acknowledges this passionate and schismatic partisan of
Lewis the Bavarian in his contest with the papacy as his teacher.
Not as though he had educated himself by means of Ockham’s
politico-ecclesiastical writings, or that he had imbibed that author’s
so-called conciliar theories. But certain philosophical and theological
views of Ockham and his disciple, Peter d’Ailly, did not fail to
influence him and several other theologians of the Augustinian
Order.

Ockham disputed the philosophical demonstrability of the existence of
God, of the freedom of the will, and of the spirituality of the soul. He
taught that these truths can be known with perfect certitude only through
faith. A proposition may be false in philosophy but true in theology. The
ultimate cause of the eternal law and of the distinction between good and
evil is solely the divine will. \textit{Per se} an unworthy individual might be found
worthy of eternal life if God has so willed it. All depends upon the will
of God (theory of acceptation); and no supernatural babitus is necessary in
the just.

It is not difficult to discern a trace of these Ockhamist errors in the
teaching of Luther. What is more important is that Luther, going beyond
Ockham, took that external imputation which the latter propounded only
as a possibility, for a reality and entirely eliminated sanctifying grace.
Luther, like Ockham, taught that the same thing need not be true in
philosophy as well as in theology. His repression of reason and his disregard
of ecclesiastical authority were characteristics of Ockham. Both led him
to assign to the emotions or to internal divine inspiration the rank of
evidence, which, independently of the teaching of the Church, assured man of
the true meaning of Holy Writ. The arbitrariness of God according to Ockham
confirmed Luther in his dread of predestination. Finally, it is easy to see
how Ockham’s disregard of true Scholasticism must have reacted upon
Luther’s attitude towards the old school.

Gabriel Biel, whose works young Luther likewise studied, kept aloof
from the Ockhamist errors, for which reason Luther attacked him and the
“Gabrielists.” Biel, under the influence of Ockham, unduly extended the
limit of man’s natural faculties in the realm of virtue, mistakenly appealing
to St. Thomas and the other great Scholastics in defense of his theory. Biel
minimized the effects of original sin, whereas Luther exaggerates them and
combats the “sophists” of Scholasticism, as though they were unanimous
in over-rating the powers of fallen man.

We are here confronted with the negative influence of Ockham. In contrast
with what he had learned at school, Luther was led to adopt an extreme
view, namely, the complete degradation of man’s natural powers
for good. This extreme antithesis confirmed him in his belief that all
things are produced by the omnipotence of God. It was for this reason
that he denounced the Scholastic theologians of the Middle Ages as well
as those of his own time as “swine theologians,” because they overestimated
the powers of man and failed to appreciate the role of grace.

Rationalism and excessive criticism had gone too far in the Nominalistic
schools. Luther was not the only one who was frightened by this tendency.
But the true antidote did not consist in the extreme position adopted by
Luther, asserting the absolute impotence of reason in matters of salvation.

The negative influence of Ockhamism on Luther also appears in his use
of Holy Writ. Despite its appreciation of the Bible, Nominalism did not
properly avail itself of the truths of Sacred Scripture in its treatment of
theological questions. Guided by a correct sentiment, Luther opposes the
study of the Bible to the preponderance of dialectics and the neglect of
positive facts. But his preference for the Bible is extreme. According to him,
the “Word,” \textit{i.e.}, the word of God, is almost the only thing that should be
considered. The “Word” should abolish the evils of the world. It was but
a step from this attitude to proclaiming the Bible as the only source of
faith, to the exclusion of tradition and the Church.

Thus Nominalism, in its Ockhamistic form, appears to be one of the
factors which cooperated in the birth of the so-called Reformation,
partly in virtue of its positive, partly as a consequence of its negative, influence.\footnote
{For a more detailed exposition of the influence of Ockhamism on Luther see Grisar,
\textit{Luther} (Engl. tr.), Vol. I, pp. 130 sqq., where the researches of Denifle are utilized. The
relations of later Nominalism to the Lutheran heresy still await complete clarification.}
