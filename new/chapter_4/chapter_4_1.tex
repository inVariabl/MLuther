\section{Interpretation of the Epistle to the Romans (1515--1516)}

In none of his other Epistles does St. Paul penetrate so deeply into
the questions of grace, justification, and election, as in his magnificent
Epistle addressed to the Christians of Rome. Luther believed
that this profound Epistle furnished the thread that would lead him
out of his labyrinth. Disregarding the tradition of the Church concerning
the meaning of the Epistle, he buried himself in its contents
and brooded over its many mysterious expressions. For him the
sacred document was to become the subject-matter of academic
lectures with entirely new ideas. How often may he not have
wandered up and down the venerable corridors of the monastery
meditating on the significance of the words of the Apostle. His
emaciated form may have become animated, his deep-set eyes may
have flashed, as he imagined to discover in the Epistle to the Romans
the desired solution of his problems. With ever-increasing confidence
he imputed to the Apostle the ideas to which he was urged for the
sake of the supposed quieting of his scruples. Simultaneously, an
arsenal of new weapons against the self-righteous Pharisees within
the Church seemed to open itself to him.

St. Paul sets forth the idea that neither the observation of the law
of nature nor that of the Mosaic law can justify man before God,
but only the grace of God now revealed through the Gospel of
Christ. In this exposition Luther erroneously discovered a denial
of the natural powers of man and the sole causality of God in His
creature—a doctrine utterly foreign to the Apostle’s mind. In the
propositions on the grace of Christ, as set forth by Paul, he discovered
an ascription of the merits of Christ equally foreign to the
mind of the Apostle—a purely external imputation without works
on the part of man. St. Paul discourses sublimely on the freedom of
the Christian believer from the disciplinary law of Moses and on the
freedom of the soul that is directed towards God. From this Luther
inferred that the true Christian was free from all law and formulated
for himself a contrast between the law and the Gospel which implied
fatal consequences. The Apostle, again, paints a striking picture of
the rejection of the Jewish nation from anterior secret decrees, which
prove his notion of predestination to hell. In a similar way his lively
imagination interpreted other doctrines of the Epistle, which he
confidently undertook to explain, despite the fact that his deficient
training rendered him incompetent for the task.

He began to lecture on Romans in the second semester of 1515.
We have his own manuscript of the lectures and a faithful copy
of it, which is preserved in the Vatican Library. Fr. Henry Denifle,
O. P., has the honor of having first edited extensive extracts from
the Roman copy in 1904. Four years later John Ficker published the
original, the existence of which was till then unknown. It belongs
to the Berlin Library.

These publications were of inestimable advantage for the student
of Luther. As contrasted with the numerous defective reproductions
of these lectures previously circulated, we have here for the first time
an authentic insight into the genesis and original form of Luther’s
teaching. Since that day, investigators have been occupied with the
task of analyzing and discussing these lectures. The principal outlines
are clear, but there are differences of opinion with respect to
various details. This is all the more natural, since Luther, when he
delivered these lectures, was still undergoing a spiritual process of
development. He sometimes contradicts his own views and retains
various elements of Catholic thought which cannot be reconciled
with his changed views. The opening sentences disclose the impulsive
vigor with which he advances towards his new position. He commences by
stating that St. Paul, in his Epistle to the Romans, was
desirous of eradicating “all wisdom and righteousness” of the terrestrial man;
that he wished to destroy, “in heart and marrow,” all
man’s works, even the best-intentioned and noblest, and all his
virtues. Instead, he wished “to cultivate and glorify sin,” that is,
to describe the permanent hereditary sin of mankind, in order to set
up an alien righteousness, \textit{i.e.}, the grace of Christ imputed to us by
faith. In this manner, he says, St. Paul inculcates the doctrine of self-annihilation
on the one hand, and, on the other, resignation to the
sole causality and omnipotence of God.
