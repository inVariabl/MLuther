\section{The Psychological Background}

At the end of this rather lengthy discussion of the genesis and
contents of the Lutheran dogma it is proper to indicate briefly how
the principal doctrines of Luther were conformed to the personal
mood of their discoverer. The new theories of grace and justification,
which were at first intended to quiet the monk afflicted
with a disordered temperament, were forthwith raised to a general
norm. Precisely because everything was so absolutely personal with
him, the discoverer of these ideas, so diametrically opposed to tradition,
plunged into them with a vim that would be incredible were
it not evidenced by his writings, especially his lectures on the
Epistles to the Romans and to the Galatians, and by his disputations. His
temperament furnishes the key to this remarkable phenomenon. He felt called
upon to reveal to ignorant and misguided
humanity the truth that had dawned on him--called by his office as
doctor of theology; called upon also to denounce all abuses of doctrine
and life. In reality the doctorate obliged him to teach according
to the mind of the Church, who had invested him with this dignity
and to whom he had promised obedience. He, on the contrary, proclaimed
that he would speak freely because he had the apostolic
commission to teach. As a doctor, he said, it was his duty to reproach
all, even those who occupied the highest places, if they were
guilty of wrong-doing.\footnote{\textit{Op. cit.}, Vol. I, p. 228.}
It does not occur to him that, if he was
desirous of effecting a genuine reform, he would have to direct his
censures to the proper places and utter them in a becoming manner,
not by blustering in the pulpit or before immature students.

In fact, he does not reflect at all, but allows himself to be carried
away by his emotions. The fatal thing was that he believed himself to be moved by God.

Lacking the true concept of the Church and of ecclesiastical authority,
the pseudo-mystic reformer believes that his ideas are inspired
from above and his steps directed by God, whose guidance
he professes to follow blindly. He is convinced that he is not seeking
temporal advantages, and we may not gainsay him when he
declares that he is and desires to remain a poor monk. At the same
time he perceives, not without a basis in fact, an excessively large
number of evils in the life of Catholics. Their presence seems
to justify, nay, to challenge his intervention. Consequently, he reasons,
the efficacious hand of God must rest upon him, particularly
since he desires only to exalt Christ.

Later, too, he always firmly believed that he was acting in conformity
with “God’s acts and councils,” though this conviction,
in matter of fact, did not persist in periods of “temptation.”
Reviewing the commencement of his career as a reformer, Luther
says that he went ahead “like a blinkered charger.”\footnote{\textit{Op. cit.}, Vol. VI, p. 163.}
Basing upon
his own example and his mystical theories he formulates the following
principle: “No good work happens as the result of one’s own wisdom;
but everything must happen in a stupor.”\footnote{\textit{Tischreden}, Weimar ed., I, Nr. 406.}
It was not stupor nor
intoxication, however, that inspired the great churchmen of the
past to give expression to ideas that moved the world or to perform
their benign deeds (such as the renewal of medieval life by St.
Francis of Assisi, St. Dominic, and Gregory VII). Their achievements
were the product of mature reflection, accompanied by humble
self-denial, fervent prayer, and close attachment to the heart of the
Church. They wrestled with difficulties without self-confidence.
Luther is so full of self-confidence that he regards every contradiction
as a confirmation of his position. For, as he repeatedly declared,
both at the commencement of his career and afterwards, a good
cause is bound to meet with contradiction; in fact opposition proves
that it is acceptable to God.

And what about responsibility? “Christ may witness, whether
the words I utter are His or mine; without His power and will
even the pope cannot speak.” Thus he writes to his fatherly friend
Staupitz after the great movement had begun.\footnote
{\textit{Briefwechsel}, I, p. 199 (May 30, 1518.)}
Despite the ravings
of our opponents, he continues, I must now appear in public, though
I have loved seclusion, and would much rather have preferred to be
a spectator of the stimulating intellectual movement of our age, than
to exhibit myself to the world. “I seek neither money, nor renown,
nor glory. I possess only my poor body, which is bowed down with
weakness and every kind of affliction. If I, whilst engaged in the
service of God, had to sacrifice it to the cunning or power of my
enemies, they would but shorten my life by one or two hours.” There
was no conquering such self-conceit. The appearance of goodness is
a powerful motive. Luther during the entire progress of his tragic
monastic development was deceived to a certain degree by the semblance
of goodness.\footnote{Characteristic parallel traits are observable at the beginning of the last century in
the religious Separatists in Bavaria, who also arrived at the Lutheran doctrine of justification
by faith alone, etc. Cf. the articles on “Boos” and “Gossner” in the \textit{Kirchenlexikon}.
}

The result of his conflicts, as above described, though potentially
dangerous to Christianity, is attributable not so much to an evil will
or to any conscious intent to destroy, but rather to his abnormal character,
to mystical “will-o’-the wisps,” and to the prevalence of unusual
abuses. It was not internal “corruption” that showed him the way;
we have no proofs for such an assumption; but he was goaded by a
combination of less culpable factors. In the background there always
threatened the terrors of a just God and eternal predestination. The
moral phenomena attending his first public appearance, the defects of
his character, and his prejudice against good works, would seem to
decide the question of responsibility against him. He incurred a clear
and terrible responsibility when he was confronted by the adverse decision
of the Church and her threat of excommunication. That he
refused to submit to the divinely appointed authority was the great
fault which entailed his ruin.
