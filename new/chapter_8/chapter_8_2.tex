
\section{The Sentence of Outlawry}

The imperial councilors, even before the opening of the diet, had
taken the point of view that the Emperor, by virtue of his own
authority, was empowered to pronounce sentence of outlawry
against Luther. The two nuncios demanded that he issue the
edict as a sacred duty, since by the terms of his royal oath he was
the sworn protector of the Church. The high-chancellor, Gattinara,
in opposition to the Saxon Elector, announced that the edict of
outlawry should be issued “with the knowledge, but not with the
counsel and consent of the princes.”\footnote{Kalkoff, \textit{Luther und die Entscheidungsjahre}, pp. 187 sq.}
On February 19, before
Luther arrived in Worms, the diet had left it to the Emperor to proclaim
“the proper mandate,” \textit{i.e.}, the sentence of outlawry in the event
of Luther’s refusal to recant. On April 20, as we saw above, the diet
expressly renewed its consent to this step. When the Emperor, without
necessity, but in a spirit of accommodation, inquired as to the
best method of procedure, the estates desired that the mandate be
submitted to them, so that “at the request of His Majesty they
might indicate their opinion in this matter.” But, in matter of fact,
the edict was not submitted to the diet, whose members had already
begun to scatter. Nor was it necessary to do so. The editor of the
recently published \textit{Reichstagsakten} writes: “There can be no doubt
that the Emperor was now justified in issuing an edict in his own
right, without further consultation of the estates.” Moreover, the
legitimacy of the imperial edict of outlawry was freely admitted by
the estates assembled in 1524 for the diet of Nuremberg, which
declared that the edict was issued “with mature deliberation and
after the Electors, princes, and other estates had been consulted.”\footnote
{Cf. Paulus, \textit{l.c.} (note 13 above), for the documentary evidence for all these assertions.}
Hence, the more recent objection to the legality of the edict of
Worms as an imperial measure are unfounded.

The edict was drawn up under date of May 8, after Luther, whose
whereabout was kept secret from everybody, had already spent a
number of days at the Wartburg, and after his safe-conduct had
expired.

This document was chiefly the work of Aleander, who unfolded
therein the whole penal system of the Middle Ages. Since the days
when Christianity issued from the catacombs, many changes had taken
place in the public relationship between Church and State and
modifications had become necesssary even in the interest of practicability,
if for no other reasons. But the eyes of Aleander, the Italian,
only saw the dangers threatened by the negligence of the bishops
and the immense growth of the Lutheran movement. For this reason
the nuncios, together with other religious-minded people, prevailed
upon the young Emperor, who was zealous in his defense of the
Church, to invoke the entire legal machinery of the Middle Ages
against heresy and the imminent revolt. Nor were the traditional
forceful phrases spared which denounced heresy as the greatest of all
evils.\footnote{Text of the Edict in the \textit{Reichstagsakten}, ed. by Wrede, II, pp. 640 sqq.}

In the introductory part of the edict Charles declared that his imperial
office obliged him to protect the Church and that he was in duty bound to
heed the Providence which had entrusted him with many countries and with
greater authority to promote the welfare of Christianity than was wielded
by any of his predecessors. The protection of religion was traditional with
his family, through which, on his father’s side, he was related to the most
Christian emperors and archdukes of Austria and the dukes of Burgundy,
and also, on his mother’s side, which originated in Spain and Sicily. Hence
it was his duty to resist the new heresies, which had originated in hell, after
the pope had solemnly condemned them and excommunicated the man who
propagated them. In vain had Luther been urged to abandon them at
Worms.

The edict goes on to enumerate the errors of the heretic and recalls his
very words that, if Hus, who had been burned at the stake, was a heretic, he
himself (Luther) was a ten times greater heretic than Hus. In addition, it
says, Luther destroys obedience to authority and publishes writings which
serve but to foment revolt, schism, and bloody dissensions. He proclaims
a brand of Christian liberty destructive of all law, the liberty of irrational
beasts. The document styles him a devil in human form and says, if Germany
and other countries are not to perish, it is the Emperor’s duty to enforce,
without delay or mitigation, “the laudable constitutions of the Christian
Roman emperors, which they promulgated for the punishment and extermination
of heretics.” Therefore, Luther is declared outlawed for the whole
extent of the empire, “with the unanimous consent and will,” as the document
has it, of the Electors, princes, and estates of the diet of Worms.
Consequently, no one was allowed henceforth to provide him with shelter,
food, drink, etc.; on the contrary, he is to be apprehended wherever he may
be found and surrendered to the imperial authorities. Those who disobey the
edict incur the penalties of high treason and will themselves be treated as
outlaws, liable to the forfeiture of all royal prerogatives, feudal tenures,
favors, and liberties which they received from the emperor and the empire.
The protectors and adherents of the heretic are to be apprehended, and their
property is to be given to those who proceed against them, to be used for
their own benefit.

All the literary productions of Luther, even if they incidentally contain
some good things, are to be burnt and shunned like poison. The plague of
anti-religious books, pamphlets, pictures, etc., composed by others, as well
as all libels against the pope, prelates, princes, universities, etc., are to be
exterminated. Books which in any manner touch on matters of faith may be
printed only after they have been submitted to the censorship of a bishop or
of the nearest theological faculty. All other literary productions require the
episcopal approbation.

The strict ordinance concerning publications was intended to
check an evil which had assumed boundless proportions. The edict
was in strict accord with the severe prescriptions of Leo X and the
Fifth Lateran Council (1515) regarding preventive censorship.\footnote
{Pastor, \textit{History of the Popes}, tr. by R. Kerr, Vol. VIII, 2nd ed., 1923, pp. 397 sq.}
The new invention of the “right noble art of printing,” as it is
styled in the Emperor’s edict, had degenerated in virtue of a deluge
of writings and pamphlets which disseminated errors, fostered agitation
, and preached ecclesiastical and social revolution. Regardless of
consequences, wood-cuts were used to heap mockery upon the hierarchy
as well as the rulers who did not sufficiently comply with the
desires of the nobility or the oppressed peasantry. Luther, by his
polemical tracts against the Church, had, from the beginning of his
career, set the example of evading the existing censorship laws.
During the diet of Worms quite 2 number of publications appeared
in favor of the new doctrines. Aleander complained that his character
was subject to defamation by publications and pictures in the
city of the high assembly. At the time when Luther himself was
condemned, and even prior to his condemnation, there were circulated
at Worms and in other places in the empire, pictorial representations of
him with the dove, the symbol of the Holy Ghost, hovering above his head.
Other pictures represented him with a halo. A
booklet on the Passion of Martinus, patterned after the Gospel narrative
of the Passion of Christ, was published, in which he was
glorified as a persecuted hero.

When the report of the edict of outlawry reached Luther at the
Castle of Wartburg, he wrote to Amsdorf:\footnote{\textit{Briefwechsel}, III, p. 151; May 12.}
“A cruel edict has
gone forth against me, but God will laugh at them” (Ps. xxxvi, 13).
Spalatin he informed\footnote{\textit{Ibid.}, p. 153; same day.}
that he was aggrieved at this procedure, not
for his own sake, but because his opponents thereby heaped disaster
upon their heads and the time of their punishment was evidently
at hand. He adds in reference to his opponent, Duke George of
Saxony, a man distinguished by the traits of a noble character:
“Would that this swine of Dresden were found worthy to kill me
during a public sermon! If it pleased God, I should suffer for the
sake of His Word. The will of the Lord be done.”

In these first letters he also rejoices in the unchained power of
the masses (\textit{moles vulgi imminentis}), who were, he said, preparing
terror for the authors of the edict and all his persecutors; it is evident,
he adds, that the people are unwilling and unable to tolerate any
Jonger the yoke of the pope and the papists.

Rendered confident by these phenomena, he continues to indulge
his scornful denunciation of the edict: “Swine and asses are able to
see how stubbornly they act \dots What if my death should prove
a disaster to you all? God is not to be trifled with.” Thus he exclaims
a few years later, when, to show his contempt, he incorporates the
entire lengthy document in his work \textit{Zwei kaiserliche uneinige Gebote}
(Two Discordant Imperial Commandments; 1524), and accompanies
them with biting comments.\footnote{\textit{Werke}, Weimar ed., XV, pp. 254 sqq.}
In a frenzy of higher inspiration he
advises that “everyone who believes in the existence of God keep
away from the commandments (of the imperial proclamation).” “If
they kill me, there will be such a slaughter as neither they nor their
children will be able to overcome.”
