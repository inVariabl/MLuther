\section{Luther on Trial Before The Empire}

After Luther had been put under the ban, Pope Leo X addressed an
earnest letter to Charles V, in which he demanded that those vested
with the proper authority should execute its secular effects.

The diet of Worms had been convoked for January 6, 1521. Elector Frederick
of Saxony, ever intent upon delaying the trial of
Luther, proposed that Luther be permitted to come to Worms
for a hearing before the diet. In a letter to Spalatin the rebellious
monk expressed his willingness to make the journey.\footnote
{\textit{Briefwechsel}, III, p. 24. Luther here solemnly advises Spalatin (and through him the
Elector) of the dedication of his life to the cause he had espoused.}
But the opposition
of the party loyal to the Church, especially that of Cardinals
Aleander and Caraccioli, the papal nuncios attending the diet, the
Elector, and subsequently also the Emperor, temporarily abandoned
this plan. When Luther was apprised of this decision, he expressed his
displeasure (\textit{cum dolore legi}); for he sedulously strove to create the
impression that he had not been accorded an adequate hearing. His
fancy was charmed by the prospect of appearing on the world stage
at Worms. How much could he not expect in furtherance of his
cause from a courageous testimony given there in the presence of
the empire! What had he to fear, protected as he was by an imperial
safe-conduct and the support of his friends among the knights?
Courage and presumption he possessed in a plentiful measure.\footnote
{On the preliminaries and Luther’s appearance at the diet of Worms see Kalkoff, \textit{Luther
und die Entscheidungsjahre}, 1917, pp. 187 sqq., and the same author’s \textit{Der Wormser
Reichstag}, 1922. Many of Kalkoff’s assertions, however, are questionable. Cf. in addition H.
v. Schubert, \textit{Quellen und Forschungen über Luther euf dem Reichstag zu Worms}, 1899,
A shorter but more reliable description in Janssen-Pastor, \textit{Geschichte des deutschen Volkes},
Vol. II, 20th ed., pp. 173 sqq. See also Grisar, \textit{Lutherstudien}, No. I: \textit{Luther zu Worms}, etc.
pp. II sqq.}

Luther’s opponents at the diet did not tarry in beginning operations
against him, On Ash-Wednesday, February 13, Aleander delivered an oration,
which lasted three hours and was received with
applause. He recommended that the papal Bull of excommunication
be promptly enforced. Among other things he pointed out that the
Saxon innovator had rebelled also against imperial statutes, and that
his agitation enkindled insurrection and civil war. At the conclusion
of the address, Charles V declared his intention to proceed at once,
and added that it was neither necessary nor expedient to grant Luther
a hearing. This was the correct point of view; for after the definitive
judgment of the Holy See, the diet was incompetent to reopen the
case, especially before the secular authority. If there was merely a
question whether Luther recanted or not, this could be decided in
the diet without summoning him. Luther himself stated that it
was not necessary for him to leave Wittenberg solely for this purpose.

Notwithstanding the activity of Cardinal Aleander to keep the excommunicated
monk away from Worms, some members of the diet
were in favor of giving him a hearing. This sentiment was nourished
by the approval which Luther’s pronouncements against the oppressive papal
taxation and the Roman procedure in the bestowal of benefices met with
in the assembly. The upshot of this dissatisfaction appears in the “grievances”
(\textit{gravamina}) voiced by the diet.

On February 19, the estates requested the Emperor to permit
Luther to appear at Worms, not, indeed, for the purpose of disputing
with him about religion, but that his recantation might be demanded
by experts appointed for this purpose. In the event of his recantation,
it was intended to interrogate him “about other points and matters.” If
, however, he refused to recant, the Emperor was to issue a
suitable mandate\footnote
{\textit{Deutsche Reichstagsakten unter Karl V.}, edited by A. Wrede, Vol. II, Gotha, 1869, pp.
316 sqq. Cf. Janssen-Pastor, op. cit., p. 197.}
declaring the concurrence of the empire in the
papal ban with its penalties.

In consequence of this request of the estates, the Emperor cited
Luther to appear in Worms, March 6, 1521.\footnote{\textit{Briefwechsel}, III, pp. 101 sq.}
The citation was handed
to him at Wittenberg, on March 26, by Caspar Sturm, the imperial
herald, who had orders to escort Luther to the diet. An imperial letter
of safe-conduct for the journey to and from the diet was issued,\footnote{\textit{Ibid.}, pp. 103 sq.}
for which reason the summons stated that he had to fear no violence or
injustice.

Before he started on his journey, Luther forwarded the printed
commencement of his explanation of the Magnificat to the future
Elector, Duke John Frederick, who was very favorably inclined
towards him, and to whom
he had dedicated this work. He also wrote
an exhortation to Wenceslaus Link on the completion of the latter’s
violent diatribe against the Italian Dominican, Ambrosius Catharinus,
who had vigorously attacked him. Then, clad in the habit of his Order,
and firmly resolved not to recant, Luther confidently set out,
on April 2, on his famous visit to the city of Worms.

His adherents saw to it that a welcome was extended to him everywhere.
His journey almost assumed the proportions of a triumphal
procession. At Erfurt he preached to a great concourse of people on
his newly discovered way of salvation. “What matters it,” he exclaimed,
“if we commit a fresh sin, so long as we do not despair, but
remember that Thou, O God, still livest. Christ, my Lord, has destroyed
sin; then at once the sin is gone.”\footnote
{Grisar, \textit{Luther}, Vol. II, p. 339; Vol. III, p. 180. Cf. the section on Luther’s journey
to Worms and his appearance at the imperial diet, \textit{ibid.}, I, pp. 379 sqq., and Köstlin-
Kawerau, I, p. 407.}
When, owing to the overcrowded condition
of the church, the galleries cracked and a panic
ensued, he forthwith adjured Satan and blamed his spite for the disturbance.
A chronicler ascribes the restoration of quiet to Luther’s
powerful command to the devil and says it was the “first miracle”
performed by the man of God. During his sermon at Gotha, the
devil cast stones from the gable of the church. According to a letter
to Spalatin, written in Frankfort, Luther likewise attributed to the
devil a severe illness from which he suffered and which seemed to
threaten the continuation of his journey. What was more serious,
however, was the news which reached him on the way of an edict
which the Emperor had issued concerning his books, that they were to
be delivered up to the authorities everywhere. This made Luther realize
that the Emperor was resolved on intervention. He said later that
this realization caused him to tremble with fear. From Oppenheim
he addressed a letter-to Spalatin, who had cautioned him; he said he
would go to Worms, even if the devils were as numerous in that town
as the tiles on the roofs.\footnote{\textit{Briefwechsel}, III, p. 120.}
At that time he also wrote: “We shall enter
Worms in spite of all the gates of hell and all the powers of the
air.”\footnote
{Cf. \textit{Briefwechsel}, III, p. 122, annotation, 5, where Spalatin’s (German) \textit{Annals} (ed. by
Cyprian, 1718, p. 38) are quoted and reference is made to Luther’s address to Sturm recorded
in the \textit{Table Talks}, No. 2609 (Weimar ed., III, No. 3357b, p. 285).}
In a tavern at Frankfort he was in such high spirits that he
played the lute in the presence of many guests. His opponent Cochlaeus afterwards
ridiculed this incident and said that here one had an
occasion to see Orpheus perform in cowl and tonsure.

At Worms there was a party, centered around the Emperor’s confessor,
a Franciscan friar by name of John Glapion, who veered to
and fro between the papal nuncio and the declared friends of
Luther relative to the latter’s trial. Glapion wanted Luther to appear
before him at the castle of Ebernburg, instead of the diet in Worms.
Luther’s friend Bucer was instructed to try to induce him at Oppenheim
to acquiesce in this proposal. Luther, however, refused to
abandon his journey to Worms, particularly since he had been apprised of
the plot which the revolutionary knights had formed
against Worms and the diet in the extreme case that violence should
be offered to his person.

In the forenoon of April 16, the flourish of the watchman’s
trumpet on the spire of the cathedral announced to the inhabitants
of Worms that Luther was entering the city. He was accompanied
by an honorary escort of about one hundred knights. A large multitude soon
gathered about him. He rode to the residence of the
Knights of St. John, where he took up his abode. After he had
alighted, he “gazed about with the eyes of a demon,” as Aleander
says (who, however, never saw him and declined to attend the sessions).\footnote
{Grisar, \textit{Luther}, Vol. IV, p. 355.}
“God will be with me,” he said. Clad in a cowl, with a leather
girdle, and a scanty tonsure about his brow, he exhibited a wretched
figure, emaciated as he was by fatiguing labors and the stress of
uninterrupted excitement. But his eyes beamed with a brilliant, deeply
glowing, and defiant lustre. A faithful representation of his appearance
in 1521 is supplied by Lucas Cranach’s etching which forms
the frontispiece of the German edition of his works. The lower jaw,
the nape of the neck, the mouth and the eyes conspire to give an
impression of defiant self-reliance. It is the best portrait of Luther
which we have, all the others being considerably “toned down.”

On the following day Luther was escorted by the marshal of the
diet and the imperial herald to an assembly of certain members of
the diet who had been summoned to meet the Emperor in the bishop’s
court. The edifice no longer exists. The Wittenberg jurist Schurf
accompanied him as his counsel, and, together with the counselors
of the Elector of Saxony, assisted him in the preparation of his address.
He entered the hall with a forced smile on his countenance,
and critically scanned the audience. He was unable, however, to
conceal a feeling of depression. He delivered the few brief words
that he was allowed to utter, according to a Spanish report, “with
great trepidation and little serenity in countenance or gestures.”
It was not surprising that the gravity of the moment affected even
his habitually defiant disposition. The speaker of the assembly, and
official of the Archbishop of Treves, the adroit John von Eck (not
to be confused with Dr. Eck of Ingolstadt), addressed two questions to
him: first, whether he admitted the authorship of the books
that had been placed before him; secondly, whether he was prepared to recant.
He acknowledged the authorship of the books, after
their titles had been read to him. As to the question of recantation,
he begged for time to reflect on account of the importance of the
matter for his spiritual welfare. He intended to delay matters, but
his expectations were disappointed when, after a brief consultation,
the Emperor granted him but one day.

After Luther had left, his appearance was being discussed, and the
Emperor said: “This fellow will not make a heretic of me.” The most
diverse opinions were relayed to Aleander in his retirement. He
wrote to Rome that some regarded Luther as deluded, others as
possessed, and others again as a man filled with a holy spirit.\footnote
{Brieger, \textit{Aleander und Luther}, pp. 143, 147; Kalkoff, \textit{Die Depeschen Aleanders vom
Wormser Reichstag}, 2nd ed., 1897, pp. 167, 171.}

In the evening Luther, assisted by his counsel, carefully prepared
a statement which he intended to make the following day. He was
“in good spirits” during his intercourse with others.

The decisive session was held in the same court, but in a more
spacious hall and in the presence of a larger audience. Luther appeared
with an escort of friends who showed a bold front. He himself was noticeably
less timid than on the previous day, and exhibited
a more deliberate reverence toward the high assembly. The official of
the archbishop of Treves addressed words of admonition to him
and inquired whether he was prepared to recant. Thereupon Luther,
in a firm tone, began the oration which has since became so famous.
He contended that his writings were partly of a religious character,
partly directed against the pope and his adherents, and partly replies
to individual opponents which had been forced from him. They
contained nothing that was censurable. He entered upon a detailed
explanation of his writings against the pope, designedly availing
himself of this occasion to complain bitterly against the Roman
tyranny “in my Germany,” as he put it, “to which I owe my services.”
He was relying on the temper of the princely audience who,
as he well knew, were ill-disposed toward the abuses prevalent at
the papal court. He also spoke of the judgments of God which
overtook the rulers of the Old Testament who resisted the Word
of God. He was not interrupted. After he had ended, Eck called his
attention to the fact that his doctrines had long ago been condemned
by the Church, that their condemnation had been reaffirmed by
the recent proclamation of the pope, and that it was unthinkable
that all Christendom had been groping about in darkness up to
then. The official concluded his remarks with the demand that
Luther should state clearly and unambiguously whether or not he
was ready to retract. Luther replied: “If I am not convinced by
proofs from Scripture or clear theological reasons, I remain convinced
by the passages which I have quoted [in my book] from
Scripture, and my conscience is held captive by the Word of God.
I cannot and will not retract, for it is neither prudent nor right to
go against one’s conscience. So help me God, Amen!” These last words
did not resound through the hall with tragic solemnity, as Protestant
biographers are wont to put it. On the contrary, they were scarcely
“audible, according to the oldest sources, because of the great uproar
and indignation which ensued, and also because of the fact that the
audience began to crowd out of the stifling hall, which was illuminated
by torch-lights. Nor did Luther’s declaration conclude with
the celebrated exclamation: “Here I stand. I cannot do otherwise.
God help me, Amen.” Long ago even Protestant scholars have
demonstrated that this sentence is unhistorical. The expression, “So
help me God, Amen,” was a formula with which it had been customary, since
the Middle Ages, to conclude solemn speeches. It was
simply a Christian paraphrase of the Latin ``\textit{dixi},'' I have spoken.\footnote
{Grisar, \textit{Luther}, Vol. II, pp. 65 sq., 75 sq.; \textit{Lutherstudien}, Vol. I, pp. 26; 42.}

After a further exchange of words with the official, Luther accompanied
by his escort, left the episcopal court. Outside, feeling
that he had been victorious, he imitated the Landsknecht, when they
celebrated a successful surprise-attack, by swinging his arms about
in the air and spreading out the fingers of his hands. “I have succeeded,”
he exclaimed; “I have succeeded!” In the tavern he repeated this demonstration,
as he greeted those who awaited him at
the bar. During the draughts of merriment, to which they abandoned
themselves, no one realized the gravity and responsibility of the
situation.

Those who had made preparations for a violent coup d’état at
Worms were also in the main satisfied. As a result of the ferment
among Luther’s sympathizers, a placard had been posted at the townhall
during the night, announcing that hostilities had been declared
upon the “Romanists,” \textit{i.e.}, the loyal adherents of the Church, by
four hundred unnamed members of the nobility, who, it was alleged,
were prepared to launch an attack with a force of 6000 men. The
revolutionary watch-word “Bundschuh,” thrice repeated, appeared
in place of the signatures. The word referred to the so-called auxiliaries
supplied by the peasant estate. It was a custom of the peasants to
wear strapped shoes. Referring subsequently to the protectors of
Luther who were prepared to do battle in his defense, Thomas Müntzer
told Luther: “You would have been stabbed to death by the
knights, if you had hesitated or recanted.” He said this to
expose the vain-glory with which Luther was accustomed to boast of
his courage at Worms, in the presence of the great men
of the empire.
That there was danger of the safe-conduct being violated, is a fable
of subsequent invention, nourished no doubt by Luther’s assertions.\footnote
{Cf. Grisar, \textit{Lutherstudien}, I, p. 88. Annotation 63 of the \textit{Table Talks}, Weimar ed., No.
5432b.}
The Emperor was determined that the promise should be kept and
that the return of the obstinate monk to Wittenberg should be
unmolested.

Prior to Luther’s departure, several days were consumed in an
endeavor to bring him round. It was done at the instigation of the
estates who feared unrest in the city and in the empire. Their decisive
declaration, issued on the twentieth of April, was to the effect that,
if Luther did not yield, they would sustain the Emperor in what
ever measures he might take against the obstinate heretic.\footnote
{Cf. N. Paulus, “\textit{War das Wormser Edikt ungesetzlich?}” in the \textit{Histor. Jahrbuch},
1918—19, pp. 269 sqq.}
The archbishop of Treves, Richard von Greiffenklau, with the assistance
of others vainly endeavored to persuade Luther to modify his stand.
Equally fruitless were the efforts of the scholarly and highly respected
John Cochlaeus, dean of the chapter-foundation of Our Lady at
Frankfort on the Main. He narrates that Luther listened to him
with tears in his eyes (which the latter afterwards denied) and
appealed to a private revelation which he claimed to have received
(\textit{est mibi revelatum}).\footnote
{Grisar, \textit{Luther}, Vol. IV, p. 258; Vol VI, pp. 143 sq.; \textit{Reichstagsakten} (see note 3
above), p. 630. Luther’s denial, \textit{Opp. Lat Var.}, VII, p. 48. The offer of disputation, \textit{Reichstagsakten},
p. 629; cf. \textit{Table Talks}, Weimar ed., No. 5432b.}
An offer which Cochlaeus made to dispute
with Luther in public before authorized judges, likewise proved
futile. Luther’s friends within the diet encouraged him. The Elector
Frederick alone, true to his usually circumspect and diplomatic habit,
held back. Never did he converse personally with his Wittenberg
professor, but only through intermediaries. He is reported to have
said: “Doctor Martinus is far too bold for me.”

Before the termination of the imperial safe-conduct, Frederick
excogitated a plan to safeguard his protégé against the dangers likely
to result from the imminent declaration of the imperial ban. Luther
was initiated in the scheme while yet at Worms. A simulated attack
was to be made upon him by the soldiers of the Elector on the homeward
journey, and he was to be taken into custody.

Accompanied by his escort, Luther left the city unobserved in
a carriage, on April 26. He had received an order not to preach on
his journey, but he disregarded it, contending that the Word of
God is untrammeled. Having arrived at Friedberg in the Wetterau,
he addressed two solemn letters defending himself, one to Charles
V, the other to the princes and estates at Worms.\footnote{\textit{Briefwechsel}, III, pp. 129, 135; both letters are dated April 28.}

These letters were
immediately published in order to create sympathy in his behalf.
In the first of them, the original of which is now on exhibition in the
“Luther-Halle” at Wittenberg, he solemnly appeals to St. Paul and
declares that he can no more deviate from the Gospel of Christ than
the Apostle, who was ready to anathematize even an angel if he
preached another Gospel. It was not the will of God that His Word
should be subject to man; in matters pertaining to salvation no one
may depend upon a mere mortal. In saying this, he was oblivious of
his own claim that he alone was able to interpret the Gospel properly
and thus show men the way to salvation. He pretended to forget the
fact that, when the Church requires men to rely on her doctrine and
her interpretation of Holy Writ, this is no demand of fallible men,
but of a supernatural institution established by God, invested with
divine authority and protected against error—an institution in which
the living Christ continues His operations until the end of time.

He who understands the true character of this divine institution will take
no stock in the assertion, current among Protestants, that
Luther represented liberty of conscience when he took his “heroic”
stand at Worms. For freedom of conscience is not violated by the
demand to submit to a divinely appointed teaching authority; on
the contrary, conscience is thereby directed to the certain possession
of higher truth, to contentment and true happiness. The statement
that Luther at Worms struggled for complete liberty of research
and autonomy of reason as the domain of future civilization, must
he read in the light of his own express declaration in his concluding
address that he was bound by the Word of Scripture, hence by the
compelling power of revelation—understood, of course, in the sense
in which he interpreted it. In brief, the diet of Worms does not
mark the birth of intellectual liberty, neither for conservative Protestantism,
nor for that Neo-Protestantism which is rapidly developing into infidelity,
nor yet for the modern world.\footnote
{For more details see Grisar, \textit{Lutherstudien}, I, ch. v: ``\textit{Luther zu Worms, ein Kimpfer
für Geistesfreibeit?}'' (pp. 28 sqq.)}

The tragic schism, caused by subjectivism in breaking away from
the ancient and venerable universal Church, is all that remains of
the incident at Worms. But if one were to abstract entirely from
the religious aspect of the rupture, the consequences of the so-called
declaration of liberty would show that no genuine benefit accrued
to Germany, which has been growing weaker and more disunited
ever since.

Shortly after Luther’s appearance before the diet of Worms, this
event was so much ‘exaggerated by his friends, that a great many
legends entwined themselves about it. As he set out for home, it
was impossible for Luther to foresee the halo with which these days
would be surrounded in later years. Among the legends referred to is
the alleged text of a prayer which he was said to have recited at
Worms. A staff which he had stuck into the ground, in testimony of
the truth of his doctrine, was said to have grown into a marvelous tree.
George von Frundsberg, the leader of the Landsknecht, was
said to have declared in the presence of Luther, prior to his appearance
before the princes and estates, that God would not abandon
the little monk if he were in the right. The nuncio Aleander is
quoted as saying: “If the Germans repudiate the rule of Rome,
we shall take care that they perish in their own blood amid internecine
struggles.” Aleander was falsely reputed to have been a baptized
Jew, or, as Luther said, an infidel who had lost faith in Christianity.
The only true element in the charges made against him is that his
past history was not blameless, because he had been infected by
the Italian Renaissance, and that he often expressed himself imprudently
in his letters and addresses. Of course, in the eyes of the
Lutherans, the representatives of the “papistical” party at Worms
were all venal scoundrels devoid of character, who condemned
Luther contrary to their own convictions and acted only for the
sake of papal favor and reward. The Archbishop of Treves is
said to have made an attempt to poison Luther at a banquet, and
Cochlaeus’ offer of disputing with Luther was a ruse for endangering his
personal safety. Luther himself contributed his share to these
legends. He later on boasted of the ineffable courage with which,
alone and forsaken at the diet, he jumped into the very jaws of
“Behemoth.” He alleged that the Emperor had outlawed him even
before his (Luther’s) arrival, by revoking his safe-conduct in virtue
of the edict condemning his books. He also claimed that the Emperor
proceeded against him after his departure by means of a surreptitious
and invalid proscription without the sanction of the diet.\footnote
{On these and other false legends cf. \textit{ibid}. VI: “\textit{Lutherfabeln vom Augsburger
Relchstag}” (pp. 36 sqq.)}

Apprehensive of the future, Luther left Worms in a carriage, on
May 4, accompanied by Amsdorf and a fellow-monk from Möhra;
he journeyed in the direction of Gotha, when five mounted horsemen, according
to a preconcerted plan, stopped the carriage near
Waltershausen, dragged Luther out, placed him on a horse, and, by
detours selected for the sake of secrecy, brought him to Wartburg
castle, near Eisenach, where they arrived about eleven o’clock at
night. Amsdorf, who had been initiated into the affair, was permitted
to continue his journey, after he had roundly abused his
assailants for the sake of pretense. The monk from Möhra escaped.
The Wartburg was the property of the Elector of Saxony; not
wishing to know where Luther was, in order to escape embarrassment,
he had permitted his counselors to select the place. For a considerable
space of time, Luther disappeared from the scene. His patron could
not have resorted to a better expedient to save him.
