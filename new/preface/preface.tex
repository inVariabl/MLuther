\chapter*{Preface} % Introduction chapter suppressed from the table of contents

The diverse and even profoundly contradictory criticisms to which
Luther has been subjected are apt to cause dismay to anyone who
undertakes to occupy himself more closely with his biography. This
divergence of opinions is a result not only of the divided confessional standpoint, but also of deviating historical evaluations of the
facts. Even the Catholic opponents of Luther have never agreed on many points.

In my work on Luther, which appeared in three volumes in the course of 1911 to 1912,
I endeavored to the best of my ability to establish a presentment of Luther and his work upon an unimpeachable basis,
derived from all genuine sources.\footnote{\textit{Luther}, 3rd ed., Freiburg i. Br., 1924--1925, Herder. English translation by E. M. Lamond, 6 vols., London 1913--1917.}
For this purpose it was necessary to treat many aspects of the subject quite comprehensively.
Others which presented fewer difficulties, could be passed over
or summed up briefly. The labor involved in the removal of untenable
conceptions prevented me from offering a uniform and well balanced biography in those comprehensive volumes.

For this reason I have always cherished a desire of submitting a
more compact description and a proper delineation of the life of
Luther in a less voluminous work based on the conclusions that
have been established, particularly as supplemented by the researches
into Luther’s history which have since been instituted with extraordinary activity, especially as a result of the Luther jubilee of
1917. Meanwhile, a large number of special studies, great and small,
have been published. Then, too, the Weimar Edition of Luther’s
works has presented, in its continuation volumes, the writings of Luther
in a new form, resulting in considerable improvement over the
previously edited texts. The Weimar undertaking has redounded in
an eminent degree to the advantage of the \textit{Table Talks}, which constitute an important source for the biography of Luther.
The progress made thus far, was utilized by me, as far as practicable, in the special supplements to the third edition of my above mentioned threevolume work.

In presenting to the public this compendious biography of Martin
Luther, which represents the fruit of several years’ labor, I may be
permitted to call attention to the fact that, in compiling it I have
been encouraged by the appreciative reception which has been accorded to my larger work.
In their discussion of this work, many
Protestants who are not confirmed controversialists, have marked
with approval my serious endeavor to observe objectivity and to give
a calm, authentic presentation of a delicate subject.\footnote{
In a volume entitled \textit{Urteile von protestantischer Seite}, the firm of Herder collected and
published a large number of these criticisms in 1912. The Swedish-Lutheran Archbishop
Söderblom, who delivered a series of lectures on Luther in May, 1923, said of the modern
Catholic investigation of Luther in general “that it has extended the horizon of the Reformation and called forth a radical revolution in the conception of Luther.” (\textit{Bayerischer Kurier},
Munich, May 9, 1923). His new conception of Luther, however, is certainly not at all
acceptable.}

Their statements show that an approximation between Catholic and Protestant
views of Luther is possible on the basis of unprejudiced historical research. The contentious and irrelevant attacks made upon my work
are of no consequence.\footnote{
Criticisms deserving of consideration were answered by me in part at the end of the
third volume of my Luther, and also in a more extensive exposition in my article \textit{Bemerkungen zur profestantischen Kritik meines Lutherwerkes} (\textit{Theologische Revue}, 1919, pp. 1 sqq.)}

Similar attacks will not disquiet me even
after the appearance of the present biography. On the other hand,
I shall gratefully avail myself of any clarification of points offered in
a possible new edition of this volume.

It may not be superfluous to mention that the present volume is
chiefly concerned with a lucid presentation of the development of
Luther, of his mental constitution and the interior impulses which
moved him throughout his life. His many frank communications concerning himself as well as his unbridled language about and against
others, almost spontaneously lead to a true characterization of him.
Luther is so communicative that the aforementioned diversity of
opinions about him becomes quite inexplicable to the attentive investigator, the more he occupies himself with his writings, letters, and
addresses. Owing to Luther’s communicativeness, I did not wish to be
sparing of the evidences which in part are inserted in the footnotes
of this volume. It was frequently possible to indicate them very
briefly by citations of apposite pages in my larger work, “Luther.”

Throughout the present work, in the annotations as well as in the text,
I have limited my observations to the great struggle of the sixteenth century
without digressing to controversial questions of the
present age which suggest themselves almost irresistibly. The question,
propounded in this book--and verily it is a sufficiently comprehensive question--rather is:
What happened in that period of upheaval
and, above all, how is the responsible author of the struggle to be
judged in his interior and exterior life? I could impose this restriction
upon myself all the more easily, since Luther is regarded by present day
Protestants more and more as a purely historical phenomenon.
According to the acknowledgment of Protestant leaders, the present
age has in general passed beyond his influence, and the little group
which still professes his peculiar doctrines, is diminishing. It may be
left to others, be they Protestant or Catholic, to draw appropriate inferences from the texts and facts here presented, which are not the
product of theological prepossession, but of an independent study of
the facts. In another work, entitled, \textit{Der deutsche Luther im Weltkrieg und in der Gegenwart}, I have frankly discussed certain of these inferences.\footnote{Augsburg, 1924, published by Haas and Grabherr.}

In the \textit{Lutherstudien}, a selection of more exhaustive contributions on the subject of Luther appearing in single numbers, I have attempted to present in detail the new movements within the Protestant pale, especially as exemplified by the recent jubilee celebrations of
the Reformation.\footnote{Lutherstudien, 6 Hefte, Freiburg, 1921 sqq.}

May the present volume be perused by unprejudiced readers.

\hfill \textsc{The Author}
%\hfill \textsc{Hartmann Grisar, S.J.}

\begin{flushleft}
Innsbruck, December 3, 1925
\end{flushleft}
