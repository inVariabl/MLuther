\section{Brighter Phases}

For the rest there are many bright spots in the ecclesiastical conditions of pre-Lutheran Germany. This is true especially of the life
of the common people, which went on in conformity with the old
spirit, nay, even became more truly religious despite all obstacles. It
is true also of the various religious orders, such as the Franciscans and
Dominicans, as well as of many portions of the secular clergy. A
modern Protestant historian writes: “We hear of grave defects \dots
And again we hear of so many monasteries imbued with seriousness
and character, of so many diligent efforts made for the improvement of
the parochial clergy, of such eager solicitude for the faithful, of such
fruitful fostering of studies within the Church, that we
hesitate to assume that vice and loathsomeness ruled absolutely. We
shall be compelled to establish the fact that gratifying and deplorable
things are to be found side by side; that there are some phenomena
which are depressing in the highest degree, but many others which are
elevating; and that the relationship which they bore to one another
was such as no one may venture to describe in numbers.”\footnote
{G. von Below, \textit{Die Ursachen der Reformation}, in the \textit{Histor. Bibliothek}, ed. by the
\textit{Histor. Zeitschrift}, München, 1917, pp. 19 sq.}

From the very beginning of his internal defection from the dogmatic teaching
of the Church, Luther had no appreciation of these
elevating and gratifying conditions. His preconceived delineation of
affairs does not constitute an objection to the brighter pages which
ought to be adduced.
As an illustration: in his commentary on the Epistle to the Romans,
where he expresses the thought that “the temporalities of the clergy ought
to be administered by the secular authorities,” he outdoes himself in unduly
generalized complaints against the clergy. Thus he says: What Paul demanded
of the servants of the Church, is “done by no one at the present
day. They are priests only in appearance \dots Where is there one who
does the will of the Founder? \dots The laity are beginning to penetrate
the mysteries of our iniquity (\textit{mysteria iniquitatis}) \dots Beyond proceeding
against such as violate their liberties, possessions, and rights, the ecclesiastical
authorites know naught else but pomp, ambition, unchastity, and contentiousness
.” They are one and all “whitened sepulchres.”\footnote{Grisar, \textit{Luther}, Vol. I, p. 284.}

Such exaggerated invectives are associated in his earliest literary productions
and letters with those fantastic descriptions which we were constrained
to adduce on previous pages.

Thus, in 1516, he wrote to Spalatin to dissuade the Elector of Saxony
from promoting Staupitz to a bishopric: He who becomes a bishop in these
days falls into the most evil company; all the wickedness of Greece, Rome,
and Sodom were to be found in the bishops. A pastor of souls was regarded
as quite exemplary if he merely pushed his worldly business, and pre-
pared for himself an insatiable hell with his riches.\footnote{\textit{Ibid.}, I, p. 57.}

Everywhere he perceived only dark gloom, because he discovered that
the Gospel as he understood it was everywhere forgotten; for, where the
“word of truth” does not reign, there can be only “dark iniquity.” “The
whole world,” he exclaims as early as 1515, or in the period immediately
succeeding, “the whole world is full of, yea, deluged with, the filth of false
doctrine.” Hence, it is not astonishing that there is prevalent in Christendom
“so much dissension, anger, covetousness, pride, disobedience, vice, and
intemperance, in consequence of which charity has grown completely cold,
faith has become extinguished, and hope has vanished.” etc.\footnote
{\textit{Sermo praescriptus praeposito in Litzka}, Weimar ed., I, pp. 10 sqq.; \textit{Opp. Lat. Var.}, I,
pp. 29 sqq.}

In view of these unrestrained and exaggerated effusions on the
decline of the Church, which pervade his whole life and are expanded into
a condemnation of all previous ages as the kingdom
of Antichrist, it is well to observe that they are inspired mainly by
his new dogmatic and pseudomystical views. They are anything but
historical and balanced judgments, and one can but marvel at the
thought that they have influenced the evaluation of the Middle
Ages for so long a time among Protestant scholars. To-day, however, well
informed Protestant writers are beginning to speak differently of Luther
’s unjustified and impassioned verdicts.

It is conceded that his discourses were based on “a one-sided and distorted
view” of things, and that he painted the history of the Middle Ages,
directed by the popes, as “a dark night.”\footnote
{Walter Koehler, 1907; Grisar, \textit{Luther}, Vol. IV, p. 116.}

With respect to medieval theology, we read that it is necessary to repudiate
resolutely “the caricature we meet with in the writings of the reformers
” and “the misunderstandings to which they gave rise.”\footnote
{Wm. Maurenbrecher, 1874; \textit{ibid}.}

A historian of the Reformation, writing in 1910, conceded that the
history of the close of the Middle Ages was “an almost unknown terrain
up to a few years ago;” “the later Middle Ages seemed to be useful only
to serve as a foil for the story of the reformers, whose dazzling colors,
when superimposed on a gray background, shone forth with greater brilliancy;”
only since Janssen has ““a more intensive study of the close of the
Middle Ages” been made, and it has been discovered that “the Church had
not yet lost its influence over souls.” “An increasing acquaintance with the
Bible toward the end of the Middle Ages must be admitted” and “preaching
in the vernacular was not neglected to the extent frequently assumed.”\footnote
{Walter Friedensburg, quoted by Grisar, \textit{Luther}, Vol. IV, p. 117.}

The first volume of Janssen’s History, despite the necessary modifications
made in later editions, clearly reveals that there was a
striking revival in many spheres of ecclesiastical life before Luther.
Popular religious literature flourished to a certain extent under the
fostering care of the new art of printing. It is impossible to assume
that such excellent and frequently reprinted works as \textit{Der Schatzbehälter
des Heils} (The Treasure Trove of Salvation), \textit{Das
Seelenwurzgärtlein} (The Little Aromatic Garden of the Soul), \textit{Der
Christenspiegel} (The Mirror for Christians), \textit{Der Seelenführer} (The
Spiritual Guide of the Soul), etc., should not have awakened a response
in the morals of the people and the general sentiment of the
age. Booklets on penance and confession, treatises on matrimony,
books on death, pictorial catechisms with instructive illustrations,
explanations of the faith and the current prayers, printed tables
with the commandments of God and a catalogue of domestic duties,
as well as many similar publications were widely disseminated among
the people. Excellent books of sermons were found in the hands of
the clergy. The classic work of Thomas 3 Kempis went through no
less than fifty-nine editions in several languages before the year
1500. The admirable pedagogical writings of Jacob Wimpfeling,
who was styled the “teacher of Germany,” were published in thirty
different editions within twenty-five years. Among the products of
the press the Bible ranked supreme. The first artistic work from the
press of Koberger (in Nuremberg) was the splendid German Bible
of 1483, which Michael Wohlgemut had oramented with more than
a hundred woodcuts. It was entitled “the most excellent work of
the entire Sacred Scripture \dots according to correct vernacular German,
with beautiful illustrations.”\footnote
{Janssen-Pastor, \textit{Gesch. des deutschen Volkes}, Vol. I, 20th ed., p. 23.}

The making of religious woodcuts and copper-plate engravings
flourished as perhaps never before. Sculptures and paintings vied with
one another in fervor, depth of thought, and beauty of execution.
Like all the artistic products of that day, they are permeated by
tenderness and sincere piety. This phase of art production is a favorable
mirror of the life of the people.

In the town-church of Wittenberg, sculpture has bequeathed to
us two splendid models in the richly ornamented baptismal font
of 1457, a creation of Herman Vischer, and in the artistically constructed
pulpit of the Luther Hall, in which Luther is said to have
preached frequently. The principal portal of the church is stiil
adorned with figures sculptured in the lovely style in vogue at the
close of the Middle Ages. In the center the enthroned Madonna with
her Child looks down upon and invites the worshipers. Sculptures
such as this attractive group of saints reveal as clearly as the popular
books just mentioned, how far the people were removed from regarding religion
as a source of horror and fear, as Luther will have
it. In life as well as in art, they, on the contrary, harmoniously
combined a loving trust in Jesus, the Divine Lord, and confidence
in the intercession of His servants, the Saints, grouped around Mary,
with the gravity of the idea of the eternal Judge, who appears
on the outside of the town-church of Wittenberg, a large statue set
in the wall. The majestic figure, with a sword protruding from the
mouth, in compliance with the Bible, inspires the beholder with a
sense of awe. It is not impossible that Luther’s morbid fear of God’s
judgment attached itself to such pictures, for the healthy piety of
the Middle Ages was wont to place them beside the monuments of
its confidence and childlike hope of salvation, as a counterpoise to
the spirit of levity.

Ecclesiastical architecture, finally, constituted a splendid field of
artistic endeavor. It was the center of all art. There is scarcely another
age in the history of architecture like the century from 1420
to 1520, in which town and country witnessed the erection of so
many houses of worship--most of them still in existence--constructed in
the devotional and joyful style of the late Gothic. The
confraternity of the German builders was the chief bearer of this
great movement, and it was one of the most popular institutions
of the time. The large sums which the faithful contributed towards
the erection of these often marvelous structures, attest the charity
and the idealism that actuated the soul of the nation.

In his journey to Rome, which took him through the heart of
Germany, Luther had ample opportunities of seeing and admiring
the artistic creations of architecture, sculpture, and painting, some
of which are still extant, whilst others have perished. But the monk
of Wittenberg had no taste for such things. There is not a sentence
of his writings or addresses which betrays any appreciation of the
mighty impetus of ecclesiastical life represented by the works of art
in churches and monasteries. In fact, neither his tongue nor his pen
reveals any genuine appreciation of art. He lived a secluded life in
his own narrow world, which fact explains the rigor of many of
his judgments.
