\section{Currents of the New Age}

Powerful movements which, proclaiming an intellectual revolution and
connected more or less intimately with the revival of the
study of classical antiquity, pervaded the Western world since the
fifteenth century, and presaged a new period in the history of mankind.
This agitation was bound to react on young Luther.

The newly invented art of printing had at one stroke created a
world-wide community of intellectual accomplishments and literary
ideas, such as the Middle Ages had never dreamt of. By the exchange
of the most diverse and far-reaching discoveries the nations came
into closer proximity with one another. The spirit of secular enterprise
awakened as from a long sleep at the astounding discovery of
new countries overseas with unsuspected treasures.

As a result of the increased facility of intellectual intercourse
and of the development of scientific methods, criticism began to
function with an efficiency greater than ever before in all departments
of knowledge. Yielding to an ancient urge, the larger commonwealths
made themselves increasingly independent of their
former tutelage by the Church. They strove after liberty and the
removal of that clerical influence whence they had largely derived
their durability and internal prosperity in the past. And in proportion
as they struggled for autonomy, the opulent cities, the knightly
demesnes and principalities, particularly in Germany, tried to throw
off the fetters which hitherto had oppressed them, and to increase
their power. In brief, we find everywhere a violent break with former
restrictions, a determined advance of subjectivism at the expense of
solidarity and the traditional order of the Middle Ages, but especially
at the expense of the supremacy of the spiritual power of the Church,
which thus far alone had preserved mankind from the dangers of
individualism.

Influenced by the spirit of the Renaissance and the awakening of
historical memories, the spirit of nationalism became more powerful
than ever before in the life of peoples. The segregation of national
ambitions became ever more pronounced, in spite of the increased
solidarity of commerce. The Germans became more keenly conscious of
being a unit which had a right to develop along its own
lines in opposition to the Latin nations. Luther very skillfully utilized
this national spirit in his public conflict. He boldly aroused this
spirit in the Germans, “his dear brethren,” who had been reduced
to servitude by the papacy, with a view to separate them from the
universal Church. If the patriotic sentiment of the Germans had
been kept within due bounds and had been animated by Christian
ideals, it would have been a great good. Aside from other considerations,
it might have led to a healthy competition with other civilized
nations. In reality, however, it descended down to the individual
principalities within the boundaries of Germany. The territorial
rulers who concurred with Luther, promoted it to the advantage of
their own power. In consequence, the empire increasingly became
a cumbersome machine, and the authority of the emperor, the august
head of the empire in virtue of his coronation by the pope, waned
visibly, especially since the imperial reforms so warmly favored by
Maximilian I (d. January 12, 1519) virtually failed and the immense
and far-flung empire consolidated under Maximilian’s successor,
Charles V, almost completely absorbed the attention of that ruler to
the detriment of his German subjects, not to speak of the impairment
which the authority of this Catholic emperor experienced
through Luther’s widely-published attacks, made partly in the direct
interest of his own ecclesiastical revolt and partly in the service of
those petty German territorial rulers who were loyal to him and
whose interests conflicted with those of the empire.

In course of time the German rulers obtained a certain ecclesiastical
régime within their respective countries, and it came to pass
that, at the close of the Middle Ages, besides the ecclesiastical princes,
the secular princes were vested, with extensive authority in the external
administration of religious affairs. They derived this authority
in part from the Roman See, which sought to protect and promote
the interests of the Church by the aid of loyal Catholic rulers; in
part they had acquired it as an inheritance from their forebears and
maintained it against the passive or active opposition of the bishops.

This ecclesiastical régime exercised by territorial princes was a
colossal danger when the religious struggles began in the sixteenth
century.

True, some of the princes, \textit{e.g.}, the well-intentioned Duke George
of Saxony and the dukes of Bavaria, employed their ecclesiastical
power successfully in defense of the existing ecclesiastical conditions.
Many others, however, especially Luther’s territorial lord, the Elector
of Saxony, constantly incited by him, and landgrave, Philip of Hesse,
made of the ecclesiastical privileges they had gained a bulwark for
the religious innovation. Thus the ecclesiastical authority of the
territorial lords formed a convenient transition to the establishment
of a Protestant ecclesiastical régime. Manifestly it was a double-edged
sword which was thus wielded by the secular arm in the
distribution of benefices, the temporal administration or partial disposition
of church property, the control of innumerable ecclesiastical patronages
and the superintendence of monasteries and ecclesiastical institutions
. It happened that large territories were torn with
ease from the faith and jurisdiction of the Church, as it were overnight.
Even in principalities that remained Catholic, the reforms
initiated by the Church authorities, \textit{e.g.}, in the monasteries, were
in many instances obstructed or interrupted by selfish rulers. And
the acts of the reigning princes were repeated in the great free cities
of the empire, and even in smaller cities, where the secular authorities
had come into possession of similar powers.

It is remarkable how this tendency of transferring ecclesiastical
functions and rights to secular rulers is noticeable in Luther’s Commentary
on the Epistle to the Romans, written at the time when he
began to drift away from the Church. The young monk there asserts
that the clergy are remiss in the performance of their duties concerning
the administration of pious foundations. “As a matter of fact,” he
exclaims, “it were better and assuredly safer, if the temporal affairs of
the clergy were placed under the control of the worldly authorities.”
The laity, he explains, are aware of the inefficiency of the clergy, and
“the secular authorities fulfill their obligations better than our ecclesiastical
rulers.” It is a question whether he perceived the far-reaching
import of his words as a kind of prelude to the coming
secularization.\footnote{Grisar, \textit{Luther}, Vol. I, pp. 283 sq.}

In any event, Luther was aware of the opposition existing between the
secular powers, and even between the common laity, and the clergy, which
smouldered in many places at that time. A certain
aversion and hostility toward the entire clergy, commencing with
the curia and the episcopate, and extending to the lower secular
clergy and the monks, had become widely prevalent and was fomented by
the secular authorities. In virtue of the pious donations
that had accumulated in the course of centuries, the Church had
become too wealthy. Thus, in the diocese of Worms, about three-fourths
of all property belonged to ecclesiastical proprietors. Everywhere the
Church possessed a plenitude of privileges which provoked
envy, as, for illustration, in the judicial forum, in her exemption
from taxation, and in the honors bestowed upon her. Jealousy and
envy engendered hatred and contentions in many places. True, an
immense share of the income of the Church constantly flowed to
charitable institutions; other sums were allotted with papal sanction to
, or else arbitrarily appropriated by, the secular authorities to
cover particular needs. Large sums were remitted to the Roman
curia in the form of ordinary or extraordinary taxes. The wealth of
the Church was alluring, and the large subsidies from Germany to
Rome especially were a constant occasion for complaint. The payments to
the papal treasury had, as a matter of fact, become too
onerous. The urgent requirements of the administration of the universal
Church, especially since the exile of the popes at Avignon,
had resulted in constantly increasing imposts levied on the faithful in
the various countries for the benefit of Rome. The annates,
the \textit{servitia} and other taxes, and the revenues derived from indulgences
had constantly increased. In Germany complaints were rife that
the material resources of the country were too heavily assessed. The
so-called “\textit{courtesans},’ \textit{i.e.}, benefice-hunters provided with Roman
documents entitling them to certain benefices, by their avaricious
practices helped to render the papal curia still more odious.

At the commencement of his controversy Luther assiduously collected
every unfavorable detail concerning the financial practices
of the curia, so as to paint a collective picture of them for propaganda
purposes. In this task he was assisted by a former official of the
Roman curia who had come to Wittenberg. True and exaggerated
reports of the pomp displayed at the court of Rome and of the papal
expenditures for secular purposes reacted upon the discontented
like oil poured into a fire.

A historical expression of this bitter feeling is furnished by the
so-called \textit{Gravamina}, official lists of complaints submitted to successive
diets by the princes and estates against the excessive burdens
and the inequality of rights. In many respects these complaints met
with the approval of men who were sincerely attached to the Church,
such as Dr. Eck. Similarly the cities had their \textit{Gravamina} against
the bishops, the citizens and town councilors against the chapters
and the other clergy. The spiritual principalities repeatedly experienced
a clash of arms as a result of the quarrels pertaining to jurisdiction or possession.

In this way it appeared--and the more recent researches concerning local
conditions confirm the impression--that one reason for
the great defection was antagonism to the papal government and
to the clergy, originating in material interests. The aversion to Rome
was all the more dangerous because it was shared by a large number
of the clergy, oppressed by taxes. These were clouds that heralded
an approaching storm. Nevertheless, the reform for which many
serious-minded churchmen clamored was not excluded, but merely
delayed. The existing discontent did not engender a desire for a
new religion, and the Catholic dogmas remained sacred. But when
Luther proclaimed his new doctrines, which implied the destruction
of ecclesiastical unity, the existing discontent accelerated the
revolution.
