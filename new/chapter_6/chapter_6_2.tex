\section{Abuses in the Life of the Clergy}

When Cardinal Nicholas of Cusa (d. 1462), a man who has
merited the gratitude of Germany, proclaimed his programme of
reforms, he indicated with complete frankness the reasons for the
corruption of the ecclesiastical system. They were: the admission
of many unworthy candidates to the clerical state, sacerdotal concubinage,
the accumulation of benefices, and simony. Towards the
end of the fifteenth century complaints had multiplied against immorality
among the clergy. “The numerous decrees of bishops and
synods do not admit of a doubt but that a large part of the German
clergy flagrantly transgressed the law of celibacy.”\footnote
{Janssen-Pastor, \textit{Gesch. des deutschen Volkes}, Vol. I, 18th ed., p. 709. On the synods,
see Hefele-Hergenrother, \textit{Konziliengesch}., Vol. VIII. Cf. Janssen-Pastor, \textit{op. cit.}, 680 sq., and
H. Grisar, \textit{Ein Bild aus dem deutschen Synodalleben im Jahrbundert vor der Glaubensspaltung},
in the \textit{Histor. Jahrbuch}, Vol. I (1880), pp. 603--640.}
A recommendation made to the dukes of Bavaria in 1447, voicing the opinion
of many friends and sponsors of a sound reform, declared that the
work of reformation had to begin with the improvement of the
morals of the clergy, for here was the root of all evils in the Church.
True there were districts where the clergy was irreproachable and
praiseworthy, \textit{e.g.}, the Rhineland, Slesvig-Holstein, and the Allgiu.
But in Saxony, the home of Luther, and in Franconia and Bavaria,
there were reports of many and grave abuses. A work entitled \textit{De
Ruina Ecclesiae}, formerly ascribed to Nicholas of Clémanges, says
that at the beginning of the fifteenth century there were bishops
who, for a money consideration, permitted their priests to live in
concubinage, and Hefele in his \textit{Konziliengeschichte} adduces a number
of synodal decrees which prohibited bishops to accept money or
gifts in consideration of their tolerating or ignoring the practice
of concubinage.\footnote
{Nicolaus de Clemangiis, \textit{De Ruina Ecclesiae}, c. 22, in Herm. von der Hardt, \textit{Magnum
Oecumenicum Constantiense Concilium} (Helmestadt., 1700), I, 3, col. 23 sq. Hefele,
\textit{op. cit.}, VII, pp. 385, 416, 422, 594; VIII, p. 97. John de Segovia, \textit{Hist. Syn. Basil.}, II
(Vindob. 1873), p. 774: “\textit{Quia in quibusdam regionibus nonnulli iurisdictionem ecclesiasticam habentes pecuniarios questus a concubinariis percipere non erubescunt, patiendo eos in
tali focditate sordescere.}”}

In addition to living in concubinage, many of the better situated
clergy were steeped in luxury and presumptuous arrogance, thus repelling
the people, and especially the middle class, which was conscious
of its own self-sufficiency.

In connection with the unduly multiplied small religious foundations without
clergy, the number of clergymen had increased to
such an extent that their very number suggests the idea that many
of them had no genuine vocation to the clerical state, and that lack
of work constituted a moral danger for many. Thus, at the end
of the fifteenth century, two churches in Breslau had 236 “altarists,”
whose only service consisted in saying Mass at altars erected by pious
donations and endowed with petty benefices. Besides saying Mass
daily, these “altarists,” of whom there was a vast multitude throughout
the country, had but one obligation, namely, to recite the Breviary. In
1480 there were 14 “altarists” and 60 vicars, besides 14
canons, stationed at the cathedral of Meissen. In Strasburg the
minster boasted 36 canonries, St. Thomas Church 20, old St. Peter’s
17, new St. Peter’s 15, All Saints 12. The number of canonries was
augmented by numerous foundations for vicars and “summissaries,”
so called because they celebrated high Mass in place of the canons.
There were no less than 63 “summissaries” at the minster of
Strasburg, not to mention 38 chaplains. John Agricola reports--although only
on the strength of an \textit{on dit} (“it is alleged”)--that
there were 5,000 priests and monks at Cologne; on another occasion
he estimates the number of monks and nuns in that city alone at
5,000. It is certain that the “German Rome” on the Rhine at that time
had 11 foundations, 19 parish churches, more than 100 chapels, 22
cloisters, 12 hospitals, and 76 religious convents.\footnote{Janssen-Pastor, \textit{l. c.}, pp. 705 sq.}

The bishop of Chiemsee traces the corruption of the clergy principally
to the fact that the spiritual and temporal rulers abused the
right of patronage, both by their appointments and their arbitrary
interference. This opinion is shared by Geiler of Kaysersberg, who
blames the laity, in particular the patrons among the nobility, for
the deplorable condition of the parishes and asserts that illiterate,
malicious, and depraved individuals were engaged in lieu of the good
and honorable.

In contrast with “the higher clergy, who reveled in wealth and
luxury,” the condition of the lower clergy in no wise corresponded to
the dignity of their state. “Beyond the tithes and stole-fees, which
were quite precarious, they had no stipends, so that poverty, and at
times avarice, constrained them to gain their livelihood in a manner
which exposed them to public contempt. “There can be no doubt that
a very large portion of the lower clergy had become unfaithful to
the ideal of their state, so much so that one is justified in speaking of
a clerical proletariat both in the higher sense, as well as in the ordinary
and literal sense.” This clerical proletariat was prepared to join
any movement which promised to abet its lower impulses.”\footnote
{\textit{Ibid.}, pp. 703--704. J. E. Jörg, \textit{Deutschland in der Revolutionsperiode 1522--1526} (Freiburg,
1851), p. 191, employs the phrase “clerical proletariat.”}

The condition created by the all too frequent incorporation of
parishes with monasteries was deplorable. Where many parishes were
incorporated with one monastery, incompetent pastors were frequently
sent, there was no supervision, and the care of souls declined.

One of the chief causes of the decline of the higher clergy and the
episcopate was the interference of the secular authorities and worldly-minded
noblemen in church affairs.

Not only were spiritual prerogatives frequently usurped by the
princes and lesser authorities, but large numbers of cathedral benefices
and diocesan sees were arbitrarily conferred on noblemen and princely
scions, so that the most influential offices were occupied in many
places by individuals who were unworthy and without a proper vocation.
“When the storm broke loose at the end of the second decade of
the sixteenth century, the following archdioceses and dioceses were
administered by sons of princes: Bremen, Freising, Halberstadt, Hildesheim,
Magdeburg, Mayence, Merseburg, Metz, Minden, Minster,
Naumburg, Osnabrück, Paderborn, Passau, Ratisbon, Spires, Verden,
and Verdun.”\footnote{Janssen-Pastor, \textit{l.c.}, p. 703.}
As a rule, the bishops who came from princely houses
were dependent upon their relatives and were drawn into secular and
courtly activities, even if their education had not radically repressed
their ecclesiastical sense, as, for instance, in the case of the powerful
archbishop of Mayence, Albrecht of Brandenburg.

An additional evil was the concentration of prominent episcopal
sees. “The archbishop of Bremen was also bishop of Verden, the
bishop of Osnabrück was also bishop of Paderborn, the archbishop of Mayence
was also archbishop of Magdeburg and bishop of
Halberstadt. George, palsgrave of the Rhine and duke of Bavaria,
was provost of the cathedral of Mayence when but thirteen and
successively became vicar capitular of Cologne and Treves, provost
of the foundation of St. Donatian at Bruges, incumbent of the parishes
of Hochheim and Lorch on the Rhine, and, lastly, in 1513,
bishop of Spires. By special privilege of Pope Leo X, conferred under
date of June 22, 1513, he, a sincere and pious man, was given possession
of all these benefices in addition to the bishopric of Spires.”\footnote{Ibid.}
“The higher clergy,” laments a contemporary in view of the worldly
bishops,” are chiefly to blame for the wretched condition of the parishes
. They appoint unfit persons to administer parishes, whilst they
themselves collect the tithes. Many endeavor to concentrate as many
benefices as possible in their own hands, without satisfying the obligations
attached to them, and, dissipate the ecclesiastical revenues in
luxurious expenditures lavished on servants, pages, dogs, and horses.
One endeavors to outdo the other in ostentation and luxury.”\footnote{\textit{Ibid.}, p. 700.}
The decline and indolence of the episcopate furnishes one of the most
important explanations of the rapid defection from the ancient Church
after Luther had set the ball a-rolling.

The religious tragedy of the sixteenth century is a perfectly insoluble
riddle except on the assumption that there was great corruption within
the Church. It is, however, a mistake to think that the
abuses were engendered by the nature of the Church, and that,
therefore, her doctrines and her hierarchy had necessarily to be
abandoned. Her exterior life, it is true, was greatly disfigured; yet
there was vitality in her soul and her salutary powers were unbroken.
Placed in the midst of mankind and exposed to the frailties of the
world in her human element, the Church, as the preceding centuries
of her existence show, is subject to periods of decline in her exterior
manifestation, without, however, being deprived of the hope of seeing
her interior light shine forth anew and her deformity vanish in God’s
appointed time. She celebrated such a renaissance after the decline of
the spiritual life in the eleventh century, in consequence of the warfare
which the great pope Gregory VII and his successors waged upon the
tyranny of the secular rulers and the numerous infractions of clerical
celibacy. She experienced a similar rejuvenation in the sixteenth
century, after the anti-ecclesiastical elements had drained off into the
new ecclesiastical system, to which they had been attracted by the
offer of emancipation from the commandments of the Church.
