\section{Preponderance of Dangerous Elements}

Having depicted the favorable aspects of the age, it is necessary
once more to revert to its shadows. They must have constituted a
source of grave danger to the Church, to judge of certain writings
of unbiased and noble-minded contemporaries, such as the indictment of
the bishop of Chiemsee, Berthold Pirstinger, published in
1519 under the title, \textit{Onus Ecclesiae}, a phrase borrowed from the
Apocalypse. True, in spite of the “burdens” imposed upon the Church
he hoped for an internal restoration of the same, based on the unchangeable
foundation of the faith. He mournfully addresses Christ
as follows: “Grant that the Church may be reformed, which has
been redeemed by Thy blood, and is now, through our fault, near
destruction!” After a dismal description of existing conditions, he
says that the “episcopate is now given up to worldly possessions,
sordid cares, tempestuous feuds, worldly sovereignty.” He complains
that “the prescribed provincial and diocesan synods ‘are not held”;
that the shepherds of the Church do not remain at their posts, although
they exact heavy tributes; and that the conduct of the clergy
and the laity had become demoralized, and so on.\footnote
{Grisar, \textit{Luther}, Vol. I, p. 48; Vol. II, pp. 45--47.}

Trithemius, Wimpfeling, Brant, Geiler von Kaysersberg, and Dr.
John Eck, joined in the lament of the Bishop of Chiemsee.

“Of the Lamentation of the Church” (\textit{De Planctu Ecclesiae})
was the title of a work which had enjoyed quite a wide circulation
before Luther’s day. It was reprinted at Ulm in 1474, and again at
Lyons in 1517. It was originally composed against the faults of
the papacy during the Avignon period, by the Franciscan Alvarez
Pelayo, a man of strict morality and whole-heartedly devoted to the
Church. The new reprints of this work addressed the contemporaries
of Luther in the severe language of Pelayo on the persecution of
the Church by those who were instituted as her protectors.\footnote{\textit{Ibid.}, p. 55.}
In another censorious composition, De Squaloribus Curiae, many justifiable
complaints were registered side by side with unfounded reproaches. The
work entitled \textit{De Ruina Ecclesiae} gained new importance
at the end of the fifteenth century.

Luther’s mental depression and distorted notions cannot be traced
to this kind of literature. Yet it is known from reliable sources that
he read other books of a similar tenor, such as the elegant works
of the pious Italian poet Mantuanus, who indignantly describes the moral
corruption of his country, extending to the highest ecclesia
cal dignitaries at Rome. In the writings of the new humanists he
found an echo, alloyed with bitter contempt, of what he himself
had heard at Rome, fortified by all the complaints of the age against
the clergy and the monks. We know that he did not approve their
attitude, in so far as it antagonized religion, nor was he himself a
humanist; but humanism with its critical activity, as developed everywhere,
especially in Germany under the influence of Erasmus, proved
to be a great help to him in his revolt against the Church authorities.

The papacy gradually came to regret the favor which it had extended to
, and the hopes it had placed in, the nascent humanism
in Rome and Italy. Among the German Humanists, Conrad Mutianus
of Gotha (d. 1526) was the chief promoter of the anti-ecclesiastical
movement. He was a man who had gone so far as to abandon Christianity
for a time. From this group originated the “Epistolae Obscurorum Virorum
,” a clever and biting satyre on monks, scholastics, and
friends of the papal curia. Crotus Rubeanus, its principal author,
had gathered a circle of younger Humanists about him at an earlier
stage of his career, among them Eobanus Hessus, Peter and Henry
Eberbach, John Lang, Spalatin for a time, and other talented men
desirous of innovation. Erfurt became the headquarters of this
group, which was very clamorous in prose and verse.

Justus Jonas also lived at Erfurt. He was a Humanist who later
associated himself for life with Luther. While yet a student of law
in 1506, he became affiliated with the Humanistic circles of that
city. He called Erasmus his “father in Christ” and, in company
with Caspar Schalbe, made a pilgrimage to him in the Netherlands.
In the same year Jonas, who was a priest and canon of St. Severus,
became rector of the University of Erfurt, an event which greatly
fortified the position of the neo-Humanists. The Leipsic disputation and
the letters of Luther aroused his enthusiasm. When Luther
journeyed to the diet of Worms, in 1521, Jonas set out to meet him
at Weimar, accompanied him to Worms, and subsequently was
called to Wittenberg as provost of the castle-church and professor
of canon law. Here, as early as 1521, having obtained his doctorate,
Justus Jonas taught theology in concordance with the ideas of Luther.
By his intimate attachment to Luther he gained the praise and friendship
of such a questionable man as Ulrich von Hutten.

Ulrich von Hutten, humanist and knight, took an active part in
the literary feud of Reuchlin against the “Obscurantists.” In 1517
he circulated the treatise of Laurentius Valla, an Italian, against
the so-called Donation of Constantine, with a view of making a
breach in the system of the Roman hierarchy and in a malicious
libel ridiculed the conduct of the celebrated theologian Cajetan at
the diet of Augsburg. In 1519 he dedicated to his patron, Archbishop Albrecht
of Mayence, a work on a cure for syphilis which he
had taken with temporary success. He had contracted this disease in
consequence of his dissolute life. Wielding a pen skilled in polemics
this adventurous Humanist launched his attacks on Rome, thereby
becoming the pathfinder of religious schism. Towards his offers of
forcible support, Luther prudently assumed a reserved attitude, preferring
the protection of his prince to the mailed fist of the revolutionary.
Politically, too, Hutten was a revolutionist. Like his
friend Franz von Sickingen, he was inspired by the ideal of a powerful
and independent knighthood. He fought with Sickingen in
the army of the Swabian League when it undertook the expulsion
of Duke Ulrich of Württemberg. Afterwards he lived in the Ebernburg,
Sickingen’s castle in the Palatinate, the so-called “Inn of Justice.”
Here he devoted himself to the composition of popular
and witty writings directed against the clergy and the princes.

The highly revered prince of the Humanists was Erasmus of
Rotterdam, at one time an Augustinian canon of Emaus at Gouda,
a scholar of prodigious learning and an epoch-making critic, whose
ambition it was, not only to introduce a new Humanistic form of
speech into ecclesiastical science, but also to reconstruct theology
along Humanistic lines, thereby exposing its dogmas to the danger
of extinction. While he wished to remain loyal to the Church, his
caustic and frequently derisive criticism of things ecclesiastical, Scholasticism,
monasticism, and the hierarchy, so influenced the minds
of his idolizing followers, both learned and illiterate, as to render
the greatest assistance to the work of Luther. His opponents coined
the phrase that his writings contained the egg which Luther hatched.
At all events, his initial sympathy for Luther was one of the causes
that induced almost the entire powerful and wide-spread neo-Humanistic
party to join the reform movement of Wittenberg, until
finally, about 1524, when it had been clearly demonstrated that the
religious struggles were redounding to the disadvantage of the sciences
, a reaction set in and Erasmus began to write against Luther.
The great services which Erasmus rendered in behalf of the text
of Sacred Scripture and his excellent editions of the writings of the
Fathers, remained undisputed and were acknowledged even by his
adversaries. In 1516 he issued his first edition of the Greek New
Testament, accompanied by a translation into classical Latin, which
was followed by his Biblical “Paraphrases.” In 1521 he took up his
abode near the printing-presses of Basle, whence, in 1529, the disturbances
caused by the new religion compelled him to remove to
Freiburg in Breisgau. Everywhere in his solitary greatness he was
an oracle of the learned. But his character was disfigured by weakness
of conviction and pronounced self-conceit. He lacked the power
of leadership, such as that trying and dangerous epoch demanded,
especially since his unfavorable characteristics were also impressed
upon his Humanistic admirers.

Besides Humanism, there were in those critical decades certain
other factors which constrain us to speak of a preponderance of
imminent dangers.

The minds of men had not yet completely divested themselves of
the consequences of the conciliar theories begotten in the unhappy
period of the Council of Basle, and of the schisms that preceded
it, with its two anti-popes in addition to the one true pope. Here
and there the Hussite theories, which had taken deep root in Bohemia,
made themselves felt in Germany. A worldly spirit and an unbridled
desire for wealth, which the newly inaugurated international commerce
and the attractive trade-routes to distant countries aroused in
the upper classes of society, were evidenced by the growing evil
of usury, against which Luther took a stand in two sermons delivered in
1519 and 1520, though he lacked “an adequate comprehension of the existing
conditions.”\footnote{Thus Köstlin-Kawerau, \textit{Martin Luther}, Vol. I, p. 279.}
In the lower strata of
society, especially among the peasant class, the long-nurtured discontent
with oppressive conditions began here and there to issue
in unrest and revolt. Lingering politico-social ideas of Hussitism
cooperated in this respect with aspirations after a higher standard
of living, awakened by the influx of wealth. The unrest was increased
by opposition to the introduction into Germany, about 1520, of
the Christian-Roman system of jurisprudence, interspersed with Germanic
principles. A fanatical preacher of social revolution in favor
of the lower classes was the prophet Hans Böhm, a piper of Niklashausen
in the Tauber valley. The “Bundschuh”--\textit{i.e.}, the strapped
shoe commonly worn by peasants--was the symbol of revolts which
broke out in many places, first in 1486, then in 1491 and 1492, but
“especially in 1513 and 1514, and again in 1517. The social revolution
of 1525, with the new Gospel of the “liberty of the Christian
man” as its background, was thus gradually prepared.

Only a man of superhuman powers could have banished the
threatening dangers of the age in the name of religion and an amelioration
of the traditional social order. Who can tell what course
events might have taken if at that time saintly men had inspired
the people with a reviving spiritual vigor, repeating the example
of ancient leaders in reshaping their age? But the teacher of Wittenberg
, who took it upon himself to direct the course of events, was no
such leader, as the sequel showed.

Many hailed the generous young emperor, Charles V, as the leader
who would conduct men out of the religious and social crisis. This
young ruler of a world-wide monarchy was animated by the best
of intentions. In all sincerity he, as emperor-elect, rendered the customary
oath of fealty to the Church on the occasion of his coronation, before
he was anointed by the archbishop of Cologne and girded
with the “sword of Charles the Great.” The oath he took bound
him “to preserve and defend in every way the Holy Catholic Faith,
as handed down,” and “at all times to render due submission, respect,
and fealty to the Roman Pontiff, the Holy Father and Lord in Christ,
and to the Holy Roman Church.” During a life replete with wars
and disappointments, Charles honestly endeavored to perform in
his beloved Germany the duties which he had assumed; but the religious
schism overwhelmed him and finally paralyzed the vigor and
determination with which he had begun his career despite constant
diversions.

The papacy, too, after Leo X, manifested many hopeful traits of
energy and efforts at reform, especially in the brief pontificate of
Adrian VI; but there was missing that towering personality on the
pontifical throne which might have averted the catastrophe. It was
necessary that the idea of a true, distinct from a false, reform of the
Church should make its way gradually at Rome and in Germany,
until it triumphed at the great Council of Trent in Charles Borromeo
and Pope St. Pius V, after the Church had sustained immense losses.
But no matter what the popes of Luther’s day might have done in
the interest of reform, in Luther’s eyes their endeavors would have
been futile; for he was firmly convinced that they had become
rulers of the kingdom of Antichrist.
