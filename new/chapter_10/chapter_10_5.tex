\section{Divine Service. Multifarious Activities}

In 1523 Luther issued his treatise “On the Order of Divine Service
in the Congregation,” published in the interest of Leisnig and other
expected churches of true believers. It was a provisional collection
of counsels--not precepts, as he himself emphasizes--for the conduct of
divine service.\footnote{Weimar ed., Vol. XII, pp. 35 sqq.; Erl. ed., Vol. XXII, p. 159.}
The Word and the arousing of “faith” are,
in his mind, the principal thing in public worship. He would have
each congregation regulate these matters by its own authority after
the model of the worship practiced in “the Apostolic age.” There
were to be daily assemblies, if not of the entire population, then at
least of the clergymen and scholars, for the purpose of praying and
reading the Bible.

Regarding the Mass, which was to be celebrated on Sunday in
connection with the Last Supper, Luther, in 1523, issued a small
Latin treatise entitled “Formula of the Mass and Communion for
the Church in Wittenberg.”\footnote{Weimar ed., Vol. XIII, pp. 205 sqq.; \textit{Opp. Lat. Var.}, VII, pp. 1 sqq.}
According to this formula, the so-called
Mass is not yet to be celebrated in German. The sequence
of parts corresponds rather closely to the Catholic Latin Mass, Even
the alb and the chasuble were to be worn by the celebrant in the
Lutheran church of Wittenberg.

The Mass commenced with the Introit, Kyrie, Gloria, and an oration,
followed by the Epistle with a chanted Graduale or Alleluja, and the Gospel.
Relative to the traditional pericopes, Luther complained that they did not
sufficiently inculcate the saving faith. The sermon was soon given the place
of honor in the middle of the function, namely, after the Credo. In accordance
with the “Formula,” there was an abrupt transition to the Preface: for,
since the sacrificial character of the Mass had been denied, the Offertory and
the prayers that followed it were omitted. The ancient Canon was omitted,
so that the Preface was immediately succeeded by the celebrant’s chanting
of the Biblical words of the institution of the Holy Eucharist (1 Cor. 11:23-25).
This chant was supposed to signify the consecration of the bread
and wine for the Last Supper. Then followed the Sanctus and the Benedictus,
the latter united with the Elevation. “For the sake of those weak in the
faith,” the Elevation was retained in the church at Wittenberg. The Pater
Noster and the Pax Domini were succeeded by the communion of the celebrant,
followed by the communion, under both species, of those among the
faithful who had announced their intention of receiving, provided that such
were present, and that they had stood the test of knowledge and worthiness
which was demanded by Luther not long afterwards. The close of the Mass
consisted of a selection of Catholic prayers after the Communion, the
Benedicamus, and a benediction couched in a Biblical phrase.

The Mass, as thus described, was the first transition to the forms
of worship still customary in the Lutheran churches. In course of
time it was simplified and toned down even more, and, in view of
Luther’s constant emphasis on freedom, exhibited many local variations.
As early as 1523 he had given it a German setting, which superseded
the Latin language, through his work, “The German Mass and
Order of Divine Service.”\footnote{Weimar ed., Vol. XIX, pp. 72 sqq. Erl. ed., Vol. XXII, pp. 226 sqq.}

Already at an early stage complaints had been heard that divine
service had become barren and ordinary, in consequence of the preponderance
of the sermon, which often developed into a tedious
polemical tirade against papism. The Catholic forms of worship,
\textit{per se} richer and more varied than Luther’s borrowings from the
ancient liturgy, were animated by the idea of the Eucharistic sacrifice.
In the minds of Catholics the entire function was dominated
by the idea of a \textit{sacrifice} of infinite worth, offered by the Son of
God through the consecrated hands of the priest--a sacrifice of
which the most ancient ecclesiastical writers, such as Justin, Ignatius
of Antioch, Clement of Rome, Tertullian, Cyprian, and Irenaeus,
speak with great reverence, professing the faith of which the Protestant
theologian, Martin Chemnitz, admits that Christian antiquity
“constantly expresses it with such nouns as \textit{sacrificium, immolatio,
oblatio, hostia, victima}, and such verbs as \textit{offerre, sacrificare, immolare}.”\footnote
{\textit{Examen Concilii Tridentini}, Vol. II, p. 782.}
In the divine service devised by Luther, the heart which
had formerly pulsated through the divine cult was missing; the
whole thing had become a corpse, which even the popular religious
hymns soon introduced, impressive though they were, failed to inspire
with life.

Beginning with 1523, Luther devoted himself to the composition
of liturgical and other hymns.\footnote
{Edited by Lucke in Vol. LIII of the Weimar edition of \textit{Luthers Werke}.}
In this field he was very successful.
His compositions are models of popularity and unadorned, natural
force. They served to edify the people and became a mighty lever
in the spread of Lutheranism. The aggressive mood is strongly
marked in some of these hymns, as may be seen from the opening
words of: “Behalt uns, Herr, bei deinem Wort, Und steur des Papst’s
und Türken Mord, Die Jesum, Deinen lieben Sohn, Stürmen wollen
von Deinem Thron” (O keep us, Lord, true to Thy Word, stay the
murders of the Pope and the Turks, who would assail Thy beloved
Son Jesus Christ and cast Him from Thy throne). This song was
written for the children, by whom “it was to be sung against the
arch-enemies of Christ and His holy Church, the Pope and the
Turks.”\footnote
{Grisar, \textit{Lutherstudien}, Heft IV (\textit{Luthers Trutzlied}), p. 47.}
Luther’s first poetical and political song was “Ein neues
Lied wir heben an,” commemorating the execution of two Dutch
Lutherans at Brussels.

Luther supplied his followers with an ample collection of beautiful hymns,
mostly adapted from the ancient Church, and highly
esteemed many of the religious folk-songs of the German people,
which embodied precious reminiscences of his youth. He is not, as
used to be affirmed, the father of the German religious or ecclesiastical
folk-song, since German songs had resounded both within
and without the Church long before his day. Owing to his efforts,
however, they flourished among his followers and in their churches
became part of the divine service in lieu of the liturgical hymns
of the ancient Church.\footnote{Grisar, \textit{Luther}, Vol. V, pp. 546 sqq.}
The first hymn-book intended for the
use of the Lutheran churches was supplied by Johann Walther in
1524. It was entitled “Geistliches Gesangbüchlein” (Spiritual Hymnbook),
and contained in five parts German hymns with appropriate
melodies. Luther wrote the preface and the collection went out
under his name. It comprised twenty-four hymns composed by himself and
was subsequently augmented by twelve others from his pen.
The hymn of defiance: “Ein’ feste Burg” (A safe stronghold our God
is still), which is regarded as the best of his poetical productions,
was composed later, in 1527--28, during a strenuous period of interior
and exterior stress.\footnote{\textit{Ibid.}, Vol. V, pp. 342 sqq.}

Johann Walther supplied the melodies, which evidenced skill and
good taste. They conformed, in part, to the traditions of Catholicism.
It is not certain that Luther is the author of even a single melody,
although up to the present time Protestant writers persist in glorifying
him as a musical composer. A report concerning his alleged compositions
made by a visitor at Wittenberg, who claims to have been a
contemporary, is a late fabrication, both in this respect as well as in
respect of the charge that Luther was a constant visitor at the
tavern.\footnote
{Grisar, \textit{Ein unterschobener Bericht über Luther als Tonsetzer und Stammgast}, in the
\textit{Ehrengabe an Prinz Johann Georg von Sachsen}, ed. by F. Fessler, Freiburg, 1920,
pp. 693 sqq.}

Among the manifold writings which Luther’s industry produced in
the years that have been reviewed so far, there is one which is entitled,
“A Christian Admonition concerning Exterior Divine Worship,”\footnote
{\textit{Werke}, Weimar ed., Vol. XVIII, pp. 414 5993 Erlangen ed., Vol. LIII, p. 315.}
directed to his followers in Livonia. It illustrates the confusion in
divine service which necessarily resulted from the religious
changes and the liberty granted by Luther. The Lutheran congregations
in Livonia were engaged in serious quarrels. In vain the voice
of Wittenberg appealed to them: “Be united in regard to these exterior
characteristics.” Luther admonishes them against the introduction of a
coercive discipline, which, he fears, would only lead to worse
dissensions. “Who can resist the devil and his satellites?” “Where the
divine Word penetrates, Satan must scatter his seeds out of sheer
envy.”\footnote{Grisar, \textit{Luther}, Vol. V, pp. 151 sq.}

Naturally there was great discord and perplexity also in other
places, not only in liturgical, but likewise in far more important
matters.

At Strasburg there was a profound schism, as is testified, among
other things, by Luther’s “Epistle to the Christians of Strasburg
against the Spirit of Fanaticism,” printed in 1524.\footnote
{Weimar ed., Vol. XV, p. 391; Erl. ed., Vol. LIII, p. 270.}
Karlstadt, whose
relations with Luther were strained because of his arbitrary ways,
had been banished from Saxony by virtue of an electoral decree issued
at the instigation of Luther, and commenced to agitate in Strasburg
against images, mural paintings, vestments worn at Mass, and other
“pagan” practices which he discovered there. He also propagated his
denial of the real presence of Christ in the Eucharist and succeeded in
surrounding himself with an active coterie of adherents. Zwingli exerted
his influence upon the movement from Zurich. Capito and
Bucer, at that time teachers of theology at Strasburg, were in agreement
with Zwingli. Certain preachers at Strasburg wrote to Luther,
asking him what should be their attitude towards the existing quarrels.
Luther hastened to have his reply printed, since he feared that like
schisms in other places would be caused by the “fanatics.” In his mind
all were fanatics who enthusiastically opposed the alleged externality
of ceremonies, or who held independent views and were not devoted
to his teaching, especially such men as Karlstadt and the “prophets of
Zwickau,” who believed that they were inspired from on high. Above
all, he attributed a fanatical spirit to those teachers who did not advance
along the same line with him relative to the doctrine of the
Last Supper.

In his letter to the preachers of Strasburg he instructs the questioners to
permit themselves to be guided, not by Karlstadt’s inconstant prophetical
notions, but solely by Christ, our Redeemer and Sanctifier, who imparts the
correct precepts and for whom he (Luther) speaks.

In the matter of the Eucharist, Luther champions the literal
interpretation of the words of Christ, “This is my body,” and makes this remarkable
confession: “Had someone told me five years ago that there was nothing
but bread and wine in the Sacrament, he would have done me a great favor.
I have suffered strong temptations, and have done violence to myself and
writhed with pain, so that I would have been glad to be relieved, because
I clearly perceived that I could thereby have administered a great blow to
the papacy.” He adds that the old Adam in him is even now only too much
inclined to deny the real presence of Christ in the Eucharist; but the words,
“This is my body,” are too plain, and Karlstadt’s buffoonery had confirmed
him all the more in his adhesion to this simple, literal sense.

He was soon to become involved in a far greater controversy regarding
this question with Karlstadt, and later with Zwingli.

Ulrich Zwingli, influenced by the writings of Luther and his own
one-sidedly humanistic training, had devoted himself to “reformatory”
ideas while yet a pastor at Einsiedeln. After his election as pastor of
the grand minster of Zurich, towards the end of 1518, he
intensified his devotion to these ideas. He had eulogized Luther as a
beacon light of Christian theology, as the fearless hero of truth, as
the man of the future. He was, however, very jealous of his own intellectual
independence as against the preacher of Wittenberg. “I
have not learned the doctrine of Christ from Luther,” he stated in
1523, “but from the Word of God. If Luther preaches Christ, he
does just what I do.”\footnote
{Grisar, \textit{Luther}, Vol. III, pp. 379 sq. Farner in \textit{Zwingliona} (1918--19) abandons the
old notion of Zwingli’s self-sufficiency.--“The humanist parson, to whom the Bible is a
book on morals, arrives at a true understanding of the Gospel through the agency of
Luther in 1519, and echoes his views.” Bossert, in the \textit{Theol. Literaturzeitung}, Vol. XIX,
p. 206}
After 1523, the example of the Wittenberg
priests who embraced matrimony exerted its influence also upon the
numerous adherents of Zwingli. As early as 1522, Zwingli himself
had contracted a so-called “marriage of conscience” with Anna
Reinhard, and, in April, 1524, he publicly led her to the altar in the
minster church. He later admitted that he was at daggers drawn with
his vow of celibacy in the days when he was still a Catholic priest,\footnote
{See Walter Köhler, \textit{Zwinglis Geisteswelt}, Gotha, 1920, pp. 22 and 31, quoting Zwingli’s
own confession in a letter to Henry Utinger, December 3, 1518. (\textit{Zwingli’s Werke}, ed. by
Egli \textit{et al.}, Vol. VII, pp. 110 sq.).}
and proclaimed the theory that the devil had introduced sacerdotal
celibacy.

The grave controversy between Luther and Zwingli on the Eucharist, to
be discussed in detail in the sequel, was occasioned by
Zwingli’s letter to Matthew Alber, a preacher of Reutlingen, dated
November 16, 1524. In this letter he first developed his interpretation
of the verb “is” in the text of the institution of the Eucharist in the
sense of “signifies.” According to his commentary of March, 1525,
he does not even wish to raise the question of the corporal consumption
of the Body of Christ.\footnote{Cfr. Grisar, \textit{Luther}, Vol. III, p. 409, n. 3.}

In the meantime Luther published his principal attack upon Karlstadt’s
doctrine of the Eucharist as well as upon those who sympathized
with Karlstadt’s ideas, in his large treatise “Against the
Heavenly Prophets, on Images and the Sacrament.” It is a work
overflowing with indigation.\footnote{Weimar ed., Vol. XVIII, pp. 62 sqq.; Erl. ed., Vol. XXIX, pp. 134 sqq. Cf. Grisar,
\textit{Luther}, Vol. III, pp. 387, 390 sq.}

He shows that his former friend and colleague is guilty of harboring “a
rebellious and murderous spirit.” Karlstadt, he says, “openly declares that
I am of no account.” “If your spirit,” thus he addresses Karlstadt and his
friends, “had been the true spirit, he would have demonstrated his office
with signs and words; but he is a murderous secret devil,” whose diabolic
nature follows from the fact that they do not know how to teach the principal
subject-matter of theology, namely, how “to obtain a clear conscience
and a joyful heart at peace with God \dots They have never experienced it.”\footnote
{Grisar, \textit{Luther}, Vol. III, p. 398.}

The fanatics, to whom Thomas Münzer and Valentine Ickelsamer belonged,
in their repeated literary attacks upon Luther, had marshaled
many effective arguments against him. The attitude of the so-called “Baptizers”
was not as foolish in every respect as Luther represented it. Recent
Protestant writers admire their modern rationalistic ideas. Luther was so
sensitive to their many effective arguments against his arbitrary conduct
that he expressed himself as follows in his book on the “Heavenly Prophets’:
“As if we did not know that reason is the devil’s handmaid and does nothing
but blaspheme and dishonor all that God says or does.” When confronted
with logical arguments he claims they are “mere devil’s roguery.”\footnote{\textit{Ibid.}, pp. 395 sq.}

Accordingly, he appeals to the Bible. This he does very effectively
in the principal part of the book, where he demonstrates the Real
Presence of Christ in the Eucharist against Karlstadt. He exhibits so
much ingenuity and erudition in proof of his literal understanding
of the verb “is” that even Catholic theologians may learn therefrom.
However, he stops halfway, excludes transubstantiation, and holds
that Christ is present simultaneously with the bread. If the anti-papistical
Luther shines forth in this exhibition, he bids defiance still
oftener to the fanatics, as, for instance, in the treatment of the elevation
during Mass. Karlstadt denounced the adoration of the species
during the elevation. Luther writes: “Although I too had intended to
abolish the Elevation; yet I will not do so now, the better to defy and
oppose the fanatical spirit.”\footnote{\textit{Ibid.}, p. 394.}
In his “Clag etlicher Brüder” (Complaint
of Several Brethren) Ickelsamer reproached him, not without
justification, with producing many anti-Catholic dogmas merely out
of spite, as he himself confesses, because the papists “had pressed him
so hard,” and not because of logical necessity or calm reasoning.
The “Complaint of Several Brethren,” originating in Karlstadt’s
circle, and composed by Ickelsamer at Rothenburg on the Tauber,
aside from many exaggerations and distortions, cast a glaring light upon
Luther’s doctrine and character.\footnote{\textit{Ibid.}, Vol. III, pp. 170 sq., 302.}
This is true in an even greater
degree of Thomas Münzer’s “Apology against the Unspiritual, Luxurious
Lump of Flesh at Wittenberg,” which he composed in 1524.\footnote
{Grisar, \textit{Luther}, Vol. II, pp. 364 sq.; Vol. III, p. 302; Vol. II, pp. 130 sq.}
The head of the Wittenberg school was regarded by the genuine
Anabaptists, who claimed to be spiritual, as one who had gone astray
and had been ensnared by the world and sensuality. Even if they went
too far in their personal attacks, they were successful in proving
that there was no evidence for Luther’s divine mission, and that he had
no right to condemn whatever ran counter to his opinion. They refused
to credit him when, with characteristic self-assertiveness, he
assured them, in his address to the preachers of Strasburg, that he had
hitherto done right and well in the main. “Whosoever asserts the contrary,”
he thought, “cannot be a good spirit.”\footnote{\textit{Ibid.}, Vol. III, p. 397.}
They reproached
him for his arbitrary treatment of a most important matter, namely,
the interpretation of the Bible, maintaining that it was not the Bible
which governed him, but the nonsense which they designated as “Bible,
Bubble, Babble.”\footnote{\textit{Ibid.}, Vol. II, pp. 365, 370 sq.}
From him they claimed to have learned to exercise
freedom in searching the Scriptures, which they said they used with
discretion. He forbade them to do so, whereas, independently of him,
“the Gospel grants freedom of belief and the right of private judgment.”
“Now settle yourself comfortably in the papal chair,” Ickelsamer tells
him, “for after all you want to listen to your own singing.”\footnote{\textit{Ibid.}, p. 377.}

The eccentric character of Thomas Münzer impelled him to advance
farthest in his fanaticism for the Anabaptist system. Since Easter, 1523,
this one-time Catholic priest ruled in Allstedt near Eisleben as preacher
of the new religion, claiming to be guided by a higher spirit. It was his
object to exterminate the impious by the use of force and to establish
a communistic kingdom, composed of all the good people on earth and
modeled upon the supposed ideal of the Apostolic age. His readiness to
apply violent measures was manifested by his destruction, with the aid
of an excited mob, of a shrine at Malderbach near Eisleben. The spiritual
and social revolution, which it was feared he would start, was proclaimed
by him in a sermon published later.

It was during this state of affairs that Luther seized his mighty pen
and in the last days of July, 1524, published his “Letter to the Rulers
of Saxony on the Spirit of Revolt.”\footnote
{Weimar ed., Vol. XV, pp. 210 sqq.; Erl. ed., Vol. LIII, pp. 255 sqq.}
It was written against Münzer.

He demanded of the princes that they should “suppress disorder and
prevent revolution.” At the same time he tried to justify his own new
religion in the eyes of the princes. He admits that he is “deficient in the spirit
and hears no heavenly voices” like the fanatics, but asserts that his cause
comes from God, whereas the devil speaks through Münzer. The fanatics, he
says, attacked his conduct; they “take offense at our sickly life”; but all
depends on doctrine, even if conduct has its shortcomings. “Let them but
preach confidently and cheerfully,” they will nevertheless succumb. He places
his hopes for the true Gospel in the constancy of the Elector, and in that
of his brother and son. He refers them incidentally and alluringly to the
prospects of material gain, when, relative to former Catholic churches and
monasteries, he writes: ``Let the territorial lords dispose of them as they see
fit.''\footnote{Grisar, \textit{Luther}, Vol. II, pp. 365 sqq.}

In his ponderous reply to Luther’s letter to the princes, which he
entitled “An Apology” (Schutzrede), Münzer complained that Luther
“exploded with fury and hatred like a real tyrant.” He (Münzer)
preached from the Bible only, but not, “please God, his own conceits.”
Luther having boasted of his courageous appearance before the diet
of Worms, and of other feats, he terms him “Doctor Liar” and “Lying
Luther.” “That you appeared before the Empire at Worms at all was
thanks to the German nobles whom you had cajoled and honeyed,
for they fully expected that by your preaching you would obtain
for them Bohemian gifts of monasteries and foundations, which you
now promise to the princes.”\footnote{\textit{Ibid.}, Vol. II, pp. 367 sq.}

The noisy criticism of the fanatics, who kept a sharp watch on
their opponent, rendered Luther somewhat more cautious with regard to
blemishes in his own life. Whatever I do, he says, is subjected
to investigation; I am a spectacle to the world (\textit{spectaculum mundi};
1 Cor. 4:9). He learned to moderate his appeal to the spirit, to
the inner voice, and to personal experience, and to attach greater
value to the so-called external word, \textit{i.e.}, the teaching of Scripture
as he interpreted it. What gave him an advantage over the fanatics
was his practical common sense, which made him feel far superior to
them.\footnote{Cf, Karl Holl, \textit{Luther}, 2nd and 3rd ed., (1923), Pp. 450 sqq. on the views of Münzer.}
He did not betray such foibles as they either in his concept
of God and in his ideas of suffering and affliction, or in the
establishment of ecclesiastical communions, or more particularly in his
social ideas, even though his teaching in these matters was quite confused.

However, it must be emphasized that the fanatics took their departure
precisely from Luther’s so-called reform ideas. They went beyond him,
partly by logically developing these ideas--a thing which
Luther did not want--and partly as a result of arbitrary distortions
and additions. At all events, the fanatics are true children of Luther;
their dreams and revolutionary projects are fruits raised in his soil.
It was a catastrophal punishment for him that he was compelled to
fight practically all his life in order to dissociate the fanatics from
his work. His rage increased with his resistance and was intensified
by jealousy. They intend to invade my field of labor and fame, he
declared in substance; they wish to wrest the leadership from me and
to appropriate what I have been unable to achieve amid bitterness and
distress. “They exploit our victory,” he says, “and enjoy it, take wives
and abate papal laws--results which they themselves did not obtain
by fighting.”\footnote{Grisar, \textit{Luther}, Vol. II, pp. 367 sqq.}

Luther was especially interested in retaining infant baptism which
the fanatical Anabaptists strove to abolish. He obstinately insisted
on the absurdity that an infant received Baptism together with the
faith, even if reason were unable to comprehend this. In 1523, he
published his “Little Book on Baptism done into German,”\footnote
{Weimar ed., Vol. XII, pp. 42 sqq.; Erl. ed., Vol. XXII, pp. 157 sq.}
in which he retains the old rite according to which the infant is thrice
breathed upon and the priest pronounces the exorcism, places salt
in the infant’s mouth, touches his ear with spittle, and anoints him
on chest and back with sacred chrism. Thus far did his policy of
accommodation go at that time.

A movement against infant Baptism had set in in Switzerland
in 1523. It was promoted by the social undercurrent of the adherents
of the new religion and provided the Anabaptist sects in Germany
with a strong impetus. Conrad Grebel, with the aid of his followers,
denied the right of the clergy to levy tithes, demanded the execution
of the priests, tried to introduce communism and to establish “congregations
of the saints” wherever the laws of God were not observed.
The semblance of rigorism and pietism was common to the Swiss
and the German Anabaptists. In reality they favored a certain emancipation
of the flesh and demanded that nuns be permitted to marry.
