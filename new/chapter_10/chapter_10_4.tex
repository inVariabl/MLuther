\section{The Congregational Church}

At the beginning of his revolt Luther entertained certain pseudo-mystical
ideas of a freely operating system of congregations.\footnote
{For the following, see \textit{ibid.}, Vol. II, pp. 102 sqq.}
He soon found it to be impossible to form an organization of those
members of former parishes who had been aroused by the new
Evangel--an organization which would be purely spiritual and reject
coercive measures, whether secular or ecclesiastical. Thus he was
necessarily compelled to change his ideal from that of a congregational
church to that of a popular church ruled by a spiritual
authority under the protection of the State. Thence, by the intrinsic
necessity of his new system, he arrived at the compulsory national
Church and his original idea of an invisible Church became gradually obliterated.

Yet he always wished to have a separate church for those who
truly believed in his Gospel, as distinct from a church which was
open to all, even to those who thought in terms of paganism. Repeatedly
he voices his longing for such a genuinely Christian church
(\textit{ecclesiola}) with a superintendent, ecclesiastical discipline, and the
ban, subject, not to the State, but to himself. He admits, however,
that he has not a sufficient number of people to join him in this
project.\footnote{See \textit{ibid.}, Vol. V, pp. 133 sqq.}
Undoubtedly the execution of this plan would have resulted
in two ambitious and mutually antagonistic groups of
churches. In general, it is evident how instability and disintegration
were the sole fruits of Luther’s abandonment of the true concept
of the Church. Some modern Protestant writers hold that Luther
did not establish any church whatever. The separation of the State
from the German Protestant ecclesiastical structure, which has taken
place in our time, and the disappearance of the supreme episcopate
of the Protestant State ruler, has placed this statement in a new
light. There is need of reconstructing the church from the foundation up.

It is interesting to review the intended institution and ultimate
fate of the congregational church originally planned by Luther.

It was to be a free covenant of brethren without binding
laws. Those who had embraced the Gospel of Luther were to remain,
without compulsion, under supreme representatives of a corporate body of
their own selection, and to call themselves Christians (not Lutherans). They
were to have their own Last Supper and their own dogmas. The
free covenant was to be an
organization for service only, with “unity of spirit,”
as Luther himself
repeatedly says, and not unity “of place, persons,
things or bodies.”\footnote{\textit{Ibid.}, Vol. II, pp. 107 sq.}
He does not desire a sectarian body. “To create sects,”
he says, “is neither useful nor helpful.” In his opinion, it was, moreover, unnecessary
in order to unite into a Christian covenant prior to the proximate end of the world, the still
faithful members of the papal empire of Antichrist--the pious who were
“terrified in conscience.”

How is the representative of the brotherhood to be selected? Luther
writes: “Those whose hearts have been moved by God” should band together
and choose a “bishop,” \textit{i.e.}, “a minister or pastor.” Even though the congregation
number only six or ten persons, they will attract others, “who
have not yet received the Word.” “They must not act of their own accord,
but must allow themselves to be moved by God.” “It is quite certain that
Christ acts through them.”\footnote{\textit{Ibid.}, p. 111.}

The first coherent statement of these ideas is to be found in a Latin
treatise of 1523, which Luther addressed to the Utraquists or Calixtines of
Prague. He entertained vain hopes of winning over this party, which still
obeyed the Catholic hierarchy. His instructions to them were also intended
for Germany, and above all for Saxony. This explains why he had the Latin
treatise published also in German.\footnote{\textit{Ibid.}, p. 112.}

About this time he purposed to establish a model congregation of free
Christians constituted from among the masses, in Leisnig, a small town situated
in the Saxon electorate.\footnote{\textit{Ibid.}, Vol. V, pp. 136 sqq.}
He addressed a treatise to the adherents of
the new Evangel in that town, the title of which is characteristic of the
impracticability of his ideal. The tract was entitled: “Reasons and Scriptural
Motives Demonstrating that a Christian Assembly or Congregation has the
Right and the Power to pass on all Dogmas and to Summon, Install, and
Depose Teachers.”\footnote{Weimar ed., Vol. XI, pp. 408 sqq.; Erl. ed., Vol. XXII, 105, pp. 140 sqq.}
The document states that, according to the Scriptures,
the universal priesthood of all Christian believers empowers every member
of the congregation to exercise independent judgment in matters of faith.
Every member may come forward and correct the erring preacher. St. Paul
says: “If anything be revealed to another sitting, let the first hold his
peace” (1 Cor. 14:30, where he speaks of the charismata of the first is
Christians). A Christian congregation is one in which the pure Gospel is
preached. It is presupposed, however, that this is the new Evangel which
Luther has brought to light and with which all are in accord who
speak the truth, since this doctrine ``has been received from heaven.''
The papists, so Luther writes to the inhabitants of Leisnig, ``ought to
yield to us and to hear our Word.''

Events at Leisnig, as everywhere else,
failed to justify his sanguine expectations. There was doctrinal confusion
and administrative dissension. When, in 1323, in his solicitude for this
town, Luther issued an introduction to their new ``fiscal regulations,''
this beneficent measure came to naught.\footnote{Weimar ed. Vol. XII, p. 11; Erl. ed., Vol. XXII, p. 105.}
The town-council confiscated the properties and endowments of the church,
but refused to co-operate in the establishment of a common poor-box.
Luther’s appeal to the Elector for aid was futile. Nothing more is reported
concerning the development of the religious life of the congregation in the
town of Leisnig.

The fate of this ideal congregation was a great disappointment to
Luther. But, as a Protestant investigator writes, “The primitive
Lutheran ideal of a congregation forming itself in entire independence
was nowhere realized \dots Thus at an early date Lutheranism took
its place among the political factors, and its development was to a
certain extent dependent upon the tendencies and inclinations of the
authorities, particularly of the ruling sovereigns of that time.”\footnote
{Walter Friedensburg, Cf. Grisar, \textit{Luther}, Vol. II, p. 333.}
