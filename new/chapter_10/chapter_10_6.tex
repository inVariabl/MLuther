\section{Free Christianity and the Freedom of the Will}

When, in the period from 1520 to about 1525, Luther approached
his doctrine of so-called free Christianity without binding dogmas,\footnote
{\textit{Ibid.}, Vol. III, pp. 9 sqq.; V, 432 sqq.}
he nevertheless revealed glimpses of an intention of adhering to
the norms of the Christian religion. He speaks with vigor about
truths undeniably based upon tradition and guaranteed by the Bible.
In brief, he desired to be a true Christian. But the inexorable logic
of his subjectivistic system produced a different result. It compelled
him with the force of gravity, as it were, to abandon his positive
foundation, notwithstanding his refusal to admit it to himself, since
he does not follow out his conclusions to the end, but turns back
after having gone halfway. It is impossible to form any other
opinion of his expectorations, so often influenced by passion and
saturated with rhetoric. Not infrequently they contain appeals to
liberty which are irreconcilable with true Christianity. Evidently
this phenomenon is associated with the impetuosity which animated
him at the outset of his career. Under unfavorable circumstances
he was not capable of moderating his temperament. In the stress of
his labors he did not weigh his words. Inclined to doubt and criticism,
his great success on the stage of the world swept him on to further
doubt and criticism.

Only when the fanatical sects' increased in strength, did a reaction
set in, which caused him to favor a more positive attitude.
In his attacks upon them, his fear of a free Christianity gained the
upper hand, prudence asserted her claims, and there came a time
when Luther reversed his views and corrected his utterances. Instead
of making internal experience and spiritual relish the sole
criterion, he now emphasized (especially after 1525) the so-called
external Word of God and the authority of the Bible, in so far
as they seemed to him clear and indisputable. He placed the rule of
faith in the foreground and conceded greater importance to good
works. The sight of the general decadence and the dissolving effects
of his sermons on Christian freedom in opposition to the law, compelled
him more and more to desist from his attacks on the motive
of fear of punishment in the fulfillment of the commandments.
When, finally, the Peasants’ Revolt began to shake the foundations of
the social and moral order, his return to positive religion became more
pronounced, or, as has been said, he turned from the free Evangel
to the legal structure of a Protestantism under state control.

Nevertheless, his early utterances on freedom and intellectual
liberty are re-echoed in his later ones. The primitive fallacy of his
system could not be eradicated. It took only certain impacts to strike
his excitable mentality and cause him once more to espouse the cause
of unrestricted liberty.

In the period of his early activity, \textit{i.e.}, before 1525, while Luther
was engaged in his struggle with the papacy and, in part, with himself,
he expressed himself as follows concerning the rights of subjectivism:

“Neither the pope nor the bishop nor any other man has the right to
dictate even so much as a syllable to a Christian believer, except with the
latter’s consent.” Formerly, under the papacy, we had “no right to form
an opinion,” but now “all councils have been overthrown.” No one, he says,
may “command what men must believe.” “If I am to know what is false
doctrine, I have the right to judge.” Let others arrive at whatever decision they
please, “I also have the right to judge, whether I will accept it or not \dots You
yourself must be able to say, this is right, that is wrong \dots God
must tell you deep down in your heart: this is the Word of God.” Autonomy,
according to him, is to be maintained at all times,\footnote{\textit{Ibid.}, Vol. IV, pp. 482 sqq.}
even towards the
sermons which every member of the congregation may criticize, reject or
accept.\footnote{\textit{Supra}, No. 4 of this chapter.}
In fact, private judgment is to be exercised even in relation to
himself (so he says incidentally); for “no one is bound to believe me, let
each one look to himself. To warn all I am able, but stop any man I can
not.”\footnote{Grisar, \textit{Luther}, Vol. III, p. 392.}
“If we all are priests, how then shall we not have the right to
discriminate and judge what is right or wrong in faith?” “A lowly man
may have the right opinion; why, then, should it not be followed?” The
Bible may be interpreted by everyone, even by a “humble miller’s maid, nay,
by a child of nine if it has the faith.”\footnote{\textit{Ibid.}, Vol. II, p. 31; Vol. IV, p. 389.}

He intoned hymns of liberty relative to the commandments, particularly
in his work on “The Freedom of a Christian,” and thus brought about dire
results because of the confusion they created. Once you have comprehended
the Word by faith, he expressly says, “all commandments are fulfilled and
you shall be free in all things.” “No one can be forced.”\footnote
{\textit{Ibid.}, Vol. III, p. 11; Vol. IV, pp. 487 sq.}

Is the reception of the Sacraments also a free matter? In 1521 Luther
declared: “Every individual ought to be free with regard to the reception
of the Sacraments. If anyone does not wish to be baptized, let him please
himself about it. If anyone does not wish to receive holy communion, that
is his precious right. Also, if anyone does not wish to confess, that is his
right before God.”\footnote{\textit{Ibid.}, Vol. III, pp. 9 sqq.}
He had already disposed of confession as a Sacrament.
With respect to Baptism and Communion, however, he subsequently defends
the necessity of receiving them. In 1521 he writes in another work: “I
approve of faith and Baptism, but no one should be compelled; everyone
ought to be exhorted and to exercise liberty in these matters.” Perhaps
his confused declarations relative to Baptism and Communion are intended
to exclude only physical coercion, whilst at the same time he completely
spurns confession as such. Yet, according to his whole system (as many
modern Protestant theologians admit), Baptism would not be necessary ben
dispensed cause “salvation is possible without Baptism”; since ``the salvation dispensed
in the Sacrament is none other than that obtained through the instrumentality
of the Word of the sermon'' (Erich Haupt). The same is to be affirmed
of the Lord’s Supper. Consequently, according to Luther, Christ instituted
Sacraments the use of which depends upon the good pleasure of men.

Such expressions--which could be multiplied considerably--lead
us to the very brink of religious radicalism. In the heat of the combat
against the dogmatic teachings of the papacy, during which, it
is true, he did not always weigh his words, Luther proclaimed complete
anarchy. Modern liberal Protestantism loves this kind of so-called
liberty. Harnack styles it a ``rich spring-tide'' in the history
of Luther's development, though, unfortunately, it was not followed
by a ``full-blown summer \dots In those years,'' he says,
“Luther was elevated above himself and apparently overcame the
limitations of his own individuality.” In contrast with the demand
of numerous contemporary Protestant theologians, who plead for
a return to Luther as he appeared in the spring of life, when, as
they allege, he was truly liberal-minded, we may quote this remarkable
statement by Frederick Paulsen, the famous Protestant
philosopher and historian: “The principle \dots to allow no authority
on earth to prescribe the faith is anarchical. And on these lines
there can be no church with the right of examining candidates for
the ministry and holding visitations of the clergy, as Luther did.”
This author furthermore says that “Luther as pope,” which he
really wanted to be, glaringly contradicts that principle. “Whoever
stands in need of a pope, had better be advised to stick to the real
one at Rome.” Fundamentally, Paulsen asserts, “an antinomy lies at
the very root of the Protestant Church,” namely: “There can be
no earthly authority in matters of faith, and: There must be such
an authority.”\footnote{\textit{Ibid.}, Vol. IV, p. 485.}

The religion which Luther cultivated in spite of his urge for
liberty was the \textit{religion of the enslaved will}. He ascribed so much
influence to the omnipotence of God and to what he calls grace,
that man’s liberty to perform moral and meritorious acts was completely
shattered. Now free-will was constantly and rightly regarded
as the preliminary condition of divine worship. “God created thee
without thy aid,” says St. Augustine, “but He does not desire to
justify thee without thy co-operation.” Luther, however, treats
man like a block of wood in matters pertaining to salvation. As
he expresses it, man is ridden like a beast by God or by the devil.

Erasmus in 1523 decided to publish an attack on Luther’s denial
of the doctrine of free-will.\footnote{Cfr. Grisar, \textit{Luther}, Vol. II, pp. 223 sqq.}
As a humanist he was particularly
interested in defending the freedom of the will. On the other hand,
Luther’s obstinate negation of free-will was one of the most vulnerable
spots in his doctrinal armor, against which an attack could
be most easily launched with the prospect of winning wide-spread
applause. It had not been an easy matter to persuade Erasmus to
take this resolution. For he had long favored Luther at least to the
extent of warmly approving his campaign not only against the
existing abuses, but also against certain perfectly justifiable religious
usages and necessary ecclesiastical institutions, which he himself
was also wont to criticize. After Luther’s open rupture with the
Church (1520) the cautious Erasmus became more and more guarded
in his utterances relative to the religious innovation.\footnote
{\textit{Ibid.}, Vol. II, pp. 249 sq.}
Amid his
epistolary compliments to Luther one may read his assurance, that
he would never separate himself from the divinely instituted spiritual
authority of the Church; that he did not wish to bother about
the clash provoked by the religious controversy, but desired to pursue
his studies unperturbed; and that God had sent the strong medicine
which Luther administered in order to purge His Church, since
Christ had been well-nigh forgotten. It required remonstrations on
the part of men in high authority, even in Rome, of the King of
England and of the Emperor, to determine Erasmus to take up his
pen against Luther.

In the spring of 1524 Luther heard that his erstwhile patron was
engaged in composing a book against him. He correctly appraised
the influence which Erasmus would exert upon the numerous humanistic
parties which had formerly favored him, but had become
estranged from his cause as a result of his violent activities. The
voice of their highly revered leader was bound to turn the scales
against him. Hence, in April, he wrote a strange letter to Erasmus,
then at Basle. He said that he had nothing to fear from an attack, but,
after flattering his antagonist for his rare qualities and merits, begs
him: “Do not write against me, or increase the number and strength
of my opponents; particularly do not attack me through the press,
and I, for my part, shall also refrain from attacking you.” “With patience
and respect,” he continues, he had observed that Erasmus, alas,
did not possess grace from above to comprehend the new Evangel.\footnote{\textit{Ibid.}, p. 260.}

Erasmus’s treatise appeared at Basle in 1524; it was written in
excellent Latin and bore the title: “\textit{De Libero Arbitrio Diatribe}”
(Discourse on Free-Will). The author triumphantly refutes the
heresy of the enslaved will, and despite his great and often timid
reserve, his critical rejection of the Biblical supports of Luther’s
theory is quite as brilliant as his use of the sacred text in defense
of the Catholic doctrine.

According to Luther, he says, not only goodness, but also moral evil must
be referred to God, which, however, conflicts with God’s nature and is excluded
by His holiness. Luther maintains that God punishes sinners who
cannot be held accountable for their misdeeds. Hence, so far as this earthly
life is concerned, laws and penalties are superfluous, because there can be
no responsibility without freedom of choice.

“In defending free-will,” thus writes the Protestant theologian A. Taube,
“Erasmus fights for responsibility, duty, guilt, and repentance--ideas which
are essential to Christian piety.” “He vindicates the moral character of the
Christian religion \dots''\footnote{\textit{Ibid.}, p. 263.}

It was not to be expected that Erasmus, who was a stranger to
Scholasticism, would enter upon a technical discussion of his topic.
Nor was it exactly necessary, although many points might have been
made more telling, as, for instance, the refutation of Luther’s doctrine
of absolute predestination. The Catholics were highly elated at the
“Diatribe” of Erasmus. Duke George of Saxony thanked him in
a letter, but, in his frank and honest way, did not suppress the caustic
remark: “Had you come to your present decision three years ago,
and withstood Luther’s shameful heresies in writing, instead of merely
opposing him secretly, as though you were not willing to do him
much harm, the flames would not have extended so far.”\footnote
{\textit{Ibid.}, pp. 261 sq.}
The
“Diatribe” also met with the approval of close friends of Luther.
Wolfgang Capito had previously declared his opposition to Luther’s
theory of the enslaved will. Peter Mosellanus (Schade) of Leipsic
had spoken so strongly against Luther’s theses and his teaching on
predestination that warning reports were sent to Wittenberg. George
Agricola, the learned naturalist, who at first admired Luther, was
repelled by his denial of free-will.\footnote{\textit{Ibid.}, p. 242.}
Melanchthon, to whom, despite
his former approval, this denial became painful in the course of time,
thanked his friend Erasmus for the moderation which he had observed.
He became more and more convinced that Erasmus was right in
certain cardinal theological points, and himself became an opponent
of determinism.\footnote{\textit{Ibid.}, pp. 261 sq.}
