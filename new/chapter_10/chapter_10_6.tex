\section{The Pen as Luther’s Weapon of Continued Attack}

Matrimony almost constantly engaged the attention of Martin
Luther after he had divested it of the dignity of a Sacrament and
Placed it on a level with ordinary civil contracts. It was imperative
to devise new regulations in order to avoid a serious decline of
morality. At the same time, in the eyes of the scrupulous, the unaccustomed
liberty of Christian believers in this department of life
had to be justified.

In 1519, Luther published a sermon “On the Married State,”\footnote
{Weimar ed., Vol. II, pp. 166 sqq.; IX, pp. 213 sqq.; Erl, ed., Vol. XVI, 2nd ed., pp.
60 sqq.}
which he had to reedit in a revised edition, because of loud complaints
against unbecoming language. He had permitted himself
to go too far in his expressions, though to some extent the transcriber
may have been to blame. The new and revised edition bore the title
“Of Married Life” and was published in 1522.\footnote
{Weimar ed., Vol. X, p. 11, pp. 275sqq.; Erl. ed., Vol. XVI, 2nd ed., pp. 510 sqq.}
In these pages,
too, he treats of sexual intercourse with a strange freedom of language
and in his accustomed fashion.\footnote
{Cf., \textit{e.g.}, the coarse expressions quoted in Grisar, \textit{Luther}, Vol. III, pp. 250 sqq.}
He declares that well-nigh all the
obstacles or inhibitions to marriage devised by the popes were to
be “repudiated and condemned.” He shakes the foundation of the
indissolubility of the marriage bond by allowing separation and remarriage
in certain cases.

This latter work contains a passage which has gained notoriety. Luther
says that if the wife refuses to render the debitum without reason, the
husband may use threatening language to this effect: “If you are unwilling,
there is another who is; if the wife is unwilling, then let the maid come.”\footnote
{Grisar, \textit{Luther}, Vol. III, p. 253. On the occurrence of this expression prior to Luther,
cf. the investigations of G. Buchwald in \textit{Beiträge zur Sächsischen Kirchengeschichte}, 1915,
n. 29, Leipzig, 1916.}
In case the wife persists in her refusal, “the marriage is dissolved,” and the
husband may “take an Esther to himself and let Vashti go, as King Assuerus
did”; he should let his wife “go to the devil” and, with the consent of the
authorities, contract a new marriage.

Because of such utterances, particularly the passage, ``then let the maid
come,'' Duke George of Saxony addressed indignant complaints to his representative
in the diet, Dietrich von Werthern. The Catholic controversialists of
the time often advert to it. Offense was taken not only at the elimination
of the matrimonial tie, but naturally also at the implied permission of
extra-matrimonial intercourse with the maid. The phrase, ``if the wife
is unwilling, then let the maid come,'' may possibly have been a proverbial
phrase in usage among the uncouth peasant class, signifying intercourse
outside of wedlock. Since, however, Luther otherwise always inveighs strongly
and decisively against extra-conjugal intercourse, it may be that he did
not intend this application of the brutal phrase, but merely gave free
rein to his pen when he cited the expression.

Luther's work ``Of Married Life,''taken in connexion with his other writings
and pronouncements, clearly reveals how widely he opened the door to the
destruction of the conjugal tie. August Bebel justly observes: ``With regard
to marriage, he [Luther] develops astonishingly radical views.''\footnote
{Cfr. Grisar, \textit{Luther}, Vol. III, pp. 241 sqq., and \textit{infra}, Ch. XVIII, n. 2.}
To this category belongs Luther's work on ``The Seventh Chapter of St. Paul
to the Corinthians'' (1 Cor. 7), published in the year 1523. In some
passages this work is characterized by a license of language which approaches
indecency.\footnote{Weimar ed., Vol. XII, pp. 92 sqq.; Erl. ed., Vol. LI, pp. 1 sqq.}

In general, lack of restraint in speech is a prominent fault
of Luther's, arising from his endeavor to employ the language of the common
people, wherein he was assisted by his descent from the peasantry. It
is not true that he found low pleasure in sexual matters. His uncouth
expressions are hardly ever employed for the sake of exciting the sexual
passion, even though Modesty must often veil her countenance. He fairly
wallows in the mire of base bodily
functions, especially in his vigorous denunciation of his opponents.
He liked to hurl the most vulgar filth into the faces of his enemies,
after he had stirred it up with something akin to glee. He was excelled
by no one in this respect either among his contemporaries
or in the succeeding so-called age of frank utterance, yes, we may
add, not even in the earlier centuries of the Middle Ages, which
were not exactly noted for delicacy of speech. In this characteristic
quality, he stands forth as a giant, even in his earlier writings, which
are now under discussion. Later on, his dexterity increased in an
astounding degree. The apologetical references of his defenders to
the indelicate language of his age, therefore, are valid only in a
very restricted measure. His contemporaries, his nearest friends
are shocked at the filthy material which is at his disposal in his temperamental
outbursts against his adversaries.\footnote
{See my \textit{Luther}, references in the index, Vol. VI, v. ``Abusive language,'' ``Unseemliness
of Luther's language.'' }


King Henry VIII, of England, who was well versed in theology, violently
attacked Luther as a heretic in 1521, in his ``Defense of
the Seven Sacraments,'' for which the Pope conferred upon him the
title of ``Defender of the Catholic Faith.'' Luther's reply to Henry's
book (1523) was published in both Latin and German.\footnote
{Weimar ed., Vol. X, II, pp. 180 sqq., 227 sq. \textit{Opp. Lat. Var.}, XI, pp. 385 sqq., and
Erl. ed., Vol. XXVIII, pp. 343 sqq. }
It abounds
in vulgar attacks upon the King. In this work he says he wishes
once more ``to uncover the infamy of the Roman harlot'' and declares
``the harlot-countenance'' of the king is brazen because of his
defense of the ``purple harlot of Rome, the drunken mother of impurity.''


This fool, he says among other things, this excrement of swine and asses,
wishes to defile the crown of my heavenly King of glory with the filth of
his body; but the dung must be cast upon him, who is nothing but a lying
rascal and buffoon, a monstrosity of a fool.\footnote{Cfr. Grisar, \textit{Luther}, Vol. II, pp. 152 sqq. }

With disgust the highly educated English chancellor, Thomas More, took
cognizance of the filthy language in this and other Latin works of Luther.
In a ``Reply to Luther's Calumnies,'' which he published in 1523 under an
assumed name, this eminent humanist wrote in a style which fortunately
was not habitual with him: ``[Luther] has nothing in his mouth but stench,
filth and dung. These he scatters about him more abusively and obscenely
than ever did churl \dots If he continues to cultivate this vituperative manner
of speech and to talk of nothing else but cloaca, latrines and excrements,
then may others do as they please; we shall turn our backs upon his filth
and abandon him to his vile discharges.''\footnote
{The Latin text in Grisar, \textit{Luther}, Vol. III, p. 237, note 1. It concludes: “\textit{Capiemus
consilium \dots sic bacchantem cum suis merdis et stercorib cacantem cactumque
relinquere.}”}


In this same work against King Henry VIII, Luther says: “Through me
Christ has commenced His revelations concerning the abominations in the holy
place.” “I am certain that I have my dogmas from heaven,” “but the devil
tries to deceive me through Henry;” “God blinds the devil, that his mendacity
is made manifest through me.” The King, he says, proves the truth
of the saying that there are no greater fools than kings and princes.

One is compelled to ask, what demagogical effects such frenzied
language was likely to produce in an agitated world, when the
respect due to civil authority was trampled under foot even in such
works of Luther as his treatise “On Secular Authority.”

Among the other polemical writings which Luther produced during these
years mention should be made of his Latin work, “Against
the Armed Man Cochlaeus,” whom he ridicules as a combative
knight because of his report of their interview at Worms and his
defense of the Sacraments.\footnote{Weimar ed., Vol. XI, pp. 295 sq.; Opera Lat. Var, Vol. VII, pp. 44 sqq.}
Cochlaeus, himself an effective controversialist,
replied in a work that bore a no less poignant title,
“Against the Cowled Minotaur of Wittenberg,”\footnote
{Köstlin-Kawerau, \textit{M. Luther}, Vol. I, p. 644.}
in which he
seriously but unsuccessfully applies to Luther the legend of the abortive
calf of Freiberg, claiming that it condemned him.

In opposition to the canonization of Bishop Benno of Meissen,
Luther, in 1524, wrote his sermon “Against the New Idol and Olden
Devil about to be set up at Meissen.”\footnote
{Weimar ed., Vol. XV, pp. 183 sqq.; Erl. ed,, Vol. XXIV, 2nd ed., pp. 247 sqq. Cf.
Grisar, \textit{Luther}, Vol. V, pp. 123.}
By means of this outrageous
sermon he intended to counteract the favorable influence which the
Catholic cause was likely to derive from the imminent canonization
of the venerable Bishop Benno and his elevation as patron saint of
Saxony, a project which was promoted by Duke George.

Eager to increase his followers, Luther at that time also cast his
eyes upon the Jews. He imagined that the Jews were inclined to
favor him and could be attracted to his cause. What gain and glory if
he should convert the people of Israel to the true Gospel! These ideas
inspired his little treatise, “That Christ was Born a Jew.”\footnote
{Weimar ed., Vol. XI, pp. 314 sqq.; Erl. ed., Vol. XXIX, pp. 45 sqq. Cfr. Grisar,
\textit{Luther}, Vol. V, pp. 411 sqq.}


In this
work, published in 1523, he relies on the prophecies which foretell the
conversion of the Jews at the end of the world. “God grant that the
time is near at hand, as we hope.” His desire to win over the Jews
remained unrealized; his hatred of Judaism afterwards induced him
to launch the most unheard-of attacks upon them.

In the midst of his fretful and many-sided labors he yet found
leisure to write works on social questions. His small tract against
usury, published in 1519, and his “Great Sermon” of 1520 on the
same subject, were followed, in 1524, by the publication of a
pamphlet, “On Mercantile Trade and Usury.”\footnote
{Weimar ed., Vol. XV, pp. 293 sqq.; Erl. ed., Vol. XXII, pp. 199 sqq. Cfr. Grisar,
\textit{Luther}, Vol. V, pp. 79 sqq.}
His writings on
usury, to which was added another work in 1540, testify to the
active interest which he took in the moral aspects of the progress
of trade and commerce occasioned by the discoveries and the more
intimate intercourse between nations resulting from them. In his
work “Von Kaufshandlung,” he again prohibits the taking of interest.
“Whoever lends in such wise as to receive more in return,
is a public and condemned usurer.” For the rest, this treatise
contains many stimulating ideas and furnishes an insight into
the conditions of the time. But the author undeniably goes too far in his
demand that the existing commerical societies ought to be abolished.
He lacked the necessary breadth of view and practical experience
to pass judgment on such a question. The desire to represent the new
doctrine as useful for a general reform was not sufficient, and
Scriptural passages, especially from the Old Testament, could not
be generalized so as to apply to all times and conditions.

Luther’s modern admirers have highly praised one of his works,
written to improve the condition of the schools. It is entitled “To
the Aldermen of all the Cities of Germany, that They Should Establish
and Maintain Christian Schools,”\footnote
{Weimar ed., Vol. XV, pp. 27 sqq.; Erl. ed., Vol. XXII, pp. 168 sqq. Cfr. Grisar,
\textit{Luther}, Vol. VI, pp. 3 sqq.; also my article on “Luther” in the \textit{Pädagogisches Lexikon},
new edition by Roloff.}
the emphasis being on the
word “\textit{Christian.}” This appeal was occasioned by the observation
that his cause was injured by the deterioration of the schools resulting
from the religious controversy he had started. It was his intention,
as he himself expressly confessed, to obtain the necessary spiritual
and secular forces for the promotion of his Gospel by
re-establishing Christian schools, \textit{i.e.}, schools in which the new religion
was inculcated. Besides practicable suggestions regarding education,
the pages of this work are burdened with inconceivably crass accusations
against the educational policy of the ancient Church, on which
he wishes to inflict a mortal wound in the name of education by
means of the Scriptures.

It was impossible for Luther, in discussing such questions as the
nature of trade and education, to abandon the controversial narrowness
which marked his ecclesiastical position.

In his literary activity, his predilection was Holy Writ. He provided
the books of the New Testament which he translated into German
with prefaces that characterize his standpoint in regard to the Bible
and theology. The most significant thing in the latter regard is his
preface to the Epistle to the Romans. It is little less than an epitome
of his theological teaching, especially as it centers around the idea
that St. Paul condemns “the entire ulcerous and reptilian complex
of human laws and commandments which drowns the world at present”
and teaches the doctrine of justification by faith alone.
\footnote{See the conclusion of the preface to Romans, 1522. Cfr. Erl. ed., Vol. LXIII, pp. 7 sqq.
for a collection of Luther's prefaces}
His
general preface to the New Testament is equally noteworthy, as it
emphasized that those portions of the Scriptures are the best which
show “how faith in Christ gives life, justice, and happiness.” Hence
his preference for “the Gospel of St. John and his first Epistle, the
Epistles of St. Paul, particularly those addressed to the Romans, the
Galatians, and the Ephesians, and the first Epistle of St. Peter; these
are the books which reveal Christ \dots They advance far beyond
the three Gospels of Matthew, Mark, and Luke \dots Compared with
them, St. James' Epistle is an epistle of straw, since there is nothing
evangelical about it.”

In this manner his criticism of the Bible proceeds along entirely
subjective and arbitrary lines. The value of the sacred writings is
measured by the rule of his own doctrine. He treats the venerable
which are doubtful, and which are to be excluded. At the same time
he practically abandons the concept of inspiration, for he says nothing
of a special illuminative activity of God in connection with the
writers’ composition of the Sacred Book, notwithstanding that he
holds the Bible to be the Word of God because its authors were sent
by God. As is well known, during the age of orthodox Lutheranism
its devotees fell into the other extreme by teaching so-called verbal
inspiration, according to which every single word of the Bible has
been dictated by God. Catholic theology has always observed a
golden mean between these extremes.

Luther always adhered essentially to his opinion of the Epistle of St. James
as quoted above.\footnote{Cfr. Grisar, \textit{Luther}, Vol. V, pp. 521 sqq.}
Relative to the other Biblical writings, his most striking
assertions will be considered in the sequel. Even at the Leipsic disputation
he had maintained that the second book of Maccabees did not belong to the
canon simply because of the difficulty presented by the passage quoted by
Eck concerning Purgatory and prayers for the departed.\footnote{Ibid.}
Later he simply
excluded the so-called deutero-canonical books of the Old Testament from
the list of sacred writings. In his edition they are grouped together at the
end of the Old Testament under the title: “Apocrypha, \textit{i.e}, books not to be
regarded as equal to Holy Writ, but which are useful and good to read.”
Under this title the Lutheran Bible retains the following arrangement even
to the present day: Judith, Wisdom, Tobias, Ecclesiasticus, Baruch, Maccabees
I and II, parts of Esther and of Daniel. Luther’s New Testament is
somewhat more conservative. After the Gospels, the Acts of the Apostles,
and the Epistles, it contains the Epistles to the Hebrews, those of St. James
and St. Jude--three epistles which Luther carped at--and, lastly, the
Apocalypse of St. John.

It is a fact that must not be overlooked that the parts of the Bible
which Luther retained were taken over from the tradition of the
past. By way of exception and as a matter of necessity, he thus
conceded the claims of tradition. Though otherwise opposed to it,
he took it as his guide and safeguard in this respect without admitting
the fact. Thus his attitude towards the Bible is really burdened
with “flagrant contradictions,” to use an expression of
Harnack, especially since he “had broken through the external authority of the
written word” by his critical method.\footnote{Cfr. Grisar, \textit{Luther}, Vol. IV, pp. 403 sq.}
And of this, Luther is guilty,
the very man who elsewhere represents the Bible as the sole principle
of faith!

If, in addition to this, his arbitrary method of interpretation is
taken into consideration, the work of destruction wrought by him
appears even greater. The only weapon he possessed he wrested from
his own hand, as it were, both theoretically and in practice.

His procedure regarding the sacred writings is apt to make
thoughtful minds realize how great is the necessity of an infallible
Church as divinely appointed guardian and authentic interpreter
of the Bible. Deprived of the guidance of the Church, with subjectivism
as his lodestar, Luther trod the path that led to an independent
religion severed from divine revelation and therefore without
foundation.
