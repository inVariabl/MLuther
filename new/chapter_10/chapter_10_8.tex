\section{Companions in Arms at Wittenberg}

Next to Luther, the most attractive personality at the University
of Wittenberg was the young humanist and theologian \textit{Philip Melanchthon},
a small and emaciated man, endowed with a pair of
glowing eyes. The reputation for learning which this precocious
youth enjoyed, powerfully attracted students. He lectured on Homer
and St. Paul’s Epistle to Titus, on rhetoric and the Gospel of St.
Matthew. Even after his transfer, in 1519, from the chair of Greek
to the theological faculty, he continued to occupy himself chiefly
with humanism, especially in his literary productions. He never
became a doctor of theology. He combined an excellent gift for
teaching with a love for youth and great sociability. His lectures
he flavored with instructive anecdotes, but never became rhetorical
or violent, like Luther. He had difficulty in overcoming his habit
of stuttering. In addition to his academic lectures he conducted a
\textit{schola privata} for beginners, whom he, an ardent friend of youth,
prepared for the higher studies. In 1524 he published for their
benefit an “Enchiridion of the Elements for Boys.” It was a reader
composed of church prayers, passages from the Bible, and secular
material, such as moral axioms of the Seven Wise Men and excerpts from
Plautus. Despite his many labors, Melanchthon knew
how to preserve his delicate health by a life of regularity and extraordinary
abstemiousness. He rose early and devoted the first
hours of the morning to his extensive correspondence. His letters
were composed with great care and a facile and fluent Latinity
similar to that which distinguishes his printed works. In the use
of the German vernacular, however, he was far less skilled than
Luther.

Melanchthon’s domestic life presented attractive features. In the
fortress-like house which he occupied, and which still stands, he
constructed a quiet “sanctuary” reserved for intercourse with his
familiars and learned friends as well as for study. It was there he was
found by a French scholar who paid him a visit, rocking the baby’s
cradle by means of a ribbon and reading a book which he held in his
right hand. The wife of this lay-theologian was the daughter of a a
burgomaster, a quiet, sensible woman, like himself very charitable
towards the poor.

The nuns who left their convents and came to Wittenberg
were not welcome in Melanchthon’s home. Aside from the fact that
there were some among them who, like Catherine von Bora, flaunted
their noble descent too ostentatiously in the presence of Melanchthon’s
wife, who belonged to the middle class, Melanchthon and
his spouse regarded with aversion the worldly conduct of these
nuns and their fawning upon Luther. Melanchthon, in a confidential
letter to Camerarius, couched in Greek, declared that they had “ensnared”
Luther “with every kind of strategy.” The conduct of many
married clergymen and fugitive monks was likewise distasteful
to him.

Melanchthon became very sad when, on a journey to his native
town of Bretten in the Lower Palatinate (Baden), he was forced
to note the moral decadence which had set in as a consequence of
the religious controversy. According to a sufficiently authenticated
tradition, he frankly encouraged his mother (either at that time or
on the occasion of a visit he paid her in 1529) not to be disturbed
by the religious controversies.\footnote{\textit{Ibid.}, Vol. V, p. 270.}
He well remembered the admonition of
his father, the strictly religious George Schwarzerd, an
armorer by trade, who, nine days before his demise, had adjured
his family never to separate from the Catholic Church. The papal
nuncio Campeggio, on the occasion of his first visit, endeavored
through the instrumentality of his secretary, Nausea, to induce
Melanchthon to return to the mother Church, but the attempt proved
futile.

In 1521 Melanchthon championed Luther’s cause by publishing
a work against Thomas Rhadinus, an Italian antagonist of the
Reformation. In this work he went so far as to demand that the
authorities of the Empire should use force against the Pope. Soon
after he penned an attack upon the Paris Sorbonne. He merited
Luther’s highest esteem by his “Loci Communes Rerum Theologicarum.” “I
esteem Philip like myself,” Luther wrote in September,
1523, “aside from the fact that he puts me to shame, nay, excels me,
in learning and virtue.”\footnote
{Letter to Billikan, September 17, 1523; \textit{Briefwechsel}, IV, p. 230.}

Despite his disapproval of the extremes to which his friend Luther
went, Melanchthon was fascinated by his personality. He was not
sufficiently self-reliant, too much of a specialist and pliant Erasmian
to adopt a firm and consistent line of conduct. He aided Luther
at all times, by polishing with his skilled pen the latter’s impetuously
uttered thoughts and bringing them into some kind of scientific
form, though he would have preferred to escape from the too turbulent
struggle to devote himself entirely to his humanistic studies, without,
of course, abandoning his “honored master” in his conflict
with the Roman Antichrist.

As early as 1525, it could be seen more clearly that Melanchthon
was animated by a desire to play the role of “Erasmian mediator”
in Lutheran theology. He thought the breach between the old and
the new theology could be healed and hoped for a conciliatory attitude
on the part of the Catholic opposition, all the more fervently
in proportion as he, who was more timid and more keenly sensitive
than Luther, was affrighted at the consequences of the ecclesiastical
revolution. Accordingly he devoted himself to the humanities in order
to prepare the way for that vague religious unity which ever hovered
before his mind. Because of his extraordinary success in the humanistic
field of learning his friends bestowed upon him the honorary title
of “\textit{Pracceptor Germaniae}.”\footnote
{Cf. Grisar, \textit{Luther}, Vol. III, pp. 319 sqq., 346 sqq., 360 sqq.}

Of quite a different stamp were the other men who stood closest
to Luther as friends and co-workers.

\textit{Justus Jonas}, provost of the church of All Saints at Wittenberg,
was likewise a disciple of Erasmus, but was far less active in behalf
of humanism than in his support of Luther.\footnote
{\textit{Ibid.}, Vol. III, pp. 413 sqq.}
As doctor of both
canon and civil law, he had taught the latter at the University of
Wittenberg, but after his promotion to the theological doctorate
through Luther’s influence, entered the theological faculty and
boldly lectured on portions of the Bible, such as St. Paul’s Epistle to
the Romans. The neo-humanists of Erfurt and the school of Mutianus
introduced this unfortunate priest to the pleasures of life. He was
of a jovial and lively disposition. Luther derived much consolation and
relaxation from his company. By translating his Latin works
into German, which he accomplished with great facility, Jonas rendered
greater services to Luther than by his scholarship. His lack of
theological erudition and profundity were compensated by a firm,
nay, fanatical attitude. Few men have treated their opponents more
disgracefully and unfairly and with worse personal invectives than
Jonas. He entitled the Latin apology of his marriage, published in
1522, thus: “In defense of Clerical Marriage against John Faber,
the Patron of Harlotry.” Among the defects of his character, according
to G. Kawerau, was a “constant, often petty concern in
the increase of his income.”\footnote{\textit{Ibid.}, p. 416.}

A third co-worker of Luther was John Bugenhagen, a native of
Wollin in Pomerania, hence called “Pomeranus,” the intruding
city parson of Wittenberg.\footnote{\textit{Ibid.}, pp. 405 sqq.}
After he had passed through the
humanistic current of his age, he attended private lectures in theology
at Luther’s side; though a priest, he had until then kept himself
rather aloof from theology. With his practical talents and energy,
which often degenerated into harshness, he, as parish priest of Wittenberg,
commenced quite early to unfold an extensive activity for
the advancement of Lutheranism, both in the electorate of Saxony
and far beyond its confines. His ability as an organizer made him
indispensable to Luther, who eulogized him as “Bishop of the Church
of Wittenberg,” as the chief support of the “Evangel” besides his
Philip, as a great theologian and a man of nerve. Because he supplied
a (rather deficient) commentary on the Psalms, Luther said that
Bugenhagen was the “first on earth who deserved to be called interpreter
of the Psalter.”\footnote{\textit{Ibid.}}
The most opposite of these epithets
was that of the “man of nerve”--\textit{multum habet nervorum}. Köstlin
rightly characterizes Bugenhagen as “merely a subordinate, though
endowed by nature with considerable powers of mind and body.”\footnote{\textit{Ibid.}, p. 407.}
His various “church regulations” were of greater importance than
his writings. Energetically and successfully he defended Luther’s
doctrine of the Last Supper when the Swiss theologians denied the
real presence of Christ in the Eucharist in Zwingli’s letter to Albert
of Reutlingen and his treatise “On the True and False Religion,”
published in March, 1525.

\textit{Nicholas von Amsdorf}, who first taught theology at Wittenberg,
became pastor and superintendent of the new religion in Magdeburg
in 1524, and as such always co-operated with Luther as his confidential
adviser.\footnote{\textit{Ibid.}, p. 405.}
Among all his friends he was most closely akin
to him in spirit and most appreciative of his mental sufferings and
struggles. He heartily concurred in the most unrestrained assertions
and outbursts of Luther, nay, possibly even outdid him. Luther
called him “a born theologian.” Later champions of orthodox
Lutheranism have glorified Amsdorf as the Eliseus of the Elias,
Luther, or even as a second Luther. The thick-set man with his
sharp features was a reckless enthusiast and became conspicuous by
his extreme views at the very outset of the struggle. In 1523 he
proclaimed it as his deliberate judgment that a Christian prince
is under obligation to bear arms in defense of the true Gospel. Luther
dedicated to him his “Address to the Nobility” and Melanchthon
his edition of “The Clouds” of Aristophanes. Amsdorf never married,
although he affirmed in one of his writings that marriage was a
divine command incumbent upon all priests.

Among Luther’s sympathizers in the secular faculties of Wittenberg University
the jurist \textit{Jerome Schurf} was the most conspicuous.
Luther also had sympathizers and active co-workers in the Augustinian
Order during the initial stages of the new movement.
Pre-eminent amongst these was \textit{Wenceslaus Link}, a man experienced
in business and fluent of speech. The Saxon congregation under
Staupitz comprised certain monasteries in the Low Countries, such
as Antwerp, Dordrecht, and Ghent. There Lutheranism took root,
especially through the efforts of two priors, Jacob Probst and Henry
Moller. The former was a native of Ypres, the latter of Zütphen
(Sutphen). Probst evaded the severe censures of the edict, first by
issuing a denial of his teachings, and later (in August, 1522) by secretly
escaping to Wittenberg. Moller likewise succeeded in reaching
Wittenberg before the censures became effective. In contrast with
these, two younger Augustinians were burned at the stake in Brussels,
the capital of the Low Countries, on July 1, 1523, in consequence
of their obstinate adherence to their heretical opinions. Their names
were Henry Vos and John van Eschen. Luther extolled them in his
hymn on the two so-called martyrs. A third Augustinian, Lambert
Thorn, who succeeded Probst in the office of preacher, was likewise condemned
to death in Brussels, but escaped with his life for
unknown reasons.

Luther derived assistance also from the ranks of the Franciscans.
Thus he was aided by the popular orators John Eberlin and Henry
von Kettenbach, and by the writers Frederick Myconius and Conrad
Pellican. Among the Dominicans who rallied round his banner was
the talented young monk, Martin Bucer, whom he had partially
won over by the disputation at Heidelberg. The German Dominicans
did not furnish him with another man of repute, neither from the
Saxon nor from the Upper German province nor from the Upper
German congregation. On the whole, the Dominicans as well as the
Franciscans consistently and decisively maintained their Catholic position
and opposed the religious innovation. They assigned their most
capable writers to enter the lists in defense of the ancient faith.
Oecolampadius left the Brigittine monastery at Altomünster and
Ambrose Blaurer deserted the Benedictine monastery of Alpirsbach to
join the new movement. Both labored with success in the interests of
the religious revolt.

\textit{Caspar Schwenckfeld}, an eccentric and fanatical layman, was a
friend of Luther for many years. He intended to bring about a reformation
of Christianity on the basis of Lutheranism along novel
lines of his own. He was born of a noble family in the duchy of
Liegnitz. When Luther took his stand against indulgences, Schwenckfeld
was already inclined to join him. In the beginning of the twenties
he tried to persuade the prince of Liegnitz to introduce the new
religion into his native city and into all Silesia. Though a layman, he
preached with unction, and his captivating manners, coupled with
an impressive appearance, enabled him to win many adherents, especially
among the nobility.\footnote{\textit{Ibid.}, Vol. V, pp. 78 sqq.}
In his endeavor to arouse men to a
realization of the seriousness of life, he took offense at the omission of
good works from Luther’s doctrine and censured the loose conduct
which he observed about him, in his “Admonition” against “carnal
liberty and the errors of the common people,” published in 1524. In
the following year he pretended to have received by private revelation
a new doctrine of the Lord’s Supper which abandoned the Real
Presence of Christ in the Eucharist. His doctrine on the Eucharist
was not approved by Luther, Jonas and Bugenhagen, whom he visited
at Wittenberg, notwithstanding the fact that he did not incline to
the rationalistic theories of Zwingli. In vain he explained to Luther,
on December 1, 1525, his supposedly deeper conception of Christ’s
abiding survival in the faithful without a Eucharist, Sacraments or
a church. In the course of his exposition he ardently advised Luther
to abandon his idea of a congregational church with general communion
and in lieu thereof establish congregations composed of revived Christians,
such as Luther himself had dreamed of at times.
Luther did not wish to break with this influential man, but a breach
did come later. According to Schwenckfeld’s statement, Luther had
admitted the plausibility of his doctrine of the Eucharist, even though
it was as yet undemonstrated, and declared: “Dear Caspar, wait a
little while.” Probably he merely intended in this fashion to get rid
of the importunate Schwenckfeld. The latter, however, was not inclined
to wait, but provoked an open controversy, in which, without
entirely denying Luther’s teaching, he frankly and severely exposed
its weaknesses, particularly in their moral implications. His example
reveals anew how the arbitrary subjectivism of Luther aroused opposition
on the part of his adherents and introduced chaos into the very
bosom of the new religion.

For the rest, Schwenckfeld was one of the few men who, having
entered into relations with Luther, refused to succumb to the fascination
of his personality. The friends whom we have enumerated
above, and many others, yielded to that exceedingly powerful charm.
There is abundant evidence of the superior force which he exercised
in his personal intercourse with others, such as John Kessler, Albert
Burrer, Peter Mosellanus, and, at a later date, Mathesius, Spangenberg,
Aurifaber, Rhegius, and Cordatus. All agree that there was
something charming about his affability, his attractive speech, and
constancy in the midst of trouble.\footnote{\textit{Ibid.}, Vol. IV, pp. 268 sqq.}
One may say that he was by
nature endowed with an immense power of suggestion, intensified by
his exterior appearance, particularly by his flashing eyes. In addition,
the influence of his personality was augmented by the glory of his
unexampled success.
