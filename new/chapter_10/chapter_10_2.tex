\section{Auxiliaries from the Monasteries}

Lichtenberg in the Saxon Electorate affords an example of how
Lutheranism gained ground by enticing the occupants of important
clerical positions to violate the vow of celibacy. The appeal to sensuality
served as a stimulant. Whereas the measures discussed before
were coercive, we have now to consider a kind of moral compulsion
whose power over mortal man was fully realized by those who set the
evil passions in motion. In this respect the letter by which Luther
prepared the way for the change of religion of Lichtenberg is an
extraordinary document. In that city there was a famous monastery
of Antonine monks, who followed the rule of St. Augustine. It was
governed by Wolfgang Reissenbusch, a doctor of laws and a former
student of the University of Wittenberg. He was bound to his Order
by solemn vows, and discharged the office of “preceptor” of the
monastery and administrator of its property. Luther’s friends reminded
him that Reissenbusch, notwithstanding his scruples, could
probably be prevailed upon to marry--a matter which he had already
discussed with him. Luther sent him an open letter, which was published
at Wittenberg under the title: “A Christian Letter Addressed
to Wolfgang Reissenbusch,” etc. In it he tries to impress the recipient,
as well as all others who are similarly situated, with every conceivable
reason to induce them to violate their vow of chastity by an immediate marriage.

As a man, he says, Reissenbusch was “created for and compelled by God
Himself to embrace the married state.” The monastic vow is void because it
demands the impossible. To keep chastity is “as little within our power as
to work miracles.” As long as one is neither an angel nor a spirit, we are
told, “God in no wise bestows or grants this privilege.” He who takes such a
vow relies “upon works, and not upon the grace of God;” he “takes his
stand upon works and commandments” and denies “Christ and the faith.”

Then follows a detailed, and, in some places, disgusting exposition of the
alleged inevitable necessity of sexual intercourse. The non-satisfaction of
the sexual instinct in matrimony had resulted in immorality among the
entire clergy and in all the monasteries. Luther overwhelms his tempted
friend in sinister language and with a demoniacal style intended to excite the
passions. “It is necessary that you be urged thereto, that you be exhorted,
driven, incited, and encouraged. Well now! dear sir, I prithee, why do
you wish to delay and meditate, etc.? It must not, it ought not, and it will
not be otherwise. Banish the thoughts from your mind and go ahead joyously!”
True he would thereby become a “matrimonial mantle covering the
disgrace” for others; but Christ, too, had become “the mantle that cloaks the
disgrace of us all.”\footnote
{Weimar ed., Vol. XVIII, pp. 270--279; Erlangen ed., LIII, pp. 286 sqq. (\textit{Briefwechsel},
V, p. 145), letter of March 27, 1525. Finally Luther says (\textit{ibid.}): “It is but a matter of a
brief hour of disgrace, to be followed by years of honor.”}

The reference to Christ is repulsive. That holy name ought rather to have
reminded him of the admonitions of Christ and His disciples, which were
the very antithesis of his own exhortations. It should have recalled to him
the words which served as a guiding star throughout all the centuries
for those who voluntarily bound themselves by a vow of chastity.
It should also have reminded him of the grace of Christ in which he who
makes this sacrifice places his sole reliance and whereby that which appears
impossible to the world is rendered easy and a source of joy. In lieu of these,
we have alluring descriptions of the irresistible force of the sexual instinct.

Reissenbusch yielded to Luther’s persuasion, bade farewell to the
Order to which he had bound himself by a solemn vow, and married
Hanna Herzog, the daughter of a poor tailor’s widow at Torgau.
With the connivance of the Elector he retained his clerical office as
“preceptor” and the endowments entrusted to his Order. For the purpose
of inviting others to imitate his example, the incident was exploited
in the press by Bugenhagen, the Lutheran pastor of Wittenberg,
who addressed to the happy groom an “Epistola Gratulatoria
de Coniugio Episcoporum et Diaconorum” which he also caused to
be published in German. Such great importance was attached at Wittenberg
to the marriage of priests and monks as an auxiliary factor
in the extension of the new Evangel.

Already at a previous stage of his career Luther had approached,
among others, the Order of Teutonic Knights, urging its members
to break their vows by marrying. Unfortunately, discipline had declined
among these Knights, so that he had reason to hope that they
would respond to his public invitation. Like the priests of this Order,
the knights, too, were bound by a voluntary vow of chastity. Their
contact with the world exposed them to special danger. The general
reform by means of which Adrian VI endeavored to check the decline
of their monastic discipline, proved to be but partially adequate in
view of the dissensions that prevailed among the rulers of the Teutonic
Order, especially since the Grand Master of the Order, Albrecht
von Brandenburg, a cousin of the Elector Frederick of Saxony, was
himself favorably inclined towards Lutheranism. After receiving a
visit from the Grand Master at Wittenberg, on November 29, 1523,
Luther wrote his “Exhortation to the Knights of the Teutonic Order
to Avoid False Chastity and Embrace Lawful Matrimonial Chastity,”
which was at once published in German.\footnote
{Weimar ed., Vol. XII, pp. 232 sqq.; Erl. ed., Vol. XXIX, pp. 16 sqq. Grisar, \textit{Luther},
I, 116 sq., 317 sq.}

In this work, the author depicts matrimony in alluring colors, as the
proper thing for their state. He tells the knights who had secret and illicit
such relations with women, ``not to despair in weakness and sin,'' because such
extra-matrimonial relations were ``less sinful'' than to ``take a lawful wife''
with the consent of a council of the Church, supposing such a permission
were given.\footnote{Grisar, \textit{Luther}, Vol. II, p. 120.}

In conclusion he says that Christ had reserved to Himself certain bishops
who would resign from their office or transform it into a genuinely episcopal
office for the sake of the Gospel. Many a bishop or abbot would marry, he
says with a significant hint to weak ecclesiastical dignitaries, if the path
were only blazed for them and wooing were no longer regarded as a disgrace
and a danger.

These alluring appeals did not fail to attain their object in the
case of those members of monastic orders to whom they were addressed. With
the same energy Luther set about the task of winning
over the episcopate, ridiculing the bishops who refused to heed
him. When, in compliance with an imperial mandate, the bishops
of Meissen and Merseburg proceeded with their visitations and called
to account the clerics who had married, he issued a tract in which
he gave full vent to his irritation against the hierarchy (1522).
His main intention was to brand the higher clergy as immoral and
to strengthen his appeals to the lower clergy to marry and preach
the pure Gospel. This tract bore the title: “Against the Falsely Named
Clerical State of the Pope and the Bishops.”\footnote
{Weimar ed., Vol. X, pp. 105 sqq.; Erl. ed., Vol. XXVIII, pp. 141 sqq.}
In this work he calls
himself an “evangelist by the grace of God,” declaring he had the
same right to style himself thus as they have to call themselves
bishops, since he was certain that Christ regarded him as such and
would testify in his behalf on the day of judgment.
Here, too, he teaches that the sexual impulse can be controlled in the
clerical state as little as fire can be deprived of its power to burn; that it
is either “all fornication” or “impure, involuntary, miserable, lost chastity.”
There is “scarcely one among a thousand who lives an upright life.” These
few are “God’s special miracles.” Pope and bishops permit innocent men
to be sacrificed “to Moloch, the fiery idol.” “Monasteries and convents are
gates of hell, where the faith (\textit{i.e.}, his faith) is not practiced
with honesty and vigor.”\footnote{Grisar, \textit{Luther}, Vol. I.}
He does not tire of censuring them because of their corrupt life.

But, he interposes, will not a revolt be the final outcome
against the episcopate? What about it? “It would be better”--thus runs his
terrible reply--“that all the bishops were murdered, that all the monasteries
and convents were uprooted, than that a single soul should perish. Of what
use are they but to live voluptuously by the sweat and toil of others?”\footnote
{Köstlin-Kawerau, \textit{M. Luther}, Vol. I, p. 517.}
As an ecclesiast by divine right he boldly issues “Doctor Luther’s Bull and
Reformation,” which begins with the solemn declaration: “All who stake
their lives, their property, and their honor, that the bishoprics are destroyed
and the episcopal régime is exterminated, are dear children of God and
true Christians, who observe God’s commandments and combat the devil’s
régime \dots All who sustain the rule of the bishops and are subject to
them by voluntary obedience, are the devil’s very own servants and militate
against God’s order and law.” To this inflammatory appeal to violence,
however, he appends the modifying clause that he does not wish to destroy
with “club and sword,” but, as “Daniel teaches (Dan. 8:25), the Antichrist
shall be broken without hand, so that everyone, with God’s Word, will talk,
teach and stand firmly against the Antichrist, until he be confounded,
abandoned, and despised, and come to grief of his own accord. That is a true
Christian agitation for which one should stake his all.”\footnote
{For contrary utterances, see Grisar, Luther, Vol. III, pp. 44 sqq. Likewise Grisar,
\textit{Lutherstudien}, VI (\textit{Kampfbilder}, Grisar and Heege, Heft IV), pp. 126 sq., and especially
pp. 137 sqq.}

Besides the extirpation of the episcopate, Luther had at heart
particularly the emancipation of the nuns. Soon after his Wartburg
days, he dedicated two tracts to the “pious children” among the
nuns, who were desirous of hearing the voice of the Gospel. One of
these, published in April, 1523, bears the title: “Reason and Reply,
why Virgins may leave Convents with Divine Sanction.” The other
is entitled: “Story of How God Aided a Nun.” In contrast with the
preceding appeal to agitate, the latter reads almost like an idyll. It
was intended to inspire other nuns to leave their convents.

The occasion of the former publication\footnote
{Weimar ed., Vol. XI, pp. 394 sqq.; Erl ed., Vol. XXIX, pp. 33 sqq.}
was furnished by twelve
Cistercian nuns, who fled from their convent at Nimbschen near
Grimma, with the assistance of a town-councilor, Leonard Koppe of
Torgau. Nine of these fugitive nuns came to Wittenberg. Among
them were Catherine von Bora and a sister of Johann von Staupitz.
According to Luther, this pamphlet was written expressly to illustrate
how all nuns should liberate their consciences and save their
souls. To the objection that such clandestine flight, combined with
a denial of the monastic vow, gives rise to scandal, he replies: “Away
with scandal! Necessity knows no law and gives no scandal \dots I
should consult my soul; let the whole world be scandalized!” It
is interesting to note Luther’s confession that he had himself, with
the aid of Koppe, planned the escape of the twelve nuns, who had
been enlightened by his writings. They were mostly daughters of
the nobility, who had been committed to the convent according to
custom and hence failed to honor the state of life which they had
embraced voluntarily. Their lodging with relatives or families in
Wittenberg was a source of no small anxiety to Luther, since he
feared that opportunities of marrying them off might not present
themselves so readily.\footnote
{Amsdorf offers his assistance in procuring husbands for them; thus he offers the sister
of Staupitz to Spalatin, adding: “But if you wish for a younger one, you shall have your
choice of the prettiest.” (Grisar, \textit{Luther}, Vol II, p. 137.)}

After several weeks, three more nuns were abducted from the
convent of Nimbschen by their families. Simultaneously sixteen
escaped from the convent at Widerstett in Mansfeld, of whom five
found lodging with Count Albrecht of Mansfeld, who was very
friendly to the Lutheran cause.

The heroine of Luther’s “Story of How God Came to the Aid of
a Nun” was Florentina of Oberweimar, who had abandoned her
convent at Neu-Helfta, near Eisleben.\footnote{Weimar ed., Vol. XV, pp. 86 sqq.; Erl. ed., Vol. XXIX, pp. 102 sqq. (year 1524.)}
She told Luther of alleged
bodily torments inflicted upon her because of her religious views.
Luther willingly believed her story and immortalized her in a publication
which he addressed to Count Albrecht of Mansfeld as a “sign
in confirmation of the Gospel”--which sign one may not overlook
with indifference.\footnote{Grisar, \textit{Luther}, Vol. III, pp. 159 sq.}
In compliance with the rule of her Order,
Florentina had completed her year of probation and taken the vows.
Having imbibed other ideas from the writings of the reformers,
she was subjected to penalties by her superioress and kept in strict
custody. But behold, O miracle, one happy day in February, 1524,
“the person who should have locked her up, left her cell open” and
she escaped! “God’s word and work,” Luther writes in all seriousness,
“must be acknowledged with fear; nor may His signs and
wonders be cast to the winds.” Ordinarily, he adds, such “miraculous
signs from God” are not properly heeded!

The birth of a deformed calf at Freiberg (Saxony) towards the
close of 1522 was regarded by Luther as a miracle wrought by God in
condemnation of the monastic life.\footnote
{Relative to the following, cf. Grisar, \textit{Luther}, Vol. III, pp. 149 sqq., and \textit{Lutherstudien},
V (\textit{Kampfbilder}, n. III), pp. 14 sqq., with two illustrations.}
He found that this monstrosity
really represented a cowled monk in the act of preaching--an evident
symbol of the divine wrath against the religious state. He published
his discovery in a treatise entitled, “Interpretation etc. of the MonkCalf
of Freiberg.”\footnote
{Printed with Melanchthon’s dissertation on the “pope-ass” in the Weimar ed. of
Luther’s writings, Vol. XI, pp. 369 sqq.; in the Erlangen ed., Vol. XXIX, pp. 7 sqq.}
The work was composed in a quasi-mystical
style. The age was very superstitious about monstrosities, but Luther’s
pamphlet was unprecedented. In view of this literary product, one
would like to wish, in the name of German literature, that the interpretation
had been intended to be facetious. In reality an attempt
has been made on the part of Protestants to explain the pamphlet
as a huge joke. But a careful perusal of it completely destroys this
hypothesis. The work on the monstrosity of Freiberg is itself a
monstrosity. A terrible seriousness breathes from these prophetical,
hyper-spiritualistic pages. The author quotes Sacred Scripture to
show that his interpretation is “adequately founded” on the word
of God. He intimates that perhaps the portent signalizes the day
of judgment, “since many portents have succeeded one another of
late.” In exhibiting to his readers a distorted illustration of the deformed
calf, he hypercritically undertakes to apply the details of
this miraculous phenomenon to monasticism. The supposed cowl
represents the worship which the monks render to the calf, \textit{i.e.}, “the
false idol in their lying hearts.” The cowl over the hind-quarters
is torn, this signifies the impurity of the monks; the legs are their
“impudent doctors”; the monster is blind because they are blind;
its ears are grotesque because of the abuse of the confessional; the
tightening of the cowl around the neck signifies their obstinacy;
the crippled horns indicate God’s intention of breaking the power
of monasticism; above all, the attitude of the calf is that of a
preacher, which means that the preaching of the monks is despicable
in the eyes of God.

Melanchthon prefaces the story of Luther’s “Monk-Calf” by another treatise,
namely, his own interpretation of the “Pope-Ass of
Rome,” a semi-legendary freak supposed to have been discovered in
the Tiber in 1496.\footnote{Grisar, \textit{Lutherstudien}, n. V. (\textit{Kampfbilder}, n. III), pp. 1 sqq., with two illustrations.}
The learned humanist was even more absorbed
in the mystical world of such portents, than Luther. The latter
subsequently approved of and praised the first part of their joint
production, in his “Amen to the Interpretation of the Pope-Ass.”

“The sublime divine Wisdom itself,” he said, “created this hideous, shocking,
and horrible image.” \dots “Well may the whole world be affrighted and
tremble.” “God manifests Himself openly in this abomination; great indeed
is the wrath impending over the papacy.”

“The multitude of signs,” which Luther beheld and interpreted, presaged
“something greater than reason can imagine,” to quote the words of his
\textit{Kirchenpostille}.\footnote{Grisar, \textit{Luther}, Vol. II, p. 150.}

Both works, that on the “Monk-Calf” and that on the “Pope-Ass,”
enjoyed the widest circulation, both jointly and separately, in repeated
German editions and in translations into foreign languages.
The illustrations were circulated as leaflets in order to gain adherents
to the Lutheran cause. The “Pope-Ass” constituted a permanent
fixture in Luther’s polemical vocabulary. As late as 1545, the picture
was selected by him for inclusion in his collection of “Illustrations
of the Papacy.”

The lively conviction with which Luther treated similar portents, in
which “God openly manifests Himself,” constitutes in some measure
an excuse for his conduct. The manner in which he labored to
promote opposition to the Church after his return from the Wartburg,
reveals a misguided combative spirit, inspired by design, acrimony,
hatred, and other reprehensible motives. At the same time, the great
power of his own prejudices must always be taken into consideration.
His excited state of mind did not permit him to measure his steps
with sufficient clearness. His eschatological notions, as revealed by
his writings on the monsters just mentioned, limited his intellectual
outlook.\footnote
{\textit{Ibid.}, Vol. III, pp. 153 sqq.; Vol. V, pp. 241 sqq.; Vol. VI, PP. 141 sqq.}

We must always remember that the historical portrait of Luther
is not devoid of favorable traits, even at the time of his severest
polemical strain after his sojourn at the Wartburg. There is, in the
first place, his external manner of life. He is remarkably unconcerned
about his dangerous status of one declared an outlaw by the empire
and is satisfied with the very modest circumstances in his decaying
monastery, which was hardly able to provide him with food and
lodging. He is always ready to advise his friends, even though he is
overwhelmed with letters. He attracts students to himself by his
winsome and unassuming ways. In his sermons he preaches a sound
morality, often with forceful emphasis and great ardor, always with
marvelous clearness, plastic metaphors, and directness of speech.
Abandoning the field of controversy for that of practical religion, he
publishes popular works, such as his prayerbook, which, as he says,
was intended to “propose a simple Christian form and mirror,”
to help the faithful to “recognize sins and to pray.”\footnote
{Köstlin-Kawerau, \textit{M. Luther}, Vol. I, p. 574.}
This is true of many other printed sermons on Biblical subjects, on the commandments,
on faith, on the Our Father and the Hail Mary.\footnote
{Kawerau, \textit{Luthers Schriften}, n. 178, 216, 242, 265, and \textit{Kirchenpostille}, n. 137, 163.}
Even more important was the continued labor devoted to his translation of the
Old Testament, of which portions appeared from time to time.

Did he still remain a monk in his exterior appearance? Dantiscus
reports that Luther, when he visited him in 1523, no longer wore
the Augustinian habit at home. He wore it when he preached, however,
until October, 1524, when it was quite threadbare, and then exchanged it
for a civilian coat, the cloth for which was presented to
him by the Elector.
