\section{The Movement Within the Empire}

After his return to Wittenberg, Luther, as we have seen, labored
almost as zealously in behalf of his new religion, as if no measures
had been taken against him at Worms. Notwithstanding the imperial
ban, his activities were but little restricted. There had been no effective
prohibition of his books, no determined prevention of the
sending of preachers and the seizure of parishes, no lack of freedom
for his personal movements, at least not within the confines of
electoral Saxony. What particularly aided his cause was the fact
that the Emperor’s attention had been almost completely diverted
from Germany immediately after the diet of Worms. The long war
with Francis I of France occupied all of his time. Henceforth, the
forces prepared to offer resistance to Luther--and of these the rulers
of the empire had an adequate number at their command--lacked
a rallying point and competent leaders. The so-called “imperial
regimen” at Nuremberg was a cumbersome body and with its constant need
of funds not adapted to evolve a uniform and effective
policy against a prince.of the reputation and political acumen of
Frederick of Saxony, the protector of Luther, and against the impetuous
methods of the theological tribune of Wittenberg, who had
succeeded in arousing the masses. The mandatory formulas of the
edict of Worms, with their medieval apparatus, may have seemed
promising in the eyes of the young emperor, who was zealous in
promoting the interests of the Church, and in those of the papal
nuncio, Aleander. In reality, however, they had lost much of their
former force in view of the changed conditions of the time.

The head of the imperial regimen at Nuremberg was the Emperor’s
brother, Ferdinand, a sincere Catholic, who acted as viceroy for the
empire and was ably assisted in the affairs that pertained to the support
of the ancient religion by the courageous Duke George of
Saxony. Among the foreign ambassadors accredited to the government the
Saxon councilor, John von Planitz, labored energetically
and successfully to promote the pro-Lutheran and dilatory policy
of his master, Frederick of Saxony. The severe measures which
the government had originally adopted to enforce the edict of
Worms, found no response, not even in Nuremberg itself. The opposing elements
succeeded in protecting themselves by indicating the
danger of a revolt on the part of the agitated masses. Above all
they pointed out that the theological questions at issue had not as
yet been definitively decided. The complaints of the Catholic members of
the government against the oppressive financial measures
of Rome likewise constituted an obstacle to decisive action.

During this dangerous state of suspense, Pope Leo X passed away,
on December 1, 1521. His successor, the pious and scholarly Adrian
VI, who had labored in opposition to the ecclesiastical revolution as
a professor at Louvain and as cardinal-archbishop of Tortosa, conceived
the noble plan of mastering the hostile movement in Germany
by means of a thorough reform within the Church, and by openly
acknowledging that the curia and the clergy had a share in the guilt.
He sent Chieregati as nuncio to the imperial diet which assembled at
Nuremberg in the fall of 1522, and commissioned him to deliver
a celebrated address on the necessity of a reform of Rome and the
entire hierarchy and clergy, which has become a unique document of
his unselfish, profound, honest, and candid character. In the strained
relations of the time, however, it failed to produce any results.\footnote
{Pastor, \textit{Geschichte der Päpste}, Vol. IV, Part 2, pp. 89 sqq. In his address, Chieregati
read his instructions, which, according to Pastor, are “a document unique in the history
of the papacy.” “We freely confess,” the document states, “that God permits this persecution
of His Church because of the sins of men, particularly those of the priests and prelates
\dots All these evils have perhaps originated with the Roman curia,” etc. (Pastor,
p. 93).}
The efforts which he made during his short pontificate of twenty months
towards ameliorating conditions in Rome were destined to fail in a
great measure, due to the opposition of worldly-minded priests and
prelates. It required time to achieve the necessary reforms. This high-minded
pope, the last of German race, saw his fondest hopes shattered
and was carried off all too prematurely by death.

Chieregati’s demand that the edict of Worms be executed, was
turned down by the imperial diet for the reason that it might provoke civil
war. In lieu thereof, the estates demanded a church council,
which was to be convoked within one year on German soil, for the
purpose of allaying the current controversies. In the meantime the
Gospel was to be preached “in conformity with the right Christian
understanding.” Luther could content himself with this resolution.
Under somewhat more favorable auspices, a new diet was opened
at Nuremberg in January, 1524.

The new pope, Clement VII (1523--1534), who, as Cardinal Julius
de’ Medici, had led an irreproachable life, made every possible effort
to suppress the religious innovations by enforcing the decrees of
Worms. As a result of the activities of his excellent nuncio, Cardinal
Lorenzo Campeggio, the majority of the delegates at the diet
acknowledged the legality of the edict of Worms and declared their
willingness to enforce it “as far as possible.” A minority, consisting
mainly of representatives of the cities, in which the reform movement
was stronger, declared that the execution of the edict of
Worms was simply impossible and repeated the misguided demand
for a “free ecumenical council,” to be held in Germany. This was
accompanied by the still more misleading demand, in the form of a
resolution adopted by the assembly, for the convocation of a national
synod at Spires in the autumn, which was temporarily to restore
order. Campeggio at once declared the latter resolution to be the
beginning of an “eternal schism.” The Pope repudiated it most
energetically, and the Emperor in forceful language prohibited the
undertaking and demanded the enforcement of the edict.

Luther now resumed the fight against the edict of Worms. He
reprinted it with a furious commentary, bearing the title: “Two
Discordant Imperial Commandments concerning Luther.”\footnote
{Weimar ed., Vol. XV, p. 254. Erl. ed., Vol. XXIV, 2nd ed., p. 220.}
He confronted it with the resolution of the diet of Nuremberg in 1524. In
it he incites the Germans by appealing to them to consider how, after
all, they were “compelled to be the asses and martyrs of the pope,
even if they had to be pulverized in a mortar.”” In heroic words he
tells the princes of his willingness to die: “What, is not Luther’s
life so highly esteemed before God, that, if he died, not one of you
would be certain of your life or rule, and his death would be a
misfortune for you all? There is no jesting with God.” So certain
was he of his spiritual mission.

At the instigation of Campeggio, the Catholic princes now adopted
practical measures by forming a defensive alliance against Luther
and his passionate threats of a revolution, which, in his opinion, could
be stemmed by him alone. The idea of an alliance was in the air.
Towards the end of June, 1524, Ferdinand with the Bavarian dukes
and most of the bishops of Southern Germany founded at Ratisbon
a union for the protection of the Catholic religion, the extermination
of heresy, and the solidification of the Empire. It was a necessity
forced upon them: by the dimensions which the religious revolt had
assumed and by the dangers which threatened the State. The destruction
of German unity was not begun at this time, as Protestant historians
maintain; rather, the subsequent decline of unity was caused
by the division that had preceded this union and by the religious
schism which was accomplished by political measures. The Emperor
welcomed the alliance of Ratisbon. It was welcomed still more cordially
by Pope Clement because of the hopes it engendered of restraining the religious
defection. For the time being, however, the Pope’s
efforts to extend the alliance of the princes into Northern Germany
proved futile.

Among those who inspired the least hope of energetic resistance
to the innovation, was the archbishop and elector of Mayence,
Albrecht of Brandenburg. This man, who adhered to a frivolous
philosophy of life and was known for his loose morals, maintained
distinguished Lutherans such as Wolfgang Capito at his court.
Luther, who knew that he could depend on him, dared to write to
the Archbishop on December 1, 1521, saying that he must not molest
the priests who had married and abolish the practice of indulgences
in the city of Halle; he added that he expected a reply within
fourteen days, and if it were not forthcoming, he would publish
his work “Against the Idol of Halle,” which he had completed, but
never printed.\footnote{Erl. ed., Vol. LIII, p. 95 (\textit{Briefwechsel}, III, p. 251).}
After three weeks, on December 31, Luther received
a reply in which the Archbishop addresses him as “Dear Doctor”
and states that he had received his letter graciously, that the indulgence
at Halle had been discontinued, that he intended to be
“a pious, religious and Christian ruler,” and, finally, that he purposed
“to be favorable and friendly” towards Luther “for the sake
of Christ.”\footnote{\textit{Briefwechsel}, III, p. 265.}
Luther now waxed more hopeful. But since he was
disappointed in his principal expectation, namely, that the Archbishop
would marry and convert his spiritual fief into a secular
state, he launched a personal attack upon him, in a letter dated June
2, 1525. He positively demanded that the Archbishop “enter the state
of matrimony and transform his bishopric into a secular principality.”
He also stated that the spiritual order was doomed to destruction beyond
recovery, and if God did not perform a miracle,
it is “terrible if a man were to die without a wife,” since God had
created him a male. At that time he was about to marry Catherine
Bora. Shortly afterwards this letter to Albrecht was printed for the
benefit of other spiritual princes of the Empire.\footnote
{Erl. ed., Vol. LIII, p. 308 (\textit{Briefwechsel}, V, p. 186).}
A kind fate preserved
Albrecht from adopting the proposal of Luther, who directed
new outbursts of ill-will against the Archbishop on several later
occasions.

In the letter which Luther wrote to Albrecht of Brandenburg,
he referred to the general degradation of the clergy manifested by
“various songs, sayings, satires,” and by the fact that priests and
monks were cartooned on walls, placards, and lastly on playing
cards. This systematic defamation was common particularly in electoral
Saxony, during the reign of Frederick, the protector of the
“Reformation,” who knowingly permitted the attacks upon Catholicism
to increase in every department of life. The deception and
duplicity which he practiced casts a dark shadow upon his character
and places his customary surname, “the Wise,” in a peculiar light.
Up to his death, on May 5, 1525, Frederick practiced double-dealing
in religious matters. He never married, but had two sons and
a daughter by a certain Anna Weller. He and Albrecht of Mayence
were the two most esteemed and powerful German princes of the
time, the one a spiritual primate, the other a prince in the temporal
order, but neither of them distinguished by high moral qualities.

A few words must be added concerning the attitude of the Saxon
Elector towards Catholicism in Wittenberg.

The support which Frederick gave to the religious innovation
produced deterrent phenomena, particularly in the stormy fight
which Luther waged against the remnants of the Mass at Wittenberg.
Notwithstanding his former declaration of tolerance toward the
“weak” and his statement regarding the avoidance of force, Luther’s
intervention against the celebration of the Mass on the part of the
last Catholic priest at the electoral chapel and the monastery church
became a tragedy of flagrant intolerance.\footnote
{For the following cf. Grisar, \textit{Luther}, Vol. II, pp. 88 sqq., and especially pp. 327 sqq.;
Vol. IV, pp. 506 sqq.}
Already in 1522, the
Elector, yielding to the pressure of Luther, abolished the customary
solemn exposition of relics. On March 1, 1523, Luther invited the
chapter to put an end once and for all to the celebration of the
Mass, otherwise the capitularies would have to be disfellowshipped
from the communion of the Church. In a second letter he seriously
threatened to discontinue his prayers for them, which might cause
unpleasant consequences before God! A romantic self-deception regarding
his influence in Heaven! When the Elector still hesitated to
give his consent and warned against disturbances, Luther appealed
to the people in a sermon in which he advised them not to lay violent
hands on the canons, and stated that the territorial lord had “no
authority except in secular matters.” A new, sharp letter of Luther
to the cathedral canons provoked the censure of the Elector; but
Luther knew he could go farther; he felt assured of the final approval
of Frederick, and nothing in that letter was more correct
than the warning issued to his enemies, that they were not certain
of the protection of the Elector. A new sermon in which Luther
fulminated against the holy Sacrifice of the Mass, was delivered on
November 27, 1524. The princes and the authorities, he exclaimed,
ought finally to force “the blasphemous servants of the Babylonian
harlot” to stop the devilish practice of saying Mass. It was scarcely
possible to restrain the people and the students from committing
acts of violence. The town-council and the university threatened
with the wrath of God the priests who still held out. Finally,
Frederick “the Wise” abandoned them ignominiously to their fate.
A vigorous word from him, reinforced by his guard, would have
silenced the opponents, at least in the city.

The canons finally bowed to the raging elements. On Christmas,
1524, Mass was suspended for the first time, never to be resumed.
Referring to the three remaining Catholic canons, Luther, in his
characteristic fashion, said that “three swine and bellies” still remained
in the church, “not of All Saints, but of all devils.”
An echo of his violent sermons against the Sacrifice of the Mass
was the tract, “On the Abomination of the Silent Mass, called the
Canon,” which he published in the beginning of the year 1525.\footnote
{Weimar ed., Vol. XVIII, pp. 22 sqq.; Erl. ed., Vol. XXIX, pp. 113 sqq. ; Grisar,
\textit{Luther}, Vol. IV, pp. 508 sqq.}
In it he attributes the merit for the deeds of violence perpetrated at
Wittenberg to the “secular lords,” who, he says, had been obliged
to intervene.

In a letter written at the beginning of May, 1525, to the Elector
Frederick, who was very ill at the time, Luther’s friend Spalatin,
Frederick’s counselor and guide in the above-described fight on
the Mass, unreservedly set forth the duties of secular rulers to
promote a religious reformation.\footnote{Köstlin-Kawerau, \textit{M. Luther}, Vol. I, p. 724.}
This applied to all territorial rulers.

A few days later, on May 5, Frederick died in his castle at Lochau
after receiving the Last Supper under both forms, as an adherent of
Luther; he was the first German prince thus to pass away. Luther
had been summoned to attend him in his last moments, but arrived
too late. In fact, he had never met Frederick personally. On the
tenth of May, and again on the eleventh, the day of the funeral
he delivered funeral orations in the castle-church, which abounded
in exuberant eulogies of his friendly and prudent protector. In a
consolatory letter addressed to John, the brother of the deceased, who
succeeded him as ruler, he renewed his former, very intimate relations.
John proved an even more determined protector than Frederick,
and with his assistance Luther was able to exterminate Catholic worship
in the electorate of Saxony.

As the Reformation was imposed in electoral Saxony by pressure
from above, so, too, in other German territories. The free imperial
cities especially hastened to take the lead in the introduction of the
new ecclesiastical régime by recourse to penal measures ‘enforced
by the civil authorities.

The tremendous development of the civil power at that time was
very advantageous to the extension of the Protestant revolt. A
number of practically independent principalities arose from the loose
structure of the empire. Due to the long absence of the Emperor,
the territorial rulers found themselves thrown upon their own resources.
The increase of power which accrued to their scepters in
virtue of the new religious system, accelerated their steps in the
direction of absolutism.

As early as 1523, Luther had dedicated his treatise “On Secular
Authority and the Extent of the Obedience due to it” to John, the
heir-apparent to the throne of the Saxon Electorate.\footnote
{Weimar ed., Vol. XI, pp. 245 sqq.; Erl. ed., Vol. XXII, pp. 59 sqq.}
It was a
sermon which he had delivered in John’s presence at Weimar and
which was published at the latter’s request, after having been enlarged
by the author. Luther later on loved to appeal to this work,
in order to show that it was he who had indicated the proper measures
to the civil governments for emerging from the oppression of the
papacy. “I would fain boast,” he says of himself, “that, since the age
of the Apostles, the secular sword and authority have never been
described so clearly or praised so splendidly, as by me, as even my antagonists
must acknowledge.”\footnote
{Köstlin-Kawerau, \textit{M. Luther}, Vol. I, p. 584. Cfr. Grisar, \textit{Luther}, Vol. II, pp. 294 sqq.;
IV, 331.}
However, what good there is in this
work had long ago been expounded by Catholic writers, \textit{e.g.}, the
demonstration from Holy Writ that the secular power exercises its
authority by the will and ordinance of God. Luther’s exhortations to
the princes in the third part are beautiful, but by no means new.

On the other hand, the new ideas contained in the second part of
the treatise on the restriction of the civil power to temporal affairs,
and to the punishment of evil-doers and the protection of good
citizens were fallacious and contradicted the theories concerning
the assistance to be furnished by temporal rulers for ecclesiastical
purposes which he himself subsequently enunciated and openly applied.
In the second part of the treatise he has in mind only Catholic
princes, his intention being to oppose a strong barrier to their
measures against Lutheranism and for the protection of the Catholic
religion. Hence the assertion that secular princes have no voice in
matters pertaining to religion. Hence, also, the separation in principle
of the kingdom of God from the kingdom of the world, which he
proposes. The world, he says, is a house of devils which requires
the sword for its government. But the true Christian believer lives
in a divine kingdom, which needs no laws and no compulsion, but is
governed solely by the Word of God. These are hazy and extravagant
ideas which led him to make such declarations as the following:
a Christian must put up with any injustice committed against him
by his fellowmen, and leave it to the authorities to protect him;
for Christ in the Sermon on the Mount taught men to resist evil.
In general, the Sermon on the Mount, with its passages referring to
the blow on the cheek, etc., affords, to the mind of Luther, not
only a guide to perfection; it supplies no mere “evangelical counsels,”
as the papists teach, but real commandments, which are known to
and observed only by those who dwell in the kingdom of Christ,
but not by those who inhabit the kingdom of the world and by the
civil authorities.\footnote
{Grisar, \textit{Luther}, Vol. II, pp. 298 sq. On Luther’s distinction between the world and
the Church see \textit{ibid.}, Vol. V, pp. 55 sqq.}
Here he meets with a dilemma when he invests
the prince with a dual personality which, on the one hand, fundamentally
degrades him to the rank of a beadle and a “jailer” of the
wicked, whilst, on the other, he must satisfy the most exaggerated
religious demands as a Christian believer.

At that time Luther never imagined that he would soon be compelled to regard
the civil rulers as the real protectors and guardians
of religion in their respective territories, whose chief duty was to
ward off the “wolves,” \textit{i.e.}, the “papistical” antagonists of the
Reformation, and to eradicate the “sacrilegious” Mass.
