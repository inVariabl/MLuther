\section{Methods of Propagation}

If one were to investigate the means by which Lutheranism established and
propagated itself, he would first of all discover the cunning
practice of concealment which was indicated towards the close of the
preceding section of this work.

The divine service was essentially altered, but suspicion was avoided
as much as possible by retaining the external form, so that the common
people, as Luther said, “would never become aware of it.” It was
to be done “without scandal.”\footnote{Grisar, \textit{Luther}, Vol. II, pp. 321 sq.}
In justification of this procedure, he
asserted that the new religious practices must be “propagated without injury
to charity.” Thus it ought to be done, he wrote on a subsequent
occasion, “if we are not to unsettle and confuse our churches
without accomplishing anything against the papists.” Melanchthon
shared this view. “The world,” he said, “is so much attached to the
Mass that it seems well-nigh impossible to wrest people from it.”\footnote{\textit{Ib.}, p. 322.}
Hence, Luther, as a matter of principle, more frequently adopts a
calculated accommodation than his impulsiveness would lead one to
expect. When Martin Weier, a young student of good family from
Pomerania, who was about to return home, asked him how he should
behave in the matter of divine worship in the Catholic surroundings
of his home and towards his father, who professed the ancient faith,
Luther, according to his own account, told him “to fast, pray, attend Mass,
and revere the saints, just as he had been doing before,”
but try to enlighten his father as much as possible; he would commit
no wrong if he “took part in the Mass and other profanations for his
father’s sake.”\footnote{\textit{Ib.}, p. 323.}
Yet, when referring to his previous practice of celebrating
Mass in the monastery, Luther declared that he had offended
God most horribly, more so than if he had been “a highwayman or
kept a brothel.”

It was, undoubtedly, a serious matter thus to deceive the people,
who, owing to the retention of the ancient ceremonies, regarded
the service of the reformers as Catholic. Cochlaeus, in a tract,
speaks of Luther’s “hypocritical deception” of the masses.\footnote{\textit{Ib.}, p. 322.}
Luther
was pleased to see that, despite the contrary urging of many an impetuous
fanatic, his principle maintained the upper hand, namely,
that anyone unable to understand the sermon, “be he a layman,
whether Italian or Spaniard, seeing our Mass, choir, organs, bells,
chantries, etc., would surely say that it was a regular papist church,
and that there was no difference, or very little, between it and his
own.”\footnote{\textit{Ib.} Attention is called to the many borrowings from the Catholic cult still found
among the Protestants of Denmark, Norway, and the duchies formerly united with
the Danish crown, as a consequence of the method of concealment introduced in the North
by Luther’s disciple, Bugenhagen.}

The progress of Lutheranism was much aided by the favorable
moral impression which the new movement made upon many, even
well-intentioned Catholics. The reformers’ frank criticism of existing
evils attracted many who longed for an amelioration. The bold
proposals to pursue other paths, brought forward by Luther with an
air of intense zeal, allured the masses and their leaders. People heard
of interior Christianity, which was to be opposed to exterior semblance,
and of a spiritual liberty which was to overcome self-righteousness. A grand
upward flight of virtue appeared to arise from
the deceptive teaching that everything was to be done at the
inspiration of a perfect love, without regard to the motive of
fear or the expectation of reward. Seductive words were heard
calling Christ the sole Saviour, who redeemed mankind without any
merit on its part, and of the unconditioned rule of grace without
human cooperation, which is really nothing but sin. These
and other errors, which were dangled before the eyes of men, influenced many whose
intentions were good, but who did not investigate more deeply. Only
by and bye the true nature of the innovation as a complete religious
revolution revealed itself to those who were able to see more clearly,
especially after the publication of Luther’s “Address to the Nobility”
and his work “On the Babylonian Captivity.”

Of those who afterwards combated the Reformation successfully
with their pens, quite a number at first were prepossessed in favor of
Luther and his proceedings, \textit{e.g.}, Cochlaeus, Zasius, Witzel, Billikan,
Vitus Amerbach, the aged Wimpfeling, and the humanist Willibald
Pirkheimer. Especially among the humanists there were deserving
men who at first favored, or at least maintained an indifferent attitude
towards, the Wittenberg reform movement, such as John Fabri,
who afterwards became bishop of Vienna, and John Faber, the Dominican prior
of Augsburg, who later opposed the movement when
it unmistakably revealed its true nature. Even Erasmus, much as he
had favored Luther’s procedure, joined the ranks of his determined
opponents in 1524. Other intellectuals were misled in joining Luther
by the semblance of reform proclaimed by him; and in the case of
some, their allegiance to him was lasting and sincere, and in some instances
fanatical. One of these literary men thus taken in was
Hartmuth (Hartmann) von Kronberg, a knight who could not go far
enough in his enthusiastic support of the Lutheran cause. Prompted
by “piety,” he petitioned the emperor to treat the pope “as an apostate
and a heretic,” if he refused to renounce his claims. In his religious
enthusiasm Kronberg wished to see all ecclesiastical goods
confiscated, and published a pamphlet in which he outlawed every
Catholic priest who remained loyal to the ancient Church and stated
that it was permissible to treat such “in much the same manner
as one treats a ravening wolf, as spiritual thieves and murderers in
word and deed.” His Protestant biographer styles Kronberg a man of
“unshakable character, though somewhat narrow-minded.”\footnote
{Cfr. Grisar, \textit{Luther}, Vol. II, pp. 325 sqq.}
Such to types were naturally rare, but many showed a steadfast devotion
the cause of Luther, which they regarded as noble.

Of far greater influence upon the masses who joined the Reformation
than the attractive force of the good or seemingly good
features of the movement was the demand for the abolition of oppressive
ecclesiastical burdens. The assertion that the commandments
of the Church were not binding opened the door to apostasy. The
abolition of confession, of the laws of fasting, of the ruling hierarchy,
and the assertion of the dissolubility of matrimony--these and all the
other gifts of the new Evangel to the free Christian was sure to captivate
many. Above all else, the easy doctrine of justification by faith
alone was sure  to meet with a friendly reception.

George Witzel (Wicel), who was a Lutheran for a while, wrote
subsequently: “Oh, what a grand doctrine that was, not to be obliged
to confess any more, nor to pray, nor to fast, nor to make offerings,
nor to give alms. With these you ought surely to have been able to
catch two German lands, not one only with such a bait \dots”\footnote
{\textit{Ibid.}, pp. 313 sq.}

Many clerics who had grown weary of the duties of their state, and
desired to marry, flocked to Luther’s banner. Convents and monasteries
opened their portals and monks and nuns who had selected the
monastic life without a vocation or who, enticed by the pleasures of
the world, had become disgusted with their vows, left the sacred
precincts and doffed their habits. The number of clerics who,
prompted by worldly motives, joined the new religion and came to
Wittenberg to receive appointments as preachers, was so great that
Luther exclaimed: “Who can deliver us from these hordes?” The example
of the married Wittenberg leaders proved exceedingly enticing.
The ranks of the priests who contracted marriage, as mentioned
above, were augmented in 1522 by Dr. Justus Jonas and John Bugenhagen.
The former, who had been forced upon the monastery of All
Saints as provost and who, in his capacity of professor of theology,
proved to be one of the most distinguished assistants of Luther, contracted
a solemn marriage with a woman of Wittenberg and defended his conduct in
a book specifically directed against Faber’s defense
of celibacy. Bugenhagen, a Pomeranian, was likewise one of
the first adherents of the new religion. He had been a priest and
teacher of a convent-school at Treptow, and took up his residence
at Wittenberg, where, in 1523, he was forcibly installed as pastor of
the local church by Luther and the town council. He had married
the previous year.

When Erasmus heard of the growing number of married priests,
he penned the sarcastic words: “Many speak of the Lutheran affair
as a tragedy; to me it appears rather as a comedy, for the movements
always terminate in a wedding.”\footnote
{Letter of March 21, 1528, on the occasion of the marriage of the monk Oecolampadius.
Cf. Theod. Wiedemann, Job. Eck, p. 246.}

In the various departments of public life the word “liberty” produced favorable
results in behalf of the Lutheran apostasy. In many
places where the municipal authorities were engaged in a struggle with
the territorial jurisdictions of the episcopate, joining the new religion
or the threat of apostasy became a slogan in the battle for civil rights.
The temptation to obtain high office and to appropriate the property
of the Church was too strong to be resisted. The desire to obtain relief
from economic pressure through the adoption of the new religion penetrated
even to the peasants. In many places the rural populace,
like the inhabitants of the cities, were affected by the mighty
agitation which was conducted by means of the spoken and the
printed word on behalf of the new ideas. Dissatisfied clerics joined in
the agitation. It was a veritable mass suggestion. The indescribable
power of Luther’s pen, his forceful language, which was carried to
the lowest classes of the population and excited all the instincts of
opposition to the Church, actually induced a sort of popular hypnosis.
The hero, such as every profound popular movement necessarily requires,
was at hand. The evil conditions existing in State and
Church furnished a fateful resonance to his voice. The apocalyptical
expressions which he employed, supported by drastic wood-cuts,
verses and songs, impressed the imagination and powerfully affected
the emotions. The seed he sowed found a soil which was all the more
favorable as it had not as yet been overrun by any other literature or
public agitation, but offered, as it were, a virginal susceptibility.
In addition to all this there was the resort to force—force on
the part of the civil authority for the purpose of introducing the new
system of religion and advocated by the theological leaders to be
exerted upon all who were exposed to their pressure and violence.
The conduct of the civil authorities, especially in the electorate of
Saxony, will be discussed more fully in the sequel. These authorities
imitated the agitators, Anabaptists and others, in the application of
violence.

Luther’s personal method of procedure was marked by forcible
measures in certain places soon after his return from the Wartburg.
In these places, as well as elsewhere, the forbearance and consideration
which he had recommended yielded to violent disciplinary penalties,
when it appeared to be to the advantage of the new Evangel.

He had, it is true, recommended to Gabriel Zwilling, his first
preacher who labored in Altenburg, that he should “liberate the consciences
of men solely by means of the Word.” He said he had promised
his sovereign (prudence had compelled him to do so) that his
adherents would observe this rule.\footnote
{“You must refrain from innovations \dots I gave my word to the prince, etc.” (See
Grisar, \textit{Luther}, Vol. II, pp. 314, 316 sqq.}
But when the loyal Augustinian canons of Altenburg, who had for generations
exercised the uncontested right of appointing the pastor of the local parish, denied the
new preacher permission to take possession of their church, Luther
addressed a letter to the city council, whose members favored him, and
as a consequence the council claimed the right of nominating Zwilling
to the position.

In this letter he maintained that aldermen existed not merely for
the sake of temporal government, but were also obliged “by brotherly Christian
love” to intervene in behalf of the Gospel. For the
rest, everyone had the right to repel ravening wolves, such as the
canons and their disturbing provost, who, ensnared by false doctrines,
unjustly collected the ecclesiastical tithes, and relied on the councils
of the Church, whereas the Scriptures did not empower a council, but
“every individual Christian” to “judge doctrine and to recognize
and avoid the wolves.” The canons were told to “observe silence, or
teach the new Gospel, or depart.”

Violence was resorted to. With the consent of the Elector, Luther’s
friend Wenceslaus Link, of the Augustinian Order, was appointed
to the position in dispute, in place of Zwilling, who was too indiscreet.
This was in the summer of 1522. In February, 1523, Link resigned
his position as vicar-general of the Augustinian congregation, chose
a wife, and was “married” by Luther himself at Altenburg over the
protest of the courageous canons, who, although sorely persecuted,
remained in the city with the other faithful clergy and personnel of
their Order.

In order to sketch the fate of Altenburg during the next few years
--a fate which became typical--it should be mentioned that Link,
in 1524, succeeded in having the municipal council forbid the Franciscans,
who were very much beloved in that city, to celebrate Mass
in public and to preach and hear confessions. At the same time the
municipal council, in a written address to the Elector, declared that,
according to the Old as well as the New Testament, rulers were
not allowed to tolerate “idolatry.” The bailiff was given a free hand. In
August, 1525, Spalatin, after resigning his office of court-preacher
upon the death of the Elector Frederick, took over Link’s position
in Altenburg and married Catherine Heidenreich on November 19.
Luther had preceded him. It was an inevitable consequence of Spalatin’s
marriage that the canons of Altenburg declared his position and
benefices forfeited. Serious conflicts resulted from this action in a
city already torn by religious dissensions. Luther demanded from
John, the new Elector of Saxony, the suppression of Catholic worship at
Altenburg, the “Altenburg idolatry,” as he styled it, among
other offensive invectives. Soon after Spalatin directed to the court a
similar demand couched in no less offensive terms, followed by a
second demand in January, 1526. It must be considered, he wrote,
“that many a poor man would readily embrace the Gospel, if that
miserable idolatry were abolished.” At the most, the canons might
be permitted “to conduct their ceremonies in the greatest secrecy,
behind closed doors, without admitting any other person.” On February 9,
Luther, referring to Altenburg, memorialized the new
Elector, who was very accommodating to him, along the following
lines:

As a ruler he was bound in conscience “to attack the idolaters” and
to suppress “the false, blasphemous cult” as much as possible. Did he
wish to be responsible to God for the criminal abominations by supplying
the foundation with tithes and property, as heretofore? Moreover,
a secular ruler should not tolerate that contradictory preachers
(\textit{i.e.}, such whose teachings were at variance) lead his subjects into
dissension and schism, whence rebellion and mutiny may ensue. “One
and the same doctrine should be preached in the same locality.” “For
this reason,” he added, “Nuremberg has silenced its monks and closed
its monasteries.”

By means of the deceptive principle of “one doctrine for the same
locality” he rests his intolerance upon a plausible foundation which
was satisfactory to the mediocre intelligence of the prince. The subversion
of the whole political order in the electorate, as well as elsewhere, was
thereby accelerated.

The solution of the Altenburg problem was thus made dependent
upon the conscience of the somewhat pietistic ruler. Should the canons,
says Luther, attempt “to apply their own conscience,” “it shall
avail them naught” because they cannot prove their position from
Scripture! And if they should complain that “they are forced embrace
a certain doctrine, that is not true; they are only forbidden to
give public scandal.” They are prevented from practicing a cult which
they themselves “must confess is not founded in Scripture.”

Such was the pitiful justification of that brutal use of force which
became habitual afterwards. But the loyal Catholics at Altenburg
offered determined resistance. When the Lutheran visitors
came to exercise their office, in 1528, the town council informed
them that there were still “many papists in the city,” yea, that the
whole district “fairly swarmed with monks and nuns.” It was an
honorable testimony to Catholic loyalty, the like of which was also
found in other places.

From the introduction of the new Evangel into Eilenburg, in 1522,
we learn what was Luther’s leading idea: “It is the duty of the sovereign,
as a ruler and brother Christian, to drive away the wolves and to
be solicitous for the welfare of his people.”\footnote{\textit{Ibid.}, p. 319.}
On the occasion of
a first and second visit to Eilenburg, Luther had discovered that the
magistrates of that place had failed to show the proper zeal. Like the
authorities of many other places, they were desirous of increasing
their own power and influence; but their Catholic conscience checked
the majority of them. The prince was to remedy this defect by the
exercise of his authority. With the aid of Spalatin, Luther at once
proposed two new preachers for Eilenburg, of whom one was to be
summoned by the town council under the influence of the court,
whilst at the same time the afore-mentioned statement about the
wolves was to be shown to the sovereign. Thus the matter was settled
in a bureaucratic manner with the co-operation of the prince. Andrew
Kauxdorf, a native of Torgau, was finally recognized by the
magistrates as preacher, entered Eilenburg in 1522, and was permitted gradually
to Lutheranize the people who refused to embrace the new religion.

Where the magistrates were unwilling, the powerful nobility, at
Luther’s instigation, frequently used violence to bring about a change.
Thus, to cite but one striking illustration, Count Johann Heinrich
of Schwarzburg became the founder of Lutheranism in his territories
in virtue of a decree authorized by Luther.\footnote{\textit{Ibid.}, p. 318.}
His father, Count Günther, who was loyal to the Church, had legally confirmed the
monks at Schwarzburg in the possession of their parishes; now, Johann
Heinrich asked Luther how he might deprive them of their rights
and possessions in favor of a preacher of the new Evangel. Luther
replied on December 12, 1522, that Count Günther had naturally
expected the monks to preach the Gospel, but if witnesses could
testify that they did not preach the true Gospel (of Luther), but
papistical heresies, the Count would have the right, nay, the duty, to
oust them from their parishes. “For it is not unlawful,” he says, “indeed,
it is absolutely right to drive the wolf from the sheepfold \dots A
preacher is not given property and tithes in order that he
should do injury, but that he should labor profitably. If he does not
work to the advantage of the people, the endowments are his no
longer.” This principle was promptly applied at Schwarzburg. The
Count seized the properties and revoked the privileges which his
father had given to the Church. Monks and parishes were subjected
to violence, and the new Evangel was introduced.

Luther’s reply concerning temporal possessions, taken in connection
with certain other statements made by him, reveals an idea truly
revolutionary in its consequences. It indicated that, if the clergy refused
to preach the new religion, in Germany and in the Church in
general, ecclesiastical possessions were no longer secure. Lutheranism
needed but to apply this principle, which, undoubtedly, it was
strongly tempted to do. If only those priests, abbots, bishops, and
other spiritual rulers were to continue in the possession of their benefices
who used them to promote the Lutheran innovation, then the
foundations of order were overturned. Wyclif and Hus had proclaimed similar
doctrines, and the Christian State had been able to defend its legal structure
against them only by taxing its energies to the utmost. It is hardly probable
that Luther realized in advance all the
consequences of his decision in the Schwarzburg affair, though practically
it had been acted upon ever since the beginning of the new
movement. Only prudent regard for the electoral court prevented
the rigorous carrying out of this decision.
