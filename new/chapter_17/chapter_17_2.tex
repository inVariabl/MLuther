\section{Abnormal Psychological Traits}

The first of the abnormal traits of Luther’s psychology was his fear
of the devil.

Numerous facts and utterances hitherto quoted attest how consciously and
persistently he moved in the imaginary sphere of the
power of darkness.

Fearfully he ever beholds this power about him. As he progresses
in age, the whole world becomes more and more the “kingdom of the
devil,” as he styles it in 1544. The devil governs it and devastates it by
his tools, “the Turk, the Jew, the pope”; but, he maintains, “he shall
be under Christ to all eternity.”\footnote{\textit{Ibid.}, V, 275.}

Before Luther’s time popular belief in the devil had assumed an exaggerated
form. It had become firmly established, especially among the
mining folk, from whom Luther descended, in consequence of the
uncanny labors which they performed within the bowels of the
earth. Even learned men were infected by this superstition. Had the
sober doctrine of Holy Writ and the Church been adhered to, the
faithful would have been preserved from many aberrations in this
sphere. Luther magnified and coarsened the maniacal ideas which his
parental home and the tendency of his age had implanted in his mind.
He confirmed this belief among the Germans, whereas, had he wished
to be a true reformer of ecclesiastical life, he would have found a
very useful field for his activity in correcting them and the other
moral diseases of his age. Up to his time, however, no one had delineated
the devil and his works with so much detail, no one had addressed
the people on the subject with such a weight of personality
and such urgent and apparently religious instructions as Luther.

In his Large Catechism, where he speaks of the damage wrought by the
devil, he tells how Satan “breaks many a man’s neck, drives others out of
their mind or drowns them in the water”; how he “stirs up strife and brings
murder, sedition, and war; also causes hail and tempests, destroying the
corn and the cattle, and poisoning the air,” etc.\footnote{\textit{Ibid.}, p. 278. The following citations \textit{ibid.}, pp. 279, 281, 284.}
 In a similar vein he writes
in his Home-Postil and in his Church-Postil, where he attributes to Satan
every evil with which mankind is afflicted. His own experiences are frequently
reflected in his excited ebullitions, especially when he touches upon
the demoniac fury with which his faith had been assailed. In connection
with the latter he asserts that the devil is more at home in Holy Scripture
than all the universities and scholars. “Whosoever attempts to dispute with
him,” he says, “will assuredly be pitched on the ash heap.” He is convinced
that madness is in every case due to the devil, who always employs mad men
as his instruments. How many people, he says, has the devil seized bodily,
especially when they had caused him trouble in virtue of a pact! According
to his own confession, Luther often was frightened by demons. There are
less of them at large now because they have entered into his enemies, the
heretics, Anabaptists, and fanatics. Luther has the faculty of furnishing
a fertile account of the various species of devils, their habitations, and
the forms which they assume.

He is most prolix, however, in his statements and reflections on the
demons’ attitude toward himself. It is he whom the evil powers have
selected to make war upon, because of his great mission. This mania
throws him into a state of misery and mental darkness and he imagines
“nightly encounters” in which he is compelled to contend with
the enemy of the Gospel; hence, all the supposed obstacles to the rapid
propagation of his religion among the papists; hence, all those remarkable
phantoms which hold his mind in their powerful spell.

When, towards the end of his life, in 1546, he was compelled to
journey to Eisleben, he writes that an assembly of devils congregated
in that town in order to prevent the establishment of peace which
he contemplated.\footnote{\textit{Ibid.}, p. 297.}
 Indeed, he believed that he actually saw the devil,
even more plainly than at the time he sojourned in the castle of
Coburg.\footnote{\textit{Op. cit.}, Vol. VI, p. 130.}
 Thus, one evening, as he stood praying at the window of
his domicile, he saw the devil perched on a nearby street fountain,
jeering at him, thus insinuating that he was his sworn enemy. He
hastened to his friend, Michael Coelius, the court preacher of Mansfeld,
who lived in the house, and amid tears apprised him of what he
had seen. Coelius narrates the incident in his funeral oration on
Luther. His physician, Ratzeberger, adds that Luther informed Coelius
and Jonas that the devil had showed him his posterior and told him
that he would achieve nothing in Eisleben.

The whole affair, naturally, was purely a hallucination. It is
reminiscent of the hallucinations which Luther had at other times, for
example, at the Wartburg and the Coburg, as a result of his excited
mental condition.

Once, at Wittenberg, he descried the devil in the garden beneath
his window in the shape of a huge wild boar, which disappeared when
he boldly jeered at it.\textit{Ibid.}, pp. 131 sq.
 On the other hand, the disputation with the
devil on the Mass, as was shown above, was but a fiction of his pen.
It may be questioned whether the frequent nightly attacks launched
by Satan always occurred without hallucinations. According to his
own statement, a vulgar “\textit{Leck mich \dots}” often banished them. He
seriously states in his Home-Postil: “The devil is always about us in
disguise, as I myself have witnessed, taking, \textit{e.g.}, the form of a hog, of
a burning wisp of straw, and such like.”\textit{Ibid.}, VI, 132.


Luther was also much annoyed by witches, whom he calls “harlots of the
devil.” “I wanted to burn them myself,” he says, “according to the usage
of the law, where the priests commenced to stone the
guilty ones.”\textit{Tischreden}, Weimar ed., Vol. IV, n. 3979.
 On another occasion he declares against the female
corrupters of milk, that “the method of Doctor Pommer is the best”--
a reminiscence of the latter’s treatment of the butter-vat. Luther’s
declarations against witches, as contained in his Table Talks, became
universally known and together with his writings, which are replete
with demoniac thoughts, fatefully contributed to the bloody persecution
of witches.\footnote
{N. Paulus, \textit{Hexenwahn und Hexenprozess, vornebmlich im 16. Jabrbundert} (1910),
pp. 488 sqq.}
Luther was very much inclined to assume that persons subject to

violent attacks of hysteria were possessed by the devil. However, he
did not wish to resort to the customary ecclesiastical exorcism with
its commands addressed to the devil, but restricted himself to prayer.
“God knows the time when the devil must depart.” The success of
prayer, however, usually appeared rather dubious.

In January, 1546, he experienced a peculiar encounter with the supposed
devil in the sacristy of the parish church of Wittenberg. In the presence of
several doctors, ecclesiastics, and students, the devil was to be driven out of
a girl of eighteen, a native of Ossitz, near Meissen, by the prayers of Luther
and his attendants. It is evident from the accounts of two participants that
the girl was in a highly hysterical condition. These two witnesses are
Frederick Staphylus, a future convert to Catholicism, and Sebastian Froschel,
Luther’s deacon. When, after the recitation of a somewhat lengthy prayer,
Luther noticed no sign of the devil’s departure, he applied his foot to the
patient to signify his disdain for the devil. The poor creature whom he had
thus insulted, followed him with threatening looks and gestures. The door
could not be opened, as it had been bolted, and the key mislaid. Since the
window, which was bolted with iron bars, did not permit of escape, Luther,
the devil’s greatest and best-hated foe on earth, says Staphylus, “ran about
hither and thither, seized with fright,” constantly pursued by the infuriated
girl, and writhed and deported himself like a person in despair. Finally the
sacristan passed in a hatchet, with which Staphylus burst open the door, and
thus liberated Luther from his desperate plight. The pious Froschel says that
afterwards reports came to Wittenberg to the effect that the evil spirit no
longer tormented the girl, as formerly.\footnote{Grisar, \textit{Luther}, Vol. VI, pp. 137 sq.}

Luther’s expectation of the end of the world is another dark trait
which pervades his life and assumes a more vivid hue in his later
years. He makes various estimates as to when the end may be expected.\footnote{\textit{Op. cit.}, Vol. V, pp. 242, 248.}

On one occasion he says it will come in fifty years. Then
again he predicts that the catastrophe will have happened by 1548.
“We shall yet experience the fulfillment of the Scriptures.” The idea
circulated widely. His eccentric pupil and friend, Michael Stiefel,
despite Luther’s opposition, anticipated the date by assigning the year
1533, the eighteenth of October, at exactly eight o’clock in the morning,
as the date of the world’s end. At this hour he assembled his trembling
parishioners in church, and, as nothing happened, was severely
censured by Luther.

The end of the world was frequently predicted by other, even by
great, men. In Luther’s case, however, the expectation is accompanied
by unusual agitation and morbid symptoms. It springs from the idea
of his vocation and success. He revealed Antichrist in the papacy, and
this revelation, according to the Bible, was to be followed immediately
by the advent of the Great Judge. This theory he sets forth in
detail and with the greatest seriousness in his tract against Catharinus,
appealing to the celebrated passage of St. Paul’s Epistle to the Thessalonians
and the misunderstood prophecy of Daniel.

There is something visionary about his proclamations concerning
the end of the world, and for this reason they deserve closer con:
sideration.\footnote
{For proofs of the following \textit{ibid.}, pp. 242 sqq. Gf. the passages
\textit{Tischreden}, Weimar ed., index to Vol. VI, 5. v. “Tag, jüngster.”}

Many signs in nature, human society, and the empire of Satan, he
held, announced the end of the world, as did also the ever increasing
brutalization of the masses and the upper classes. In his fantastic interpretation
of the monk-calf, he adduces the horrors of the papacy
in corroboration of his prophecy. In old age he says of himself: “Let
the Lord call me hence, I have committed, seen, and suffered sufficient
of evil.” At that time he was able to find even a certain consolation
in reflecting on the end of the world, and to speak of the “dear
Judgment Day,” which was to liberate him from the difhculties which
surrounded his work, and from the struggles within his own soul. The
great successes which he finally experiences, appear to him to signify
the last flaring up of the light. “The light is approaching extinction;
it still makes a mighty effort, but thereafter it will be extinguished in
a twinkle.” Oppressive dreams of the impending judgment visit him.
He overcomes the impression by hoping vigorously and longing for his
departure from this vale of tears. Thus a mixture of dread and consolation,
of fear and satisfaction is prevalent in his expectation of the
end of the world.

So sure is he in his calculations that, in view of the brevity of time
still allotted, he does not concern himself particularly about the discipline
of the Church in the future, \textit{e.g.}, about the institution and
order of public worship.\footnote{Cfr. Köstlin-Kawerau, \textit{M. Luther}, Vol. II, p. 522.}

The spiritual exhaustion which overwhelmed him towards the end
of his life had a share in this indifference. It also belongs to the dark
side of his spiritual life. Luther exclaims in his apathy: “Let everything
collapse, stand, perish, as it may. Let matters take their course
as they are, since, after all, matters will not change\dots Germany
has had its day and will never again be what it once was.” Thus he
writes in 1542.\footnote
{In a letter to the preacher Probst at Bremen, on March 26, 1542. \textit{Briefwechsel}, XIV,
p. 218; Grisar, \textit{Luther}, Vol. V, p. 226; the following passages, \textit{ibid.}, pp. 230 sq., 246
sqq.}
He is “tired of this life,” he writes in the same year;
“all thoughts concerning plans and rendering aid begone from me!
All is vanity.” And again in 1543: “We will let things take their
course as they may.” He regards his words: “Let happen what may”
as inspired by Christ, who will seize the reins Himself. The ailments,
too, which he suffered, contributed to this despondency. He was
afflicted alternately by oppression of the heart and by violent and
whirling sensations in the head, accompanied by a ringing in the
ears, and by sufferings caused by gallstone and other maladies.

“Distemper, melancholy, and severe afflictions,” says Ratzeberger,
oppressed him. This physician believes that the mental sufferings
which Luther sustained contributed to his death.\footnote{Grisar, \textit{op. cit.}}
 Luther is profoundly
grieved at his inability “to proceed effectively” against the
papists, “so great is the immensity of the papistic monster.”\footnote{\textit{Ibid.}, IV, 344.}

He sees the advent of the Tridentine Council and he execrates and curses
it.\footnote{\textit{Ibid.}}

With avidity he gives credence to the fable that the Emperor and
the Pope had despatched ambassadors to the Grand Turk with gifts
and an offer of peace, ready to prostrate themselves before the infidel
ruler in long Turkish garments. He says that this is “a token of
the coming of the end of all things.”\footnote{\textit{Ibid.}}

Indeed, in the disturbed
state of his mind he believed the most incredible things. He felt that
he was repeatedly saved from the danger of being poisoned by the
papists, who pursued him with deadly intent. He preached in poisoned
pulpits without injury to himself. Witches endeavored in vain to
destroy him and his family. Hired incendiaries convulsed the districts
which adhered to him; but the devil raved in vain.

Dissatisfaction with Wittenberg finally induced him to abandon
that city forever. Although he had bidden adieu to his Wittenberg
hearers on a former occasion, and although he wanted to depart from
the ungrateful city in the beginning of 1544, he did not carry out
his plan until the end of July, 1545. After careful preparation\footnote{\textit{Ibid.}, pp. 341 sqq.}
 he
betakes himself to Zeitz, whence he addresses a letter to his wife, declaring
that he will never come back and requesting her to return to
the estate of her family at Zulsdorf and to restore the Black Monastery
to the Elector. He says he is resolved to beg for his bread in his old
age. From Zeitz he repairs to Merseburg, where, on August 2, he confers
upon the canon of the cathedral chapter, George von Anhalt, a
so-called ordination as bishop of the diocese that had been abolished
by Duke August of Saxony. From Merseburg he proceeds to Leipsic,
where he preaches on the twelfth of August. He was prevailed upon
to return to Wittenberg only after most strenuous efforts put forth
by Melanchthon and Bugenhagen, who had been appointed emissaries
of the city, the University, and the Elector.

After his return he felt better for a while. Owing to the devotion
shown him by his followers, his mental depression yielded to a lively
spirit of enterprise. It was a sudden transition, such as not infrequently
occurred in his interior life, an idiosyncrasy of psychopathic sufferers.\footnote
{John Joseph Mangan, in \textit{Life, Character, and Influence of Desiderius Erasmus of
Rotterdam} (1927), at pp. 87-88 of volume two, makes the following observations on
Luther, which confirm Father Grisar’s position. He says: “As our study of Erasmus has
led us to decide definitely that he was a neurasthenic, so our study of Luther has con-
vinced us that he was a psychopath, if not always, then most assuredly at intervals.”--
The author advances a number of proofs in substantiation of his assertion,
very interesting, as some of them are based upon medical observations. (Tr.)}
The enemies were made to sense the old Luther in the new
polemical literature which he produced. But melancholia once more
returned, though apparently not accompanied by his former temptations.

Because of his suspicion of the teaching of others, association with
him is described as intolerable, since he always suspected deviations
from his own doctrinal position and would brook no differences of
opinion. Melanchthon, who held different opinions on various points,
\textit{e.g.}, on the Eucharist, complained bitterly and wrote afterwards that
he was forced to put up with “an ignominious servility” in his association
with Luther. He compares himself with the unfortunate
Prometheus chained to the rock and describes Luther as the demagogue
Cleon and the impetuous Hercules.\footnote{Grisar, \textit{Luther}, Vol. V, pp. 252 sqq.}
 Forced to linger, as it were, in
the cave of Cyclops, and not feeling secure against the secret wrath
of Luther, he also desired to leave Wittenberg and announced his
readiness “to slink away”--such is his expression.\footnote{\textit{Ibid.}, VI, 347.}

One particularly prominent trait in the spiritual life of Luther is
his extraordinary capacity for self-delusion. The inward necessity of
continually justifying anew to the world, and no less to himself, his
pretended calling, the overwhelming ambition of belittling his
antagonists and augmenting the number of his own followers and,
finally, his inevitable and constant perplexities, resulted in most curious
expressions of self-delusion, which sometimes contradicted the
views he entertained at other times.

Thus the moral corruption developing under the new evangel on occasion
would appear terrible to him only because the gospel which he preached was
pure and holy, and light intensifies the shadows. As there is no light in the
papacy, he contends, its horrible evils and vices are not so noticeable. He ascribes
the corruption of his followers to the devil, who would discredit the
evangel, but on other occasions admits that it was caused by his solsa fides
doctrine which implied the inefficacy of good works.\footnote
{Cfr. Grisar, \textit{Luther}, Vol. IV, pp. 210 sqq.; Vol. VI, pp. 331 sq.}

A few examples will show how arbitrarily his mental processes contravened
the ordinary laws of experience and the rules of logic, in order to
conform with the idea of his vocation, which dominated him with morbid
force.

God continuously performs miracles in confirmation of his doctrine. In
response to the prayers of the Lutherans, He destroys the enemies of His
teaching. “By my prayers I have brought about the death of Duke George
of Saxony; by means of our prayers we intend to encompass also the death of
others.” With the greatest apparent naiveté he finds his doctrines confirmed
by the ancient doctors of the Church, such as St. Augustine, whereas in
matter of fact they state the exact contrary. He cannot comprehend why
the whole world does not agree with him and says that malice alone prevents
the papists from accepting what is so evidently right; or, rather, they
accept it in secret, as the pope and the Roman curia do, but they do not
wish to honor him and the truth. The most horrible infidelity is rampant
under the papacy. The papists do not heed “that God incessantly attacks
them with many wounds, plagues, and signs.” Has He not confirmed the
Lutheran religion by the sudden death of Zwingli?

I am in duty bound to proclaim to the world what I “feel inwardly
through the spirit of God.” Indeed, I permit myself to be guided by God
``as the wind and the waves propel the ship.'' Opposition and the din of
battle are but the seal of divine approbation on my work. In the Peasants’
War he boldly asserted that God commanded him to proclaim that the peas-
ants were to be slain like dogs. In a spirit of defiance he afterwards vouches,
with regard to his whole doctrinal system, that he will abide “by the first
mandate of the vocation received from on high,” by the “firstfruits of the
received spirit,” even “if God or Christ should announce the contrary”(!)
He may not yield, since, in his own imagination, his victory over the papacy
is already assured. “This majesty has fallen,” “it has been destroyed” by the
``spirit of the mouth of the Lord.'' “The Church will be without a pope,”
since he will fall just as the Turk. “Even now he is singing Eli, Eli,” because
he has been struck. Soon men will say, Expiravit--"he has breathed his
last.” “My adversaries will have neither the Word nor the Cross.” They all
avoid the Cross.

“Here I, a poor monk, and a poor nun must come to the rescue.
We two comprehend the article of the Cross and raise it up; for this
reason the Word and the Cross are sufficient; they give us assurance.”

Thus the series of illusions outlined above is concluded with the
remarkable tableau showing how Luther and his Kate raise up the
the Cross in the sight of the world--they who had bound themselves
by a solemn vow before the altar of God to embrace His Cross in a
life of voluntary chastity, poverty, and obedience. How different the
reality! To what extent the Cross disappears in Luther’s conception
of the married state and sexual life will be seen in a subsequent
section.
