\section{From Freedom to Violent Intolerance}

The reversion of his early attitude on religious liberty and spiritual
independence also belongs to the remarkable features of Luther’s later
life. This contradiction must be emphasized because rationalist Protestants
of the present time like ta quote Luther’s earlier utterances,
which seem to imply the destruction of all positive Christian belief.\footnote{\textit{Tischreden}, Weimar ed., Vol. V, n. 5515,}

It is well known that Luther by no means intended to substitute
rationalism for revealed Christianity. Nevertheless, in the first years of
his polemics he advanced to such an extent in his antagonism to Catholic
doctrine as apparently to antagonize all dogma. Relative to practice,
he at that time permitted the obligation and dignity of the law
to recede in an unwarranted manner. It is known that the liberty of
opinion which emerged in consequence, and the loose morals of his
adherents, as disclosed especially by the visitations of 1527--1528, led
him to employ greater caution. The law and the fear of divine
chastisement were inculcated by him more strictly, and belief in the
revealed doctrines which he regarded as certain was emphasized with
great severity. The controversy with Zwingli caused Luther to draw
up a formal series of articles of faith. The Augsburg Confession furnished
a certain basis, which was afterwards expanded and sanctioned
by the publication of the so-called Schmalkaldic Articles. The controversy
aroused by the Antinomianism of Agricola once more restored
the law to its legitimate position and set new limits to liberty
of teaching. Present-day Protestant theologians speak of a restriction
of teaching and doctrinal torpidity on the part of the aging Luther.
Many would prefer to progress along the path which he blazed in
his early career as a reformer, holding that a healthy evolution can
only take place where there is complete freedom, and lamenting that
Luther’s promising spring was succeeded by no summer.\footnote
{Grisar, \textit{op. cit.}, Vol. VI, pp. 237 sqq.}

It is not true, however, that Luther’s declarations in favor of liberty
ceased altogether in his later years, His principle of the untrammeled
formation of a personal religious conviction on the part of
each individual in accordance with his private interpretation of the
Bible, was never abandoned. After he had rejected the teaching
authority of the Church, instituted by God for the preservation of
dogma, there was really nothing left for him to do but to uphold
an unlimited subjectivism. He contradicts himself, therefore, when he
demands unconditional adherence to an objective sum of truths. All
the more so, as he boldly sets up his own personal opinions in place of
the venerable authority of the Church. True, he tried to conceal this
substitution as much as possible. He assures his adherents that they
must follow Christ, not him, but at the same time proceeds, in part
unconsciously, on the assumption that his mandates must be obeyed.
But, it may be asked, where is the necessary guarantee of truth if his
judgment, hence the view of one person, is to be depended upon as to
what constitutes the teaching of Christ, and what part of the ancient
teaching and revelation is to be retained in the prescribed articles of
the new evangel?

The reversions to a more positive attitude, occasionally noticed
in Luther, were, therefore, fundamentally only a more brutal emphasis
of the Wittenberg doctrine, which was his own. For him the change
was a necessity; and even though it appears as grossly inconsistent, it
nevertheless redounded to the future benefit of Protestantism by
assisting it in carrying on during the coming centuries, until the
skepticism of a new era and the advent of the so-called historical
method and the modern spirit of independence shook Lutheranism
with destructive force.

In his tribunal at Wittenberg Luther strove to protect his truth
by all means. He hurled the most violent invectives not only at the
papists, but also at the “heretics” in his own camp, who dared to
deviate from his theological teaching. He assails not only the Zwinglians,
but also Karlstadt, Bucer, Capito, Grickel (Agricola) and Jeckel
(Jacob Schenck) among many others.

“They are knaves,” thus he inveighs against them. “They would readily
assail and surprise us, just as if we were blind and ignorant of their
methods.”\footnote
{\textit{Tischreden}, Weimar ed., Vol. III, n. 2896b. The other passages cited in small print
will be found \textit{apnd} Grisar, \textit{op. cit.}, Vol. VI, pp. 279 sqq.}
“By God’s grace I am more learned than all the sophists and
theologians of the schools.” But “they have a high opinion of themselves,
which, indeed, is the cause and well-spring of all heresies, for, as St. Augustine
says, ‘Ambition is the mother of all heresies.” “It all comes from obstinacy
and conceit and the ideas of natural reason, which puffs itself up.”

Whilst in this frame of mind, it appears to have been impossible for
Luther to realize that he actually condemned himself by these declarations
uttered at various times.

The heretics--he says on another occasion--cannot be sure of their cause.
First of all, they ought “to be certain of their mission.” “One ought to be
certain of it before God, whilst by all means one ought to be able to say
before the people: ‘If anyone knows better, let him say so; I will gladly
yield to God’s Word when I am better instructed.’”

He asks them: Where are your miracles? With 2 boldness that is truly
extraordinary he demands miraculous signs from the sectaries. He himself
needs none in attestation of his teaching; for his mission is an ordinary one,
whereas their pretended mission is extraordinary. Were he to petition God,
he says, God would endow him with “the gift of raising the dead, or of
performing other miracles.” However, he does not ask God for these gifts,
since “the rich gift of interpreting the Scriptures is sufficient” for him.

In his reaction against the doctrinal liberty which he himself inaugurated
Luther goes so far as to advocate compulsory measures against
those who differ with him. In 1525 he had enunciated the principle:
“The [secular] authorities are not to hinder anyone from teaching
and believing what he pleases.”\footnote{This and the following quotations \textit{ibid.}, VI, pp. 248 sqq.}
In 1530--originally induced thereto
by the agitation of the Anabaptists--he demands the exercise of force
and bloody repression on the part of his Elector and the other
Protestant rulers. Not only are those Anabaptists who rebel against
the authority of the State to be dealt with harshly, but also those who
are not in rebellion. “These also are not to be tolerated, but are to
be treated as public blasphemers.” To deviate from his teaching is
equivalent to “public blasphemy” and deserving of death. “The authorities
shall hand over knaves of that ilk to their proper master, to
wit, Master Hans” (\textit{i.e.}, the hangman). The Sacramentarians and the
papists, too, being blasphemers, must not be tolerated.

At the end of October, 1531, Melanchthon, who was known for his
humane disposition, developed in detail the reasons for employing the
sword against the Anabaptists. These reasons apply to all who “reject
the office of public preaching and teach that men can become holy
in some other manner, without sermons and ecclesiastical worship.”
Luther subscribed his name to these reasons, saying, “It pleases me.”
In his sermons on St. Matthew, which were delivered about this time,
he says: “It is not allowed to everyone to excogitate his own ideas,
formulate his own doctrine, permit himself to be called Master and
dominate or censure anyone else”; “it is one of the greatest and most
injurious vices on earth, whence all factious spirits originate.”

Thus sectaries, especially Anabaptists, were executed in Electoral
Saxony, the rulers appealing to the Wittenberg theologians and jurists
in justification of their procedure.\footnote{\textit{Ibid.}, pp. 254 sq.}
 Luther never revoked his intolerant
views; on the contrary, he constantly intensified them as he
approached the end of his life.\footnote
{\textit{Ibid.}, pp. 256 sqq. Cfr. N. Paulus, \textit{Protestantismus und Toleranz im 16. Jahrbundert}
(1911), Ch. I.}

Certain formidable barriers which had been erected by him and
Melanchthon at the faculty of Wittenberg, were intended to safeguard this
supreme tribunal and citadel of pure orthodoxy against
the incursions of “heretical” opinions. The statutes of the theological
faculty, which were probably drawn up in 1533, invested that body
with the right of deciding all matters of faith. The proper observance
of this provision was guaranteed by the fact that Luther presided
over the faculty uninterruptedly from 1535 until his death.\footnote
{Grisar, \textit{Luther}, Vol. VI, pp. 262 sq.}

Moreover, after 1535, there was prescribed, at the instigation of the Elector,
an “Ordination Oath,” preceded by a theological examination,
for all preachers and pastors sent out by the University. In the certificate
cate of ordination of Heinrich Bock (dated May 17, 1540, and signed
by Luther, Bugenhagen, Jonas, and Melanchthon) it is set forth that
Bock had undertaken to “preach to the people steadfastly and faithfully
the pure doctrine of the gospel which our Church confesses.”
It is also stated that he adheres to the “consensus” of the “Catholic
Church of Christ.”
\footnote{\textit{Ibid.}, p. 265.}
“Catholicity” here is understood in a sense
which does not ordinarily attach to the word. “Ordination” merely
consisted in the declaration that the candidates were authorized to
serve as ministers.

Naturally many opponents within his own camp reproached Luther
with lack of liberty in the exercise of the ministry. They charged,
and not without justification, that the Wittenbergers proposed “to
breathe new life into despotism, to seat themselves in the chair, and
to exercise jurisdiction just as the pope had done heretofore.”
\footnote{\textit{Ibid.}, p. 315.}
Luther was dubbed “the Pope of Wittenberg” (\textit{Papa Albiacus}), an
epithet which became increasingly popular when his talented and
scholarly opponent, Sebastian Franck, whose writings enjoyed a wide
circulation, developed Luther’s subjectivism to its logical conclusions
and combated the Lutheran ecclesiastical system, demanding unrestricted
liberty. This intrepid challenger was everywhere pursued by
verdicts and demands for execution on the part of Luther and
Melanchthon.\footnote{\textit{Ibid.}, pp. 266 sqq.}
Simon Lemnius (Lemchen), a Protestant humanist of
Wittenberg, was another public opponent of Luther’s theological
despotism. Banished from Wittenberg in 1538, he avenged himself by
the publication of 2 caustic “Apology,” the complete text of which
became known to historians but recently. In it he unmercifully
castigates the spiritual tyranny exercised by Luther. “He sits like a
dictator at Wittenberg and rules”; thus the “Apology,” “and what
he says must be taken as law.”\footnote{\textit{Ibid.}, p. 288.}
However, it must be noted that
Lemnius, because of other attacks upon the conduct of Luther’s circle,
did not bequeath to posterity the reputation of a respectable
controversialist, his attacks being very frivolous
in diction and content and also untrue. He composed a revolting poem in which he
depicts Luther as suffering from dysentery. Luther retorted with a
“Merd-Song” of his own, in which he paid his respects to Lemnius in
language that was no less vulgar than his opponent’s.\footnote
{\textit{Tischreden}, Weimar ed., Vol. IV, n. 4032. Cfr. Grisar, \textit{Luther}, Vol. VI, p. 288.}

In spite of such attacks Luther maintained his tribunal at Wittenberg.
“Whosoever shall despise the Wittenberg School,” he declared,
“is a heretic and an evil man; for in this school God has revealed His
Word.”\footnote{\textit{Tischreden, ibid.}, n. 5126; Grisar, \textit{op. cit.}, Vol. V, p. 170.}
And he adhered to this solemn pronouncement. In 1542 he
went so far as to demand that the leading citizens of Meissen, who
had embraced his doctrine, should “signify their approval of everything
which has hitherto been done by us and shall be done in the
future.”
\footnote{Grisar, \textit{ibid.}, VI, p. 279.}

Luther’s intolerance also animated his co-workers, especially Melanchthon.
\footnote{\textit{Ibid.}, p. 265.}
A. Hänel in the \textit{Zeitschrift für Rechtsgeschichte} passes
the following judgment upon the latter: As far as Protestantism is
concerned, “liberty of belief” was “denied at every point.” In fervent
words Melanchthon sanctioned the execution of the “heretic” Michael
Servetus by Calvin in 1554 as “a pious and memorable example for
all posterity.”\footnote{Grisar, \textit{op. cit.}, VI, pp. 266 sqq.}
It has been previously noted that he wished God
would send a “bold assassin” to “dispatch” the heretical King Henry
VIII of England.\footnote{Grisar, \textit{op. cit.}, VI, pp. 269 sqq.}
Martin Bucer, to mention but one more of Luther’s
associates, asserted that the civil authority “was obliged to abolish
false doctrine and perverse public worship,” and that all the bishops
and the clergy must obey it as the sole existing authority.\footnote
{\textit{Ibid.}, Vol. IV, p. 12. Respecting Melanchthon, Bucer, etc., see the proofs in Paulus,
\textit{Toleranz.}}
The new religion was, as a matter of course, capable of enforcing
such demands only by availing itself of its intimate connection with
the secular authority. By surrendering the religious discipline to the
civil government, the Protestant Church became a compulsory State
institution, a nursery of despotic encroachments upon the spiritual
domain. Luther himself says of it: “Satan is still Satan; under the
papacy he pushed the Church into the world sphere and now, in our
day, he seeks to bring the State system into the Church.”\footnote{Grisar, \textit{op. cit.}, Vol. VI, p. 320.}

The Church, whose invisibility and purely spiritual power Luther
had hitherto so forcibly emphasized, in his hands became a visible institution,
which asserted its visibility all too strongly, and became
accustomed to marshal all temporal forces and to insure its preservation
with the aid of the secular arm. The tragic fate of Luther’s theory
of the Church has led many a Protestant scholar to assert, quite truly,
that there is no room for a church in Luther’s system. It is even
doubted whether he intended to found a church in any true sense of
the word.\footnote{\textit{Ibid.}, p. 307, quoting Martin Rade.}
Protestants have frankly exposed the inherent contradiction
between his pronouncements on religious authority and the duty
of secular rulers and his persistent assertion of individual liberty and
the claims of his gospel.\footnote
{\textit{Ibid.}, pp. 321 sq. Cfr. Th. Pauls, \textit{Luthers Auffassung von Staat und Volk}, Bonn, 1925,
a book which unjustifiably credits Luther with too much systematization.}

Luther’s retreat from the position which he had originally assumed
merely contributed to a clearer disclosure of the contradictions inherent
in the principal ideas of his system. The intrinsic contradictions
are especially manifest in the sphere of morality. For this reason,
we will devote a special chapter to the ethical aspects of Luther’s
doctrine and practice.
