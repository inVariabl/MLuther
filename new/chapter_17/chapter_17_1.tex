\section{Duration and Waning of Temptations}

Luther’s vigorous, nay, coarse language was not infrequently intended
by him to drown the interior scruples about his conduct and
teaching.

He called his phobias “temptations” (Anfechtungen). From 1527
to 1528, a particularly stormy period in his life, they attained to
an extraordinary intensity. After a brief calm, during his sojourn in
the castle of Coburg, the gloomy spells returned. He tells us that at
that time affliction and sadness of spirit seized him to such an extent
as to produce a contraction of the heart.\footnote
{Cfr. Grisar, \textit{Luther}, Vol. V, pp. 346 sqq., for the passages quoted in the text.}
Thereupon the struggle
gradually abated. Of his serious illness during the diet of Schmalkalden,
1537, he says: “I would have died in Christ, without any
temptations and very composedly.” Recollecting this same affliction
in 1540, he said to his friends: “At the close of life, all temptations
cease; for then the Holy Spirit abides with the faithful believer, forcibly
restrains the devil, and pours perfect rest and certainty into the
heart.”

He ascribed the acquisition of his strong faith to the terrible storms
he had experienced. Heretics, on the other hand--so he assures us--
were devoid of strong faith, even if they died for it; they possessed
only obstinacy, inspired by the devil.

For two weeks he experienced a “spiritual malady,” as he styles it in
1537, during which he practically lived without food and sleep. He
consoled himself, however, by having recourse to the Apostle Paul,
who was also “unable to comprehend” what was proper. When such
“spiritual temptations present themselves,” he says, “and when I
add: ‘cursed be the day on which I was born,’ then there is trouble.”

Again, in 1537, when exhausted in consequence of overwork and
indisposition, he protests that he was willing to die; for now he was
“exceedingly happy and peaceful of heart.”\footnote{\textit{Tischreden}, Weimar ed., Vol. IV, n. 4777.}
Nevertheless, the fears
of the spiritual struggles at once re-arose in his mind--a mental
state in which man regards God as his enemy and feels himself as if
pierced by a lance. Then “one does not know whether God is the
devil or the devil God.”\footnote{Grisar, \textit{Luther}, Vol. V, p. 352.}

On October 7, 1538, he again says that he is in the throes of a
mortal agony. In this same year he complains that the devil “accused
him severely before God.” But soon after he declares that while he is
assailed by doubts and stumbles at times, this is often the case with
Christians; “even though I stumble, yet I am resolved to stand by
what I have taught.” Only fanatics, he thinks, never stumble, “they
stand firm.”

In order to overcome his phobias, Luther makes use of the most
diverse remedies, some of which have already been referred to in
this book. “Quickly inveigh against the papists,” was his slogan,
especially during temptations concerning the doctrine of justification.
He excites in himself a “bold anger” or some other distracting
emotion. According to his own confession, he also sought a remedy in
sexual enjoyment with his Kate and in jovial intercourse with his
friends. “A strong drink of beer,” according to his directions, is apt
to be of benefit when melancholy thoughts afflict one. Of course,
devotional and theological helps are also to be applied. “When I seize
the Bible, I have triumphed.” Only the correct text did not always
suggest itself to him or excitement rendered it lame when it was
found. The thought: “Thou shalt become a great man through temptation”
was invoked, but often proved inoperative.\footnote{\textit{Ibid.}, p. 355.}

The defects of his concept of faith strongly re-acted upon his
scruples and temptations. According to that doctrine, life powerfully
inspired by faith lacked sufficient support. Faith meant the acceptance
of revelation, but more frequently and aggressively, a confident trust
in God. The acceptance of revelation was made difficult for him, nay,
logically impossible by the arbitrary way in which he impugned the
books of the Bible, which form the basis of revelation.\footnote
{\textit{Op. cit.}, Vol. IV, pp. 387 sqq.}
In his interpretation
of those parts of the Bible which he acknowledged as
authentic, Luther opened wide the gates of a subjectivism diametrically
opposed to faith.\footnote{\textit{Op. cit.}, Vol. V, pp. 356 sqq.}
In dealing with faith as fiduciary confidence
in the mercy of God he was exposed to the oppressive experience that,
notwithstanding his boldness, this faith was unstable in its presuppositions.
It depended on vacillating emotions. The appropriation of
the merits of Christ, the cloak of His justice, was a very difficult and
mostly unattainable matter for a conscience weighed down by guilt.
Moreover, he himself had declared that man was not free to do good,
but God alone could infuse a feeling of the possession of the merits
of Christ into the heart. But who could vouch for the operation
of God? Owing to his theory of predestination, Luther and his followers
did not even know whether they were destined for eternal
punishment by the mysterious will of the Most High, despite their
acquired feeling of certitude. How, then, were doubts and disquietude
to be cured? One realizes that the temptations suffered by Luther
must have found a fertile soil in his doctrinal system.\footnote{\textit{Tischreden}, Weimar cd., Vol. IV, n. 5462; Grisar, \textit{Luther}, Vol. V, p. 361.}


He confesses, in 1543, that he did not feel quite sure that his was a
steadfast, fiduciary faith, but that it still lagged behind that of
ordinary believers. “I cannot believe it,” he said in 1540, “and yet
I teach others \dots I know it is true, but I am unable to believe it
\dots Oh, if only a man could believe it!”\footnote
{\textit{Tischreden, ibid.}, n. 4864; Grisar, \textit{op. cit.}, V, 360 sq. The following passages, \textit{ibid.}, pp.
361 and 368.}
By means of these words,
he naturally does not intend to deny his faith, but to describe the
freshness of that religious fervor which distinguishes a true Christian
and which he, in the days of his youth, had observed everywhere
among Catholics. Not a day does he waste, he writes in 1542; “but
the devil is an evil spirit \dots as I do not fail to realize day by day;
for a man waxes cold, and the more so, the longer he lives.”

Even in his last sermon at Eisleben, he speaks of “the sin which
still persists in us, and which compels us not to believe. We, the best
of Christians, also do the same \dots In view of the weakness of faith,
we feel trepidation and anxiety.”

Toward the close of his life, Luther’s temptations became fewer.
At least the discussion of them becomes constantly rarer in his writings
and conversations. It seems that he had succeeded, to a certain
degree, in lulling them to sleep. The application of the antidotes
which have been enumerated above, may not have been ineffectual.
The gymnastics of which he had made use to stifle his conscience
produced an unenviable result: lassitude and indifference appeared
simultaneously. “Towards the close of life,” he says, “such temptations
cease, whilst other maladies remain.”

The other maladies which continued to afflict him, were his morbid
states, which at all times had co-operated with his temptations and
had at least contributed to strengthen them. His permanent heart
trouble, as is known, often resulted in precordial distress; and his
overwrought nerves exacted their tribute in the form of mental
suffering. Thus, in connection with other bodily infirmities, an intolerable
psychological condition developed, namely, a tormenting
sense of fear, which restlessly sought and found an object in the unrest
of his conscience. As a result, his “temptations” often assumed an
intensity akin to that of the death agony, a phenomenon which would
hardly be capable of explanation without the presence of bodily infirmities.\footnote{\textit{Ibid.}, V, p. 333.}

Inversely, the poor man’s physical condition was undoubtedly affected
by his struggles of conscience. On the other hand, he
assures us that there were frequent periods of temptation which he
sustained in a state of perfectly good health.

Without any doubt, the phobias originated not merely in disease,
as has been maintained; but disease and spiritual attacks combined
to assail his soul, and his conscience had to bear the brunt of the attack.\footnote{Grisar, \textit{Luther}, Vol. V, pp. 321 sqq.}

Indeed, his violent apostasy from the Catholic Church, then universally
acknowledged, could not have taken place without a lengthy
and profound agitation of conscience. It cannot be repeated too
often that Luther’s terrific assault upon the papacy was inevitably
accompanied by a life of interior storm and stress, which could
scarcely be allayed, especially in a man who had enjoyed the interior
peace of the Church for so long a time. It is of such struggles of
conscience as the real objects of his temptations that the unfortunate
man speaks when he tremblingly asks himself the questions: Are
you alone gifted with understanding? Has mankind been in error
until your advent? Did the Almighty really abandon His Church and
acquiesce in her being immersed in error? Even his consciousness of
great success and the eulogies of his adherents were bound to prove
ineffectual in view of such terrible thoughts.

If he was less frequently assailed by storms of conscience in his
advanced years, this is due in part to the exhaustion which finally overpowered
him and which produced a certain apathy. Other dismal
features of his mental life associated themselves with this condition,
which, in its totality, constituted a truly abnormal state of soul.
