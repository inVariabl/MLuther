\section{Before the Ban}

The friendly invitation which had been extended to Luther to
visit Rome, occasioned by the distorted reports of Miltitz, was barren
of results. The papal brief addressed to the “beloved son” was suppressed
by the Saxon court, and Luther never knew of it. Nor was
this a misfortune; for it would only have been the occasion of misinterpretation
and derision. Luther was unflinchingly resolved to
carry out his programme and the Saxon Elector artfully and perseveringly
continued his maneuvers to shield the rebellious monk.

It was to the advantage of both that Miltitz with his unfortunate
efforts at reconciliation did not disappear from the scene. He invented
the proposal that Luther should present himself to the
archbishop-elector of Treves as a non-partisan judge. In the course
of these negotiations he pretended that this archbishop had been
legitimately appointed by Cardinal Cajetan to render a judicial verdict
in the matter. It was a phantom that did not materialize. Extremely strange
and dubious is the statement which Frederick of
Saxony attributed to the archbishop of Treves, that Pope Leo X was
prepared to make Luther a cardinal if he would retract his errors.

Relative to the policy of Leo X in the great questions of that day,
especially in that of the election of a German emperor, there is no
doubt that it strongly affected his attitude toward the Saxon Elector
and, mediately, toward the latter’s protégé. For some time Leo had
favored the promotion of Frederick of Saxony, the most esteemed
and influential German prince of the time, to the dignity of emperor
to succeed the deceased Maximilian I. Subsequently he promoted the
cause of the youthful Charles V, even though his election to that
dignity would have placed him in dangerous proximity to the
capital of Christendom as King of Naples.

The most decisive step taken by Luther after the Leipsic disputation and
prior to his condemnation by Rome, was the publication, in
June, 1520, of his work, “On the Papacy at Rome.”\footnote
{\textit{Werke}, Weimar ed., Vol. VI, pp. 285 sqq.; Erlangen ed., XXVII, pp. 86 sqq.}
He now intended to strike the Church a mortal blow by disproving the doctrine
of the primacy. The occasion was the written attack made on him by
a learned Franciscan of Leipsic, Augustine Alfeld, professor of Biblical
science in the monastery of his Order in that city, who had
publicly espoused the cause of the divine right of the papacy. The subject
had aroused the sympathy of Catholics, who correctly perceived
that here was the center of the conflict and the decisive battle-field
of the future. The other errors would be defeated if the authority of
the papacy was firmly established from Holy Writ and the tradition
of fifteen centuries. In his hastily compiled reply to the “highly
celebrated Romanist of Leipsic,” Luther undertook to refute this
“monkish booklet,” as he styled it, on a broad basis. In his own book,
which he had composed in German for the masses to read, Luther
expounded his doctrine that the Church does not require a pope,
that a visible head is inconsistent with its nature, and that the attachment
of that head to a definite place such as Rome, is inconsistent
with its character as a spiritual, invisible kingdom, a congregation of
all the faithful who cannot be differentiated individually according to
their interior dispositions. The true Church of Christ,
who is its only Head, is made manifest only by certain signs, namely,
Baptism, the Eucharist, and the preaching of the true Gospel (as
purified by Luther). The power of the keys is conferred upon all
Christians collectively, including the laity, and does not consist in the
absolute sovereignty of any spiritual government, but is solely the
assurance to be awakened by Christians of divine forgiveness and
mercy for consciences in need of consolation in their brethren. Here
the old starting-point of Lutheranism emerges once more: Quiescence
of the timorous soul as the supreme end.

In the course of his work Luther indulges in unprecedented invectives against
the covetousness of the pope, whom he constantly
implicates in the shocking villainies of the Romanists. He passionately
invokes the patriotism of the Germans and, in particular, the economic
and nationalistic ambitions of the princes and the nobilityagainst
the ecclesiastical régime of the Italian curia. At Rome, he
says, they speak of the drunken Germans who must be fooled. They
regard us as beasts. They say: “The German fools must be separated
from their money by all manner of means.” “If,” he finally exclaims,
“if the German princes and nobility do not bestir themselves very
soon, Germany will become a desert or it will have to eat itself.”
This pamphlet contains many violent invectives but no solid argument. A
Protestant critic (Vorreiter) correctly judges that it is
filled with “consummate sophistry” and “defies the most elementary
logic.”

The courageous Alfeld was denounced by Luther as “an uncivil
ass which cannot even bray!” But he did not permit himself to be
frightened. Subsequently he penned still other solid treatises on the
debated questions of the day.

About this time Luther was engaged in the composition of his
exhaustive “Sermon on Good Works”--a work which is very important for a
deeper understanding of his mental development.\footnote
{\textit{Werke}, Weimar ed., VI, pp. 202 sqq.; cf. IX, pp. 229 sqq.; Erlangen ed., XVI (z2nd ed.),
pp. 121 sqq.}
He says that it treats of “the greatest question that has arisen,” and that
the publication of the book appeared to him to be more necessary
than any of his sermons or smaller works.\footnote{See the dedication.}
For, on account of his doctrine that man is saved by faith alone he was loudly accused of
being opposed to good works. He now wishes to restore them to
honor, and, at the same time, glorify faith, as he understood it, as
the pillar of good works. He dedicated this “Sermon,” which had
grown into a book in consequence of his rapid industry, to the
brother of his territorial lord, Duke John of Saxony, who was very favorably
inclined to him and sought edification in religious books. Thus,
in a certain sense, this book was a parallel to his consolatory tract
“Tessaradekas,” which he had shortly before dedicated to the sick
Elector Frederick. Luther’s precarious position at Wittenberg moved
him to correspond with the court which protected him. Throughout
this tract he appears to be very solicitous about true virtue. It has
been characterized by Protestants, even in recent years, as “the first
description and demonstration of evangelical morality by the reformer.”\footnote
{J, Köstlin, \textit{M. Luther}, I, 5th ed., p. 291.}

This is an added reason for subjecting it to a closer inspection.
Luther’s alleged true conception of Christian liberty had its origin
in the Epistles of St. Paul, particularly in his commentary on
Galatians, published in 1519. The liberty of the believing soul be
came his favorite theme. Somewhat later, at the time of the proclamation
of the ban, he developed it in his well-known treatise, \textit{On the
Liberty of a Christian}. For the sake of his attitude toward good
works, he pushes his conception of freedom from commandments and
obligations somewhat into the background. In this book he teaches:
Although the true Christian is subject to no law, faith alone being
necessary for him, whilst everything else is voluntary, yet because
of this very faith, and impelled by it, he submits to the commandments,
which are necessary on account of the weak sinners; indeed, a
Christian is constantly occupied with good works. His faith urges
him to do good. Where there are no good works, there is no true
faith. Faith, however, according to Luther, is never anything else but
an uninterrupted trust in God’s mercy through the merits of Christ.
This joyous confidence he regards as the sole source of morality. The
sentences which we have quoted and which, in part, are echoes of the
work’s mystical aberrations have about them something deceptive and
apparently attractive. But his glorification of faith, \textit{i.e.}, trust inspired
by faith, is permeated by that forcible struggle for the quieting of his
own inner needs and fears which led him to cling to the
doctrine of faith as the sole means of salvation.

The fundamental deficiency of his theory of good works cannot
escape the critical eye.

In the first place, he says, good works are only such as have been
commanded by God. Such a thing as the voluntary assumption of a
moral act that is not commanded by God does not exist for him.
Consequently, the main artery of the perfect life is severed. There is
no foundation for the intense pursuit of virtue or for heroism. The
saints of the Bible or of Church history, whose wondrous deeds
were not inspired by divine command, were simply fools.

Hence, according to Luther, good works flow spontaneously from
confident faith in the blood of Christ. But neither his own life nor
that of others confirms this doctrine.

The self-righteous, of whom the world under the rule of the
papacy is full, are supposed to know nothing of faith in the blood
of Christ. They err, according to Luther, because, without Christ,
they invest their own works with value for salvation. Luther’s utterly
false arraignment of the Catholic doctrine of good works is here
again resuscitated in drastic form.

His assertion that the self-righteous papists sinned against faith in
the blood of Christ, was bound to meet with speedy and decisive retorts
on the part of Catholics. The Humanist Pirkheimer, after renouncing
Lutheranism, which he had favored for 2 while, wrote: “we know that we
are justified gratuitously by the grace of God and the atonement which is
in Christ Jesus, through which we obtain the remission of sins”; we are
not justified by our works alone, but by the death of the Son of God;
nevertheless, “we cannot have life without good works, and if they are
performed for no other reason, then they should be performed out of gratitude
to God and His only begotten Son.” Others very appropriately referred
Luther to the known bases of good works, namely, the necessity of doing
penance and atoning for sins, the need of supplicating God’s help in the
affairs of temporal existence, and above all else the command of the Bible
that we must gain Heaven through the merits of good works performed
with grace. Everyone was familiar with the concept that the love of God
must sustain and ennoble all good works.

In the formulation of this new doctrine, Luther is governed by the
idea, conceived during his mystical period, that those works only can
be called good which proceed from the motive of absolutely perfect
love. Hence he likes to portray how this love, as an activity in man
and an efficient motive urging him to goodness, is joined with the
fiduciary faith that springs from his gospel. For the rest, moral spontaneity
is suppressed in his system. The contradiction is obvious.

Man, according to Luther, is not free to do good. God alone is the cause
of everything. Even reason does not attain to the truly spiritual, and the
co-operation of man in working out his salvation is, according to a casual expression
of his, only a figment of ``that mad harlot,'' the brain.\footnote
{\textit{Werke}, Weimar ed., Vol. XVII, I, p. 58.}
Luther pretends to ignore the doctrine of faith (\textit{fides formata caritate})
as taught by the Catholic Church. But in opposition thereto, in his above-quoted
sermon, glorifies faith because love abides in faith and results
in good works. Thus generally, in his practical writings he abandons his
theories of the unfree will and the faith produced by God, invariably
demands self-activity, and represents love as an element of faith. This contradiction apparently remains hidden to him because of his quite intelligible
effort to appear as a promoter of good works, as the founder of true morality in contrast with the self-righteousness of the papal system.

“The Catholic conception of faith by its emphasis on good works is something
so natural, something that so obtrudes itself upon the Christian and
natural conscience, that we need not be surprised if Luther, in contradiction
with his reformatory principle, frequently testifies to this truth.”\footnote
{Jos. Mausbach \textit{Die katholische Moral und ibre Gegner}, 1911, p. 27.}
Luther himself says on one occasion: “The entire nature (of man) seeks
good works, when he is not subject to temptation.”\footnote{\textit{Opp. Excget.}, XX, p. 188.}
The new system had been suggested to him by his so-called temptations.

Certain extreme expressions employed by Luther were derived from this
same false principle. Thus he says in 1520: “Let us beware of sin,
but much more of laws and good works; and let us only observe well the
divine promise and faith; then good works will readily follow.” Now,
good works are prescribed by God in Sacred Scripture. Here we perceive
a new contradiction in Luther’s teaching. One of the most eminent theologians
of modern Protestantism aptly wrote: “If it is asked why God, who
connects salvation with justification by faith, prescribes good works and
wishes to be honored by them, the arbitrariness of this disposition cannot be
concealed \dots Nor it is evident that good works should serve every man
as the ratio cognitionis of his own justification.” According to Luther, he
adds, the value of good works must not be considered at all, and yet his
doctrine bases the certain consciousness of justification on a measure of good
works. “Confronted with these contradictory judgments of moral action,
shall anyone find that repose which justification is supposed to guarantee
him?” (Albrecht Ritschl).

But, to continue with the characterization of the much discussed
Sermon on Good Works, the author, after having announced his new
principles, takes up the Ten Commandaments one by one. He wishes
to demonstrate how faith works itself out through good works in
the case of each commandment. This part of his treatise is full of sound,
inspiring ideas, which are identical with the ancient Catholic teaching.
Nor does it lack charm and warmth. Duke John and many other
readers might be greatly edified by the popular exhortations contained
in this work. Though Luther, for instance, does not acknowledge
any commandments, such as that of fasting, he asserts that
faith leads the devout Christian to chastise his flesh in order that concupiscence
might be broken thereby; likewise he intimates that the
religious observancé of Sunday rest must serve to allay the passions.
On the other hand, he adds, on his own responsibility, that physical
labor on Sunday, according to Christian liberty, is not
to be regarded as prohibited; all days should be holy days and work days.
There follow certain recommendations, e.g., to be patient under the
tyranny of an unjust ruler, since corporal sufferings or material losses
cannot injure the soul; rather, to suffer injustice improves the soul
and injustice is not so dangerous by far to the spiritual as it is to the
temporal authorities.

Here he again seizes the opportunity to inveigh against the abuses
of the spiritual authorities from which the German* nation was supposed
to suffer.

“Behold, these are the real Turks.” “It is not right that we should support
the servants, the subjects, yea the rogues and harlots of the pope to the loss
and injury of our souls.” In these words he favored the view of obedience
as conceived by the Elector of Saxony and his court. Now he appeals to the
princes, the nobility, and all public authorities to defend themselves “with
the secular sword” against the burdens imposed by Rome and its clergy,
“since there is no other help or remedy.” The bishops and clergy opposed to
the machinations of Rome were full of fear. Hence, “the best and only
remaining remedy would be for kings, princes, nobles, cities, and communes
to put a stop to these abuses.” These particular Turks “kings, princes,
and nobility should attack first of all;” they should “treat this same clergy
like a father who had lost his reason” and is imprisoned with all honors.\footnote
{\textit{Schlusswort des Sermons vom Neuen Testament, Werke,} Weimar ed., VI, pp. 353 sqq.
Erlangen ed., XXVII, pp. 141 sqq.}

With all honors, he says, and continues: “Thus we should honor the Roman
authority as our supreme father; and yet, since it has become mad and
irrational, we should not allow it to carry out its purpose.” It was, of
course, but a figment of Luther’s imagination that the expected measure
would be taken “with all honors.” The purposely selected phrase “supreme
father” has no weight, but merely indicates the respect which the cautious
court of Saxony still tried to foster toward the pope.

With still less restraint Luther summoned the people to revolt in a
short reply which he published at that time to the Epitome of the
papal court theologian Prierias. As an indication of the contempt
with which he regarded this book he had it reprinted in its entirety,
adding a series of violent attacks,\footnote{Werke, Weimar ed., VI, p. 328; Opp. Lat. Var., II, pp. 79 sqq.}
in the course of which he made
a formal appeal to bloodshed. This appeal was couched in such violent
language that the early Protestant editors of his works did not dare
to publish it in its entirety; it betrayed a passion no longer master of
itself.\footnote
{Cf. H. Grisar, ``\textit{Cur non manus nostras in sanguine istorum lavamus?}'' in the \textit{Histor.
Jahrbuch} of the Görres Society, XLI, 1921, pp. 248-257 (\textit{Luther-Analekten, V}).}

“If the fury of the Romanists goes on thus,” he writes, “it seems to me
there is no other remedy left to the emperor, the kings, and the princes than
to attack this pest of the earth by means of arms, and to decide the matter
with the sword instead of words. For what else do these lost people, bereft
of reason, do than what Antichrist will do according to the prophecy? They
hold us to be more unfeeling than all blockheads. If we punish thieves with
the gallows, robbers with the sword, heretics with fire, then why do we not
equip ourselves with every weapon and proceed against these teachers of corruption,
these cardinals, these popes, and this whole swarm from Roman
Sodom which corrupts the Church of God without end? Why do we not
wash our hands in their blood? We would thus free ourselves and our own
from the most dangerous universal conflagration in existence. How fortunate
those Christians, wheresoever they may be, who are not compelled to live
under such an Antichrist of Rome, as we unfortunate wretches!”
Naturally Luther did not persist in the bloody designs which he
conceived in 2 moment of sudden excitement. It was an impossibility
to follow his crazy call. In a calmer hour he afterwards wrote to
Spalatin words which were destined for the Saxon Elector: Not by
force and murder should the gospel be contended for, but Antichrist
is preferably to be crushed by means of the Word; if, as he fears,
a revolt against the clergy should break out, he himself would be
quite innocent of the calamity, because he--so he now avers--advised
the nobility to have recourse, not to the sword, but to “edicts”
against the Romanists.\footnote{Grisar, \textit{Luther}, II, 54 sq.; January 16, 1521.}

Luther’s vehement appeals to the sword and his call for blood,
which followed closely upon the publication of the work of Prierias,
were not caused by the tone of the Roman tract, but by the clearness
with which that writer expounded the doctrine of the primacy of
the Holy See. The calmly written tract was really but the table of
contents of a larger work which Prierias intended to publish. It was
remarkably free from offensive and inflammatory language. In his
blind rage, however, Luther directed his whole opposition against
spiritual submission to the pope, whom he denounced as Antichrist. The
principal objection, so far as he was concerned, were not
the acts of robbery imputed to Rome, but the submission which
the pope, in the name of his divine primacy, demanded for his false
doctrines, deviating from those of Luther. This point is brought
out still more clearly by his characterization of Prierias’ composition
as a heretical, blasphemous, Satanic, Tartarean poison which
spreads over the whole earth, He--Prierias--infinitely outdoes
Arius, Manichaeus, Pelagius, and all the other heretics. If Rome teaches
the doctrine Prierias attributes to her, then fortunate Greece,
fortunate Bohemia, fortunate all who have severed their relations with,
and wended their own way out of, this Babylon. “Now Satan has
taken even the hitherto impregnable fortress of Sion, Sacred Scripture,
the tower of David. Fare thee well, thou unfortunate, lost, and
blasphemous Rome, the wrath of God has come upon thee.” “We
wished to heal Babylon, now let us abandon it, that it may become
the abode of dragons, ghosts, specters, witches, and that it be what
its name indicates--an eternal confusion, full of the idols of avarice,
perjurers, apostates, priapists, robbers, simonists, and countless hordes
of other monsters, a new pantheon of impiety!”

Those who censured his frenzy he referred to the inflammatory
and vehement language of his opponents. But all these combined did
not approach the horrible bitterness and the resounding fury of
his own effusions. As the Epifome of Prierias was free from reproach
in this respect, so, too, the writings of Eck, Emser, Alfeld, and the
earlier productions of Tetzel were actually moderate in comparison
with Luther’s. Rome itself had not proceeded against him with excommunication.
The Holy See and the bishops had as yet taken no
steps against him, which might have conjured up such frenzy. His
outbreaks proceeded from his temperament. The cause lay deeper
than is usually supposed; it has been partly revealed in the previous
remarks about his psychology. It will be made still more apparent
by the events of the momentous year 1520.

In the meantime events led to other outbursts of his vehement
polemic. In conformity with their position and custom, the faculties
of Cologne and Louvain had rejected a2 number of propositions extracted
from his writings. He published a Latin Responsio, in which
he proposed to demonstrate the vanity and nullity of such academic
verdicts in general. Until they would refute him, he says, he would
regard their condemnation as one does the imprecations of 2 drunken
wench. The professors of Louvain and Cologne he characterized as
“asses” in a letter to Spalatin.\footnote{Weimar ed., VI, pp. 174 sqq.; \textit{Opp. Lat. Var.}; IV, pp. 176 sqq.}

In other writings he undertakes to discuss practical questions. The
little book on “How to Confess”\footnote{Weimar ed., VI, pp. 157 sqq.; \textit{Opp. Lat. Var.}, IV, pp. 154 sqq.}
asked the question whether it
was obligatory to confess all secret sins known only to the sinner
himself. The author denounces the practice of auricular confession
in general as a means of avarice and tyranny.

In a \textit{Sermon on the Blessed Sacrament}\footnote
{Weimar ed., II, pp. 742 sqq.; Erlangen ed., XXVII, p. 28, of the year 1519, with
the supplements in Luther’s “\textit{Verklärung etlicher Artikel},” Weimar ed., VI, p. 78;
Erlangen ed., p. 70, of the year 1520.}
he acknowledges the real
presence of Christ in the Eucharist, but questions whether it takes
place in virtue of transubstantiation, or whether the bread is present
simultaneously with the body of Christ. The manner in which he presents
the subject amounts to a denial of the sacrificial character of
the Eucharist. One may not assume, he says, that this Sacrament
is a work \textit{per se} pleasing to God (\textit{ex opere operato}); he maintains,
on the contrary, that the work of salvation is wrought by the faith
of Christians who are united about this Sacrament, and by that believing
conviction of salvation which is nurtured and inflamed by
this holy bread as the sign of Christ’s Testament. It would be becoming
and goodly, he adds, if this Sacrament were dispensed to
Christians in the twofold form of bread and wine, and not merely
under one species; for it was instituted by Christ under both and
should be so ordained by a council of the Church. In this exposition,
he did not consider the weighty reasons which had determined the
Church in the course of centuries to administer this Sacrament under
one form only--a measure which is advocated by many Protestants
to-day for hygienic reasons. The Church has always taught that
Christ is present whole and entire in, and consumed under, each form,
and that the Sacrament is but “partially” present under the form
of bread.

The demand for both forms was destined to prove a powerful
means for the introduction of the new religion.
The first opposition to this presumptuous demand was made by
the bishop of Meissen, John von Schleinitz. In a decree published on
January 24, 1520, he ordered the sequestration of Luther’s work
\textit{On the Sacrament}, and commanded the clergy to inform the people
of the reason. why, as recently as the fifth Lateran Council, the
Church adhered to the decrees which ordained the administration of
the Eucharist under one form only. It is worthy of note that this
was the first public declaration of a German bishop against Luther.
The decree was issued in the name of the bishop with the advice
and approbation of the cathedral chapter of Meissen. This procedure
angered Luther and he at once published two replies, one in German, the
other in Latin. To be able to attack the bishop with less
restraint, he asserted that the author of the decree, which had been
issued at Stolpe, could not possibly be a bishop; that the stupid note
had made its appearance quite appropriately during the Lenten
season, and that the author probably lost his reason in the carnival.
He insisted that the use of the chalice was “an open question.”

Luther, as so often happened to him when engaged in controversy,
went even further, with fatal results. He discussed the abrogation of
clerical celibacy in 2 manner enticing to the clergy who had taken
sides with him or who were undecided in their attitude. Thus he
says: “What if I were to say that it appears proper to me that a
council should once more permit priests engaged in the cure of souls
to have wives? Behold, the Greek priests have wives, and what good
man to-day would not, out of sympathy for our own priests, wish
that they would enjoy the same liberty, in view of the dangers and
scandals that beset them?”

The court of the Saxon Elector now became deeply concerned over
Luther’s attacks upon the Saxon episcopate and the aggravation of
the controversy within the Church. In order to restrain the assailant
in the name of the Elector, Spalatin entered upon a more intensive
correspondence with Luther, in the course of which the latter revealed
even more clearly the illusions under which he labored. He said
he did not comprehend the counsels of peace which Spalatin addressed to
him. Why did Christ make him a doctor? He was acting in
conformity with the will of God and his vocation. He must permit
himself to be directed by God, as the ship is driven by winds and
waves. The Word of God could not progress without strife, profound
agitation, and danger. Let Spalatin caution his raving opponents to
be considerate towards him, lest the filth which they had stirred up
emit an even greater stench. If he, Luther, permitted himself to be
led by human wisdom, it would be a different matter; but his cause
had not been prompted by the judgment of men. God carried him
along; let Him see to it what'He would accomplish through him;
he himself had not chosen the task which he must now perform.
It was true that he was more violent than necessity called for; but
who can observe moderation when he is angry? He feels that his
blood boiling, but he is at least plain and frank, etc.\footnote
{\textit{Briefwechsel}, II, p. 294: “\textit{Quis rogavit Dominum, ut me doctorem crearet? Si creavit,
habeat sibi aut rursum destruat, si poemitet creasse,’ etc--Ibid.}, p. 323: ``\textit{Esto novum
et magnum sit futurum incendium; quis potest Dei consilio resistere”? Ibid.}, p. 325:
“\textit{Non patiar damnatum errorem in evangelio Dei pronuntiari etiam ab umiversis angelis,”
etc.--Ibid.} p. 327: “\textit{Ne praesumcres, rem istam tuo, meo, aut ullius bominum iudicio
coeptam.}”--Ibid. p. 328: “\textit{Verbum Dei, ut Amos ait, ursus in via et laena in silva \dots
Sic Deus me rapit; qui viderit quid faciat per me},” etc.--Cf. Grisar, \textit{Luther}, Vol. III, pp.
109 sqq.}
It was not
difficult for him to admit his violent emotion; but how was he going
to prove that he was guided by God? He made no attempt to prove
this; he simply felt the divine guidance as anyone else might.
During all these trials he looked with unrest and anxiety toward
Rome, whence the ban could not fail to come.
He judged that it would be advantageous for him to prepare the
minds of the people for this event by publishing a German tract
entitled \textit{A Sermon on the Ban}.\footnote
{Weimar ed., VI, pp. 63, 75 sqq.; Erlangen ed., XXVII, pp. 51 sqq.}
He had previously issued a similar
work, On the Validity of the Ban. His cardinal theme is this: An
unjust excommunication (such as he is looking forward to) does not
separate 2 man from communion with Christ, nor deprive him of
the intercession and “all the good works of Christendom.” It is rather
“a noble and great merit before God, and blessed is he who dies unjustly
excommunicated.” “Christians should learn,” he continues,
“to love the ban rather than to hate it, just as we are taught by
Christ not to fear, but to love punishment, pain, and death.” Although
he preaches respect for ecclesiastical authority, he expatiates
excitedly on the prevalent abuses of the ban and says one may not
be astounded that at times the ecclesiastical judges are bloodily
avenged by evil-doers, through God’s permission. The secular authorities
should not tolerate certain abuses of the ban in their countries
and among their people.

The Roman curia had too long deferred determined action for
political reasons. Luther’s writings, which issued in rapid succession
from Wittenberg, some of them in inflammatory German for dissemination
among the masses, others in learned Latin for the outside
world, had ample time to prepare the way for the coming defection.
The lamentable delay of a firm decision was attributable to the negligence
of the German episcopate, no less than to the illusions to which
the learned circles.in Germany and abroad, who had been educated in
a one-sided Humanism, were subject concerning the reform movement.
Everything conspired to enable the great controversy to vest
itself all too long with the character of an undecided issue.
