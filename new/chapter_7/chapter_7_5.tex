\section{The Fire Alarm}

On December 10, 1520, Luther solemnly burnt the book of canon
law, as he had previously announced.\footnote
{For details see Grisar, \textit{Lutherstudien}, No. I (1921), pp. 5 sqq.}

In response to a poster which Melanchthon had nailed to the door
of the Wittenberg church, a large number of students and teachers
assembled before the Elster Gate in the forenocon of that day. A
funeral pile was built, upon which were placed the papal law books
together with some Scholastic and anti-Lutheran treatises. The principal
act was the burning of the book of canon law. The papal Bull
was not even mentioned in the invitation. Only after the pile of
books, to which a master set fire, had commenced to burn, while
the students were chanting jeering songs, Luther stepped forth and
cast a printed copy of the Bull of excommunication into the fire, saying
“Because thou hast destroyed (\textit{conturbasti}) the truth of God,
may the Lord consume thee in this fire. Amen.” The words were
almost a literary repetition of the report of the judgment visited upon
the person of Achan, as related in the Book of Josue (VII, 25). A
less reliable tradition reports Luther as having said: “Because thou
hast grieved the holy one of the Lord, eternal fire consume thee!”
and the anti-Lutheran party erroneously identified Luther with “the
holy one of the Lord,” whereas the expression probably referred to
Christ, who is called the Holy One of the Lord in the Bible.

The burning of the book of canon law (as this act should be called
in preference to the burning of the Bull of excommunication) was a
sign of conflagration, symbolical of the ecclesiastical revolt which
was commencing in Germany and thence was to spread throughout
the world and result in civil revolution. The young men who were
present, in their enthusiasm for Luther, did not suspect the farreaching
significance of the act. They sang farcical songs, and long
after Luther and his friends had departed, continued their buffoonery
about the funeral pyre, which they stirred up ever and anon. In the
afternoon students rode about the streets of Wittenberg in a carriage,
jeering a ridiculous imitation of the papal Bull. On the next day,
however, Luther clearly explained the proceedings to his hearers, saying
that really “the pope himself, \textit{i.e.}, the Roman See,” should have
been consigned to the flames, instead of the book of canon law. The
exuberant spirit of the students vented itself anew against the pope in
the course of the ensuing carnival. Besides other mischievous pranks, a
student vested himself like the pope, paraded in great pomp with a
masked Roman curia and, according to a previous agreement, was pursued
through the streets by a boisterous mob and finally arrested and
brought to judgment. Luther gives an account of this affair with
characteristic satisfaction.\footnote
{According to Köstlin-Kawerau, \textit{Luther}, I, p. 375; cfr. Grisar, \textit{Luther}, Vol. II, pp.
51 sq.}

In consequence of the intervention of Rome, the number of students at the
University had at first decreased slightly. But soon afterwards Spalatin
found that Luther’s lectures were attended by nearly
four hundred hearers, Melanchthon’s by from five to six hundred.\footnote{Grisar, \textit{Luther}, Vol. II, pp. 50 sqq.}
Despite the Bull of excommunication the fame of Wittenberg increased greatly
throughout Germany. Many never obtained a correct
knowledge of the Bull, and still less of the reasons for the condemnation.
After the termination of the respite which had been granted
Luther for the purpose of recantation, the Pope promulgated a new
Bull, “Decet,” dated January 3, 1521, which announced that the ban
had gone into effect and contained exhortations to the faithful. On
June 12, Luther’s writings and a wooden statue of him were consigned to
the flames at Rome. In conformity with ancient custom the
scene was enacted in the Campo dei Fiori, where the statue of Giordano
Bruno now stands.\footnote
{Kalkoff in the \textit{Archiv für Reformationsgeschichte}, XI (1914), pp. 165 sq. Cf. \textit{Zeitschrift
für Kirchengeschichte}, XXV (1904), P 578.}

The proceedings at Rome found no particular echo in Germany,
for which indolence the episcopate was chiefly to blame. Too many
of the noble lords who held episcopal sees had other interests at heart
than meeting hostile attacks upon the Church, which some of them
failed to understand, while others feared to promulgate the papal
Bull and take corresponding measures, Yet their dioceses were
founded upon that very canon law which Luther had consigned to
the flames and which formed the basis of the entire ecclesiastical life
of the past and of all western civilization. If the flames of Wittenberg
could not enkindle the zeal of the bishops, the latter should at
least have learned a lesson from the following work of Luther: \textit{Why
the Books of the Pope and his Disciples have been burnt by Dr. Martin Luther.}
In this work, which appeared in both German and
Latin,\footnote{Latin text in Weimar ed., VII, pp. 170 sqq.; Erl. ed., \textit{Opp. Lat. Var.}, V, pp. 257 sqq.;
German text in Weimar ed., VII, pp. 161 sqq.; Erl. ed., XXIV 2, pp. 151 sqq.}
fundamental attacks upon the Church alternate with revolting
misrepresentations. Never did the lack of ecclesiastical loyalty
in the majority of the episcopate prove more fatally injurious to
the German Church than in those decisive days. Not until December
30 was the Roman Decree read from the pulpit at Augsburg. At
Freising, its publication was postponed to January 10. Eichstätt likewise
neglected to act till January, and, as in the case of other bishoprics,
minimized the document by means of guarded clauses. Even
Meissen and Merseburg delayed its dissemination. Ratisbon prudently
waited for developments. Passau offered covert resistance. The Upper
Rhenish bishops, such as the bishop of Spires, for a long time took no
notice of what had happened. The University of Ingolstadt scarcely
manifested any interest. The University of Erfurt was openly hostile.
The University of Vienna, in opposition to its theological faculty, declined
to carry out the provisions of the Bull, and, in justification of
its refusal, referred to the dilatory conduct of the archbishops of
Salzburg and Mayence. In brief, Dr. Eck met with most unpleasant
experiences in connection with his efforts to promulgate the papal
Bull.

The learned Erasmus confirmed Frederick of Saxony, who inclined
to Lutheranism, in his anti-ecclesiastical attitude. Owing to actual or
feigned indisposition, the Elector tarried at Cologne at the time of
the coronation of Charles V at Aix-la-Chapelle, October 28, 1520.
Erasmus made personal representations to him there, contending that
Luther should first be tried by learned and pious judges at a place free
from suspicion. In judging of the controversy, Erasmus made a sarcastic
remark to Frederick: “Luther has sinned in two respects--he
has assailed. the crown of the pope and the belly of the monks.”\footnote
{Kawerau, \textit{Reformation und Gegenreformation}, 3rd ed., 1907, p. 30.}
In conformity with Erasmus’s suggestion, Frederick saw the two
papal legates, Aleander and Caraccioli, who had been sent to the new
king and to the emperor-elect.

These legates of Leo X were more successful with Charles V, who,
though only twenty years of age, was a real monarch. He caused
the Bull against Luther to be proclaimed in the Netherlands, his
hereditary patrimony. At Louvain and Liége the writings of Luther
were publicly burned. The same scene was enacted at Cologne, after
Elector Frederick had left that city in November. Aleander succeeded
in having the same thing done at Mayence, though amid difficulties
and opposition. The strict traditional demands of the Bull and the
altered views of the age frequently collided. Even men devoted to
the Church voiced their opposition to the forcible removal of heretics.
Eck himself was decidedly in favor of the ancient imperial laws
which demanded the execution of heretics. He dealt with this question
in his widely circulated \textit{Enchiridion adversus Lutteranos}, which
abounded in quotations from the Fathers and theologians.\footnote
{Cap. 26: \textit{De haereticis comburendis}.}

Jerome Emser, who, like Eck, advocated the enforcement of the
penal laws, also attacked Luther in print. The replies which he received
from the latter were coarse and ironical, as may be seen from
the pamphlets \textit{An den Bock zu Leipzig} and \textit{Auf des Bocks zu Leipzig
Antwort} (1521).\footnote{Weimar ed., VII, pp. 262 sqq., 271 sqq.; Erl. ed., XXVII, pp. 200 sqq., 205 sqq.}

When the time for Easter confessions approached, in 1521, Luther,
taking into account the critical situation of his readers relative to
their confessars, published his \textit{Instructions for Penitents on Probibited
Books}. It was a model of his apparently considerate, yet inciting approach
to the practical questions involved in the conflict.\footnote{Weimar ed.; VII, pp. 290 5q.; Erl. ed., XXIV², pp, 204 sqq.; Grisar, \textit{Luther}, Vol. II,
pp. 59 sq.}

He clearly perceived that for many the Easter confession of that year
was to be decisive. Therefore he instructed his readers to entreat their
confessor if he should question them, “with humble words,” not to bother them
about the books of Luther; they should simply say that the popes had often
changed their opinions after promulgating a prohibition. If they were denied
absolution because of their refusal to promise to leave the prohibited books
alone, they should not be disturbed in conscience, “but be joyous and certain
of absolution, and also receive the Sacrament [Communion] without any
fear.” The more courageous penitents, who had “a strong conscience,” were
told to rebuke their confessor to his face for his narrowmindedness. If communion
was refused, they should first “humbly beg for it,” and, if that were
without avail, “abandon Sacrament, altar, priest and churches,” for they
all teach that “no commandment may be made or may exist in contravention
of God’s Word and your conscience.”

Luther was inventive in the promotion of his cause. In his eagerness
to avail himself of whatever appeared likely of serving his ends,
he, towards the close of 1520, made use of a notorious fable attributed
to Bishop Ulrich of Augsburg, by providing a Wittenberg reprint
of it with a preface of his own. This publication was intended
to be an effective weapon against the celibacy of priests and religious.
In this letter the saintly bishop is represented as narrating how some
3000 (according to others, 6000) heads of infants were discovered in
a pond belonging to St. Gregory’s nunnery at Rome. The manuscript with
the letter had been sent to Luther from Holland. It is
one of the clumsiest forgeries which issued from the ranks of the opponents
of Gregory VII, who strictly enforced the ancient law of
clerical celibacy. Emser called Luther to task for publishing this questionable
letter, and he replied that he did not place much reliance
upon it. Nevertheless, thanks to his patronage, the fable was allowed
to continue on its harmful career and was zealously exploited.\footnote
{Grisar, \textit{Luther}, Vol. IV, pp. 89 sq. What Haussleiter, in the passage cited, establishes
as a conjecture (that the Wittenberg edition of this fabulous letter and its preface were
the work of Luther) was confirmed by Otto Clemen in the \textit{Theol. Studien und Kritiken},
(XCIII 1920-1921), pp. 286 sqq. Luther appeals to the spurious letter of Bishop Ulrich
in his Table Talks, Weimar ed., IV, no. 3983, p. 55; cf. also IV, No. 4358, p. (p. 258, and 123),
No. 4731, p. 456. His commentary on Genesis, in the passage quoted by Haussleiter (p. 123),
likewise refers to this letter. For the text of the forgery, see \textit{Monumenta Germ. Hist.,
Leges}, I, pp. 254 sq.}

What was the attitude of the Augustinians towards Luther in these
fateful days? The foundation on which the German Congregation of the Order
rested, was deeply undermined by Luther’s conduct and the undecided attitude
of Staupitz. John Lang, Luther’s
confidant, who had succeeded Luther in the rural vicariate, advanced
his cause by preparing the minds of the brethren for Luther’s ideas.
Wenceslaus Link effectively aided him. In the chapter which met at
Eisleben, August 28, 1520, Staupitz resigned the vicariate when the
storm broke over his head.\footnote{Kolde, \textit{Die deutsche Augustinerkongregation}, p. 327.}
Fourteen friars abandoned the monastic
life with Lang. Link succeeded Staupitz in the government of the
congregation, now tottering to a fall. The monks who remained loyal
were decried by their pro-Lutheran comrades as sanctimonious Pharisees,
and their position became very difficult. At a meeting of the
brethren at Wittenberg, on Epiphany, 1522, it was announced that
begging would be officially prohibited. Moreover--and this was more
important--it was resolved that, in view of the liberty granted by
the Gospel, any monk might leave the monastery, but he must “proceed
without scandal, lest the holy evangel be subject to insult.”\footnote{\textit{Ibid.}, pp. 378 sq.; Grisar, \textit{Luther}, Vol. II, p. 337.}
Among those who remained faithful at Erfurt were Nathin and
Usingen, Luther’s former teacher, and a valiant defender of the
Church, who retired to the Augustinian monastery at Würzburg in
1525. Luther in the meantime continued to live outwardly as a monk
in the monastery at Wittenberg. He did not yet attack the validity
and binding power of the monastic vows.

Staupitz, utterly disappointed, but still favorable to Luther and
his doctrine, retired to the Benedictine monastery of St. Peter at
Salzburg. He repeatedly exhorted his protégé not to go too far. In
1522, he even informed him in a letter that he ought to know that
his (Luther’s) activities were being “praised by those who keep
houses of ill-fame.”\footnote{Grisar, \textit{op. cit.}, Vol. II, pp. 151 sq.}
Luther, on his part, bombarded him with letters, in which he importuned
him to join his party whole-heartedly
and openly. Owing to the restraint exercised by his environment, and
undoubtedly also to the influence of Cardinal Matthew Lang, archbishop
of Salzburg, who appointed him abbot of the above-mentioned
Benedictine monastery, Staupitz finally made a fairly satisfactory.
though ambiguous profession of faith and openly rejected Luther’s
abuse of Christian liberty. He died in 1524, and was buried in the
monastery graveyard. A large artistic slab, bearing his coat-of-arms,
covers his remains in the chapel of St. Vitus, and an elegant epitaph,
composed in the style of the age, proclaims his eulogy. The monastery preserves
his portrait as a Benedictine abbot in a little known
fine oil-painting, the work of a contemporary artist.\footnote
{Another, less reliable portrait of a later period is to be found in the series of abbots
depicted on a wall of the monastery.}

The image of this man, who for many years favored Luther, reminds
the beholder of the words which Luther, in a spirit of uncanny
satisfaction at the turmoil prevailing in Germany, wrote in one of his
last letters to Staupitz: “The confusion rages splendidly (\textit{tumultus
egregie tumultuatur}).\footnote{January 14, 1521; \textit{Briefwechsel}, I, p. 70.}
It seems to me that it can be quelled only
by the break of doomsday; so eagerly are both parties involved in it.
Marvelous things are ready to happen in the history of the time \dots
I have burned the books of the pope and likewise the Bull. In doing
so, I was at first seized with fear, and prayed; but now I take greater
pleasure in this act than in any other act of my life; they are undoubtedly
a greater plague than I imagined!”

In the same letter he writes to his spiritual father thus: “At Augsburg
[in the days of the trial] you said to me: ‘Remember that you
have begun this cause in the name of our Lord Jesus Christ.” \dots Until
now it has really been but a trifling! More serious developments are
at hand, and what you have once said is being verified: If God does
not do it, it is impossible. Manifestly everything is in the hands of
God. No one can deny it.”
