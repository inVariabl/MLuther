\section{After the Promulgation of the Bull}

At the end of September, the papal condemnation of Luther was
promulgated in the places designated in the Bull. Dr. Eck, who had
been charged by the Pope with this commission, was accused by
Luther’s adherents of being moved by personal hostility and vindictiveness.
There is no proof of such motives extant, though the selection
of another person on the part of Rome would probably have
been more advantageous. Yet, how many persons were there in Germany whom
the Holy See could have engaged for so delicate an undertaking?
As the special confidant of Rome, and as one familiar with
German conditions, Eck was authorized and commissioned to designate publicly
some of the most important partisans of Luther as having incurred the same
papal condemnation. Among the six thus selected
were Luther’s colleague, Andrew Karlstadt, the celebrated
Humanist, Willibald Pirkheimer of Nuremberg, and Lazarus Spengler, the
town clerk of that city. On October 3, Eck sent the papal
Bull to the University of Wittenberg together with a polemic treatise,
composed by himself, on the Council of Constance and on certain
doctrines which Luther had advocated in his \textit{Address to the Nobility}.

On September 28, Luther, having obtained definite knowledge of
the arrival of the Bull in Germany, wrote a letter to Canon Gunther
von Bünau of Merseburg, in which he declared his intention to ridicule
the \textit{bulle} or \textit{ampulla} as an empty bubble.\footnote{\textit{Briefwechsel}, II, p. 482.}
About the middle of
October, he published a small tract, \textit{On the New Eckian Bulls and
Lies}, in which, contrary to his better knowledge, he asserted that
there was no Bull, intimating that it must be another lie or
forgery on the part of Eck.\footnote
{\textit{Werke}, Weimar ed., Vol. VI, pp. 579 sqq.; Erlangen ed., XXIV², pp. 17 sqq.}
This was a political move, designed to
evade the blow for the time being, and to create sentiment in his
behalf.

Soon afterwards he completed another highly significant German
treatise, of which Miltitz sent a printed copy to Pirkheimer, on November
16, while Luther was working on a somewhat enlarged Latin
version. It was his famous treatise \textit{On the Liberty of a Christian}.
This treatise also appeared in German.\footnote
{German version, Weimar ed., Vol. VII, pp. 20 sqq.; Erlangen ed., Vol. XXVII, pp.
175 sqq. Latin version, Weimar ed., VI, pp. 39 sqq., \textit{Opp. Lat. Var.}, IV, pp. 206 sqq.}
Protestant writers esteem
it as the third so-called “great document of the Reformation,” coordinate
with the Address to the Nobility and the treatise On the
Babylonish Captivity. The little work, we are told, embodies Luther’s fundamental
ideas and, in its “matured and exquisite presentation,”
ranks equal to “the noblest productions of German mysticism.”\footnote{Köstlin-Kawerau, \textit{Martin Luther}, I, p. 363.}
Luther sent a Latin version to the Pope, accompanied by a
letter which he also prefixed to the original edition. This step was
taken at the advise of the fantastic Miltitz, who also caused Luther
to date this letter as of September 6, whereas it had actually been composed
on October 13. It was intended to create the impression that
the document was composed prior to the promulgation of the papal
Bull, and hence was not influenced by his condemnation. In this way
Luther was able to assume a more peaceable air, with the result that
he gained sympathizers and disposed the masses against the ban.

It was peace, he told the Pope, that he had aimed at from the beginning;
he had earned the “favor and gratitude” of Rome for his resistance to Eck
and other knaves. He had always spoken most respectfully of the person of
Pope Leo, whose reputation no one could assail; in his generosity the Pope
should restrain his enemies,.among whom Eck was the worst on account of
his insane ambition. “But, that I should recant my doctrines, that cannot
be”; “the Word of God, which teaches all liberty, shall not be made captive.”

The author again indulges in his usual descriptions of the corruption of
the papal court; the papacy, he says, is no longer in existence; poor Leo sits
like a sheep amidst wolves, and like Daniel among the lions; the Roman
Church is 2 “den of assassins worse than all others, and a house of knaves
more roguish than all the rest.”

This letter, ostensibly intended for the Pope, but in reality for the
masses, on whom Luther definitely counted, constitutes the introduction
to his work \textit{On the Liberty of a Christian}, by means of
which he purposed to set forth “the sum-total of the Christian life,”
consisting in faith, \textit{i.e.}, trust in God. Such faith produces a splendid
good, the true freedom of a Christian. He says he had acquired this
faith himself amid great and manifold interior storms and by means
of this book now intends to acquaint all “simple folk” with the value
of freedom, that treasure contained in faith. He proposes to address
only plain people. In the first and principal part of the work he seeks
to demonstrate that a Christian, in virtue of his faith, is “a free lord
over all things and subject to no one.” In the second part, he asserts
that “a Christian is the servant of all things and subject to everybody.”
The two divisions are intended to be complementary to each other.

There is a dual nature in every Christian, he maintains, one interior or
spiritual, the other exterior or corporeal. Freedom is the property of the
former, servitude that of the latter. In describing the spiritual freedom
which flows from faith, he opposes the alleged servile Catholic doctrine of
good works, and in doing so makes use of expressions which reveal his entire
new system of justification by faith alone.

“Thus, by this faith all your sins are forgiven you, all the corruption
within you is overcome, and you yourself are made righteous, true, devout
and at peace; all the commandments are fulfilled and you are set free from
all things.”

“This is Christian liberty \dots that we stand in need of no works for the
attainment of piety and salvation.”

“The Christian becomes by faith so exalted above all things that he is
made spiritual lord of all; for there is nothing that can hinder his being
saved. He may snap his fingers at the devil, and need no longer tremble
before the wrath of God.”\footnote{Grisar, \textit{Luther}, Vol. II, p. 28.}

The second part treats of the Christian as a willing servant who is subject
to everybody. In as far as the Christian lives in the society of his fellowmen,
he says, he must exercise himself in discipline and assist others as a
matter of charity. Such good acts are a direct result of faith. Luther first
discourses on the manner in which a Christian must discipline himself against
the lusts of the flesh. This is followed by a discussion on the works
done for the neighbor. All works are done out of the highest and purest
love of God. They cause a “pure, joyful life” to dominate the soul of the
Christian. Good works are not, however, performed (as the Roman Church
teaches) in order that “man may become pious in the sight of God”; they
are not meritorious and do not lead to salvation; we must not look to good
works and think that we do well in performing many of them.” “Good
works never make a devout man, but a devout man performs good and
pious works.” He performs them even though he lovingly trusts that Christ
has fulfilled all the commandments for him, and that for this reason a
righteous man is in need of no law and no good works.\footnote
{Luther had expressed similar ideas in his commentary on the Epistle to the Romans,
and hence this present tract contains nothing new. Moreover, the same distinction between
the interior and exterior life is found almost literally in Tauler, whose works were known
to Luther,--only that Tauler leaves good works intact, nay, expressly emphasizes their
meritoriousness. Cf. Grisar \textit{Luther}, Vol. I, pp. 229 sq.}

Luther believed that he had adequately replied to the objection of
those who said: “If faith alone is sufficient to produce piety, why are
good works commanded? Would it not be better to be of good hope
and do nothing?” Nevertheless, this objection naturally persisted
wherever the new religion became dominant, and to the end of his
life Luther combated the lax practice which prevailed among his followers
as the result of his depreciation of good works, which assumed
the pseudo-mystical semblance of an exaltation. The abolition of the
laws of the Church as a despicable means of coercion of necessity
avenged itself on the masses, who promptly abused the newly proclaimed
freedom of the Christian. Luther in his system forgot the
earnest and emphatic exhortations of St. Paul, whose teachings he
pretended to revive. Paul taught the necessity of zealously performing
good works from a motive of both fear and love and in the hope
of heavenly reward and forgiveness of sins. He crowned his exhortations
with the words: “Be ye steadfast, immovable, always abounding in the work
of the Lord, as knowing that your labor in the Lord
is not in vain” (1 Cor. XV, 58). Thus the faithful will be free, according
to his teaching; not indeed, here below, but in the life to
come, they will enjoy perfect liberty of spirit, there where the victory
over death hath been given to us through our Lord Jesus Christ
(\textit{ibid.}, 56 sq.).

A well-known Protestant writer, speaking of the liberty which
Luther wishes to discover in the hearts of believers, says: “The sublime
and beatific image of the liberty of the Christian has descended
upon the earth in a different manner than Luther once visioned it in
golden clouds.” “His ideal,” the same author adds, “is not organizatory
or regulative,” but “incomprehensible and almost unlimited”’;
we shall always be suspicious of such an ideal when there is question of
performing the daily duties of life.\footnote
{V. Naumann, \textit{Die Freibeit Luthers}, 1919, PP 44, 15.}
Other modern Protestant
scholars, such as Köhler, Tröltsch, Wernle, and Bess, have frankly
criticized this \textit{opusculum} from the same standpoint.\footnote
{B. Bess, \textit{e.g.}, declares (\textit{Zeitschrift fir Kirchengesch.}, XXXVII [1917--1918], p. 526):
“We must not omit to say that the fundamentation of morality in the second part signifies
a limitation of Luther.” R. Otto (\textit{Das Heilige}, 7th ed., 1922, pp. 236 sq.) directs
attention to the fact that the good thoughts, how to remain attached to God in trustful
confidence, were extensively developed long before Luther in the Catholic mystical writings
of the Middle Ages, e.g., by Albertus Magnus (rectius by the Benedictine John von Kastel),
\textit{De Adbaerendo Deo}.” This author even says: “If we did not know that ‘The Liberty of a
Christian’ is the work of Luther, we should probably classify it with Catholic mystic
literature.” Luther’s errors, however, prevent this classification. The celebrated saying,
“\textit{Ama et fac quod vis},” is quoted by the Catholic mystics of the Middle Ages in a very
different sense than that which Luther ascribes to it. On the teaching of St, Thomas
Aquinas concerning indifference to exterior works, see Jos. Mausbach, \textit{Die christliche Moral},
p. 225 (tr. by A. M. Buchanan, \textit{Catholic Moral Teaching and its Antagonists}, New York,
1914, pp. 256 sq.)}
Some of them disagree as to the meaning of Luther’s work. His Catholic contemporaries
(\textit{e.g.}, Hoogstraten and Murner) attacked this imaginary
freedom as a dangerous and corruptive phantom. Hoogstraten says
that Luther’s work on \textit{The Liberty of a Christian} is worse than the
other products of his pen, because of its seductive and insinuating
style and because it is fundamentally destructive of the doctrine of
good works.

The fervent mystical note which Luther frequently strikes in this
work recalls his former monastic readings, especially from Tauler,
and indeed is captivating. It exercised an almost hypnotic effect on
the better class of his adherents. The doctrine of pious Christian liberty,
though no one really understood it, became a shibboleth of marvelous
power. It was a genial trait in Luther’s literary career that he
knew how to interlard his impetuous speeches with winning sentiments
and appeals.

The above-mentioned tendency towards radicalism is also ascribed
by Protestant authors to Luther’s \textit{Liberty of a Christian}. At the time
of the Luther jubilee of 1917, the \textit{Christliche Welt} of Marburg said:
“From this freedom to Goethe’s ideal of humanity is indeed only a
short remove.”\footnote{1917, p. 690.}
Of the liberty achieved by Luther generally the
Heidelberg theologian v. Schubert says: ““Whilst by their struggles men
gained liberty in the supreme questions of conscience, they blazed
a path for all intellectual liberty.” It was by following this road that
mankind arrived at enlightenment.\footnote{\textit{Grundzüge der Kirchengesch}., 3rd ed., 1906, p. 234.}
It will be profitable to discuss this matter a little more fully.

A remarkable picture presents itself when the three so-called great
documents of the Reformation are considered as a whole under this
aspect. On the one hand, there is the religious sentiment which is
apparently spread over the third and wafts its fragrance back upon
the two that have preceded it. On the other hand, there is an individualism
which is opposed to revealed religion and every form of
ecclesiastical solidarity.

In the Address to the Nobility, the autonomy of the individual in judging
of religious doctrines, is advocated. “If we all are priests,” Luther says,
“how then shall we not have the right to discriminate and judge what is
right or wrong in faith? \dots We should become courageous and free” in
the presence of traditional doctrines. We should “judge freely, according to
our understanding, of the Scriptures,” and “force” the popes “to follow
what is better, and not their own reason.”\footnote{Grisar, \textit{Luther}, Vol. II, p. 31.}

Reference should be made to the expression previously quoted from the
\textit{Babylonian Captivity}, concerning the divine spirit which enlightens every
man and imbues him with absolute certitude. Here we have the enthronement
of subjectivism.

The doctrine of private judgment inspired by the “whisperings” of God,
as Luther subsequently put it, destroys revealed religion and renders impossible
the existence of a religious communion with a common creed. Nor
does it pause before the body of Sacred Scripture. Luther himself, in the
last-mentioned work, undermines the canon of the Bible by his distinction
between those writings which manifest the truly Apostolic spirit, and, \textit{e.g.},
the Epistle of St. James, which contradicts his teaching on good works.

We do not mean to assert that Luther had a clear perception of the
road he trod to religious nihilism. He wished to be and to remain a
believing Christian, and must be vigorously defended against certain
of his Neo-Protestant admirers who, in the interests of infidelity,
represent the author of the Freedom of a Christian as a conscious
champion of an undogmatic Christianity, especially in the period of
his youthful vigor and the supposed Lutheran fervor. But we may
well ask whether many of the expressions which he used in the
first flush of revolt are not diametrically opposed to the binding duty
imposed by every form of revelation, as well as to his dogmatic attitude
in later years. It may also be questioned whether the demand for
freedom for the individual and the right of private judgment may
not have assisted in laying the foundation of a mere religious humanitarianism.

It is true, as Harnack says, that Kant, Fichte, and Goethe “are all
hidden behind this Luther.”\footnote
{Cf. the passage from Ad. v. Harnack quoted in our \textit{Luther}, Vol., II, p. 32. Chapter
XXXIV of the fifth volume of my large work on Luther, first section, treats of Luther on
his way “towards a Christianity devoid of dogma,” based mainly on Protestant opinions.}

A few words more in explanation of Luther’s attitude. Luther
desired and was compelled to justify his colossal revolt from the creed
of a Church that had existed upwards of a thousand years. How could
he, a lone individual, expect to succeed in his opposition to millions
of the wisest and most excellent men of the past? He simply asserted
that no communion has an established right to claim exclusive possession
of the faith, but that every individual who correctly interprets
Sacred Scripture, has precedence. Thoughtful critics perceived
in this proposition the dissolution of all religion, but Luther says:
Follow me, for I am certain, and accept the fragments of faith which
I leave to you. That he believed and willed to be perfectly certain,
must be conceded; for the protracted struggle he waged with himself,
in order to attain certainty, could not, in view of his psychological
disposition, be devoid of results. But when he perceived that many,
influenced by his rare qualities, his firm stand, and the force of his
language, admiringly accepted his opinions, the notion that he was
sure of his ground became deeply embedded in him, even though
he was not always able to cope with the doubts that assailed him, as
his own confessions, made both subsequently to and contemporaneously with
that so-called spring-time of the Reformation, show.
Nevertheless, in the introduction of his “Address to the Nobility,”
he finds himself “compelled to exclaim and to shout.” He believes he
is promoting only the glory of Christ.

Hence his concluding words, “God has forced me by them [my adversaries]
to open my mouth still further.” He meets with opposition, but is
only confirmed thereby. “Though my cause is good, yet it must needs be
condemned on earth and be justified only by Christ in heaven.”\footnote
{Grisar, \textit{Luther}, Vol. 11, p. 37.}

“I feel that I am not master of myself (compos mei non sum),” he
writes to a friend a few weeks later. “I am carried away and know not by
what spirit.”\footnote
{To Konrad Pellikan, end of February, 1521. Grisar, \textit{op. cit.}, Vol. II, p. 52.}

About this time he unburdens his mind to Staupitz, thus: “Our dearest
Saviour, who has immolated Himself for us, is made an object of ridicule.
I conjure you, should we not fight for Him, despite all dangers, which are
greater than many believe? \dots With confidence I have sounded the bugles
against the Roman idol and true Antichrist. The word of Christ is not the
word of peace, but of the sword \dots If you do not wish to follow me, then
at least suffer me to go on and be carried away [\textit{sine me ire et rapi}].”\footnote
{\textit{Briefwechsel}, III, p. 85; Grisar, \textit{op. cit.}, Vol. II, p. 54.}

Staupitz, however, deluded and undecided, had himself uttered the words:
“Martinus has undertaken a difficult task and acts magnanimously, illumined
by God.”\footnote{Köstlin-Kawerau, \textit{M. Luther}, I, p. 371.}

Touching on the cares of his impetuous activity, which allowed him no
time for reflection at the most decisive moment of his life, Luther writes
in a letter: “Labors of the most varied kind carry my thoughts in all directions.
Twice a day I have to speak in public. The revision of my commentary
on the Psalms engages my attention. At the same time I am preparing sermons
for the press. I am also writing against my enemies, opposing the Bull
in Latin and in German, and working at my defense. Besides this I write
letters to my friends. I am also obliged to receive visitors at home.”\footnote
{Grisar, \textit{Luther}, Vol. II, pp. 52 sq.}

In this last-quoted letter he states that he was opposing the papal
Bull of condemnation in Latin and in German. Encouraged by the
attitude of the University and the Elector’s court, which declined to
promulgate the Bull, Luther at once followed up his work \textit{On the
New Eckian Bulls and Lies} with his \textit{Adversus Execrabilem Antichristi
Bullam}, also published in German, though somewhat altered,
under the title \textit{Wider die Bulle des Endchrists.}\footnote
{Latin text in the Weimar ed. of Luther’s works, VI. pp. 597 sqq.; Erlangen ed., \textit{Opp.
Lat. Var.}, V, pp. 134 sqq.; ‘German text in Weimar ed., VI, pp. 614 sqq.; Erl. ed., XXIV²,
pp. 38 sqq.}
In the Latin text he
reiterates his doubts on the authenticity of the Bull, and pronounces
anathema upon the authors of this “infamous blasphemy.” He volunteers
to die, in the event that this damnable tyranny should actually
be consummated in him. In the German version he exclaims with
demoniacal fury: “What wonder if princes, nobles and laity should
smite the heads of the pope, bishops, priests, and monks, and drive
them from the land?”

On November 17, he renewed his appeal to a general council of
the Church.\footnote{Weimar ed., VII, pp. 75 sqq.; Erl. ed., Opp. Lat. Var., V, pp. 121 sqq.}

In compliance with an order of his Elector, he forthwith undertook to compose
a more learned defense of his condemned forty-one
articles. It appeared in Latin and in German, about the middle of
January, 1521, under the title, \textit{Grund und Ursach aller Artikel (Assertio
Omnium Articulorum)}.\footnote
{Latin text in Weimar ed., VII, pp. 94 sqq.; Erl. ed., Opp. Lat. Var., V, pp. 156 sqq.;
German in Weimar ed., VII, pp. 308 sqq.; Erl. ed., XXIV², pp. 55 sqq.}
“Who knows,” he says in the German
edition, “whether God has not raised me up and that it behoves
mankind to fear Him, lest they contemn God in me? Do we not read
that God usually raised but one prophet at a time?” In an attack on
free-will contained in this book he expressly teaches that “everything
happens necessarily,” because ordered and effected by God. The
Bible is expressly represented as the sole source of faith.
