\section{Conclusion of the Roman Trial}

The written representations which Dr, Eck made at Rome after the
Leipsic disputation, and the efforts of the Dominicans at last caused
final proceedings against Luther to be initiated at the curia. They
began before Eck’s arrival in the Eternal City. The report of the disputation
and the opinions of the theological faculties of Louvain and
Cologne did not fail to make a deep impression on the Pope and his
entourage. Crotus Rubeanus testifies that even prior to this, and despite
various delays, the Lutheran affair was by no means regarded
with levity by the Roman curia. With the aid of these official documents,
the difficulties were at last surmounted which ignorance of
the German language had placed in the way of a proper appreciation
of the numerous productions of Luther’s pen.

On January 9, 1520, the first consistory met to initiate definite
proceedings. Cardinal Cajetan, a theological expert of the first rank,
and Cardinal Pietro Accolti, usually called Anconitanus, a distinguished
jurist, were particularly active in the formulation of the
charges against Luther and his Elector, against whom proceedings
were at that time contemplated as an abettor of heresy. With the
active participation of these two Cardinals, the matter was debated
from February to the middle of March by a commission of theologians
and representatives of monastic orders. A smaller commission, under
the presidency of Pope Leo himself, then drafted the Bull, containing
41 theses of Luther which had been selected for condemnation. Eck,
upon his arrival at the end of March, rendered substantial assistance
by the clarifications which he brought. The erroneous opinion of
many who were unacquainted with the question at issue may be
gathered from the fact that, as late as May, 1521, some at Rome held
that the publication of a solemn Bull against Luther would be more
damaging than useful, and that the scandal in Germany would only
increase if it appeared that so much importance was attached to thee
new errors.

Knowledge of the proceedings filtered out and reached Luther at
Wittenberg. He became alarmed. Among the peculiar opinions which
accompanied the report was this, that special difficulties existed, because
the necessary proofs against Luther had to be gathered from
Holy Writ.\footnote{Grisar, \textit{Luther}, Vol. I, p. 48; Vol II, pp- 45--47.}

When, at the end of April, the commission had finished its labors,
its conclusions were presented to the Cardinals for their final decision.
The matter was decided after four consistorial sessions. Cardinal Cajetan
had recommended that the forty-one theses be specified
in strict theological form, together with the condemnation that attached
to each thesis, such as heretical, false, scandalous, etc. The majority,
however, were in favor of condemning the theses as a whole,
without definitely qualifying each separately, following the example
of the Council of Constance when it condemned the errors of Wiclif
and Hus. The commission, moreover, unanimously resolved that the
forthcoming Bull should be principally a condemnation of the false
teachings of Luther, whereas a solemn excommunication from the
Church with its attendant temporal consequences was to be pronounced against
Luther personally only after the expiration of a
certain period of time to be granted to him for the purpose of reconsidering
his position.

At the final consistory, Leo X definitely resolved to promulgate the
Bull which commenced with the words: “\textit{Exurge Domine et iudica
causam tuam}.” It was dispatched on June 15, 1520.\footnote
{Text in Raynaldus, \textit{Annales}, a. 1520, n. 51 and a somewhat inferior reprint in the
\textit{Bullar. Rom.}, ed. Taurin., t. V, pp. 748 sqq. Another reprint, with odious and alien interpolations,
in Luther’s \textit{Opp. Lat. Var.}, IV, pp. 264 sqq. The 41 condemned theses are
reprinted in Denzinger’s Enchiridion, 26th ed., 1928, pp. 257 sqq.}


Most solemn is the introduction to this memorable document, which, in
lieu of an address, was superscribed with the customary formula: ``\textit{Leo episcopus,
servus servorum Dei, ad perpetuam rei memoriam.}” “Arise, O Lord”
--thus it begins--``arise and distinguish Thy cause, be mindful of the insult
which foolish men have heaped upon Thee without intermission \dots Arise,
O Peter, and, mindful of the pastoral office entrusted to thee by God,
be thou solicitous of the Holy Roman Church, the mother of all churches
and the teacher of the faith which thou hast consecrated with thy blood
at the command of God \dots Arise thou also, O Paul, we beseech thee,
thou who by thy teaching and martyrdom hast become the refulgent light
of the Church \dots Let the whole multitude of the Saints arise, and the
universal Church, whose true understanding of Sacred Scriptures is
contemned and trampled under foot,” etc.

There follows the condemnation of the forty-one theses, which are specifically
designated, without mentioning Luther’s name.
Then the Bull proceeds directly against Luther and prohibits his writings,
in which “these (forty-one) theses and many other errors are contained.”
These writings are to be forthwith sought out everywhere and burnt publicly
and solemnly in the presence of the clergy and the people. Luther
himself is ordered to recant officially, or to appear personally at Rome within
sixty days for the purpose of recanting, being given the assurance of a safe
papal escort; otherwise the solemn excommunication was to become automatically
effective against him with all the consequences established by law.
The term of sixty days was to be computed from the time when the Bull was
publicly nailed to the doors of the Lateran, the apostolic chancery, and
the cathedral churches of Brandenburg, Meissen, and Merseburg.

Luther is reminded of the former citation, when he had been promised a
letter of safe-conduct, a friendly reception, nay, even compensation for the
expenses of his journey, and of his defiant attitude for more than a year,
regardless of the ecclesiastical censures which he had incurred by his appeal
to an ecumenical council in violation of the constitutions of Pius II
and Julius II, who had prohibited such an appeal under penalties fixed for
heretics. Hence, proceedings could be instituted against him forthwith as
one “notoriously suspect of heresy, nay, a true heretic.” Still, mindful of the
mercy of almighty God, all the insults which he had heaped upon the Pope
and the Apostolic See would be forgiven him if he would repent, and the
Pope would receive him back lovingly, as the prodigal son in the parable
was received by his father.

At the same time, however, Luther is firmly reminded of the consequences
attendant upon disobedience as implied in the great excommunication and
prescribed by medieval and canon law. In this respect the Bull is strictly in
line with ancient tradition. It mentions, furthermore, that it was the “illustrious
German nation” which had distinguished itself by its loyal and
energetic defense of the faith in the past; that the German emperors, with
the approbation of the popes, had promulgated the severest edicts for the
expulsion and extirpation of heretics throughout their realm, that they had
forfeited the territory and sovereignty of all who protected heretics or refused
to expel them. (This was a strong reminder to the Elector Frederick
of Saxony.) Although there were no certain prospects of success, the
Bull was intended as a reminder of the ancient and severe norms of the
Christian family of nations, now confronted with the greatest menace to
religion that history had ever recorded. Accordingly, all accomplices of
Luther, as well as all who received him, were subject to “the penalties
provided by law” for insubordination. Under pain of spiritual penalties, all
who were invested with authority, in the spiritual as well as in the secular
realm, including the highest Christian princes, were commanded to apprehend
the monk of Wittenberg, if he proved obstinate, and his abettors, and
to have them sent to Rome, or, at least, to expel them from their domiciles.
All localities in which the excommunicates resided, were to be under the
interdict, \textit{i.e.}, closed for divine service as long as they abided there, and
for three days after their departure.

Upon the termination of the above-mentioned respite of sixty days, a
proclamation was to be issued in all churches to the effect that Luther and
all who remained disobedient were to be shunned by all as declared and
condemned heretics; and the Bull “\textit{Exurge}” was to be read and posted
everywhere. All transcriptions of the Bull which were made and subscribed by a
notary public as well as all printed copies coming from Rome and bearing
the seal of an ecclesiastical prelate, were to be regarded as authentic.\footnote
{The only extant copy of the three originals of the Bull was found in 1920 in the State
Archives of Württemberg at Stuttgart. About nineteen original prints are known, of which
an account is given by Schottenloher in the \textit{Zeitschrift für Bücherfreunde}, N. F., Vol. IX,
No. 2, p. 201.}

Whether Dr. Eck approved all of these penal paragraphs, borrowed from medieval
and curialistic tradition, we have no means of
knowing. Relative to the selection and arrangement of the forty-one
condemned propositions he later expressed certain wishes. Of the Bull
in its entirety he heartily approved and assumed the commission, conferred upon him by the Pope, of promulgating it in Germany and
securing its observance wherever possible.

Among the principal doctrines which were declared heretical or
otherwise worthy of condemnation in the forty-one propositions, are
Luther’s errors concerning the utter impotence of man to do good,
fiduciary faith, justification and grace, the hierarchy and the Church,
the efficacy of the Sacraments, Purgatory, Penance, and indulgences.
The denial of the authority of the pope and of general councils--
the centre of the position which Luther adopted--was sharply and
decisively rejected. Thus, this doctrinal utterance of the Holy See
was a. magnificent manifesto for the orientation of all Christendom.
Once more the Apostolic See, in spite of the disturbances caused in
the papal curia by the Renaissance and political intrigues, proved
itself a beacon light amidst the errors that beset society at this critical
juncture. For the rest, the Pope in the Bull “\textit{Exurge}” does not descend
to a refutation of the various condemned errors, but adheres to the ancient
custom of the Apostolic See. Luther’s doctrines had been tested
in the light of Sacred Scripture and tradition and the Pope solemnly
appeals to the promise of the Divine Saviour to abide with His Church
“all days, even to the consummation of the world,” and protect her
from error. He appeals, finally, to the power conferred upon the Holy
See in Blessed Peter and to his authority as the head of the faithful,
which bound him to provide for the peace and concord of the
Church.

If modern sensitiveness takes offense at some forceful phrases of the
Bull, it should be borne in mind that these are almost all passages from
Holy Writ which, in conformity with the usage of the papal court,
were invoked against heresy as the greatest of all evils. Thus in the
introduction of the Bull, which is so frequently criticized by Protestants
, the Supreme Pontiff interweaves a number of Biblical texts,
such as: Fools cast opprobrium upon God, foxes sought to devastate
the vineyard of the Lord, “the boar out of the wood hath laid it
waste, and a singular wild beast hath devoured it.”

The condemnation of Luther’s two propositions concerning the
execution of heretics and the Turkish wars has likewise given offense
to modern writers, but without reason.\footnote
{The condemned 33rd thesis of Luther says: \textit{``Haereticos comburi est contra voluntatem
Spiritus}.'' On the meaning of this condemnation, which has given such grave offense to
modern writers, see N. Paulus in the \textit{Histor.-pol. Blätter}, Vol. 140 (1907), pp. 357 sqq.
Dr. Paulus says that, although the Bull “\textit{Exurge}” is a so-called ex-cathedra decision, not all
the condemned theses are “heretical.” Relative to the 33rd thesis, it is sufficient to assume
the qualification “scandalous” (\textit{scandalosa}), which means not that the proposition is false,
but only that it has some other quality deserving of disapproval. Even if the predicate
“false” were to be applied to it, the meaning nevertheless would not be that to burn
heretics was a work pleasing to the Holy Ghost, but only that it was not contrary to the
will of God, etc. But the censure “scandalous” or “objectionable” suffices. “Such propositions
as are objectionable or provocative of scandal at any given time and rightly censured as such,
may well cease to be objectionable at another time and under different circumstances, and
in that case the censure ceases without further ado as no longer possessing a purpose”
(p. 364). This will be conceded by all, even those who, with the older theologians, look
upon the medieval penalties for heretics as founded upon the public conditions
then prevalent in State and Church. Paulus reminds his readers that Luther himself, in later life,
and the Protestant theologians of the 16th century, acknowledged and demanded the death
penalty for heretics.--The 34th condemned proposition of Luther is on a par with the
33rd: “To go to war against the Turks is to resist God, who punishes our iniquities
through them.” “To-day circumstances are quite different” from what they were in the
age of the Turkish wars and of the Bull ``\textit{Exurge}.'' (Paulus, p. 367).}
