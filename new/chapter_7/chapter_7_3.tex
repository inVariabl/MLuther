\section{The So-Called Two Great Reformation Pamphlets of Luther}

The reports of his forthcoming condemnation by the Holy See
threw Luther into feverish agitation. Notwithstanding his new ideas,
the solemn ban constituted a blow which stirred the depths of his
soul, and which was calculated to alienate from him for all time
many of his adherents among the German people and abroad. In this
frame of mind, moved by fear and hatred, he composed two vehement
polemical pamphlets which were intended to meet the threatening
evil and to weaken, as much as possible, the effects of the sentence
about to be passed upon him. For he had no intention of bowing his
proud head.

The first, written in German, was entitled, ``\textit{[An Address] to
the Christian Nobility of the German Nation on the Improvement
of the Christian Estate.” The second, composed in Latin, bore the
title, “On the Babylonish Captivity of the Church}.''

In “To the Nobility,”\footnote
{Werke, Weimar ed., VI, 381 sq.; Erlangen ed., XXI, PP. 274 sqq. Cfr. Grisar, \textit{Luther},
Vol. II, pp. 26 sq., 31 sq.}
he addressed the estates of the Empire,
especially the governing high nobility, and with inflammatory words
summoned them to intervene against the sins and crimes which the
Roman curia was perpetrating against the Germans and all Christendom.
With the most circumstantial detail he delineates the real or
alleged abuses, the perfectly infernal measures which, so he alleged,
were being adopted at Rome for the sake of gaining wealth and
power, at the expense of German purses and German honor. But he
also unreservedly proclaims doctrines destructive of the Church as
such. Thus he proclaims that the distinction between the clerical and
the secular state is but a hypocritical invention, since all men are
priests in virtue of Baptism; that the hierarchy must be removed, if
necessary by force; and that the civil authorities have power over it.
The office of the temporal authority, to punish the wicked and to protect
the good, he exclaims, should be exercised throughout Christendom
without let or hindrance, even though it strike priests, bishops,
or popes. It is the boast of Luther’s Protestant biographers that he
thus “laid the foundation for the right of civil authority within
Christendom.”\footnote{Köstlin-Kawerau, \textit{Martin Luther}, I, p. 322.}
It would be more correct to say: he indicated the
path that led to an intolerable amalgamation of spiritual and temporal
power. The Church is reduced to servility. In case of necessity, according
to Luther, the civil authority even has the right to convoke a
council; indeed, “as a true member of the whole body, and one able
to do so most effectively, it should provide a truly free council, which
no one is so competent to achieve as the temporal sword, especially
since they are now also fellow-Christians, fellow-priests, and fellow-spirituals.”\footnote{\textit{Ibid}.}

Twenty-five sections of this polemical tract Luther devotes to
the evils of the ecclesiastical estate, and an additional section to the
injuries inflicted upon the temporal public life. Hence, the deficiencies
of the Church, which he assumes the right to reform, constitute the
principal burden of this pamphlet. Essentially, however, it contains
scarcely anything of importance which he had not previously set
forth or at least touched upon. The facts which it stated had already
been acknowledged by individual churchmen and even in public
gravamina. The Protestant historian Johann Haller remarks: “This
is probably the least original of Luther’s writings.”\footnote
{\textit{Die Ursachen der Reformation}, Tübingen, 1917, p. 5.}
On the other
hand untrue assertions are propounded with so much exaggeration,
that the very language in which they were expressed must have prejudiced
against them all who thought dispassionately. At every turn
the scare-crow of the papal Antichrist is visible in the background.
This Antichrist, according to the prophecy of Daniel (XI, 8, 43),
must acquire the treasures of the world and destroy everything. Indeed,
“most of the popes were devoid of faith”!

One of the tricks employed by Luther to gain adherents was his declaration
to the monasteries that he judged they should “become free, everyone
to remain as long as he pleased,” whereas now every monastery was a life
prison for its poor inmates.

His audacious attack upon sacerdotal celibacy, to which “the poor
priests,” as he puts it, were bound by canon law, must have been an equally
effective means of augmenting his strength among the clergy. He would
“freely open his mouth, no matter whether it displease pope, bishop or anyone
else,” and demand that priests be not compelled to live without a legitimate
wife, as they now are by virtue of an institution introduced by the devil
through the pope. By means of this device, he asserted, the pope subjected
the clergy to his avaricious power. “O Christ, my Lord, look down, let the
day of judgment come and destroy the devil’s nest at Romel There is seated
the man of whom Paul has said [2 Thess. II, 3 sq.], he shall be lifted up
above Thee, and sitteth in Thy temple, shewing himself as if he were a
God.”

Thus he justifies his provocative summons to the high addresses of his
book, whom he exhorts: “If we are to attack the Turks, let us begin at
home, where they are most harmful. If it is just that we hang thieves and
decapitate robbers, why should we allow to go unscathed Roman greed,
the greatest thief and robber who has been or may ever be on earth?”

Luther’s “Address to the Nobility” became the most widely read
of his works and has remained so up to the present time. Even where
the rest of his books have long ago fallen into oblivion, this work is
still read as a masterly product of the terrific force which that tribune
of the people wielded by his popular invective. It was a trumpet of
war that resounded throughout Germany, as Luther’s friend, John
Lang, expressed himself in a letter. Toward “others who were close
to him, Luther had to defend himself against the charge that he
sounded the call to revolt.”\footnote{Theo. Kolde, \textit{Martin Luther}, Gotha, 1884, Vol. I, p. 256.}
Conservative Protestants severely criticize this “demagogical book.” “It
is now fairly conceded,” says one
of them, “that Luther, in this book, exceeded the bounds within
which it was his duty to keep.” These critics are horrified at its “revolutionary
admixture.” The course pursued by Luther, says, \textit{e.g.},
F. J. Stahl, “was verily gigantic, even in its negations.”\footnote
{Stahl, \textit{Die lutherische Kirche und die Union}, 2nd ed., Berlin, 1860, pp. 17 sq. Stahl
also coined the phrase “revolutionary admixture.” H. Vorreiter is even more severe in his
criticism (\textit{Luther’s Ringen mit den antichristlichen Prinzipien der Revolution}, Halle,
1860). According to him-Luther’s “Address to the Nobility” is far more destructive than
constructive. His refusal to sever connections with the revolutionary Frankish knights
was a decisive deviation from the path of sound and successful reform (pp. 300 sqq.; 369
sqq.; 377 sqq.; 392 sq.). Leo, Kliefoth, and other Protestants have expressed themselves in a
similar manner, and they are not alone, but have supporters on the Protestant “left.”}

Luther at that time hoped for an intervention of the worldly
power. But it failed to eventuate, and as a consequence, his mind,
now aroused by illusions, suffered a disappointment which he soon
admits. For this reason he begins to pursue his object by other means.\footnote
{Cf. P. Drews, \textit{Entsprach das Staatskirchentum Luthers Ideen?} The views of Karl Holl
(\textit{Luther}, 2nd and 3rd ed., pp. 326 sqq.) on the “Address to the Nobility” are to be
received with caution, as this writer interprets the writings of Luther with a view of
justifying his conduct. Luther was neither always deliberate in his actions nor did he
always “remain true” to himself, as Holl would have us believe.}
Superficially considered, the reforms proposed in the “Address to the
Nobility” appear very fruitful, but when they are closely examined,
they prove to be largely the fruit of the prepossessions of an inexperienced
monk. In the course of this work his attacks gradually
grow more violent and his style more acrid. This is to be ascribed
to the circumstance that, in the course of the work, the author became
more and more exasperated at the reports from Rome demanding
his surrender.

It is probable that the encouragement of the neo-Humanists and
the offers of the revolutionary knights, which reached him about this
time, contributed to the presumptuous tone of the book. Above
all the frivolous and rebellious Ulrich von Hutten endeavored to
make common cause with him. Hutten had written to Melanchthon
that Franz von Sickingen, a famous mercenary chieftain and notoriously
the greatest swashbuckler of the age, who harbored revolutionary
ideas similar to those of Hutten, was prepared to protect
Luther in his castles if necessary. The Franconian knight Sylvester of
Schaumburg also promised to aid Luther until his case was decided
and wrote to him that he would place at his disposal 2 hundred
noblemen for his protection.\footnote{Grisar, \textit{Luther}, Vol. II, pp. 4 sqq.}
Luther would not have run true to
form if such promises had not inspired him with increased boldness.
On July 17, 1520, he wrote to Spalatin: “Schaumburg and Franz
von, Sickingen have insured me against the fear of men; the wrath of
the demons is now about to come.”\footnote{\textit{Ibid.}, p. 5.}
And in a letter to his friend
Wenceslaus Link, the Augustinian, he thus expresses his triumphant
confidence: “To such an extent is the fury of the Romans disregarded by
the Germans.”\footnote{Letter of July 20, \textit{ibid}.}
He now counseled the court of the Elector
of Saxony to write the Pope that Luther had many friends in
Germany who would protect him, despite all bans that might be fulminated
against him, in the event that he should be driven from Wittenberg.

It was his intention to leave Wittenberg, he averred, in order not
to embarrass the Saxon Elector. Nevertheless he was quite certain
that ‘Frederick would declare that the university could not dispense with
him, and that the controversy would have to be decided
by a council. The representations which the Elector had meanwhile
made to Rome did not effect any interruption in the proceedings.
When writing to Staupitz, not long after, Luther boasted that not
only Hutten and many others had written valiantly in his defense,
“but also our prince proceeds wisely, faithfully and, at the same time,
steadfastly.”\footnote{Grisar, \textit{Luther}, Vol. II, p. 8.}

Prior to the arrival of the Bull of excommunication in Germany
Luther, at the advice of the Elector, addressed a letter to Charles V,
in which he sought to induce him to extend his protection to him,
entirely innocent as he was, against the machinations of his enemies.\footnote
{August 30, 1520; Grisar, \textit{op. cit.}, Vol. II, p. 20.}
This letter with the supplement (“Oblation or Protestation”) which
Luther appended to it, is an example of that political art of concealment
of which the correspondence of Luther with Spalatin and the
court of Frederick offers so many examples.\footnote
{Cf. Grisar, \textit{Luther}, Vol. II, pp. 15 sqq.}
Luther at once published
it together with the “Oblation,” in Latin for the benefit of
readers in other countries.

In the “Oblation” Luther asserts his submission to the holy Catholic
Church, as whose devoted son, he says, he wishes with the help of God to
live and to die.

To the Emperor, however, he writes that he was forced to go before the
public contrary to his wishes, that the hidden life of the cell was the supreme
ideal of his life, and that his only desire was to serve the truth against those
who in their frenzy disdained it. “In vain do I plead for forgiveness, in vain
do I offer to observe silence, in vain do I propose conditions of peace, in vain
do I demand to be better instructed.” To obtain such instructions and to
be convinced by proofs supplied by competent judges, he now appeals to
the emperor, before whom, as the king of kings, he humbly appears as an
“insignificant flea.”

Charles V tore up this letter with his own hands at the diet of
Worms. It deserved no better fate, especially in view of the subsequent
events.

Towards the end of August, when Luther had signed his letter to
the Emperor, he had already in print a part of a second polemical
tract, which ranks worthily beside his “Appeal to the Nobility.” It
was his Latin work \textit{De Captivitate Babylonica}.\footnote
{\textit{Werke}, Weimar ed., Vol. VI, pp. 484 sqq., Erlangen ed., \textit{Opp. Lat. Var.}, V, pp. 13 sqq.;
Grisar, \textit{Luther}, Vol. II, p. 27.}
In the introduction
to this work he declared the papacy to be the empire of Babylon. and
repudiated the hierarchy and the entire visible Church. According to
this book, the Church has been delivered into Babylonian captivity
because her doctrines have been falsified and her Sacraments held in
bondage. To her falsehoods he opposes his own teachings, derived
from the Word of God. He is, for the present, concerned chiefly with
dogma. Because of its Latin garb, this book, intended for scholars
and foreigners, is not composed in the style of its predecessor, but is
prolix and ponderous, as it is intended to be a scientific attack upon
the doctrines of the Church.

In his denial of particular dogmas the author advances beyond his
previous position, quite in conformity with the principle laid down
by him, that if one meets with contradictions, one should advance
all the more boldly! The opponents are to be confused and overwhelmed
by new assertions.

Above all, the holy Sacraments are to be rescued from the captivity
of the papacy. It is an unfortunate and injurious error to hold that
there are seven. There are only three, namely, Baptism, Penance and
the Eucharist, and these are efficacious through faith alone. As regards
the Eucharist, the doctrine of Transubstantiation must be rejected;
the bread remains unchanged, only Christ becomes simultaneously present
with it. Christ did not prescribe the reception of the
Sacrament. The denial of the chalice to the laity is a mutilation of the
Eucharistic banquet. The Mass is no sacrifice; nay, not even a meritorious
work. The commandments of the Church are contrary to
freedom. The Church may not invite vows. Since the so-called Sacrament
of Matrimony is a fraud, the entire marriage law must be
abolished. The celibacy of the clergy is a damnable institution. And so
forth.

It is not worth while to follow up these assertions in detail. They
are the fragments of the foundation which Luther had wrecked by his
denial of the authority of the Church.
It is more important to establish the fact that he is approaching
a complete disintegration without being aware of the fact. With
closed eyes he blows up ecclesiastical and religious subordination and
the certain outward tradition of positive truth. “Neither the pope,
nor a bishop, nor any one else,” he declares in his zeal for destruction,
“has the right to impose even a syllable upon any Christian without
his consent.” He deduces this freedom of the faithful solely from
Baptism and its obligations toward God and laments “that few know
this splendor of Baptism and this boon of Christian freedom.” He
holds, moreover, that faith originates only in the interior sense of the
individual who reads the Bible under the activity of God. There is
no need of Church authority; God’s Word suffices to make everyone
interiorly certain. He himself claims to be conscious of such certainty.
It is his opinion, though he does not express it too freely, that all inquirers
will agree with him if they permit themselves to be properly
directed from above. This was not the case, however, as may be seen
from his subsequent bitter complaints about the hordes of sectarians
and fanatics within his own church. It is a trait of the power of self-deception
which was characteristic of him and which, at times, almost
resembles naïveté. The book concludes with invectives against the
“despotism, craftiness, and superstition” of the pope, whose adherents
are characterized as filled with “stupid ferocity.”

What impression did these pages make upon foreign Catholic
readers, who were unable to understand the spirit of this “Teuton”?
Ambrose Catharinus, a Dominican, who at that time (1520) had
already composed his “Apology against Luther’s infamous pest of
doctrines,” regarded Luther as the “intellectual monster” of Wittenberg
. The German Franciscan, Thomas Murner, a satyrically inclined
antagonist of Luther, undertook to ridicule his \textit{Babylonish Captivity}
in a German translation. In the same year, 1520, another German
translation of the same work was published by an adherent of the
new religion, under the title, \textit{Von der Babylonischen gefengnusz der
Kirchen}.

Luther had given to his Latin work the subtitle \textit{Præludium, i.e.},
prelude. This prelude and the concluding words indicated that it was
to be regarded as the forerunner of another. The author had meanwhile been
apprized of the tenor of the Bull directed against him.
With the approach of the storm his soul seemed to be endowed with
demoniacal power. He concludes by stating that when the Bull arrived, he
would, with the help of Christ, issue a sequel “such as the
Roman See has hitherto neither seen nor heard,” and that the present
work might be regarded as a part of his future retraction, lest tyranny
appear to have puffed itself up in vain.
