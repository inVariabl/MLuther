\section{“The Bold, Lusty Lie”}

This subject demands a special note. It was part of Luther’s theological
system. Strange as this statement may sound, yet his attitude
towards lying is based upon principles which he formally defended.\footnote
{For the following cf. Grisar, \textit{Luther}, Vol. IV, pp. 80-178.}
He taught, and endeavored to demonstrate on diverse occasions, that
lying is permitted as a matter of expediency or of necessity, provided
that it redounds to the advantage of the new Evangel or to
the real benefit of others. He excludes only the lie that works an
injury. He proposed this theology of lying as early as 1524. Points
of contact with the past were not entirely wanting, notwithstanding
the contrary teaching of St. Augustine on the unlawfulness of lying
in every form. But never before was the lawfulness of lying brought
into a system.

By degrees, Luther reduces the lie of convenience or necessity to a
virtue. “Lying is a virtue,” he says, “if it is indulged in for the purpose
of preventing the fury of the devil, or made to serve the honor,
the life, and the welfare of one’s fellowmen.” He likewise regarded it
as permissible if intended to secure a personal advantage pleasing to
God or, in general, to promote His glory.\footnote{\textit{Ibid.}, pp. 108 sqq., 116 sqq., 131 sqq.}
 In confirmation of his
attitude he repeatedly appeals to misconstrued examples from the Old
Testament.

In the long war which he waged upon the Catholic Church, and
which he believed to be for the glory of God, he so habituated himself
to the application of his fundamental principle, \textit{viz.}, that everything
was permissible in the warfare against Antichrist, as to feel no reluctance
in resorting to notorious falsehoods. Indeed, it is probable that,
owing to his peculiar practice of auto-suggestion, he finally believed
his unfair and offensive inventions, in consequence of his frequent
repetition of them, especially since they offered him an apparent composure
in his qualms of conscience. The author has collected a veritable
arsenal of untrue assertions made by Luther, especially against the
“papists,” in his larger work on Luther, in which he has also offered a
psychological explanation of the strange phenomenon of Luther’s
mendacity and tried to give an insight into the infectious results of
his lying.\footnote{\textit{Ibid.}, Vol. IV, pp. 80 sqq.}
