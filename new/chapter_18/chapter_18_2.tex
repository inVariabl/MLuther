\section{Matrimony and Sacerdotal Celibacy}

We shall not revert here to the examples which have been previously adduced
to show how Luther availed himself of the lure of
the married state in order to gain adherents among priests and religious.
Nor shall we refer to his aforementioned writings on his
favorite theme of matrimony, such as his treatises “On the Matrimonial
Life,” “On Things Matrimonial,” and his interpretation of
the seventh chapter of the First Epistle to the Corinthians. We will
consider only a few of the general principles and viewpoints prominent in these works.

Luther’s doctrine on matrimony cannot be treated as a complete
system because of the author’s numerous contradictions and vacillations.
It is manifest from many ardent expressions that Luther regarded Christian
matrimony as an exalted state of life. His own marriage,
which had been contracted in defiance of the Church laws,
afforded him frequent opportunities of eulogizing matrimony as an
institution ordained by God. He waxes enthusiastic in emphasizing its
chief purpose, namely, the procreation of children for the welfare of
State and Church, even though at the same time he continually exaggerates
the danger of incontinence as its most urgent motive. In a
hundred passages he describes how married life operates as a stimulus
to good works, how it protects the faith of husband and wife, awakens love,
and fosters discipline and domesticity. He delineates family
life in such captivating terms that the single life appears quite unattractive.
The moral features which he delights in mentioning in
the course of these descriptions have led Protestant writers to say
that Luther’s views on matrimony spell the very apex of morality.
Why disregard the sensual and dangerous aspects of his teaching and
example?

It was fatal to Luther’s teaching on matrimony that it was the
product of a twofold struggle--that against the state of virginity
and that against the authority of the Church.

The sexual and sensual admixture of his doctrine derives from his
antagonism to the state of virginity. His hostility to the sacerdotal
and monastic states leads him to degrade celibacy beyond all measure.\footnote
{Cfr, Grisar, \textit{Luther}, Vol. III, pp. 241 sqq., and the excellent work of S. Baranowsk
\textit{Luthers Lehre von der Ehe} (1913), pp. 34 sqq.}
In order to assail the Catholic position more effectively he asserts,
as the starting-point of his own doctrine, that the sexual instinct in
man operates as an irresistible law of nature and tolerates no
restriction in the form of vows, which can result in unchastity. In
his invectives against the vow of chastity human dignity and decency are
set aside. The means of grace offered by the Church for the
successful conquest of the sexual instinct and for leading a higher life,
are ignored. Nature, in his opinion, compels practically all men to
embrace the matrimonial state. It is a “miracle” if anyone is able to
live continently.\footnote{Cfr. Grisar, Luther, Vol. III, pp. 246 sqq.}
For the rest, he goes much farther than the
Catholic Church by declaring the sexual instinct sinful in itself. He
refuses to acknowledge that the involuntary movements of the sexual
instinct are no sin, and that it is virtuous to resist them for the sake
of God or to keep a deliberate vow.

His position was influenced not only by his antagonism to sacerdotal
celibacy and the religious vow of chastity, but likewise by his
fight on the authority of the Church. Her venerable and salutary
traditions for the protection of the matrimonial state and family life,
which she ever esteemed most highly, counted for nothing in his
eyes. He rejects without investigation the teaching of the Church
on matrimonial impediments and divorce. An artificial Biblical or a
merely natural argument is sufficient for him to open a wide road to
what he calls Christian liberty. The efforts made at the present time
to abolish marriage as a social institution were unconsciously inaugurated
by Luther when he denied the time-honored authority
of the Church in matrimonial matters.

In his system, which did not recognize matrimony as a sacrament,
the Church, which regulates and administers the sacraments, was replaced
by the secular authority. His original endeavor to regulate
matrimonial matters with the aid of his preachers and pastors, was
defeated by a multiplicity of problems and controversial cases. The
State usurped authority in this sphere and Luther favored this tendency.
He could not consistently have done otherwise after he had
declared matrimony to be a purely secular affair. But in doing so he
nevertheless created a contradiction which cannot be spanned; for,
on the one hand, he praises the matrimonial state as “most holy,” and
on the other, he divorces it from the Church, to whom holy things
are subject.\footnote{Baranowski, \textit{op. cit.}, p. 177.}

In vain one seeks to find in Luther a true concordance between the
service of the world and the service of God in matrimony. There is
discord and antithesis everywhere. At times he extols sexual intercourse
in matrimony as a lofty divine service; then again he characterizes it
as religiously indifferent, nay, even as stained with sin. He exhorts
parents to be one in their prayers and in the Christian education
of their offspring, yet asserts the validity of marriage between Christians
and pagans, because this does not affect the faith. He arbitrarily
relaxes the ties of matrimony, and at the same time unduly contracts
the duties of the domestic sphere; for the subjection of the wife to
the authority of the husband, and that of the children to the will of
the parents, as advocated by him, appears to exceed the bounds of
what is lawful to personal conduct and individual self-determination.\footnote
{\textit{Ibid.}, p. 209. Similarly an examination of Luther’s utterances on vocation would
show how confused were his views of marriage and celibacy, and also of the secular and
the spiritual vocation. On “vocation” in the Middle Ages and Luther’s idea of it see
N. Paulus in Histor. \textit{Jahrbuch}, Vol. 32, pp. 725 sqq., and Vol. 45, pp. 308 sqq.}
His attitude on the latter subject provoked the celebrated,
lengthy and violent controversy on the validity of marriage contracted
without parental consent, which consent he designated as
necessary for a valid union of the children.

Even more subversive were his principles concerning divorce.\footnote
{Grisar, \textit{Luther}, Vol. IV, pp. 3--79; Baranowski, \textit{Luthers Lehre von der Ehe}, pp. 115 sqq.}

As early as 1520 Luther refused to assert the indissolubility of the
marriage tie. As time went on, the complaint raised against him became
ever more justified, that (to quote his own words) “he arbitrarily trifled
with the dissolution and the confirmation of matrimony.”\footnote{\textit{Briefwechsel} VIII, p. 398.}

Though he regards divorce as a serious matter, injurious to the Christian
polity and the State, he finds that adultery is an immediate
ground for divorce, with liberty to remarry. After breaking with
the Biblical doctrine and with tradition in this critical matter, he
subsequently proposed a second ground for divorce, namely, willful
desertion. He did this in order to come to the aid of those unfortunates
who had been forced to adopt celibacy. Other grounds for divorce
recognized by him are persistent irascibility and violent
incompatibility of temper. If either one of the parties concerned cannot
restrain himself, he says, “let him (or her) woo another in the
name of God.”\footnote{Baranowski, \textit{op. cit.}, p. 124.}
This was the extent to which the idea of the irresistibility
of the sexual urge had led him. As a matter of course,
Luther permits divorce (\textit{divortium}) where the life of one conjugal
partner is jeopardized by the other; but the passage in question does
not clearly indicate whether or not he means a complete dissolution
of the matrimonial bond. It frequently appears that he is little concerned
with the important distinction between a complete dissolution
of the marriage bond and a mere separation from bed and board. He
extends the so-called Pauline Privilege to Christian couples and to
cases where one party urges the other to “unchristian conduct,” to
“theft, adultery or any unrighteousness towards God.” But, as regards
these matters, he also repudiates the civil authority, which ought to
devise remedial measures. He holds that physical impotence not only
dissolves matrimony where it previously existed, but also when subsequently
contracted, even in the case of marriages blessed with
offspring.

There are two other grounds for divorce which he admits. In the
case of obstinate refusal to render the \textit{debitum}, the injured party may
enter upon a new marriage. “If you are unwilling,” to quote the easily
misconstruable and insidious assertion of Luther, “then another
shall; if the wife is unwilling, then the maid shall come.” At all
events, Luther holds that, in case of refusal, the secular authorities
ought to intervene. Finally, he also regarded serious illness, especially
leprosy, as an adequate ground for divorce, at least in the internal
forum of conscience, thereby proposing a principle which, according
to a recent Protestant critic, ``is apt by its consequences to shake the
institution of matrimony to its very foundations.''\footnote
{\textit{Theol. Studien und Kritiken}, 1881, p. 445.}
If this be true
of this one ground, what must be said of the collective effect of all
the grounds that have been mentioned?

In the practical application of these ideas Luther mingled the
strangest contradictions. He requires the verdict of the civil authorities
for the validation of divorce, yet regards marriage as already dissolved
and to be treated as dissolved in secret. He grants the right of
remarriage to one party and at the same time denies it to the other.
In the forum of conscience he concede grounds for divorce which he
refuses to defend in public, and so forth.\footnote{Baranowski, \textit{op. cit.}, p. 131.}

In his narrow purview he discovers ecclesiastical domination and
pretensions, if not avarice, in the traditional diriment impediments
upheld by the Church. He particularly rejects the dilatory impediments.\footnote{Weimar ed., Vol. X, ii, p. 287.}

“Freedom,” he says, “may not be abolished by
the superstition and stupidity of others.”\footnote{\textit{Ibid.}, VI, p. 558.}
 Suffice it to mention that,
on the basis of the Old Testament, he regards only the second degree
of consanguinity and the first degree of affinity as impediments arising
from relationship, and that on other occasions he indirectly exempts even
the first degree of affinity. The precedent of an Old Testament patriarch
counts for more with him than “100,000 popes.”\footnote{\textit{Ibid.}, XVI, p. 405.}

The precedent of the patriarchs also confused his views of the
unity of marriage. While he would not tolerate the introduction of
bigamy, he nevertheless, as indicated elsewhere, allowed a certain scope
to it.\footnote
{Grisar, \textit{Luther}, Vol. IV, pp. 13 sqq.; Baranowski, \textit{op. cit.}, p. 162 sqq.; Rockwell, \textit{Die
Doppelebe Philipps von Hessen} (1904), pp. 247 sqq.}

As early as 1520 he gave expression to this sentimental inclination
of his, preferring bigamy to divorce in case of necessity.\footnote{Weimar ed., Vol. VI, p. 559.}

In 1524 he expressed himself more decisively toward Chancellor
Brück: “I admit that I am unable to prohibit a man from marrying
several wives; it does not contradict Holy Writ.” But scandal and
sound ethics, he adds, establish objections to the practice.\footnote{\textit{Briefwechsel}, IV, pp. 282 sq.; January 27, 1524,}
 He
repeats these statements frequently. In the subsequent account of the
bigamous marriage of Landgrave Philip of Hesse, the consequences
of Luther’s and Melanchthon’s attitude will be seen. This one case
clearly reveals the fact that the so-called reformers “lacked a comprehensive
insight into the true ethical nature of matrimony.”\footnote{This pertinent observation is by Baranowski, \textit{op. cit.}, p. 168.}

How different in this respect are the Middle Ages, particularly the
time which preceded Luther, with its numerous popular treatises on
matrimony as a sacrament--writings which abounded in attractive
profundity and solid theological content. In the tender delineations of
domestic life found in these “marriage booklets” justice is accorded
the human side of the marital relationship and protection is afforded
the sublimity and purity of this institution by the faithful reproduction
of the precepts of the Church.\footnote{Grisar, \textit{Luther}, Vol. IV, pp. 135 sqq.}

The Protestant claim that the esteem of womanhood originated
with Luther is entirely unfounded. The dignity of woman, her social
mission, the esteem and veneration which her position demands, could
not be more thoroughly effective than in a society firmly founded
on a religion which appreciated and extolled virginity equally with
motherhood, and honored the supreme type of motherhood in Mary,
Virgin and Mother, the protectress of Christendom.\footnote{\textit{Ibid.}, pp. 131 sqq.}

It is impossible to follow the utterances of Luther on marriage and
sexual matters without ever and anon being repelled by the vulgarity
of his language and his sensuality.\footnote{\textit{Op. cit.}, Vol. III, pp. 264 sqq.}

For example, here is a sentence which surely does not honor womanhood:
“The word and work of God is quite clear, \textit{viz.}, that women were made to
be either wives or prostitutes.”\footnote{\textit{Ibid.}, p. 243.}
 We add a few others: “Had we opportunity,
time, and occasion,” he says in his bombastic manner, “we should all
commit adultery”; he thus intends to indicate the power of concupiscence,
and then continues: “We are so mad, when once our passions are aroused,
that we forget everything.”\footnote{\textit{Ibid.}, p. 245.}
 Marriage ought to be contracted by “a boy
not later than the age of twenty, and a girl when she is from fifteen to
eighteen years of age. Then they are still healthy and sound, and they can
leave it to God to see that their children are provided for.”\footnote{\textit{Ibid.}, p. 246.}
 “Even though
one may have the gift to be able to live chastely without a wife, yet he
ought to marry in defiance of the pope, who insists so much on celibacy.”\footnote{\textit{Tischreden}, Weimar ed., II, n. 2129b; cf, a; see also Grisar, \textit{Luther}, Vol. III, p, 246.}
--“Were all those living under the papacy kneaded together, not one
would be found who had remained chaste up to his fortieth year.”\footnote{\textit{Ibid.}, p. 251.}
--I am satisfied that the saints stick in the mud just like we do.”\footnote{\textit{Ibid.}}

Certain indecorous German utterances of Luther are reproduced in Latin
in the present writer’s more exhaustive work.\footnote{\textit{Luther}, Vol. III, pp. 251 sq., n. 3.}
 In a letter of December 6,
1525, Luther speaks of his marriage and that of Spalatin in a manner that
is not fit for reproduction. The older editors (Aurifaber, De Wette, Walch)
omitted the passage in question.\footnote{\textit{Ibid.}, p. 269, n. 2.}

It is a notorious fact that the undignified vulgarity of Luther’s language,
spiced with sexual allusions, attains its height in the objurgations which he
metes out to the papacy and the Roman Church.\footnote{\textit{Ibid.}, pp. 265 sqq.}
Thus the pope is compared
with the detestable pagan god, Priapus. In giving vent to such utterances,
Luther, as his excuses demonstrate, was quite conscious that he had
transcended even the freedom which his coarse age was wont to grant. In
1541 he writes against the loyal Catholics:\footnote{Grisar, \textit{Luther}, Vol. VI, p. 331.}
 “You are the runaway apostate,
strumpet Church as the prophets term it”; “you whoremongers preach
in your own brothels and devil’s churches”; “your conduct is such as if the
bride of a beloved bridegroom were to allow every man to abuse her at his
will.” Unable to satiate himself with this image, he continues: “This whore,
once a pure virgin and beloved bride, is now an apostate, a vagrant, a whore, a
house-whore,” etc. “You old whores bear in your turn young whores, and so
increase and multiply the pope’s Church, which is the devil’s own.” “You reduce
many true virgins of Christ, who have been regenerated by baptism,
to arch-harlots.”\footnote{\textit{Ibid.}, p. 332; cfr. III, pp. 270 sqq.}
