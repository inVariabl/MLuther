\section{The Bigamy of Philip of Hesse}

On December 9, 1539, Martin Bucer visited Luther in Wittenberg
and presented him with a request by the Landgrave Philip of
Hesse for an opinion sanctioning his intended bigamous marriage with
Margaret von der Sale.\footnote
{For proofs pertaining to this section let it suffice to refer in general to my lengthy
exposition in \textit{Luther}, Vol. IV, pp. 13-79. In a few cases only are references given here.}

Luther and Melanchthon were alarmed at the disclosure made by
Bucer in accordance with his written directions. The argument which
the Landgrave advanced in his petition was that, in consequence of
the immoral life he had hitherto led, he was constrained by his conscience
to take unto himself another wife, in addition to the one he
already had, as a substitute for the “debauched women” with whom
he had hitherto consorted. It was his desire that the new marriage, as
well as the formal opinion of Luther and Melanchthon permitting the
same, be “publicly proclaimed to the world” by gradual stages, so that
his second wife “be not regarded as a dishonorable person.” What
alarmed the two leaders of Protestantism most was Philip’s threat
that, in the event of their non-acquiescence, he would appeal to the
Emperor; in other words, that he, in complete contradiction to the
attitude which he had hitherto observed, would endeavor by means
of concessions to obtain the favor of the most hated and most powerful
opponent of Lutheranism, in order to procure from him toleration of his
step, notwithstanding the severe imperial laws which prohibited
it. Plainly, Philip contemplated a fatal betrayal of the Protestant cause.

Luther and Melanchthon had previously expressed themselves in
favor of the permissibility of bigamy in particular cases.\footnote{Grisar, \textit{Luther}, Vol. III, pp. 259 sqq.}
 They believed
that there was justification for it in the Old Testament, citing
the cases in which polygamy was permitted by God in exceptional instances.
In certain cases, Luther, too, as he himself says, had counseled
the contraction of bigamous marriage, when, for instance, husbands
complained of severe and incurable disease of their wives or of their
refusal to render the \textit{debitum}. Philip of Hesse was acquainted with
these public pronouncements as well as with Luther’s proffer of bigamy
to Henry VIII of England. Martin Bucer cited all these facts
with persuasive eloquence.

On the other hand the Wittenberg theologians were well aware
that polygamy had been abolished by the divine Founder of the
Church in the New Testament. Even though they were prepared to
tolerate exceptions in extraordinary cases, they yet held that polygamy
ought not to be reintroduced generally. Thus Melanchthon asserted
that the words of Christ, “they two shall be one flesh” should
be observed as a “universal Christian law.”\footnote
{Cf. Rockwell, \textit{Die Doppelebe Philipps von Hessen}, p. 194; Camerarius in \textit{Corp. Ref.},
III, pp. 1077 sq.; Grisar, \textit{Luther}, Vol. IV, pp. 62 sq.}

Philip’s demand for permission to contract a bigamous marriage
appeared to open the gates to polygamy. Hence, the embarrassment
which seized both Wittenberg reformers at the unheard-of proposition of
the Landgrave.

Philip, who had just partly recovered from a severe venereal attack, had
cast his eyes upon Margaret, the seventeen-year-old
daughter of the lady in waiting of his sister, the Duchess Elizabeth
of Sachsen-Rochlitz. He obtained a promise from Margaret’s ambitious mother
that his desire would be gratified, but only on condition
that Margaret would become his wife and true landgravine, not
merely a despised concubine. This unsavory plan, coupled with the
aforesaid condition, was furthered by the Protestant pastor of Melsungen,
John Lening, an apostatized Carthusian monk, who was
reputed to be leading an immoral life himself. The afore-mentioned
sister of the landgrave, however, violently opposed the bigamous
marriage as soon as she became aware of it, not indecd for any ethical
motive (her own moral conduct as a widow was blameworthy), but
because she regarded her brother’s marriage to the daughter of her
governess as a disgrace to the reigning family. Philip’s declaration
that, after a dissolute life, he was constrained to ease his conscience by
means of a new, duly contracted marriage, was but a pretext which
served as a cloak to cover his unrestrained sensuality. He despatched
the complaisant physician Sailer, also a Protestant, to win over for
his plan the theologian Bucer, whom he had selected to conduct the
negotiations at Wittenberg. Sailer writes that Bucer was “highly
amazed” when he communicated to him the invitation to visit Philip
regarding this affair, but that he finally consented to come and act
as mediator, to avert the defection of the Landgrave from the Protestant
cause. Bucer undertook the mission and received written instructions
from the prince, of which the text is still extant. The authentic
text of the reply made by Luther and Melanchthon is preserved in
the government archives at Marburg.\footnote
{Printed in Luther’s \textit{Briefwechsel}, Vol. XII, pp. 326 sqq. An excerpt in Grisar, \textit{Luther},
Vol. IV, pp. 19 sq.}

On December 10, the day after the arrival of Bucer, they delivered
to him the fatal document which had been composed with remarkable haste
by the skilled pen of Melanchthon. It styles itself a “testimonial,” and
states that the contemplated marriage is not contrary to
the law of God and may be entered upon by the Landgrave because of
a “necessity of conscience.”\footnote
{The document declares that, in a very special case, “a husband” might ``take another
wife with the advice of his pastor.''}

The deponents demand that the new
marriage, as well as the “testimonial,” should remain secret, in order
to avoid scandal and to prevent polygamy from becoming general.
They might have foreseen that this desire was destined to remain unfulfilled
in view of the declared intention of the autocratic Landgrave
to divulge the entire matter.

The document is not devoid of sound, moral exhortations, but, on the
other hand, the alleged divine “dispensation” to contract a bigamous marriage
is treated as a sort of initiation of the petitioner into a “retired state”
with the intimation that the marriage with Margaret would entail “no particular
scandal,” since the people would regard her as a concubine, and
concubines were not uncommon in many courts. At the very beginning of
the document, the Landgrave is invited to continue to act as loyal protector
of the new religion and to hold himself aloof from the imperial party. The
conclusion contains an angry remark of Luther charging the Emperor with
being utterly devoid of faith and aiming only at mutiny in Germany; pious
Christians, Luther contended, are forbidden to associate with him.
It is evident that the acquiescence of the two Wittenberg reformers was
dictated by their desire to retain Philip in his réle of protector of their party.

The other protector of the Protestant cause was the Elector John
Frederick of Saxony. Bucer, gratified with his success, at once repaired
to his court in order to communicate Philip’s plan and the
Wittenberg ``testimonial'' to John Frederick and to put him in a
favorable mood by various political promises. Due to his powers of
persuasion, Bucer succeeded in obtaining the promise of the Elector
“to give his fraternal aid at all times” to Philip in this matter.\footnote{Grisar, \textit{op. cit.}, IV, pp. 23 sq.}

On December 23, Philip was in possession of the “testimony” of
the two theologians and the favorable reply of the Elector. He caused
the theological opinion to be subscribed by his own Hessian theologians,
in order that it might carry greater weight. It was signed by
Lening, Melander, Corvinus, and three other Protestant ministers. The
solemn nuptials were celebrated on March 4 in the chapel of the castle
of Rotenburg on the Fulda, in the presence of Bucer, Eberhard
von der Thann, who represented the Elector of Saxony, and various
other witnesses. Melanchthon also, after a heated argument with the
participants, graced the occasion with his presence. Thus, with the
aid of the theologians, Philip had taken a step which was fraught
with serious consequences.

The new princess was sent to the castle of Wilhelmshshe, because
the Landgrave was still intent upon secrecy. But the impossibility of
concealing the marriage soon became manifest. Too many knew the
secret. Thus, when Philip sent a barrel of wine to Luther, as a mark
of his gratitude, and also remembered Catherine with a gift, the
mayor of Lohra openly discussed the destination of the wine in the
presence of all the peasants and declared he “knew for certain the
prince had taken a second wife.” The courts and the aristocracy were
informed of the marriage principally through the sister of the Landgrave,
Elizabeth von Rochlitz, who was greatly agitated over and
vehemently protested against it. Amid tears she proclaimed that
Luther and Bucer were consummate villains. The ducal court of Saxony also
was apprehensive and indignant. The Elector now began to
fear that the Emperor would interfere, on account of the general scandal,
especially since the news had reached King Ferdinand and Rome.

It was recalled that, as recently as 1532, the code of laws known as
“Carolina” had prescribed “capital punishment” for bigamists.

Due to the universal indignation, the Landgrave, awaking from his
dream, began to speak of a reconciliation with the Emperor, nay,
even with the Pope. Bucer and his apprehensive Hessian theologians,
who were joined by Schnepf, Osiander, and Brenz, urged Philip to extricate
himself from his embarrassing situation by publicly passing off
Margaret von der Sale as his concubine, and not as his wife, and to
have a new and suitable contract drawn up instead of the matrimonial
certificate inscribed at Rotenburg; thus, they thought, he might be
able to silence the hostile court of Dresden and other opponents. The
Landgrave declined, saying that God never permitted lying and that
he expected a change of public opinion from the publication of the
“testimony” of the Wittenbergers.

This dreadful threat and the whole embarrassing situation promptly
became known at Wittenberg, and on June 10, 1540, Jonas wrote to
George von Anhalt that Melanchthon was “very much perplexed
and Doctor Martin full of thought.”\footnote{\textit{Ibid.}, p. 36.}
 Luther’s predicament increased
when his own elector became very apprehensive and indicated to him through
Chancellor Brück that he had gone too far,
as universal bigamy might result from his conduct. Luther hit upon
a way of extricating himself from this dilemma by suggesting that his
“testimony” to Philip of Hesse be represented as a secret advice given
in the confessional and consequently subject to the seal of confession.
He wrote to his ruler that, even if the Landgrave would publish the
document, he would not be ashamed of his Biblical standpoint nor of
his advice in an extreme case of conscience, “even should it come before
the world.”\footnote{\textit{Ibid.}, p. 37.}
 Nevertheless, the danger of publication continued
to be a source of terror to him. It does not redound to his credit that
he assured the elector that he was not aware at the time he drafted his
“testimony” for Philip of Hesse that the noble lady of Eschwege
was also at the disposal of the petitioner as a concubine and that he
did not expect a new princess, but had hoped that the Landgrave
would only “keep an honorable maiden secretly in clandestine marriage
to satisfy the great necessity of his conscience, even though it
had an illegitimate appearance before the eyes of the world,” since
he (Luther) had given the same advice to various pastors and bishops
relative to their housekeepers.

There could be no question here of the seal of confession, though
Luther cites the words “confession” and “advice given in confession”
as often as three times in this letter. In matter of fact neither the
Landgrave nor anyone at Wittenberg thought of confession. What
Philip desired was not absolution, but something quite different. And
where was there an auricular confession in the ecclesiastical sense
which would have entailed the seal? Where was the Landgrave’s willingness
to perform any action demanded by the secrecy of the confessional, in
lieu of the publicity desired by him? Only a natural
obligation of secrecy might arise, just as in the case of any delicate
and confidential transaction; but this obligation was annulled by the
conduct of Philip, who did not care that the sordid reasons for his
``necessity of conscience'' became even more widely known than they
already were.

In this dilemma Luther, on June 27, recommended to the Hessian
courtier Eberhard von der Thann that, if hard pressed, the Landgrave
should deny the whole affair and declare to the Emperor that he had
merely taken a concubine.\footnote{\textit{Ibid.}, p. 40.}
 About the middle of July Luther wrote
in a similar vein to another Hessian councillor, who has been identified
as the Chancellor John Feige, asking him to state that the Landgrave had
contracted no secret union and assuring him that he had
answered inquirers by stating that “the Landgrave’s other marriage is
all nonsense.” This, continues Luther, he was justified in doing, on
the theory of the secrecy of confession. At the same time he warned
the Chancellor that he would strongly resent it if Philip would undertake
to make public his (Luther’s) “testimonial,” and that he (Luther) would
know how to “extricate’ himself from the quandary. He
admitted the impossibility of defending the bigamous marriage “before
the world \textit{iure nunc regente}.”\footnote{\textit{Ibid.}, p. 471 sq.}


Luther’s agitation at this time is reflected in his familiar discourses,
especially the Table Talks.\footnote{\textit{Ibid.}, p. 43 sqq.}
“I am not pleased with what has happened,” he
laments; “would that I could alter it!” “Would that it might not become
more aggravated!” Since this trial has been imposed upon us by God, “we
must put up with the devil and his filth.” etc. “The papists may deride us;
they, however, merit still less pardon on account of their infidelity.” In his
perplexity he consoles himself with the impending decline of popery. In his
habitual manner, he devises acrimonious witticisms: “What do the papists
intend to make of this incident? They kill men, whereas we labor in behalf
of life, and take several wives.”

Luther’s chief source of worry is the fear that the Landgrave might come
to an understanding with the Emperor and desert the party of the reformers.
In mentioning this danger, he exclaims: “He is a strange man”; “he was born
under a star; he is bent upon having his own way.” It is noteworthy that
both Luther and Melanchthon repeatedly suggest the prevalence of hereditary
madness in the family of the Hessian ruler. On one occasion Luther
said: “This is a fatal curse in his family.” Melanchthon said that “this [the
bigamous marriage] is the beginning of his [Philip’s] insanity.”

The haughty Landgrave had undoubtedly at first believed that he would
have the whole Protestant world behind him in his bigamous adventure, and
that, protected by public opinion, he could afford to ignore the supreme
court and the Emperor. The disappointment which he experienced and his
subsequent clashes with Luther were all the more apt to impel him to seck
a reconciliation with the Emperor.

The fear lest Philip should desert their party and the disgrace resulting
from the Landgrave’s bigamous marriage affected Melanchthon to such an
extent that he became seriously ill in Weimar on
his journey to the religious conference to be held at Hagenau. Luther
hastened to his bedside, and as a result of his strong exhortations,
Melanchthon speedily recovered. In Luther’s eyes this was a benevolent
dispensation of Providence, which he describes in his correspondence
as a “manifest miracle of God.” The fanciful embellishment which
he gave to the incident when narrating it, has left its traces in his
friend Ratzeberger’s account.\footnote{\textit{Ibid.}, p. 48.}
Melanchthon now advanced the excuse
that he and Luther had been “deceived” by Philip when they
formulated their “advice.” That it was a disgraceful matter he concedes.
In publishing Melanchthon’s letters, Camerarius printed that
of September 1, 1540, addressed to him by Melanchthon, only with
omissions and additions. The genuine text was not made public
until 1904.\footnote
{Melanchthon says therein among other matters: “Either love gained the upper hand
[in the case of the Landgrave] or [it was] 2 beginning and prelude to the insanity, which
exists in the family.”}

Philip of Hesse’s bigamy led to an official conference of theologians
and councillors from Hesse and the electorate of Saxony,
which commenced on July 15, 1540, at Eisenach. Luther, too, put in
an appearance. He vigorously opposed the intention of the Landgrave
not to permit the new marriage to be represented as a form of concubinage
and coasequently to publish the “testimony” of the Wittenbergers and the
fact of his marriage at Rotenburg. In the event that his
opinion were to see the light, he (Luther) was prepared to admit that
he had “played the part of a fool,” to confess his disgrace and beseech
God to restore his good name. His idea was either to retract, or
to publish the lie that Philip’s second wife was a mere concubine. According
to the minutes, he declared on the first day of the Eisenach
conference: “What harm could it do if a man told a good, lusty lie
in a worthy cause and for the sake of the Christian Church?”\footnote
{Grisar, \textit{Luther}, Vol. IV, p. 51; excerpted from Philip’s \textit{Briefwechsel}, ed. by Lenz, pp.
373, 375.}
On July 17 he said: “To lie in case of necessity, or for convenience, or in
excuse, would not offend God, who was ready to take such lies on
Himself.”

Philip, indignant at Luther’s attitude, addressed to him a letter
in which Luther’s threat of retracting the advice and of saying that
he had “acted foolishly” was denounced as “a bit of folly.” “Nothing
more dreadful has ever come to my ears,” he writes, “than that it
should have occurred to a brave man to retract what he had granted
by a written dispensation to a troubled conscience \dots If you can
answer for it to God, why do you fear and shrink from the world?”\footnote
{Grisar, \textit{op. cit.}, Vol. IV, pp. 55 sq.}
He finally asks Luther to proceed vigorously against the vices rampant
in his own circle and to invoke the ban (which he himself had caused
to be introduced) “against adultery, usury, and drunkenness,” which
are no longer regarded as sins. Sarcastically he adds of his new wife:
“I confess that I love her \dots that I should have taken. her because
she pleased me is only natural, for I see that you holy men also take
those that please you.”

Luther was unable to appease the wrath of Philip in his reply of
July 24, in which he permitted himself to pen the following provocative
words: “When it comes to writing, I shall be quite competent to
wriggle out of it and to leave Your Grace in the lurch.”\footnote
{It is significant that in this same letter he threatens to take the Emperor to task because
“he raves against the truth of God.”}
To which
the prince replied that it was a matter of indifference to him whether
Luther extricated or implicated himself by means of his pen; let him
but reflect that the marriages of the Wittenberg preachers were not
recognized by the law of the empire, because they had been monks
and priests; he, however, looked upon Margaret as his “wife according
to God’s word and your advice; such is God’s will; the world
may regard my wife, your wife, and the other preachers’ wives as it
pleases.”\footnote{\textit{Op. cit.}, p. 59.}

In the same letter he makes grave charges against the Elector John
Frederick of Saxony, in order, if possible, to bring him around. He
accuses him of having committed an atrocious crime (sodomy) under
his (Philip’s) roof at Cassel and again at the time of the first diet
of Spires. He mentions this matter also in a letter to Bucer (dated
January 3, 1541),\footnote{\textit{Op. cit.}, pp. 202 sqq.}
 in which he expresses the belief that he ought
to speak in definite terms of this crime because at that time Justus
Menius, the “superintendent” of the Elector, boasted of the virtues
of his master and threatened to attack the bigamous marriage of the
Landgrave in print. Many an ugly rumor was current about the immoral conduct
of the Saxon Elector, who was addicted to excessive
drinking. Both the Landgrave and the Elector, says the Protestant
biographer of Luther, Adolf Hausrath, “did their best to make mockery
of the claim of the Evangelicals that their gospel would revive
the morality of the German nation.”\footnote{\textit{Op. cit.}, p. 203. Hausrath, \textit{Luthers Leben}, Vol. II, p. 391.}

Bucer wrote from Marburg to Landgrave Philip, in 1539, of the
effect which these and other examples of persons in high station was
sure to have on the masses: “The people are lapsing into barbarism,
and the lascivious state of affairs goes on increasing.” And, in a letter
written in the same year, Luther applies the expression “a horrible
Sodom” to the conditions then existing in Wittenberg and in the Electorate
of Saxony.\footnote{Grisar, \textit{op. cit.}, IV, pp. 201, 208.}

This side-light on contemporary conditions is indispensable to
understand the history of the bigamous marriage of the Landgrave of
Hesse.

In a vigorous pamphlet against Philip of Hesse and Luther, written
in November, 1540, Duke Henry of Brunswick, an active opponent of
Luther and the new theology, proclaimed that the Landgrave had incurred
the severe penalty prescribed by the imperial laws, as a result of his bigamous
marriage authorized by the biblical experts of Wittenberg. Luther replied to
him acrimoniously and abusively in his pamphlet entitled “Wider Hans
Worst.” For this reason, a reply which the Duke of Brunswick published
in May, 1541, characterized Luther as “that most insidious arch-heretic,
that impious arch-miscreant and hopeless knave.”\footnote{\textit{Ibid.}, pp. 61, 63 sqq.}

About the same time John Lening--that physical and spiritual monster,
as Luther and Melanchthon call him--who had been the first to promote the
bigamous marriage of Philip, undertook a serious defense of the Landgrave’s
conduct which was agitating all Germany. He did this in a book entitled
“Dialogue of Huldericus Neobulus,” which Philip caused to be printed at
Marburg. The “dialogue” but vaguely refrains from advocating the universal
practice of bigamy. The Wittenbergers believed that Lening aimed
at legalizing polygamy. Luther prepared a refutation, which, however, was
not published because of the:intervention of his elector, who did not wish
to add fuel to the fire. Later on he (Luther) himself deemed it better “not
to strengthen the clamor” by additional writings and “to have the filth
stirred up under the noses of the whole world.”\footnote{\textit{Ibid.}, pp. 64 sqq., 67.}

It was not to be marveled at that the obstinate Landgrave, who had
never possessed any profound Protestant convictions, having been
left in the lurch by Luther, finally resolved to abandon his protectorate
over the new theology, and appealed to the Emperor, to whom he
made liberal offers which were unfavorable to the Protestant party,
but by means of which he expected to arrive at a settlement and to
escape the penalty which he had incurred.

The politicians of the imperial court found Philip’s offers acceptable.
He was permitted to retain Margaret von der Sale, though she
was not to be regarded as his wife. All his other mistakes were pardoned.
In return, he promised to support the recruiting of soldiers
by the imperial forces and to remain neutral in the Emperor’s impending
campaign against Jilich. As a result of his change of attitude, the
Schmalkaldians were forced to sever their connections with the King
of France and to forego the assistance of Denmark and Sweden. As a
consequence of this move and of Philip’s resignation of his command,
the power of the Schmalkaldic League was paralyzed.\footnote
{G. Kawerau, \textit{Geschichte der Reformation und Gegenreformation}, p. 146; Grisar, \textit{Luther},
Vol. IV, pp. 76 sq.}
It was the
severest blow which could be inflicted upon the political position of
the religious innovators. The way was prepared for the triumph of
Charles V over the leaders of the Protestants in the Schmalkaldic War,
which was waged soon after Luther’s demise (1547). It is not impossible
that the wily Hessian Landgrave, when he allied himself with
the Emperor, perceived the confused and desperate condition of the
Protestant cause and that his change of front was inspired also by a
tactical reason. It was legally confirmed by the Treaty of Ratisbon,
June 13, 1541.

Luther was bitterly requited for his unfortunate decision of December
10, 1539, wherein he permitted himself to be governed “by political
ideas and political manipulations,” instead of acting under the
inspiration of “the unvarnished truth and an incorruptible conscience,”
as the Protestant historian Julius Bohmer expresses it. The
same historian declares that in this entire affair Luther showed himself
to be “weak, nay, flabby in his moral judgments.”\footnote{\textit{Ibid.}, p. 71.}

Another Protestant, the historian Paul Tschackert, characterizes
the Hessian affair as “a dirty story,” which is and must remain “a
shameful blot on the German Reformation and on the life of our
reformers.” Theodore Kolde, in a work which is otherwise decidedly
favorable to Luther, holds that “the attitude which the reformers
took up [towards this affair] at a later date, is even more offensive
than Luther’s advice itself. He refers to the lie which Luther recommended
and which he was prepared to tell, according to his own
public declaration. “With devilish logic,” says Adolf Hausrath, the
Protestant biographer of Luther, “one false step induced them [the
Protestant ecclesiastical leaders] to take another which was even
worse.”\footnote{The quoted and other passages of Protestant historians, \textit{ibid}., pp. 72, 78 sq.}

To mitigate these abundant condemnations, an attempt has been
made by Protestant writers to hold the Catholic Church and the ideas
of the Middle Ages at least partly responsible for Luther’s attitude.
These writers cite the reformer’s opinion in the case of Philip of Hesse
as an advice given as a secret matter of conscience, under the seal of
confession. The “egg-shells of a previous period of Church history”
are said to have clung to the Wittenberg doctor in his “testimonial”
to the Landgrave and the ensuing negotiations. It is sufficient to note
that these ideas are an invention of Martin Luther, for the “secret
of confession” which he claimed, never existed in the Catholic Middle
Ages.\footnote{\textit{Ibid.}, pp. 72 sq.}

Certain established facts, which are generally overlooked in the
Protestant condemnation of this affair, are more important than the
refutation of the “egg-shell” theory. In the first place, there is a close
connection between the “testimonial” of December 10, 1539, and
Luther’s fundamental attitude towards the Bible. It was only because
he disregarded ecclesiastical tradition in the interpretation of Holy
Scripture and had accustomed himself to introduce his own ideas into
the sacred text, that he was able to discover that the New Testament
permitted bigamy in exceptional cases. His attitude towards the authority
of the Church must also be taken into consideration. Only
because he substituted the subjective opinion of an individual, \textit{i.e.},
his own, for the teaching and governing authority of the Church,
which he had repudiated, was he able to propose his own erroneous
opinion as a moral guide. Finally, Luther arrived at his lamentable
accommodation because Lutheranism was compelled to seek the aid
of the secular rulers to insure the permanency of the new Evangel.\footnote
{Spontaneously the attention is here directed to the entirely different attitude of the
Catholic Church towards Henry VIII’s attacks upon the sanctity of matrimony and his
introduction of the schism.}
Hence, it is obvious that the incident casts a shadow upon the entire
interior structure of Lutheranism, and that it cannot be regarded
simply as an accidental disfiguration.
