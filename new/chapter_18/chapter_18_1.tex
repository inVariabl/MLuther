\section{Luther’s Ethical Teaching in General}

The ethical system of Luther was vitally influenced by his conception of
the gospel, which in his opinion was essentially only forgiveness of sin,
a cloak covering guilt, the quieting of an “affrighted
conscience.” To gain a sense of confidence was the starting-point of
the new doctrine. Luther’s supreme gain was to acquire certainty of
salvation through an active faith in the appropriation of the merits of
Christ. This thought is the guiding star also of his ethics.

Protestants say that Luther erected ethics upon its genuine foundation,
which had been ignored up to his time. This claim, however, is
disproved by certain leading declarations of his, which raise the question
how a true ethical system could originate under such conditions.
Luther taught that man is not a free agent, but a mute “pillar of
salt,” either controlled by the grace of God which is operative within
him, or subject to the domination of the devil, without any activity
on his part. His reason in religious matters resembles a lunatic.” In
consequence of an ineradicable original guilt, sin persists in man’s
inordinate concupiscence; even the just man, \textit{i.e.}, he who is regarded
as just by God, remains a sinner. Sin is merely covered up by fiduciary
faith in the blood of Christ. The “golden cloak of grace” due to the
merits of the Redeemer does everything. Good works are devoid of
supernatural merit and have no significance for Heaven. Every man
is predestined for Heaven or hell by a hidden decree of God.\footnote{Cfr. Grisar, \textit{Luther}, Vol. V, pp. 3 sqq.}

It is fair to ask: What moral inducement is there in Luther’s hypothesis
to combat the perversity of human nature? Is there any moral
responsibility? Can there be any such thing as Christian morality? At
a time when the renown of Luther was not so great as it is since the
last decades of the nineteenth century, the Protestant theologian K.
F. Stiudlin openly declared that “no genuine Christian ethic could
exist” on the basis of Luther’s principles.\footnote{\textit{Geschichte der Moral}, Göttingen, 1806, p. 209.}
Many other Protestant
authors share this view. As a matter of fact, Luther’s work “On the
Enslaved Will” marks the death of ethics.

He inculcates humility because sin is in man and man is completely dependent
upon God, without any capacity or volition on
his part. But this sort of humility is no basis for a system of ethics.

Luther errs in his fundamental presupposition that the assurance
of possessing salvation, contained in fiduciary faith, will move man to
observe true morality, and particularly to perform acts inspired by
love, which alone are pleasing to God. Lack of morality demonstrates
that one has not the right kind of faith, which, in his system, takes
the place of Christian perfection and virtue.

Practical Christianity is relegated to the background in Luther’s
system, for the only obligatory works of divine worship are faith,
praise, and thanksgiving. The others are to be “directed towards our
neighbor”; yet there are no good works except such as have been commanded
by God. Indeed, without faith the good works which man
performs amount to sin, just as the virtues of the pagans were
\textit{splendida vitia}.\footnote{Cfr. Grisar, \textit{Luther}, Vol. V, pp. 47 sqq.}
“Faith” causes the Church and the world to be two
entirely different empires, so completely separated that the exterior
office of a Christian, \textit{e.g.}, of a ruler, has nothing to do with
his Christian belief. To strive for perfection as Catholics do, is folly. There are
no evangelical counsels, and even the most pious believers are sinners.
Saints must be “good, hearty sinners”--an expression which recalls to
mind his declaration: “Sin boldly, but believe more boldly.”

But in spite of these intellectual aberrations, the sermons and writings
of Luther contain a rich treasure of ethical doctrines. He so
urgently and eloquently exhorts men to the practice of virtue that
his voice is scarcely distinguishable from that of the ancient Church.
Many illustrations of this have been furnished in the foregoing chapter.
Thus, in point of morality Luther actually pursued a far better
course than his theological opinions would lead one to expect. His
lack of consistency proved a decided advantage. In his ethical teaching,
as in other respects, he did not carry his avowed principles to
their logical conclusions. He desired to be helpful to others in his own
way as a spiritual director and to demonstrate that the new Gospel
was morally sound and profitable.

Luther never attempted to formulate a system of ethics, and his theoretical
principles would have rendered the attempt futile. But Protestantism has reason
to congratulate itself that its founder, even without a system, scattered so
many seeds of Christian morality in his emphatic and popular way, though it
should not be overlooked that he derived his supply from the heritage of the
ancient Church, upon which he drew freely. Protestant writers have lamented the
fact that Luther bequeathed to his followers no systematic introduction to the
devout life so that even to the present day Protestantism lacks any definite
rule of piety. Julius Kaftan laments that Luther slighted the doctrine of piety
and that of “redemption from the world,” in the narrower sense. The salvation
“bestowed by Christ is not merely justification and forgiveness of sins,” but
rather the “everlasting possession” to be reached by a Christlike life.
Justification is but the road to this possession. The Church has other “vital
interests.”\footnote{\textit{Op. cit.}, Vol. V, pp. 89 sqq.}

In his writings no less than in his life Luther neglected the true
methods of self-reform. Catholic authors, on the other hand, such
as St. Bernard and Gerson, from whom Luther derived enlightenment at a
former period of his life, showed that true piety is based on
self-denial. In Luther’s opinion self-denial is of far less importance
than the ready surrender of the “fretful” so-called traditional prejudices
of renunciation and restraint in worldly affairs. The ill-considered expression:
“What matters it if we commit a fresh sin?”, since
there is forgiveness in faith, supplies us with a profound insight into
his mentality.

Retirement, examination of conscience, and solitude, were to be
shunned, according to his view. Quietude, he says, “calls forth the
worst of thoughts.”\footnote{\textit{Ibid.}, p. 93.}
In his directions for praying, one misses the salt
of sorrow and contrition; they lack the fragrance of true humility
and are far removed from that charity which resigns itself to God
and submits with equanimity to the divine will, especially with reference
to one’s vocation in life. To render his prayers fervent, he must
spice them with curses against the “papists.” There is anger and passion
in his practical attitude; megalomania, jealousy and irritation in
his appeal to violence; there are, finally, examples of untruthfulness
and dishonesty in his polemics, so that Erasmus is constrained to exclaim:
“You pretend to be a teacher of the Gospel!”\footnote{\textit{Ibid.}, pp. 112 sqq.}

His teaching on “self-improvement and the reformation of the
Church,” considered in the light of Luther’s conduct, furnishes numerous
other and no less damaging objections to his ethical doctrine.\footnote{\textit{Ibid.}, pp. 84 sqq.}
Suffice it to say that reform, to be effective, should have commenced
with a sincere improvement of morals, enforced by the example of
his own life. Instead, he commenced with arbitrary changes of doctrine.
It was necessary at that time to counteract the subjectivism
and skepticism produced by the Renaissance. Luther, on the contrary,
encouraged this evil. He favored the divergent tendencies of the
nations and took no account of the new tasks imposed on the Church
by the discovery of new countries. Allowing the masses to read the
Bible was no compensation for the want of truly great objects of
reform.

The Bible in consequence of the use that was made of it rather became the
means of theological and social confusion. It seemed, as
Luther himself declared, as if everyone was desirous “of boring a hole
wherever his snout happened to be.”\footnote{\textit{Ibid.}, p. 129.}
In his pessimism, with which
he infected the world, he descries how even in the first centuries “the
devil had broken into Holy Scripture and caused such a disturbance
as to give rise to many heresies.”\footnote{\textit{Ibid.}, p. 130.}

As a reformer, he did not direct the unfavorable currents of his
age into better channels, but to a certain extent permitted himself to
be carried off by them. Thus, in the beginning of his career, he
adopted the pseudomysticism which pervaded his age. From contemporary
humanism, he not only adopted disrespect for authority
and the spirit of rebellion, which he augmented, but also promoted
the excessive use of authority on the part of ambitious princes at the
expense of their subjects. Finally, in his writings and addresses on
marriage and sexual questions, he cultivated the crude naturalism of
the Renaissance with an abandon that was astounding, particularly in
his fight upon sacerdotal celibacy and monasticism. At times, says the
Protestant philosopher Frederick Paulsen, this naturalism causes
Luther to speak “as if abstention from the works of the flesh spelled
rebellion against the will and command of God.”
