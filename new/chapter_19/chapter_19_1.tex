\section{The Eve of the Religious War. The Council}

In the early forties of the sixteenth century Germany was in a
strained condition, politically and religiously, and no signs of an early
turn for the better were visible. The Turks, more menacing than ever
before, continued their assaults upon the eastern frontiers of the Empire.
In 1541 they gained a foothold in Hungary, whence they threatened
destruction against Germany. In the west, the King of France was
entering upon his fourth war against Charles V, which dragged on
from 1542 to 1544.

To fill the measure of domestic misery caused by the religious
schism, the Bundestag of Schmalkalden, in March, 1540, declared
against the toleration of Catholic worship. The fruitlessness of the
religious conference of Worms and of the diet of Ratisbon (up to
May 22, 1541) convinced the Catholic spokesmen, John Eck, Julius
von Pflug, and John Gropper, as well as all other intelligent observers,
that the hope of religious reunion was vain.

The diet of Ratisbon closed with the armistice of the so-called
Interim. Two later diets, that of Spires (1542), and that of Nuremberg
(1543) endeavored to unite the nation against the Turkish peril.
But the Protestants, because of their extravagant religious demands,
obstinately refused to come to the assistance of the Emperor, as they
were in duty bound to do. The victory of Charles V over the Duke of
Cleve, notwithstanding the support which the latter received from
France and from the Elector of Saxony, proved to be an advantage for
the Catholic cause. In July, 1542, however, that cause was injured by
the violent and victorious irruption of the Schmalkaldian forces into
the duchy of Brunswick, which resulted in the imprisonment of Henry
of Brunswick, the Catholic Duke “Heinz,” whom Luther pursued
with implacable hatred (October 20, 1545).

Halle, the see city of Cardinal Albrecht of Mayence, which had so
long defended itself against Lutheranism, was Protestantized by Justus
Jonas through the efforts of Luther, and Albrecht was compelled to
establish his residence at Mayence and Aschaffenburg. He transferred
his celebrated collection of relics to these and other cities. The antinomian,
Agricola, who, to escape the wrath of Luther had taken refuge
with the Elector of Brandenburg, supported the measures which
that prince employed to promote the Reformation. The religious upheaval
continued unimpeded in the duchy of Saxony. The bishop
of Meissen, John von Maltitz, complained as early as 1540 to Bishop
John Faber, of Vienna, that the Catholic press of his diocese had been
reduced to silence, and the prescriptions of the Lutheran “inspectors”
were enforced, whilst the bishop was unable to bestir himself. “I fear,”
he wrote, “that the wrath of God will be visited upon the Pope, the
Emperor, and King Ferdinand, because they permit the decline and
suppression of the Catholic faith.” The deeds of violence, he continued,
were too potent; divine services could no longer be conducted
at Meissen and the sermons in the local church were preached by a
Lutheran minister who had been forced upon the people.\footnote
{L. Cardauns, according to Vatican manuscripts in the sixth volume of the \textit{Nuntiaturberichte}
(1910), pp. 233 and 237 sq.}

It was a lamentable accommodation by which, in 1543, the new
religion was permitted to invade the bishoprics of Münster and Osnabrück.
The venerable see of Cologne, the “German Rome,” was assailed
by the innovators with every prospect of success under the administration
of the weak-kneed and uneducated elector, Herman von
Wied.

In his reports to Rome, the far-seeing papal legate, Morone, uttered
bitter complaints--and not without reason--concerning the attitude
of the great majority of German ecclesiastical princes. He charged
that many flourishing parts of the German Church had been lost in
virtue of their dilatoriness and worldly-mindedness and predicted that
the ruin would assume unlimited proportions if the bishops did not
arouse themselves and offer resolute resistance.

In consequence of the deficiency of the German resistance, the Emperor
also lapsed into a complaisance which displeased the more zealous
Catholics. Notwithstanding his loyalty to the Church, he succumbed
to the influence of the subversive movement which had been going
on for many years. His advisers, even Granvella, were not competent
to cope with the difficulties of the situation. At the diet of Spires, in
February, 1544, the empire made undue concessions to the innovators,
although the position of the Emperor had been fortified by the return
of Landgrave Philip of Hesse to his former allegiance. In the
recess of June the tenth, the religious controversy was adjourned by
leaving the adherents of the new theology in substantial possession
of their usurped ecclesiastical rights and properties. It was resolved
that a “free Christian council of the German nation” should shortly
attempt a settlement. The very term, “a free council,” was bound to
excite the apprehension that the papal authority would be eliminated.
If the council were restricted to the German nation and conducted as
a mixed spiritual-temporal assembly, composed of Catholics and
Protestants, as was to be feared, it was bound to give rise to the greatest
anxiety. In either event, the expectations connected with a legitimate
general council of Christendom, as planned by the Church authorities,
could not be realized.

On November 9 of the same year Pope Paul III convoked the ecumenical council
which had so often been deferred because of the
belligerency of the times. It was to meet at Trent on the fifteenth of
March, 1545.

The Bull of convocation proclaims that it was intended to put
an end to the religious schism, to effect the reform of the whole of
Christendom, and to achieve unity in the defense of the Cross against
the Crescent. On the twelfth of December of the same year, the council
was opened by a splendid address by Cardinal Del Monte, the future
Julius III. Attended by only thirty-four voting members, the council
faced the prospect of being compelled soon to transfer the seat of its
deliberations elsewhere. In March, 1547, the state of public affairs
forced it to continue its deliberations at Bologna. The opening of the
Tridentine Council took place at almost the same time when the congress
of the Schmalkaldic League, assembled at Frankfort on the Main,
issued its protest against the ecumenical council, which had so often
been demanded by the Protestants themselves.

On August 24, 1544, the Pope had addressed grave remonstrances
to the Emperor because of the latter’s attitude toward the religious
schism in Germany.

In a Brief transmitted to Cardinal Morone, which was communicated also
to King Ferdinand and the Catholic estates of the empire, he wrote that he
was deeply grieved to see that the recess of the diet of Spires excluded from
the deliberations of religious affairs the Pope, who from the foundation of
the Church had been invested with supreme authority in all such matters.

The Emperor had reason to fear the penalties set forth in Holy Writ against
those who infringed upon the rights of God and His representatives. The
Emperor is not the directive head of the Church, but only its protector. The
increasing complaisance of Charles had made the reclamation of the separated
brethren all the more difficult. He must withdraw the concessions
which (in a spirit of undue clemency) had been made to the enemies of the
Church, otherwise, the Pope could no longer be content with mere admonitions.\footnote
{The text is most accurately reproduced in Ehses, \textit{Concil. Trident.}, Vol. IV, pp. 364 sqq.;
cfr. Pastor, \textit{Geschichte der Päpste}, Vol. V (1909), pp. 504 sq.}

Owing to his Christian convictions, Charles was not prepared to satisfy
the opponents of the papacy by allowing matters to develop into a rupture
with Paul III. In defense of himself, he caused an oral reply to be made
to some of the charges contained in the papal Brief. “After calm deliberation
he could not but perceive that the complaints which were preferred by the
Pope with so much determination were not without justification.”\footnote
{Thus Pastor, \textit{l.c.}, p. 507.}

In 1556, Charles V, under the pressure of his office, which weighed
heavily upon him, transferred his crown to Philip II, and retired
to the Spanish monastery of San Yuste, where he spent the remainder
of his life.

In opposition to the Catholic estates, and still more in view of the
approaching council, the Protestants endeavored to unite their forces
and organize more effectively. A result of this endeavor was the so-called
“Wittenberg Reformation,” which on the fourteenth of January,
1545, Luther and the theologians associated with him at Wittenberg presented
to the Elector of Saxony in response to the latter’s
request for an official demonstration. Melanchthon had drafted the
document in such a cautious and apparently mild form that Chancellor Brück
wrote that there is “no trace of Doctor Martinus’ boisterousness” in it.
This “Wittenberg Reformation'' treats successively
of doctrine, the sacraments, the office of preaching, the episcopal government,
etc. It pretends that the bishops ought to be retained--
provided they embrace the new theology! This was a well-known and
favorite dream of Melanchthon’s. He does not speak of forcing new
bishops into office. Notwithstanding the fact that the authors of the
document appeared to be moderate in their language, the pronouncement
characterizes the papacy as “idolatry.” Whatever might savor
of concession is rejected. Only a few external forms of the old Catholic
worship are tolerated, mainly as a means of deceiving those who
were accustomed to them.\footnote
{On the “Wittenberg Reformation,” see Grisar, \textit{Luther}, Vol. V, pp. 385 sqq.; Vol. III,
pp. 448 sq.}
