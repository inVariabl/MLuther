\section{``Consecration'' of the Bishop of Naumburg and Dedication of the Church at Torgau}

The manner in which Luther at that time intended to proceed with
the installation of bishops for his churches, was exemplified in the
celebrated case of the “consecration” of Nicholas Amsdorf as “Evangelical
bishop” of Naumburg.\footnote
{For the following see Grisar, \textit{op. cit.}, Vol. V, pp. 192 sqq.}

The legitimate bishop of this city, Philip, palsgrave of the Rhine
and duke of Bavaria, had died in 1541, after exposing his people entirely
too much to the infiltrations of Protestantism. The selection of a
successor precipitated a crisis, owing to the arbitrary and violent
measures taken by the elector, John Frederick of Saxony, who was resolved
upon obtaining possession of this ancient and wealthy bishopric. The see
of Naumburg was directly subject to the imperial
government and enjoyed the personal protection of the emperor. The
elector’s powers were limited to certain privileges of arbitration. After
Bishop Philip’s death a Lutheran preacher, Nicholas Medler, in compliance
with the elector’s instructions, began to function as ``superintendent
of Naumburg'' and to deliver sermons in the cathedral.
Soon after the admirable and scholarly provost, Julius von Pflug,
was legitimately elected by the chapter. The elector forcibly prevented
Pflug from entering upon his new office. He prosecuted the
conquest of the city and the territory with such vehemence that his
procedure seemed rash even to the Wittenberg theologians. Luther
wrote to the elector: “What cannot be carried off openly, may be won
by waiting.” Pflug, however, was not recognized as bishop by the
Wittenbergers, who at first intended to erect only a Lutheran consistory
in the city of Naumburg, but changed their mind and favored
the appointment of a new bishop by the elector. Up to this time, no
steps had been taken anywhere in Germany towards the direct installation
of a Protestant bishop. Melanchthon in particular hesitated to go to such
lengths, as his “Wittenberg Reformation” evinced.

The curious procedure at Naumburg remains practically an isolated
case.

Luther and his friends wanted to force the issue by the election
of Prince George of Anhalt, canon of Magdeburg and Merseburg, who
shared the Wittenberg views. The elector, however, feared that, owing
to his position, this prince might not prove an easy tool in his sovereign’s
hands. Not until 1545 did Prince George attain to the “episcopal dignity,”
when Luther “consecrated” him bishop of Merseburg,
which see had been occupied in 1541 by August, the brother of Duke
Maurice of Saxony, and had since then been governed by George as
spiritual administrator. The “consecration” of Prince George of Anhalt,
besides that of Amsdorf at Naumburg, was the only act of the
kind which Luther performed.\footnote{\textit{Ibid.}, pp. 193 sq.}

In lieu of George von Anhalt, the Saxon elector destined Nicholas
von Amsdorf for the see of Naumburg. Amsdorf at the time was actively
engaged in preaching Luther’s doctrine at Magdeburg. Luther
thoroughly approved of the appointment and eulogized Amsdorf in a
letter to the elector as “learned and proficient in Holy Scripture, more
so than the whole crowd of papists; also a man of good morals and upright
life.” Melanchthon was not so favorably disposed towards this
hot-blooded theologian, nor did he approve of his conduct, but charged
him with having had adulterous relations with the wife of his deacon
at Magdeburg.

In order to give the proceedings connected with the “consecration”
the Protestant councillors of at Naumburg a semblance of legitimacy,
that city and of Zeitz, and the Protestant members of the gentry,
were asked to vote for Amsdorf. Many of them had sworn everlasting fidelity
to the Catholic Chapter under the former bishop. Luther
solved their scruples by advising them that no oath taken by the
sheep to the wolves could be of any account, and that “no duty could
be binding which ran counter to God’s commandment to do away
with idolatrous doctrine.”\footnote{Köstlin-Kawerau, \textit{Martin Luther}, Vol. II, pp. 554 sq.}

On January 20, 1542, Luther himself “consecrated” the new Protestant
“bishop” in the sacred precincts of the cathedral of Naumburg.
This magnificent structure, one of the most beautiful medieval cathedrals
of Germany, a noble type of the later Romanesque and Gothic
style, was begun as early as 1045 under Bishop Cadalus, and through
the mouths of its numerous and wonderful statues of saints and pious
princes seemed to utter a loud protest against the unheard-of proceedings
within its walls. Many guests of high rank assembled in the choir
between the magnificent rood-loft, with its richly ornamented passages,
and the altar. The elector had arrived two days previously, accompanied
by an escort of 200 mounted knights, all clothed in black.
At his side were his brother, John Ernest, and Duke Ernest of Brunswick.
The councillors and knights of Zeitz and Naumburg acted as
electors and witnesses. In response to an address by Nicholas Medler,
they signified their choice in the name of the congregation, to which
the people assented by saying “Amen.” Luther deemed it very important
to emphasize on this occasion his old idea of the importance
of the congregation in contrast with the arbitrary action of the elector.
The entire ceremony was to proceed in accordance with the example of
the earliest centuries of the Church, in which the bishop,
as it were, married the congregation of the faithful who had elevated
him. The accompanying ancient rites, however, were omitted; above
all the transfer of legitimate ecclesiastical jurisdiction to the new
“bishop.” In place of the neighboring bishops who, in conformity with
the practice of antiquity, were wont to participate, three superintendents
and an apostate abbot were invited to perform the rite of the
laying on of hands. Melanchthon, in his usual weakness, had also complied
with the summons to be present.

Luther preached to the superintendents of the Church on the text
of St. Paul, where he teaches that “the Holy Ghost hath placed you
bishops to rule the church of God.” After the sermon Amsdorf knelt
before the altar, surrounded by the four assistants, and the “Veni
Creator” was sung. Luther reminded the future bishop concerning his
episcopal duties, and on the latter giving a satisfactory answer, in
common with the four others, he laid his hands on his head; after this
Luther alone offered a prayer for him. This was followed by the “Te
Deum,” sung in German. The “consecration” took place in a manner
similar to the ordination of preachers, namely, by imposition of hands
and prayer.

Shortly after the ceremony Luther wrote to a friend that it was “a
daring act” on the part of the “Heresiarch Luther,” which “will arouse
much hatred, animosity, and indignation against us”; and that he
was hard at work “in hammering out a book on the subject.”\footnote{Grisar, \textit{Luther}, Vol. V, p. 195.}
This
book was to justify the “consecration” at Naumburg and to present
the procedure as a model. It was composed at the request of the elector
and appeared in 1542, under the title \textit{Exempel, einen rechten christlichen
Bischof zu weiben} (Example How to Consecrate a Genuine
Christian Bishop). In many respects it is opposed to the cautious and
mild “Wittenberg Reformation,” which knows naught of such an
“example of how one might wish to reform bishoprics and organize
bishoprics in a Christian manner.” Luther states in this book that the
new bishop was ordained “without any chrism, without even any
butter, lard, fat, grease, incense, charcoal, or any such-like holy
things.” He dismisses his opponents with insolent remarks, even resorting
to language borrowed from the category of human evacuations, and adduces
in the following form a list of Catholic apologists:
“Doctor Sow (Eck), Witling, Blockhead, Dr. Dirtyspoon (Cochlaeus),
Lick-dish, Urinal, Heinz, Mainz, etc. Extensive repetitions,
moreover, are characteristic of the book. Luther had evidently overwritten
himself. His disgust with life, due to bad health, the forerunner of his
fatal sickness, spoiled his work.

In his later correspondence with the new “bishop,” Luther refers to
Amsdorf’s bitter complaint that practically nothing was being done by the
elector to establish order in the ecclesiastical régime of Naumburg. Luther
bewailed with him that the government “so often take rash steps, and then,
when we are down in the mire, snore idly and leave us in the lurch.”\footnote{\textit{Ibid.}, p. 197.}
“All
Germany,” he says, “presents a terrible scene of demoralization and decadence.”

It is Christ’s business to see to this,” he writes on another occasion, “since
He Himself by His Word has called forth so much evil and such great
hatred on the part of the devil.” In a pseudomystical strain he consoles
himself and Amsdorf by reflecting that it would be safest to allow oneself to be
“carried along” by the guidance of God, who has created the bishop and
disposed everything that has happened. “The blinder we are, the more God
acts through us.” We can only look forward hopefully to the end of the
world.

Amsdorf, the whilom Catholic priest, found little pleasure in his
episcopal status. He and his preachers received but a meager sustenance.
Soon after Luther’s death he was deposed as a result of the decisive victory
of the imperial forces over the Schmalkaldic League.

He had to retire to Magdeburg, where he did his best to develop that
city into a strong citadel of ultra-Lutheranism. In Magdeburg he also
wrote his notorious book: “That Good Works are Injurious to Salvation.”\footnote{\textit{Ibid.}, p. 198.}

Due to the change of events that followed the battle of Mühlberg,
in which the Protestant forces were defeated, the venerable cathedral
of Naumburg once more decked itself out in festive garments to welcome
its legitimate bishop, Julius von Pflug. But the bishopric had
been Protestantized, and Catholics were permitted to celebrate divine
service only in the cathedral at Naumburg and the collegiate church
of Zeitz. A painting in the cathedral, though unfortunately much
damaged, preserves the features of the noble Bishop Pflug, who, after
a life replete with hardships and disappointments, died in 1564 and
was interred in Zeitz. He was particularly saddened by his experiences
with the Augsburg Interim of 1548, in the drafting of which he had
participated. Pflug’s expectation of obtaining papal consent to sacerdotal
marriages and the lay-chalice, for the sake of reconciling the
non-Catholics, proved futile. Other remedial measures were necessary.
It pained him to see how many members of his clergy lived in
concubinage in this era of mental and moral confusion--a fact that
explains the propositions which he himself and others submitted
for the sake of obtaining far-reaching concessions from the Holy
See.

After having been so successful in seizing the bishopric of Naumburg,
the elector of Saxony sought to obtain control of that of
Meissen also. Luther and Amsdorf assisted him in this undertaking.
In consequence, a feud (known as the “Wurzer Fehde,” after the
town of Wurzer) arose between him and the youthful Duke Maurice
of Saxony, an adherent of the new religion.\footnote
{On the “Wurzer Fehde” and Luther’s attitude toward the same see Grisar’s \textit{op. cit.},
Vol. V, pp. 200 sqq.}
In order to terminate
the controversy, both parties agreed to divide the spoils. The elector
appropriated that part of the bishopric which lay about Wurzen,
which he forthwith Protestantized, whilst the remaining part, including
Meissen, fell to Duke Maurice, who also exercised violence in religious
matters. Both in Meissen and in Wurzen treasures of Catholic
art were removed from the churches and dissipated with that brutal
barbarity which was employed against so many churches abounding
in rare and noble works of art.\footnote
{For illustrations see Grisar, \textit{op. cit.}, Vol V, pp. 203 sqq.; II, 352 sq.; Vol. IV, p.
196; Vol. VI, 277 sq.}
In his feud with the elector, Luther
denounced Maurice of Saxony as ``a mad bloodhound.'' For various
reasons, this prince developed such a degree of antipathy to Protestantism,
that he abandoned the Schmalkaldic League and commenced
to gravitate toward the Emperor. As a result, the elector and Luther
became all the more intimately allied.

When, in 1544, John Frederick of Saxony had completed the
castle-church at Torgau on the Elbe, the first newly erected
Protestant church in Saxony, he invited Luther to officiate at its dedication.
The ceremony, which Luther performed as if he were a bishop, paralleled
the pseudo-consecration of the Protestant bishop of Naumburg.
The ancient church ritual was set aside for the “service of the Word.”
The intimate participation of the congregation in the act, which
was supposed to take place in the name of the people, was once
more clearly expressed. Luther’s sermon, which replaced the rite of
consecration, was published by Caspar Creutziger (Cruciger).\footnote
{Erlangen ed., Vol. II, pp. 218 sq.; Köstlin-Kawerau, \textit{M. Luther}, Vol. III, pp. 573 sq.}

In it he develops, among others, his doctrine that every Christian believer
is a priest. He so strongly emphasized the unity of his act with
the congregation that here, too, one is constrained to think of opposition
against the elector. In this address he incidentally ascribes
to the Christian congregation the right of transferring the celebration
of Sunday to some other day of the week, if circumstances
demand it. He boldly stated, without fear of contradiction, that
man is the master of the Sunday, not Sunday the master of man.\footnote{Erl, ed., Vol. II, p. 223.}

Another novelty was the singing of a fugue, composed by John Walther,
which contained the words: “\textit{Vive Luthere, vive Melanchthon, vivite
nostrae lumina terrae} (Long live Luther, long live Melanchthon, long
live the lights of our earth), with an ostentatious amplification of
their merits.\footnote{Text, \textit{ibid.}, p. 219,}
