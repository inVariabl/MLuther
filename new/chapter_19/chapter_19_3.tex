\section{Attacks Upon the Archbishopric of Cologne and the Duchy of Brunswick}

In 1543 Luther joyfully boasted to Duke Albrecht of Prussia that
the archbishopric of Cologne had embraced his Evangel in earnest,
despite the strenuous opposition of the canons of the cathedral, and
that matters were progressing famously in the duchy of Brunswick.\footnote{Grisar, \textit{Luther}, Vol. V, p. 166.}

Herman von Wied, archbishop and elector of Cologne, was a
worldly-minded man, uninstructed in ecclesiastical matters, of whom
Charles V once said to Philip of Hesse: “Why does he start novelties?
He knows no Latin, and, in his whole life, has only said three Masses,
two of which I attended myself. He does not even understand the
Confiteor. To reform does not mean to bring in another belief or
another religion.”\footnote{\textit{Ibid.}, V, p. 166.}

After 1539 Bishop Herman was persuaded by the preacher Peter
Medmann, and still more by Martin Bucer, to effect a change in religion
in the archdiocese of Cologne. The powerful resistance which
he encountered in his cathedral chapter as well as in the city council
led Luther to speak of “seven devils, who sit in the highest temple,
whom God will overthrow, who breaks down the cedars of Lebanon.”\footnote{\textit{Ibid.}, p. 233.}

In the courageous defense of the Catholic cause at Cologne,
three men distinguished themselves: John Gropper, a member of the
secular clergy, a true guiding spirit of the archbishopric in its hour
of danger; Eberhard Billick, the learned and indefatigable provincial
of the Carmelites, and Peter Canisius, a member of the newly-founded
Society of Jesus.

The situation became critical when Melanchthon, in 1543, settled
at Bonn and, in conjunction with Bucer, devoted his energies to promoting
the plans of the misguided archbishop. Melanchthon wrote a
pamphlet against Billick, in which he accuses the Rhinelanders of
“idolatry.” In order to draw the people away from the Church, he,
aided by Bucer, published the so-called “Cologne Reformation,”
whereby he gained over many Catholics. The Catholic leaders on
their part addressed urgent representations against Herman von
Wied to the Holy See, and also endeavored to procure the intervention
of the Emperor. Herman von Wied, however, issued a summons for a provincial
diet to be held at Bonn on January 6, 1546,
for the purpose of definitely introducing a new form of religion,
which, to the great chagrin of Luther, but conformably with the
designs of Bucer and Melanchthon, was a compound of Lutheranism
and Zwinglianism.

The youthful Canisius at that time rendered excellent services to
the Church of Cologne, a prelude of his future activity as “Apostle
of Germany.” He sojourned at Cologne with several brethren of his
Order, for the sake of study. When Emperor Charles visited Cologne,
in August, 1545, Canisius, then a deacon, eloquent of speech, well
informed, and courageous, appeared before that monarch at the urgent importunities
of Gropper, and warmly requested him to protect
the religion of his Rhenish subjects. The ensuing exhortations and admonitions
of the Emperor had no influence upon the archbishop.
Before the assembly of the dreaded diet of Bonn, Canisius, already
highly esteemed by the citizens of Cologne, was asked to make a
hurried trip to the Netherlands, to plead with the Emperor for assistance.
He succeeded in obtaining an order which inhibited the archbishop from
making any decision in religious questions prior to the
next diet. Because of the persistent danger, Canisius, in February,
1546, undertook another journey, and visited the Emperor at Nymwegen,
the birthplace of Canisius. In the name of Cologne, he requested
the removal from the city of certain clamorous adherents of
Archbishop Wied, and an imperial letter to the magistrates, encouraging
them in their opposition to the religious innovation.\footnote
{The visits of P. Canisius to the Emperor are mentioned by O. Braunsberger, \textit{Petrus
Canisius}, 2nd and 3rd ed., 1921, p. 32. Relative to the last journey, Braunsberger remarks,
without indicating his source: “first revealed through the latest researches.”}
The efforts
of the Catholic leaders of Cologne resulted in the deposition and excommunication
of the Archbishop on April 16, 1546. After this
Herman von Wied disappears from history. He was succeeded by a
zealous Catholic, Adolf von Schaumburg. Cologne was saved to
the faith and continued to be a citadel of the Catholic religion and
the heart of the Catholic Rhineland. In the same year, 1546, the year
of Luther’s death, Canisius, who was afterwards canonized by the
Church, was ordained to the priesthood and published a Latin edition
of the works of St. Cyril of Alexandria and a revised text of the
writings of Pope St. Leo the Great, in order to demonstrate the traditional
teaching of the Church.

In Northern Germany, Duke Henry of Brunswick-Wolfenbüttel
was the militant leader of the Catholic league against the Schmalkaldians.
This position made him highly obnoxious to Luther and his
friends, but their hatred was augmented by his defense of the Catholic
cause in writing. His forceful attacks upon the bigamous marriage
of Philip of Hesse provoked Luther to write his pamphlet
“Wider Hans Worst,” which was directed against Duke Henry.
Luther followed this up with a bitter incitement against the “bloodhound
and incendiary Heinz,” as he styled the militant Duke.

For the rest, his criticism of the private life of Duke Henry was
not unfounded. Not only his controversial methods, but also his
moral character were quite objectionable. But the charge of arson in
the territories of the new religion, which Luther made against him,
was the product of a morbid imagination.\footnote{Grisar, \textit{Luther}, Vol. IV, pp. 293 sq.}
 It is rumored, he says, that
“Heinz” has dispatched “many hundreds of incendiaries against the
Evangelical estates.” He maintained in all seriousness “that the Pope
is said to have given 80,000 ducats towards it.” History is ignorant
of any such transaction. Certain alleged confessions extorted on the
rack have no significance.

In 1542, when Duke Henry attempted to enforce the ban which
the Imperial Supreme Court had declared against the city of Goslar,
the war which the Elector of Saxony and the Landgrave of Hesse had
long prepared against him, finally broke out. Luther was aware at
that time of prophecies which foretold the end of the “son of perdition,”
--a “warning example, instituted by God, for the tyrants
of our time.”\footnote{\textit{Op. cit.}, Vol. V, p. 236.}
 The two allied Protestant princes took possession of
the Duke’s territory and committed many atrocities. They introduced
Protestantism there with the aid of Luther and especially Bugenhagen.
Due to Henry’s determination to regain his principality, a new war
broke out in 1545, which ended in even more striking success for his
enemies, who took the Duke captive.

When, in deference to the Emperor, Philip of Hesse and others
favored the release of Duke Henry, Luther published an open letter
to Philip and the elector,\footnote
{Erl. ed,, Vol. XXVI, ii, pp. 251 sqq.; De Wette, \textit{Briefe}, VI, pp. 385 sqq. Cfr. Grisar,
\textit{Luther}, Vol. V, pp. 394 sq.}
in which he characterizes the idea of
setting free that “mischievous, wild tool of the Roman idol,” as an
open attack not merely on the Evangel, but even on the manifest will of
God, as clearly displayed in the recent war, which had
been waged “by His angels.” In this remarkable document Luther
rose to the pinnacle of his morbidly mystical conception of life: God
Himself has kindled this conflagration against his adversaries, God,
who calls Himself a consuming fire. His friends, as well as he, had
always prayed and clamored for peace, but “the pope and the papists
would gladly see us all dead, body and soul, whereas we for our part
would have them all to be body and soul happy, together with us.”\footnote{\textit{Ibid.}, p. 395.}

Luther represents this document as his own exclusive and personal affair.
“Nevertheless,” says Köstlin-Kawerau, “it is not to be doubted”
that Luther “performed a task that had been ordered with the intention
of influencing the Landgrave.”\footnote{Köstlin-Kawerau, \textit{Martin Luther}, Vol. II, p. 612.}
Chancellor Brück had inspired
him with the ideas set forth in that document, for it was extremely important
to the elector to prevent the release of Duke Henry.

The last armed champion of the Catholic cause in Northern Germany succumbed
with Henry of Brunswick. Not until the Schmalkaldic War of 1547, after
Luther’s death, did a favorable change
come, at least temporarily.
