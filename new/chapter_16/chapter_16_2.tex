\section{Religious Poetry and Church Hymns}

A new sphere was opened to Luther’s successful endeavors in the
field of hymnology. In the days of his youth he had become familiar
with the hymns of the Catholic Church and learned to appreciate the value
of congregational singing. He realized that poetry and
songs within and without the church are apt means for conveying
religious thoughts to the hearts of the people. Quite naturally he made
use of this means in the furtherance of his gospel. By bis successful
poetical compositions he created a preéminent and efficacious position
for the religious hymn within the Protestant cult. It supplemented
the sermon and the defective liturgy within the church and aroused
the minds of the faithful with a religious and also a militant fervor
outside the church walls.

The thirty-fifth volume of the Weimar edition of Luther’s works
contains all the hymns composed by him, as collected by Lucke.\footnote{Cf. also Erl. ed., Vol. LVI, pp. 291 sqq.}

The series commences with “Ein neues Lied wir heben an” (composed
in 1523) and, “Nu freut euch, liebe Christen gemein.” In the following
year Luther was most prolific in the production of church
hymns. His “Enchiridion geistlicher Gesänge,” published in that
year, consisted of twenty-five hymns, of which fifteen were his own
work. Somewhat later in the same year, Luther and John Walther,
a cantor stationed at the court of Torgau, published a “Geistliches
Gesangbiichlein” in Wittenberg. This hymn-book, as revised in 1529,
contained several new hymns, notably the celebrated “Ein’ feste Burg
ist unser Gott” (A mighty fortress is our God)\footnote{Grisar, \textit{Lutherstudien}, n. 2: “Luthers Trutzlied.” Cf. Lucke, Weimar ed., Vol. LIII.}
 and also, “Verleih
uns Frieden gnädiglich” and “Herr Gott, dich leben wir.” Other
hymns of Luther originated in the period from 1535 to 1546. The
battle-hymn, “Erhalt uns, Herr, bei deinem Wort, Und steur des
Papsts und Tiirken Mord” (Preserve us, O Lord, in Thy Word, and
check the atrocities of Pope and Turk), was written in 1537.

Luther did not set any of his hymns to music. The melodies were
partly supplied by Walther and many of them are adaptations of earlier
melodies or chorals familiar to the people from Catholic days.
The singing of anthems or secular songs (chorals) by the younger
members of Luther’s household was a favorite means of recreation
after meals. Luther gives expression to his delight thereat in a poem,
“To Lady Music,” which prefaced an edition of one of the abovementioned
hymnals. The conclusion of the poem ardently eulogizes
the nightingale because of her praise of God: “She sings and flits in
praise of Him, And naught her ardent soul can dim; Thus, too, my
lyre would sound His praise, And thank Him through the endless
days.” In another preface Luther develops an excellent discourse on
the educational value of spiritual hymns. In his opinion they should
assist the young “in getting rid of amorous and carnal songs.” All
should be convinced that “not all the arts are overthrown by the
gospel, \dots but I would like to see all the arts, especially music, serve
Him who has bestowed and created them.”

Ratzeberger, the physician, tells of another motive of Luther’s
predilection for the art of music. Luther, he says, “discovered that
he was relieved of great depression by music during his temptations
and melancholy spells.”\footnote{Grisar, \textit{Luther}, Vol. II, p. 171.}
 In matter of fact, the soothing strains of
the church hymns often helped to assuage the storms that agitated
his soul. In a letter to the composer Senfl, who was in the service of
Duke William of Bavaria, and whose motets he esteemed very highly,
he acknowledged that music had often “refreshed his spirit and relieved
him of great troubles.”\footnote{\textit{Ibid.}}
 He requested Senfl to set to music
the text of the Psalm, “In peace I will sleep and rest” (Ps. IV, 9);
for this verse afforded him consolation for his approaching death.
He was as weary of the world as it was of him. The letter to Senfl
incidentally embodied an attempt on Luther’s part to regain the
favor of the Bavarian court. Luther, who was desirous of obtaining a
foothold in Bavaria, evidently attached great importance to the
friendship and activity of this highly esteemed composer.
