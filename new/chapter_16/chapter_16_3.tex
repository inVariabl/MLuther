\section{The Table Talks}

Luther’s Table Talks are of great importance as sources for his
personality and work. These original effusions from his communicative
lips are a profound revelation of his inmost being and often cast a
bright light on the events of his age and life.

Formerly the utilization of the Table Talks was a difficult matter,
as there were only inadequate editions extant, such as the old German
compilation of John Aurifaber, the Latin Colloguia of H. Bindselil,
originally collected by Antony Lauterbach, and other collections.
There was wanting a critical compilation going back to the oldest
textual tradition, as embodied in the transcript of Luther’s utterances
made by various individuals. This task has now been performed by a
painstaking Protestant investigator, Ernst Kroker. In six large volumes
he has collected 7,075 speeches or addresses, reproduced in the
most exact possible order and in accordance with the sequence of the
recorders and the time.\footnote{\textit{Tischreden}, Weimar ed., Vols. I-VI (1912--1921).}
 In this manner, the value of the Table Talks
as historical sources is very much enhanced.

The students who had found lodging in the cells of the former
Black Monastery were wont to assemble daily about Luther’s table,
where the places of honor, next to Luther and his Kate, were occupied
by invited guests from afar, or by friends who lived in the city
of Wittenberg. The students listened attentively to the conversations,
in most of which Luther acted as spokesman. Many instructive matters
were discussed here, many notable thoughts uttered for the benefit of
the education of his auditors, designed to be imparted later on to
their friends and acquaintances. The students soon grew accustomed
to take down in shorthand, either in Latin or in German, as much
of the conversation as they could. Luther observed this, but did not
protest; on the contrary, he frequently asked them to write down
this or that utterance. Kate once jestingly remarked that the copyists
should be obliged to pay for this privilege, just as they were obliged
to pay for their lectures at the university.\footnote
{Cf. the exposition in Grisar, \textit{Luther}, Vol. III, pp. 217--241, for which, however,
Kroker’s edition was not yet available.}

Nevertheless, the conversation was quite unconstrained. Luther
often uttered remarkable opinions on Biblical passages, on theological
or philosophical doctrines, on individuals of both camps, the Protestant
as well as the Catholic, or on his own experiences, on natural
phenomena, on matters pertaining to the present or the future. The
conversation became especially animated when strangers participated
therein. Before all else, however, the audience liked to listen to Luther
himself, the honored and admired oracle of his younger disciples.
Cordatus, one of the copyists, repeatedly expressed his displeasure,
when the loquacious Kate or the talkative Jonas did not allow Luther
sufficient opportunity for speaking.

The first direct copies were made by the students in their rooms. They
were somewhat polished and either jealously preserved or circulated for the
instruction of others. There were collectors who compiled reports which
were derived from diverse sources and originated at various times. Anton
Lauterbach’s collection, made with great diligence, is the largest. In it the
talks are grouped according to the topics discussed and the various points of
view expressed. Besides Lauterbach’s printed work, many such collections
have come down to us, whereas none of the original papers written at table
have been preserved. The general fidelity of the old copies is vouched for not
only by the character and the purpose of the authors, but also by the observation
that, where the same discourse is reported by various parties, there is
usually substantial agreement, despite grammatical or other formal variations.
Misunderstandings of one or the other copyist, omissions, even in important
matters, mistakes owing to inadvertence in the course of reproduction, are, of
course, not excluded. This circumstance must be taken into consideration
when the Table Talks are quoted.\footnote{See Kroker’s introduction to Vol. I of his edition.}


Kroker in his edition supplies the parallel passages and furnishes pertinent
emendations. Hence, these literary remains of Luther’s table must be regarded,
in general, as an adequate historical source concerning his character
and life. Kroker rightly rejects, for example, the objections of Otto
Scheel to important passages which differ from the latter’s theories.\footnote{\textit{Tischreden}, Weimar ed., Vol. V, pp. XIV sqq.}


Of course, it must not be overlooked that the Table Talks are
ephemeral--“children of the moment.” While they correctly and
vividly reproduce the ideas of the speaker, minus the cool reflection
which prevails in the writing of letters and still more of books, they
contain frequent exaggerations and betray-a lack of moderation. The
lightning-like flashes which they emit are not always true. The momentary
exaggerations of the speaker at times beget contradictions
which conflict with other talks or literary utterances. Frequently humorous
statements were received as serious declarations. Humor and
satire of a very pungent kind play a great part in these talks.

The recording of the Table Talks commenced with the year 1531,
or possibly 1529. They are continued, with interruptions, in longer or
more abbreviated and detached communications of the students up to
the last meal taken by Luther.

In point of time the transcripts of Vitus Dietrich and those embodied in
the collection of Dietrich and Rorer are the first.\footnote
{\textit{Ibid.}, Vol. I, p. XVI. Cfr. Kroker in the \textit{Lutherstudien}, edited by the collaborators of
the Weimar edition (1917), pp. 178 sqq.}
These are followed by
three groups of copyists and collectors. The older group consists of John
Schlaginhaufen (for years 1531 and 1532), Cordatus (after 1531), Lauterbach,
Weller and Corvinus. The middle group, who compiled the Tischreden
from 1536 to 1539, consists of some of the above-mentioned writers. Among
these Lauterbach is especially noted for his diaries, which cover the years
1538 and 1539. The later group, from 1540 to 1546, is composed of John
Mathesius, who supplies an excellent source of information, Caspar Heydenreich,
Jerome Besold, Magister Plato, John Stoltz, and John Aurifaber.\footnote
{Cfr. \textit{Tischreden}, Weimar ed., Vol. I, p. XI and the introductions to the various parts in
the following volumes.--The transcripts of Besold are to be published in the Weimar ed.
by J. Haussleiter. Cf. his treatise in the \textit{Archiv für Reformationsgeschichte}, Vols, XIX sqq.}
In addition to this, each one of these groups embraces various smaller manuscripts,
such as those of Pastor Khumer and George Rorer.

The most ample information is furnished by Anton Lauterbach, who has
arranged his notes in topical order.\footnote{These notes form the essential, nay, almost literal content of Bindseil’s Colloguia.}
 The most exact reporter, however, is
George Rorer, the versatile secretary of the committee which revised Luther’s
translation of the Bible.

In the course of the present work we have cited many a typical
passage from the Table Talks. These, and the multifarious discourses
themselves, display extraordinary versatility and profound feeling.
Even though we are compelled to criticize these Talks severely, it
must be acknowledged that Luther’s utterances are permeated by
many sound, stimulating, and pious thoughts.\footnote{Cfr. Grisar, \textit{Luther}, Vol. IV, pp. 262 sqq.}
 Thus there are beautiful
expressions on the attributes of God, particularly His love and
mercy, on the duties of the faithful and their obligations in everyday
life, on the cure of souls, on preaching and education, on charity,
on the vices of the age, on the virtues and vices of great men, past and
present, and so on. It was as much the purpose of the Table Talks
to benefit the hearers spiritually as to cheer them up and to amuse
them. If we take up at random numbers 5553 to 5577, in which
Mathesius, availing himself of Heydenreich’s notes, supplies his readers
with detailed information, we may well marvel at the abundance
of profound and practical ideas. In discoursing on the blindness of
the Jews and the night of God’s wrath against them, for example,
Luther becomes so deeply moved that he folds his hands in prayer
and exclaims: “O heavenly Father, let us remain in the light of the
sun, and permit us not to become recreant to Thy Word!”\footnote{\textit{Ibid.}, 111, 225 sqq.}
 It is not
to be wondered at that Protestants have published many anthologies
of interesting and instructive passages taken from Luther’s Table
Talks. Their good features, with which alone most Protestants are
familiar, have contributed to a general overestimation of the Table
Talks.

Voluminous collections of the Table Talks were published at an
early period. That by Aurifaber appeared in 1566 at Eisleben and
went through several editions. It was reprinted by K. Förstemann and
H. Bindseil in 1844\footnote
{M. Luthers \textit{Tischreden oder Colloquia.} (Based on Aurifaber’s text, but collated with
the redactions of Stangwald and Selnecker.)}
and found its way into the Erlangen edition
of Luther’s collected works as late as 1854 sqq.\footnote{Vols. LVII-LXII.}
 This version is defective,
not only because of frequent arbitrary rearrangements of the
subject-matter and changes in the style of the original text (which
changes were made for the sake of fluency or clearness), but also on
account of an attempt at rendering certain utterances less objectionable
and at toning down extremely blunt expressions.

The learned historian J. G. Walch (died in 1775), in common with
other Protestant scholars, regretted the publication of the Table Talks.
He says that passages in Luther’s colloquies “were revealed which
should have remained unpublished” and surmises that his indiscretions
were the result of “a perversion of the human will.”

On the other hand, many friends of Luther were edified by the
Table Talks. Among the original copyists, for instance, Cordatus
places them at the head of Luther’s writings and regards them as
“more precious than the oracles of Apollo.” Mathesius recalls with
gratitude the “many precious things” he heard at Luther’s table, and
certifies that the ex-monk never uttered “an improper word.”\footnote{Cfr. Grisar, \textit{Luther}, Vol. III, pp. 224 sq.}
In his more recent and popular edition of Luther’s Table Talks, Forstemann
declares them to be the most important part of Luther’s spiritual legacy
because in them “the stream of his genius flows clearest.”
According to Bindseil and Müllensiefen, in their introduction to the
\textit{Colloquia}, Luther’s Table Talks display “the noblest flower of his
nation,” and the repulsive and uncouth passages, while, of course,
not entirely excusable, contribute to the “complete characterization
of the great man,” since they show the “furrows and faults that
formed a part of his personality.”\footnote
{The cited passages are given more completely in Grisar, \textit{op. cit.}, Vol. III, pp. 223, 228,
221, 228 sq., 222.}

The “furrows and faults” revealed in the Table Talks are, as a matter
of fact, so prominent that they overshadow the better features and
the eulogies which we have quoted are well-nigh beyond understanding.
Among these glaring defects are innumerable unjust accusations,
polemical exaggerations, and crying distortions of the Catholic
faith. This applies above all to Luther’s immoderate and blunt
expressions, not to say the vulgar obscenities with which he assails
the pope, the monastic orders, the Mass, etc. The earlier champions
of the Catholic cause are to be pardoned for having again and
again adverted to the sad phenomenon of this filth in their characterization
of Luther. The Table Talks were known to the Catholics
of a later period mostly through the selections thus made from them
by the earlier controversialists. Hence, the opinion formerly entertained
by many Catholics that the Table Talks were mainly a collection
of obscenities. This opinion is as much an exaggeration as
the Protestant eulogies mentioned above. To convince oneself that
the colloquies abound in vulgar and obscene passages, side by side
with excellent features, one need but read a few pages of them at random
or peruse the excerpts which the author of the present book
felt it necessary for the sake of historical truth to reproduce in his
larger work on Luther.\footnote{Grisar, \textit{Luther}, Vol. III, pp. 228 sqq.}

Suffice it to remark here that the sphere of the ventral functions
constitutes the most fertile soil of his amplifications and comparisons.
The students around his table frequently indicate improper remarks
in their manuscripts by signs, such as I or X, where their pen hesitates
to express the dirty word. As noticed before, Luther employs such expressions
with predilection in his references to the pope and Catholicism. The hatred
which inspires his shameless utterances makes them
all the more repulsive. In the entire scope of German letters there is
nothing that may be compared with these excrescences of Luther’s
eloquence, least of all among the representatives of religion or the
heralds of the religious reformation, to whom, of course, he wishes to
belong, though it is quite true that his century was distinguished for
its coarseness.\footnote{\textit{Ibid.}, pp. 236 sq.}
 Caspar Schatzgeyer, one of the mildest among the
Catholic apologists, in rebuking Luther’s coarseness and vulgarity,
says that he befouls the face and garments of his foes with such a mass
of vituperative filth (\textit{conviciorum stercora}) that they are forced to
save themselves by flight from the intolerable stench and dirt.
“Never,” he says, “in any literary struggle has a larger array of
weapons of that sort been seen.”\footnote{\textit{Ibid.}, p. 297.}

Luther used these weapons also against some of those who professed
the new faith. Thus he censures the nobility who refused to provide
an income for the Protestant ministers. They exasperate us unto
evacuation, he says, and continues: “Then \textit{adorabunt nostra stercora}
\dots we are as ready to part as \textit{ein reiffer dreck und ein weit
Arssloch.}”\footnote{\textit{Ibid.}, p. 233.}
Naturally the devil who inspires him and others with
doubts and fears, receives his share of abuse. The manner in which
he teaches his hearers to despise Satan is too revolting to be quoted.
A hint is supplied by the previously mentioned butter-vat of Bugenhagen.
On the other hand, it should be noted, if this be necessary, that
it is not his object to arouse sensuality. His language is coarse, not
lascivious; it arouses disgust, not the evil passions of man’s lower
nature.

In that age and in the succeeding century the unhappy after-effects
of this coarseness led to a certain corruption of the German
language, due to the fact that it came from the mouth of a man who
was so highly admired as Luther, and that his Table Talks were read
in every home. The dregs of vulgarity which had been stirred up by
Luther were for a long time a sad dowry of the so-called “Grobian
Age” and the polemical literature against Catholics.
