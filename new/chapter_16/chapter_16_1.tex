\section{Engaging Characteristics}

In reviewing the life of Luther in the former Black Monastery of
Wittenberg, our attention is first attracted to his relations with Catherine.
Although there were weighty objections to their marriage from
the Catholic point of view, and although it was severely censured by
the jurists who upheld the canon law of the Church, it nevertheless
presented a favorable exterior appearance. It had to be admitted that
peace, harmony, and mutual good will governed the union of the
ex-monk and the former nun. So far as known, neither ever violated
the pretended marriage. Luther expressed himself in words of
gratitude and appreciation for the aid and comfort which he derived
from his wife, even though, on occasion, he scourged her willfulness
in partly serious and partly facetious language.

Luther’s children were compelled to learn and practice their religion.
As they grew up, they, on the whole, caused no dishonor to the
family. They were not endowed with any special talents, nor did
they distinguish themselves in their positions in life.
The home life of the family was subject to considerable unrest,
caused by the fact that relatives and students occupied the former
monastic cells and ate at Luther’s table. In addition, quite a few
strangers visited Wittenberg, who wished to see and converse with
Luther. Moreover, the agitation caused by Luther’s controversies,
which so visibly vibrates in his correspondence, quite naturally affected
his domestic life, as his Table Talks frequently testify.

On the other hand Luther’s family life displayed many attractive
traits. Thus, when his daughter Magdalen, a sweet and pious child,
died at the age of thirteen, Luther was seized with a sorrow so profound
as to move even the modern reader to tears.\footnote{Cf. Köstlin-Kawerau, \textit{M. Luther}, Vol. II, p. 596.}
Thanks to his
letters, his admirers are likewise enabled to participate in the happy
hours he spent in his family circle. Luther is frequently pictured as
a happy father sitting with his family under the Christmas tree. But
the Christmas tree was not introduced till several centuries after his
death. Luther’s family life at Wittenberg is usually celebrated by
Protestant biographers as the model and archetype of that of an
evangelical pastor. But we must not forget--to mention only one
point--that the Reformer’s home, being the center of a tremendous
religious conflict, cannot have been so devout and tranquil as we are
asked to believe.

Luther desired every father of a family to interpret the Bible to his
family and to address them on religious matters in accordance with
a prescribed plan. He himself set the example. When, in 1532, sickness
prevented him from preaching in church, he preached to his
household in the Black Monastery. This custom gave rise to his
“\textit{Hauspostille},” \textit{i.e.}, book of instructions for home
use. It was intended as a guide for others and undoubtedly did much good. It was
first edited in 1544 by Vitus Dietrich.\footnote{Weimar ed., Vol. LII; Erl. ed., Vols. I--VI.}
Of larger scope and wider influence
was the “\textit{Kirchenpostille},” a collection delivered in public.
Of these, he published the sermons for the winter semester in 1540.
The sermons for the summer semester were brought into shape and
published by Cruciger in 1545.\footnote
{Cf. Weimar ed., Vol. VII, p. 463; 10, I, 1; 17, II, 21; 22. Erl. ed., Vol. VII, 2 ed.,
p. 134; Vol. VIII, pp. 11, 173; Vol. IX, p. 1; Vol. X, 2 ed, p. 133; Vol. XI, p. 1913
Vol. XII, p. 1.}
A very large number of Luther’s
sermons had been circulating in separate editions or in smaller compilations
since their delivery.

Luther’s sermons are invariably distinguished by great freshness
and practicality. They display a forcefulness of diction and a diversity
of thought, the like of which is scarcely met with elsewhere. He
possessed sufficient talent to become a second Berthold of Regensburg.
It must be admitted, however, that the addresses are often monotonous
on account of frequent repetitions and show lack of preparation and reliance
upon the author’s innate gift of speech. Sometimes
his auditors were bored by his noisome and tedious attacks upon the
ancient Church and her doctrines. Despite these defects, however,
Luther’s sermons were so diligently copied that a large number of
copies, made by various individuals, are still extant. The new Weimar
edition of Luther’s works reproduces them all, thereby occasionally
bestowing unmerited honor on addresses which were delivered without
due preparation and order.

In their originality some of his better sermons, and also some of the
inferior ones, are reminiscent of Luther’s maxim: “Ascend the pulpit,
open your mouth, and then stop.”\footnote
{Thus Kroker (\textit{Tischreden}, Weimar ed., Vol. VI, p. 643) translates the saying (\textit{ibid.},
Vol. IV, n. 5171a): “\textit{Ascendat suggestum, aperiat os et desinat}” (cf. \textit{ibid.}, n. 5171b.)}
Luther frequently addressed similar
maxims to his preachers.

Despite his facility in the use of words, the voluminousness of his
sermons is a source of amazement. He was anxious to produce moral
effects no less than to confirm his new doctrine and to eradicate popery.
Relative to morality, he felt a profound obligation to counteract the
decline of ethical standards, which was a concomitant of the
new freedom proclaimed by his gospel. His very desire to preserve the
good repute of his religious innovation impelled him to issue frequent
warnings and reproofs. Moreover, as he had abolished the holy Sacrifice
of the Mass, the office of preaching was advanced to the most
prominent place in his church. Everything was made dependent
upon the “Word,” which was supposed to be experienced interiorly
and to persist without the aid of the Catholic means of grace and the
weight of ecclesiastical authority.

Luther intended to introduce the interdict when to his sorrow he
saw how weak was the influence exercised by his Wittenberg pulpit
and how scandals grew apace. He always had felt the need of some
kind of ecclesiastical discipline, though at the same time he dever
gave up the idea of a “church apart of true believers,” who having
expressly obligated themselves to observe religion and morality,
should take their stand alongside the partly heathen masses of the
national Church.\footnote{Cfr. Grisar, \textit{Luther}, Vol. V, pp. 133 sqq.}
The plan proved impracticable. As the Protestant
theologian Drews says, Luther himself “was uncertain and
wavered in the details of his plan. He had but little bent to sketch out
organizations even in his head; to this he did not feel himself
called.”\footnote{Grisar, \textit{op. cit.}, Vol. V, p. 140,}
This was also the reason why his proposal to introduce the
ban, which he made in 1538, and again in a sermon at Wittenberg
on February 23, 1539, came to naught. He was compelled to lament:
“They refuse to hear of excommunication.”\footnote{\textit{Ibid.}, p. 186,}
Which utterance recalls
the words of the Elector: “If only people could be found who would
let themselves be excommunicated!” And yet there was question only
of the so-called minor excommunication, namely, exclusion from
divine worship, or at least from the Lord’s Supper, and prohibition to
act as sponsor at baptism. It was Luther’s intention that not only the
ecclesiastical authorities, but the entire congregation, should inflict the
ban, just as was the rule in Hesse, under the “Regulations for Church
Discipline” drawn up for that country.

Luther, to be sure, was not unwilling to exercise severity. Thus he
writes to Antony Lauterbach at Pirna: “I am pleased with Hesse’s
example of the use of excommunication. If you can establish the same
thing, well and good. But the centaurs and harpies of the court will
look at it askance. May the Lord be our help! Everywhere license and
lawlessness continue to spread among the people, but it is the fault of
the civil authorities.”\footnote{On April 2, 1542; \textit{Briefwechsel}, Vol. XV, p. 131; cfr. Grisar, \textit{Luther}, Vol. V, p. 188,}


In a sermon of February 23, 1539, wherein he vigorously developed
the idea of the lesser excommunication, he maintained the duty of
the entire congregation to co-operate in the enforcement of the ban.
After the public denunciation of an obdurate member, the congregation
was to lift its voice in prayer against him, assist in the formal
expulsion, and participate in the readmission of the excommunicate
to public worship.\footnote{\textit{Ibid.}}
When he saw that his zeal was not appreciated,
Luther threatened public offenders all the more violently with harsh
treatment after death: “Let them go to the devil, and if they die,
let them be buried on the rubbish-heap like dogs.” Whoever obstinately
remains away from the Lord’s Supper lives “in a self-inflicted
ban” and is to be delivered to the civil authority.\footnote
{Grisar, V, p. 189, where more passages are given. \textit{Tischreden}, Weimar ed., Vol. IV,
n. 5174; Vol. V, n. 5438.}

The civil government was obliged by law to lend its aid to support
ecclesiastical discipline. Luther favored this procedure; for “facts have
shown”--thus he wrote to Spalatin in 1527--“that men despise the
evangel and insist on being compelled by the law and the sword.”\footnote{Grisar, \textit{Luther}, Vol. VI, p. 262.}

In 1529 he demanded that even those who had no religion yet should
“be driven to attend the sermon” in order that they may know what
is right or wrong.\footnote{\textit{Ibid.} pp. 743 sq., where the following passages may be found.}
 According to his Small Catechism, the masses
must be “held and driven to the faith.” Particularly should they be
held to attend catechetical instruction, as he advised Margrave George
of Brandenburg. At Wittenberg those who persistently neglected to
attend the sermons were threatened with “banishment and the law.”
The court ordained that there be “universal attendance at church.”
In 1557 we hear of a fine imposed upon violators; in case of poverty
they were “to be punished by being fastened to the church or a
prison by means of an iron collar.” The oppressive policy of the State
Church of Saxony resulted from the force of circumstances and the
endeavor to achieve some kind of union among Protestants. The
State transferred its rule to the spiritual sphere, which usurpation,
even at its inception, provoked loud protests from Luther and many
preachers. Nevertheless, the process of evolution could not be arrested.

History is obliged to chronicle many instances where Luther displayed
great courage for the sake of preserving religious discipline
and ecclesiastical customs. A case in point is that of Hans von Metzsch,
a haughty captain and governor of Wittenberg, who led a dissolute
life. In 1531 Luther notified this powerful man that he was excluded
from ecclesiastical communion and forbade him--though not publicly--to
receive the Lord’s Supper. When Metzsch married his
mistress in accordance with the prescribed regulations, a reconciliation
was effected. Nevertheless, in 1538, Luther again censured the
governor, this time with increased vigor, because of his affronts
against public worship and the preachers. He pronounced invalid the
absolution which the deacon Froschel had granted to him, and, in a
statement served upon him by two deacons, demanded that Metzsch
reform and become reconciled with the Church and with those whom
he had offended. At the same time he apprised him that he would
incur excommunication if he refused to conform with these demands.
Metzsch was also threatened with major excommunication on the part
of the prince in case he continued recusant. The subsequent course of
events is doubtful; it appears, however, that some kind of peace was
again patched up.\footnote{Köstlin-Kawerau, \textit{M. Luther}, Vol. II, pp. 438 sq.}
In the following year, Luther inveighed from
the pulpit against a citizen of Wittenberg who had approached the
Lord’s Supper though he had committed a murder and was unreconciled
with the Church. He insisted that this man render strict satisfaction
before being readmitted to church.

Mention has been made on a previous page of the courage which
Luther displayed at the time of the pestilence. Unmindful of the
danger of contagion, he remained at his post, although many proved
themselves deserters. He resolutely endeavored to be of service to the
afflicted and to encourage his clerical assistants in persevering by the
power of his example. As early as 1527, during those critical days
when Wittenberg and its environs were ravaged by the epidemic, he
composed a treatise: “Whether One Should Flee from Death,” which
contained beautiful and encouraging thoughts calculated to comfort
the afflicted.\footnote
{Weimar ed., Vol. XXIII, pp. 333 sqq.; Erl. ed.,, Vol. XXII, pp. 317 sqq.; Köstlin-
Kawerau, \textit{M. Luther}, Vol. II, pp. 171 sqq.}

Courageously and lovingly he used his influence on many occasions
to secure redress for those who were the victims of injustice.\footnote
{Köstlin-Kawerau, \textit{op. cit.}, p. 420.}
Because of the esteem in which he was held, and his willingness to
minister to others, his aid and intercession with the Elector were frequently
invoked. His petitions, as a rule, were effective. His protestations against
oppression, even though they assumed most vigorous
forms, were usually heeded at court. On one occasion he called
himself the supporter of the poor and defender of their rights. At
times it happened that, in his short-sightedness, he permitted himself
to become interested in cases where justice was on the side of the
other party. Frequently he displayed undue credulity and anger. A
case in point--it was on the occasion when he assumed the honorary
titles quoted above--is furnished by his advocacy of the cause of
Hans von Schénitz of Halle, who had been legally executed by the
Elector Albrecht of Mayence for serious crimes which he had committed.
The brother of the executed man together with one Louis
Rabe succeeded in convincing Luther that the Archbishop, whom
Luther cordially hated, was guilty of murder. In 1535, and again in
1536, Luther published two letters against Albrecht concerning this
case. In 1538 he composed a treatise on the alleged Schénitz scandal,
in which he forcibly vented his indignation.\footnote
{Grisar, \textit{Luther}, Vol. V, pp. 106 sq.; Köstlin-Kawerau, \textit{op. cit.}, Vol. II, pp. 419, 422;
see also the note in the third German ed. of Grisar’s \textit{Luther}, Vol. III, pp. 1009 sqq. (this
note is not contained in Lamond’s English translation).}

A merchant from Kölln on the Spree, Hans Kohlhase, who had
failed to obtain a favorable verdict in a lawsuit, became enraged and
declared a private feud against the entire commonwealth of Electoral
Saxony. It was a procedure incomprehensible to our age, which finds
its explanation only in the then prevalent conditions. With the aid
of a mercenary mob from Brandenburg, Kohlhase began to “rob,
burn, capture, and hold to ransom,” according to his own formal
announcement. Conflagrations, attributed to his revengeful spirit,
broke out in Wittenberg and its environs. The Elector was disposed
to effect an amicable settlement and Kohlhase sought Luther’s advice.
He received a trenchant reply, in which Luther vigorously espoused
the cause of law and order and demanded that vengeance be left to
God. At the same time he addressed ardent religious exhortations to
the offender. Kohlhase, however, being dissatisfied with the offers of
the Elector, continued his depredations. Luther prophesied in his Table
Talks that Kohlhase would be drowned in his own blood. He was
executed at Berlin, March 22, 1540, being broken on the wheel on
account of excesses committed in Brandenburg. Fable has seized upon
the story of this gruesome adventurer angd his relation to Luther.
Popular biographers of Luther still love to relate how Kohlhase, in
disguise, knocked at Luther’s door one dark night and, being admitted
by the latter, explained his quarrel in the presence of Melanchthon, Cruciger,
and others, became reconciled with God and his fellowmen, and promised
to abstain from violence in future.\footnote
{Grisar, \textit{Luther} (Engl. tr.), Vol. V, pp. 117 sqq.; \textit{Tischreden}, Weimar ed., Vol. IV,
n. 4088, 4315, 4536.}

To some extent this legend (originating with a chronicler who
gives no authority for his statements) is reminiscent of the conciliatory
attitude which Luther assumed toward his enemy Karlstadt when the latter,
plunged in direst need after the Peasants’ War,
approached Luther in 1525 and asked him to intercede for him with
the Elector, so that he might be permitted to return to the country.
Luther, having obtained Karlstadt’s promise that he would change
his doctrines, magnanimously procured for him the permission he
craved.\footnote{Köstlin-Kawerau, \textit{M. Luther}, Vol. I, pp. 718 sq.}


Luther was frequently generous towards the poor, even beyond
his humble means. His simple way of life made it possible for him to
practice benevolence. He lived frugally and was satisfied with but
little of this world’s goods. This fact was generally known and, in
view of his meager income, the court and the city gladly helped him
along with gifts of money and food. Poor students were the chief
beneficiaries of his solicitude; for he was very much concerned that
the support of the students of the University of Wittenberg should
be assured. His amiable disposition and sociability strongly attracted
the students, who were charmed by his fame no less than by his robust
physique and characteristically flashing eyes.

Luther’s lectures, which began in the early hours of the morning,
were carefully prepared and embodied practical directions for the
usually large audiences which attended them. His graphic interpretations
of the Bible and his meditations, which at times bordered on the
mystical, were interspersed with frequent sallies (not always phrased
in choice language) against Catholic dogmas, the papists, and the
enemies in his own camp, the so-called \textit{Schwärmer} (fanatics). When
they disagreed with him, he did not spare even the most highly esteemed
Fathers of the Church. With a self-consciousness that made a
profound impression on the short-sighted young men to whom he
lectured, he exalted his own opinions above those of others. Proofs
of this are amply supplied by his Commentary on St. Paul’s Epistle to
the Galatians and his exposition of Genesis, which he commenced in
1535, but which was published by someone else, and not very accurately.\footnote{Weimar ed., Vols. XLII-XLIV.}


There is extant also a collection of proverbs made by Luther, which
did not, however, appear in print until 1900.\footnote{Weimar ed., Vol. LI, pp. 645 sqq.}
 It reflects his efforts
to preserve and increase the treasury of German proverbial sayings
of which he was wont to avail himself so liberally. Besides these, we
have the theological disputations held before the faculty under
his direction and amid constant interruptions on his part. These disputations
(1535-1545) were edited in a stately volume by Dr. Paul
Drews in 1896. They show many traces of Luther’s passionate nature
and are characterized by rude diction and an attempt to go to extremes
in expression as well as content.
