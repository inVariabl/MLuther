\section{The Hearing at Augsburg. Miltitz}

The German Dominicans reported to the Roman curia fresh accusations
against the impetuous Augustinian. A certain rivalry between
these two great and influential bodies may have to some extent
prompted this procedure; but it was not the decisive motive. Older
Catholic authors, relying too confidently on contemporaneous verdicts,
have mistakenly endeavored to trace the origin of the religious
schism to the jealousy that existed between the Dominican and the
Augustinian orders, inasmuch as the former espoused the cause of
Tetzel and his indulgence sermons, whereas the latter rose in defense
of Luther and his courageous utterances. Whatever may have been in
the background, as far as the attitude of Rome was concerned, the
documents which were before the papal curia, namely, the Wittenberg theses,
the disputations conducted there, the Heidelberg theses
of Luther, and his “Resolutiones,” decided the issue.

The Pope, acting in conformity with the law, through his fiscal
procurator and the auditor of the camera, cited Luther to Rome,
where he was to present himself within sixty days. At the same time,
the theologian of the sacred palace (\textit{magister sacri palatii}), the
Dominican Sylvester Mazzolini, a native of Prierio (therefore called
Prierias) , was entrusted with the task of preparing an expert opinion.
As he had evidently been exactly informed of the case some time
before, he completed his task within three days. His printed opinion
was a complete apologia for Tetzel and his teaching on indulgences.
He employed unnecessarily bitter expressions against Luther--following
a custom in vogue in those times of controversies against heretics. He
placed special emphasis on the power of the pope and his right
of definitively deciding all ecclesiastical questions, which he set forth
in most forceful terms.\footnote{\textit{Op. cit.}, Vol. IV, pp. 373 sq.}
When Sylvester’s pamphlet with the citation
to Rome reached Luther, he realized to his consternation, as he tells
us, how serious the situation really was. He made a hasty, all too
hasty, decision and published an indignant response (\textit{Responsio}) to
Prierias.\footnote{\textit{Op. cit.}, Vol. IV, 374 sq.}
Relative to the citation, he on the very next day issued a
written appeal for help to the Elector, requesting that dignitary
to insist that the case be tried in Germany. Frederick “the Wise”
was in attendance at the diet of Augsburg at the time.

This diet was also attended by the papal legate, Cardinal Cajetan
de Vio, a highly respected Dominican theologian. The Elector told
him that he would not let his university professor go to Rome. In
the meantime (August 9), the Emperor Maximilian had sent a
vigorous letter to the Pope, assuring him that he would execute the
decision of the Holy See against Luther with all his energy. Luther’s
efforts to prevent this course had proved futile. Assured of imperial support,
the curia decided to accelerate the procedure against
the growing evil. It was also intended to comply with the request of
the Elector to have the hearing conducted in Germany in order to
secure the help of this powerful prince in the war against the Turks.
For this reason Cardinal Cajetan received orders from Rome to
summon Luther to appear before him in person to recant his errors.
The order was accompanied by another, to the effect that Luther,
in the event of his refusal to recant, should be apprehended forthwith and delivered to the Roman authorities. The Elector and the
provincial of the Saxon Augustinians received simultaneous orders
to assist in apprehending Luther in case such a measure should become necessary. On September 11, the legate received a further
document from Rome, empowering him to conduct the case against
Luther according to his own discretion.

Luther appeared before Cajetan at Augsburg, on October 12.
The first hearing and the subsequent meetings were fruitless. With
moderation and dignity Cajetan demanded the retraction of two
theses of Luther’s: one denying the treasury of merits gained by
Christ and the Saints, which was the foundation of the doctrine of indulgences;
the other, contained in the ``Resolutiones,'' asserting
that faith alone renders the Sacraments of the Church efficacious. He
disregarded the other theses because, although false, they did not
so patently offend against theological truths. As to the treasury of
merits, many theologians, among them Cajetan himself, held that it
had already been defined by Pope Clement VI. In any event, like the
independent efficacy of the Sacraments, it constituted an important
doctrine of theology.

Despite all his kindness and determined earnestness, Cardinal
Cajetan’s efforts proved futile. Luther manifested arrogance and offensive
obstinacy. Dismissed with the threat of excommunication,
he announced to the Cardinal that he would appeal from his tribunal
“to the Pope, who would be more correctly informed.” To others
he said that this was but a preparation for an appeal to a general
council, which was bound to follow in case the Pope, “in the plenitude
of his authority, or rather tyranny,” would reject his appellation. Luther
did not tarry for a reply from the hesitating Cardinal,
but secretly fled from the city and hastened back to Wittenberg.

The Cardinal had also made advances to the Elector, in order to
influence his attitude regarding the election of a German king in
conformity with the intentions of Rome. Frederick subsequently
asserted that he had obtained from Cajetan the promise that Luther
would be returned to Germany in any event. It is certain that the
scholarly and gentle prince of the Church was no match for the
cunning diplomacy of the Elector.

The formerly popular legend of Cajetan’s haughty treatment of
Luther is now admitted to be unhistorical even by Protestant writers.
The Cardinal is described as “humble, just, and self-sacrificing” and
his conduct towards Luther as dignified; he is admitted to have been
“an earnest and, in his judgment concerning the abuses prevalent
at the curia, a strict and free-spoken thinker.”\footnote{Paul Kalkoff, \textit{Luther und die Entscheidungsjahre}, pp. 57, 157; \textsc{Idem} in \textit{Kirchengesch}.
\textit{Forschungen, E. Brieger dargebracht}, 1912, and in the \textit{Theol. Studien und Kritiken}, 1917,
p. 246. Similarly Hermelink in the \textit{Theol. Rundschau}, 1917, p. 141}
Luther, on the
contrary, accuses him of being “most wofully ignorant” and of having
treated him like a lion.\footnote{\textit{Briefwechsel}, I, p. 282.}

The question which now tormented Luther was whether he would
be safe at Wittenberg. He thought of going to Paris, where the
theological faculty of the University was engaged in a quarrel with
Rome. But his friend Spalatin, a preacher at the court of the Elector
Frederick, provided protection through that ruler. Luther entered
into a lively correspondence with Spalatin, through whom he assured
the prince that he would gladly go into exile rather than embarrass
him through the fury of his enemies. In the interim, on November
28, he appeared before a notary and two witnesses, and drew up a
solemn appeal to a general council. In the lengthy formula he declared
his intention to do or say naught against the Roman Church, the
teacher and head of all the churches, nor against the authority of
the Pope--as long as the latter were well advised (bene consultus).
The affidavit was signed, as the subscription of the notary attests, “in
the chapel of Corpus Christi, situated in the cemetery of the parish
church.” The little church was a pretty structure erected in honor
of “Christ’s holy body,” such as adorned many a churchyard in those
days.

Luther’s appeal to an ecumenical council, like his former appeal,
was inadmissible and ineffective. According to canon law, an appeal
to a council was a penal offense. This provision was justified by the
answer to the question: Has any individual who wishes to create a
schism within the Church the right to convoke all the bishops of
the world to a council, prior to his submission?

On December 18, the Elector addressed a letter to Cardinal
Cajetan, in which he disclosed to the latter the line of action he had
resolved upon with reference to his protégé, and to which he
always adhered. Luther’s doctrine, he said, had not as yet been proven
heretical; Luther was prepared to appear before a university for a disputation
and formal examination; hence, nothing could be done to
him at Wittenberg. But Rome proceeded directly, though, out of
regard for the Elector and his participation in the great questions
of ecclesiastical policy which were then pending, it proceeded with
notable slowness. At first an attempt was made to influence Frederick
by sending him the Golden Rose blessed by the Pope. It was customary
to send it annually to a prince as a mark of distinction. The
presentation was to have been made by the Roman notary and titular
chamberlain, Karl von Miltitz. The selection of this Saxon nobleman
was not a happy one. Miltitz undertook his commission with great
pomp, but in the end executed it in a very ineffective manner. He was
an incompetent man and a seeker of benefices.

In order to persuade Frederick to deliver up Luther, Miltitz, of
his own accord, adopted wrong methods. In the Dominican monastery at Leipsic
he overwhelmed Tetzel with unmerited and bitter
reproaches, which are said to have hastened his death. Luther consented
to make a doubtful promise to the importunate agent, who
exceeded his commission, namely, to observe silence if his opponents
did the same. There could be no question of a general silence on the
Catholic side in view of the ever increasing dangers that threatened
the Church; and, on the other hand, Luther was far from expecting
his opponents to observe silence, or from being silent himself. Under
the influence of Miltitz, Luther at that time published a curious work
under the title of “\textit{Unterricht}” (Instructions), etc., which contained
both affirmations and negations, in order to conciliate his opponents.

Miltitz sent boastful reports of the success of his efforts to the
Roma curia, and they were not entirely devoid of effect. The death
of Emperor Maximilian, on January 12, 1519, and the fact that
Frederick of Saxony had some prospects of becoming emperor, supplied Leo
X with a reason for new delays. Finally the Pope, in a friendly
brief (\textit{Paterno affectu}), issued March 29, 1519, summoned Luther
to Rome to receive personal instructions and abandon his erroneous
doctrines. It cannot be proved that the treatment accorded Luther
was severe and ill-considered. When the brief arrived, steps had already
been taken by Luther for the Leipsic disputation, which destroyed every
hope of arriving at an understanding. At Rome this
measure occasioned the termination of the trial which had already
been too long drawn out.
Luther’s pen was not exclusively devoted to attacks. With impetuous activity
he had in the meanwhile composed a series of tracts
which, beside those mentioned above, were dedicated in part to a
glorification of his cause, and in part written to pastoral requirements.
His popular religious writings were intended to invest him
with the indispensable reputation of a man who was solicitous solely
about the welfare of souls. This activity gained for him a large following
among religious-minded people. Among other things he published,
in that period of stress, a serviceable explanation of the Our Father, a
short instruction on confession, a condensed explanation of the
Decalogue, and an interpretation of Psalm 109 (Vulg. 110). Even % roman -> arabic numerals
before this he had entered the field of popular literature with an
exposition of the seven penitential Psalms, a sermon on the Ten Commandments,
and some other smaller writings.

His history of the Augsburg trial (\textit{Acfa Augustana}), on the other
hand, as well as his edition of the “Replica” of Sylvester Prierias
against his “Responses,” were polemical. By publishing a reprint of
the “Replica” of the Master of the Sacred Palace, he intended to
represent Prierias as a man entirely devoid of importance and worthy
of disdain.\footnote{Grisar, \textit{Luther}, Vol. IV, p. 375.}
