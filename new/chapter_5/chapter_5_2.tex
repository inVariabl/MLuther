\section{The Hearing at Augsburg. Miltitz}

The German Dominicans reported to the Roman curia fresh accusations
against the impetuous Augustinian. A certain rivalry between
these two great and influential bodies may have to some extent
prompted this procedure; but it was not the decisive motive. Older
Catholic authors, relying too confidently on contemporaneous verdicts,
have mistakenly endeavored to trace the origin of the religious
schism to the jealousy that existed between the Dominican and the
Augustinian orders, inasmuch as the former espoused the cause of
Tetzel and his indulgence sermons, whereas the latter rose in defense
of Luther and his courageous utterances. Whatever may have been in
the background, as far as the attitude of Rome was concerned, the
documents which were before the papal curia, namely, the Wittenberg theses,
the disputations conducted there, the Heidelberg theses
of Luther, and his “Resolutiones,” decided the issue.

The Pope, acting in conformity with the law, through his fiscal
procurator and the auditor of the camera, cited Luther to Rome,
where he was to present himself within sixty days. At the same time,
the theologian of the sacred palace (\textit{magister sacri palatii}), the
Dominican Sylvester Mazzolini, a native of Prierio (therefore called
Prierias) , was entrusted with the task of preparing an expert opinion.
As he had evidently been exactly informed of the case some time
before, he completed his task within three days. His printed opinion
was a complete apologia for Tetzel and his teaching on indulgences.
He employed unnecessarily bitter expressions against Luther--following
a custom in vogue in those times of controversies against heretics. He
placed special emphasis on the power of the pope and his right
of definitively deciding all ecclesiastical questions, which he set forth
in most forceful terms.\footnote{\textit{Op. cit.}, Vol. IV, pp. 373 sq.}
When Sylvester’s pamphlet with the citation
to Rome reached Luther, he realized to his consternation, as he tells
us, how serious the situation really was. He made a hasty, all too
hasty, decision and published an indignant response (\textit{Responsio}) to
Prierias.\footnote{\textit{Op. cit.}, Vol. IV, 374 sq.}
Relative to the citation, he on the very next day issued a
written appeal for help to the Elector, requesting that dignitary
to insist that the case be tried in Germany. Frederick “the Wise”
was in attendance at the diet of Augsburg at the time.

This diet was also attended by the papal legate, Cardinal Cajetan
de Vio, a highly respected Dominican theologian. The Elector told
him that he would not let his university professor go to Rome. In
the meantime (August 9), the Emperor Maximilian had sent a
vigorous letter to the Pope, assuring him that he would execute the
decision of the Holy See against Luther with all his energy. Luther’s
efforts to prevent this course had proved futile. Assured of imperial support,
the curia decided to accelerate the procedure against
the growing evil. It was also intended to comply with the request of
the Elector to have the hearing conducted in Germany in order to
secure the help of this powerful prince in the war against the Turks.
For this reason Cardinal Cajetan received orders from Rome to
summon Luther to appear before him in person to recant his errors.
The order was accompanied by another, to the effect that Luther,
in the event of his refusal to recant, should be apprehended forthwith and delivered to the Roman authorities. The Elector and the
provincial of the Saxon Augustinians received simultaneous orders
to assist in apprehending Luther in case such a measure should become necessary. On September 11, the legate received a further
document from Rome, empowering him to conduct the case against
Luther according to his own discretion.

Luther appeared before Cajetan at Augsburg, on October 12.
The first hearing and the subsequent meetings were fruitless. With
moderation and dignity Cajetan demanded the retraction of two
theses of Luther’s: one denying the treasury of merits gained by
Christ and the Saints, which was the foundation of the doctrine of indulgences;
the other, contained in the ``Resolutiones,'' asserting
that faith alone renders the Sacraments of the Church efficacious. He
disregarded the other theses because, although false, they did not
so patently offend against theological truths. As to the treasury of
merits, many theologians, among them Cajetan himself, held that it
had already been defined by Pope Clement VI. In any event, like the
independent efficacy of the Sacraments, it constituted an important
doctrine of theology.

Despite all his kindness and determined earnestness, Cardinal
Cajetan’s efforts proved futile. Luther manifested arrogance and offensive
obstinacy. Dismissed with the threat of excommunication,
he announced to the Cardinal that he would appeal from his tribunal
“to the Pope, who would be more correctly informed.” To others
he said that this was but a preparation for an appeal to a general
council, which was bound to follow in case the Pope, “in the plenitude
of his authority, or rather tyranny,” would reject his appellation. Luther
did not tarry for a reply from the hesitating Cardinal,
but secretly fled from the city and hastened back to Wittenberg.

The Cardinal had also made advances to the Elector, in order to
influence his attitude regarding the election of a German king in
conformity with the intentions of Rome. Frederick subsequently
asserted that he had obtained from Cajetan the promise that Luther
would be returned to Germany in any event. It is certain that the
scholarly and gentle prince of the Church was no match for the
cunning diplomacy of the Elector.

The formerly popular legend of Cajetan’s haughty treatment of
Luther is now admitted to be unhistorical even by Protestant writers.
The Cardinal is described as “humble, just, and self-sacrificing” and
his conduct towards Luther as dignified; he is admitted to have been
“an earnest and, in his judgment concerning the abuses prevalen
the curia, a strict and free-spoken thinker.”\footnote{Paul Kalkoff, \textit{Luther und die Entscheidungsjahre}, pp. 57, 157; \textsc{Idem} in \textit{Kirchengesch}.
\textit{Forschungen, E. Brieger dargebracht}, 1912, and in the \textit{Theol. Studien und Kritiken}, 1917,
p. 246. Similarly Hermelink in the \textit{Theol. Rundschau}, 1917, p. 141}
Luther, on the
contrary, accuses him of being “most wofully ignorant” and of having
treated him like a lion.\footnote{\textit{Briefwechsel}, I, p. 282.}

The question which now tormented Luther was whether he would
be safe at Wittenberg. He thought of going to Paris, where the
theological faculty of the University was engaged in a quarrel with
Rome. But his friend Spalatin, a preacher at the court of the Elector
Frederick, provided protection through that ruler. Luther entered
into a lively correspondence with Spalatin, through whom he assured
the prince that he would gladly go into exile rather than embarrass
him through the fury of his enemies. In the interim, on November
28, he appeared before a notary and two witnesses, and drew up a
solemn appeal to a general council. In the lengthy formula he declared
his intention to do or say naught against the Roman Church, the
teacher and head of all the churches, nor against the authority of
the Pope--as long as the latter were well advised (bene consultus).
The affidavit was signed, as the subscription of the notary attests, “in
the chapel of Corpus Christi, situated in the cemetery of the parish
church.” The little church was a pretty structure erected in honor
of “Christ’s holy body,” such as adorned many a churchyard in those
days.

Luther’s appeal to an ecumenical council, like his former appeal,
was inadmissible and ineffective. According to canon law, an appeal
to a council was a penal offense. This provision was justified by the
answer to the question: Has any individual who wishes to create a
schism within the Church the right to convoke all the bishops of
the world to a council, prior to his submission?

On December 18, the Elector addressed a letter to Cardinal
Cajetan, in which he disclosed to the latter the line of action he had
resolved upon with reference to his protégé, and to which he
always adhered. Luther’s doctrine, he said, had not as yet been proven
heretical; Luther was prepared to appear before a university for a disputation
and formal examination; hence, nothing could be done to
him at Wittenberg. But Rome proceeded directly, though, out of
regard for the Elector and his participation in the great questions
of ecclesiastical policy which were then pending, it proceeded with
notable slowness. At first an attempt was made to influence Frederick
by sending him the Golden Rose blessed by the Pope. It was customary
to send it annually to a prince as a mark of distinction. The
presentation was to have been made by the Roman notary and titular
chamberlain, Karl von Miltitz. The selection of this Saxon nobleman
was not a happy one. Miltitz undertook his commission with great
pomp, but in the end executed it in a very ineffective manner. He was
an incompetent man and a seeker of benefices.

In order to persuade Frederick to deliver up Luther, Miltitz, of
his own accord, adopted wrong methods. In the Dominican monastery at Leipsic
he overwhelmed Tetzel with unmerited and bitter
reproaches, which are said to have hastened his death. Luther consented
to make a doubtful promise to the importunate agent, who
exceeded his commission, namely, to observe silence if his opponents
did the same. There could be no question of a general silence on the
Catholic side in view of the ever increasing dangers that threatened
the Church; and, on the other hand, Luther was far from expecting
his opponents to observe silence, or from being silent himself. Under
the influence of Miltitz, Luther at that time published a curious work
under the title of “\textit{Unterricht}” (Instructions), etc., which contained
both affirmations and negations, in order to conciliate his opponents.

Miltitz sent boastful reports of the success of his efforts to the
Roma curia, and they were not entirely devoid of effect. The death
of Emperor Maximilian, on January 12, 1519, and the fact that
Frederick of Saxony had some prospects of becoming emperor, supplied Leo
X with a reason for new delays. Finally the Pope, in a friendly
brief (Paterno affectu), issued March 29, 1519, summoned Luther
to Rome to receive personal instructions and abandon his erroneous
doctrines. It cannot be proved that the treatment accorded Luther
was severe and ill-considered. When the brief arrived, steps had already
been taken by Luther for the Leipsic disputation, which destroyed every
hope of arriving at an understanding. At Rome this
measure occasioned the termination of the trial which had already
been too long drawn out.
Luther’s pen was not exclusively devoted to attacks. With impetuous activity
he had in the meanwhile composed a series of tracts
which, beside those mentioned above, were dedicated in part to a
glorification of his cause, and in part written to pastoral requirements.
His popular religious writings were intended to invest him
with the indispensable reputation of a man who was solicitous solely
about the welfare of souls. This activity gained for him a large following
among religious-minded people. Among other things he published,
in that period of stress, a serviceable explanation of the Our Father, a
short instruction on confession, a condensed explanation of the
Decalogue, and an interpretation of Psalm CIX (Vulg. CX). Even
before this he had entered the field of popular literature with an
exposition of the seven penitential Psalms, a sermon on the Ten Commandments,
and some other smaller writings.

His history of the Augsburg trial (\textit{Acfa Augustana}), on the other
hand, as well as his edition of the “Replica” of Sylvester Prierias
against his “Responses,” were polemical. By publishing a reprint of
the “Replica” of the Master of the Sacred Palace, he intended to
represent Prierias as a man entirely devoid of importance and worthy
of disdain.\footnote{Grisar, \textit{Luther}, Vol. IV, p. 375.}

\section{The Discovery in the Tower}

An essential element was still missing in the new theology, as it
appears in Luther’s exposition of St. Paul’s Epistle to the Romans, in
his early disputations, and in the writings which he had thus far published.

He was unable to discover an adequate answer to the distressing
question: How can we be personally certain that Christ’s merits are
imputed to us, and that we are in a state of grace? The Church told
him that whoever sought justification by true penance, should be
certain of it. Faithful souls in a normal state were not tortured by
doubts on this point; but they did not conceive this certitude as
really a certitude of faith in the strict and proper sense, as an
object of revelation, which would have been erroneous. However,
the teaching of the Church and her practice did not satisfy the restless
soul of Luther; nor was he content with the results of his own
study. His demand of perfect surrender (\textit{humilitas}) to almighty
God, coupled with resignation to whatever He might decree, appeared insufficient
even to himself to engender that perfect certainty
of the state of grace for which he longed. The reason was that his
God was the arbitrary God of Ockhamism.

Hence, he conceded, in painful language, the endurance, yea the
increase of his fear of a wrathful and avenging Deity. The word
justitia, he said, had ever persecuted him and often entered into his
soul like a flash of lightning.\footnote{\textit{Op. cit.}, Vol. I, pp. 374 sqq.}
Fear agitated the morbid substratum
of his soul. He assures us that he felt most distressed at the time he
was about to deliver his second series of lectures on the Psalms. It
was synchronous with his appeal to a general council, in the winter
semester of 1518 to 1519. Beginning with 1516, one is able to see
how, step by step, he gradually advances toward the conclusions
which he had laid down in his second exposition of the Psalms,
namely, the dogmatic certitude of personal justification. In the
works which he published in 1518 he confidently announces this
result. The publications which embody this conclusion are the second
commentary on the Psalms (Operationes in Psalmos), the shorter
explanation of the Epistle to the Galatians, and the sermons on “twofold
and threefold righteousness.”\footnote{\textit{Op. cit.}, Vol. I, pp. 386 sq.}
In what manner did he arrive
at this conclusion? The answer is supplied by the so-called discovery
in the tower.

On various occasions during his later life, Luther spoke freely
of this capital discovery. Thus in the preface to his Latin works,
edition of 1545,\footnote
{\textit{Opp. Lat. Var.}, Erlangen ed., I, pp. 15 sqq. Cf. Grisar, \textit{Luther}, Vol. I, pp. 388 sqq.}
he describes how his discovery in the tower of the
monastery was connected with a passage in the Epistle to the Romans
(I, 17): “For the justice of God is revealed therein [\textit{i.e.}, in the
Gospel], leading men from faith unto faith [\textit{i.e.}, unto the believing],
according as it is written: The just shall live by faith.” (CE.Hab, 11, 4).

“Until now,” Luther says in the preface, “the words, ‘the justice
of God is revealed in the Gospel,” were an obstacle to me. For I hated
the words, ‘justice of God,” which I had been taught, in conformity
with the usage and custom of all doctors [!], to comprehend philosophically,
namely, of the so-called formal or active justice, by which
God is just and punishes the sinners and the unjust. Although I was
a monk without reproach, I felt myself to be a sinner in the sight of
God, suffered the greatest spiritual unrest, and could not consolingly
imagine God as reconciled by my atonement. Consequently, I did not
love, but rather hated the just God who punished sinners.” The ancient
law of the Decalogue already threatened sinners with dire punishment, and now, as he understood that passage, God intended to
proclaim His anger and avenging justice through the Gospel. “Thus
I raved (\textit{furebam}), and my mind conjured up terrors and confusion.
Importunately I sounded the text and thirsted to know its purport.”
While in this frame of mind, the significance of the words, “the just
man liveth by faith,” suddenly became clear to him. He saw that the
“justice of God” was identical with the justice which the just and
holy God bestows by means of faith upon those who are to be justified and
did not denote avenging justice, as everybody else imagined.
He had discovered, not active but passive justice, as he phrases it.
“I felt completely reborn and believed I was entering paradise
through open portals \dots Henceforth I praised the word justice
with as much love as I had formerly pursued it with hatred.” He
concludes that he was confirmed in his interpretation by reading the
works of St. Augustine.

Relative to Luther’s assertions concerning the sequence of these
events, two things should be noted. The assertions were made long
after the event. The most detailed, which we have just cited, was
made twenty-seven years later, after an agitated life spent in controversies.
It is natural, therefore, that the revelation he claimed he
had received is no longer as prominent as in other passages of his
writings.\footnote{\textit{Op. cit.}, Vol. VI, pp. 504 sqq.}
The definitely expressed content was the alleged efficacy of
faith alone, namely, the absolute certitude of personal justification to
be obtained from “sola fides,” \textit{i.e.}, the confidence engendered by faith.
As a result of his later experiences and owing to the progress of his
doctrines, this idea appears somewhat obscured in Luther’s subsequent
account. The time and place were more clearly fixed in his memory.

It is not true that “all the doctors” up to his time understood Rom.
I, 17, of the avenging justice of God, and that Luther was the first
to perceive the correct meaning of the phrase, namely, the concept
of justice by which God makes men just. This assertion is reiterated
in Luther’s commentary on Genesis and was popularized by Melanchthon in
his short biography of the Reformer.\footnote
{\textit{Commentar. in Genesim}, see chap. 27, \textit{Opp. Exeg.}, VII, p. 74. Melanchthon, \textit{Vita
Lutheri (Corp. Ref.)}, VI, p. 159.}
The very contrary
is true. Denifle has reviewed all the ancient commentators in a
careful monograph\footnote
{\textit{Quellenbelege Luther und Luthertum: Die abendländischen Schriftausleger bis Luther
über lustitia Dei (Rom. I, 17) und Jusiticatio,} pp. XX and 380 with quotations with 65
exegetes (Mayence, 1905).}
 and shows “that not one Christian commentator
from the days of Ambrosiaster up to the time of Luther, interpreted
the Pauline passage in the sense of an avenging justice or an angry
God, but that all understood it as referring to the justifying God,
His justifying grace, and the former exegetes had spoken of justification
of faith.”\footnote
{Thus Denifle summed up the results of his investigation in \textit{Luther und Luthertum},
2nd ed., pp. 387 sq.}

Hence, Luther had not made a new discovery,
but taught the acquisition of justice in a far different manner. Denifle
also showed how tradition contradicts Luther and corroborates the
ancient teaching of the Church that justification is obtained only
through faith animated by charity (\textit{fides caritate formata}) and not
through the Lutheran formula \textit{sola fide}. Luther’s assertion about the
teaching of the ancient commentators can only have originated in the
fact that he had not read, or else had not understood some of them we
know he had read. His subsequent utterance is a sign of the self-delusion
into which he gradually fell under the influence of self-interest.
It was while he was somberly meditating on Rom. I, 17, that, at
the end of 1518, his mind was enlightened in a tower at the southeastern
corner of the monastery, next to the garden. In the second
story of this tower there was a so-called hypocaust, \textit{i.e.}, a furnaceroom,
and beneath it the toilet (\textit{cloaca}) of the monks. The hypocaust
served Luther as a study.\footnote{E. Kroker in the \textit{Archiv für Reformationsgeschichte}, 1920, pp. 300 sqq;}
He mentions the tower and the
\textit{cloaca} in 1532, in a passage of his Table Talks, where he speaks of
the place of his illumination. The conversation was recorded by
his pupil and friend, John Schlaginhaufen, who wrote down the short
conversation at table for his private collection of Table Talks.\footnote
{\textit{Tischhreden}, Weimar ed., II, p. 177, no. 1681, The word \textit{cloaca} is represented
by the letters \textit{cl.}}
Schlaginhaufen, since 1531, resided at Luther’s house, the former
Augustinian monastery, as an expectant for a position as pastor. In
the interval between July and September of the following year, Luther spoke
in his presence of the terrors he had suffered at the thought
of divine justice. While in the tower, he said, he had pondered the
words: The just man lives by faith. His spirit rose and the conclusion
flashed upon him: Therefore, it is God’s justice which justifies and
saves us. “Those words became more gratifying to me. On this \textit{cloaca}
the Holy Ghost inspired me with this apt interpretation.”

The two references of Schlaginhaufen to the Holy Ghost and the
tower are repeated in the same connection by other contemporaneous
collectors of Table Talks, who were not present at the conversation,
but had Schlaginhaufen’s manuscript before them. Thus, Conrad
Cordatus reports Luther’s words as follows: “The Holy Ghost inspired
me with this solution in this tower.”\footnote{\textit{Ibid.}, III, p. 228, no. 3232a.}
He, too, lived in the same house
with Luther, was familiar with the place, and adds in the introduction
to Luther’s words that the “privy” of the monastery was there. George
Rörer, also a pupil of Luther and a most reliable collector of his Table
Talks, quotes Luther as saying: “The spirit of God has inspired me
with this interpretation on the \textit{cloaca}.”\footnote
{\textit{Ibid.}, II, p. 177, n. I. Here the word \textit{cloaca} is written out in full.}
Anton Lauterbach reports
that Luther concluded his description of this event thus: “The Holy
Spirit revealed the Scriptures to me in this tower.”\footnote
{\textit{Ibid.}, I, p. 228, no. 3232c (Bindseil, \textit{Colloquia}). Likewise Kaspar Khumer (\textit{ib.} no.
3232b): \textit{“Diese Kunst hat mir der Heilige Geist auf dieser cloaca auf dem Torm gegeben}.”}
The repeated
use of the pronoun “this” permits the inference that it was thought
that Luther indicated the tower with his finger. The hypocaust is
mentioned only by Lauterbach at the beginning of the Table Talk,
thus: “Once when I was reflecting in this tower and hypocaust.” The
question may be raised why he inserted the word \textit{hypocaust} in
Schlaginhaufen’s story.

It is of little moment, whether the enlightenment came to Luther
in the \textit{cloaca} itself, as seems to have happened, or in the hypocaust,
which was his study.\footnote{Kroker assumes the hypocaust to be the place. (\textit{Jahrbuch} etc.; see note 34.)}
In fact, it is of even less moment than might
appear from the elaborate discussions of Protestant authors who
favor the elimination of the word \textit{cloaca} from the narrative. The
matter was quite indifferent to Luther and his aforementioned pupils; only
the timid Schlaginhaufen seems to have taken offense at it,
since he does not write out the word in full, but only insinuates it with
the letters \textit{cl.} That he understood that the \textit{cloaca} was meant when
Luther pointed out the place, is not subject to doubt, according to the
Protestant author of the new critical edition of the Table Talks,
though some still place a different and deviating interpretation on
the letters \textit{cl.}\footnote{Kroker in the \textit{Jahrbuch der Luthergesellschaft}, I (1919), pp. 112 sqq.}, assumes that
Schlaginhaufen had misunderstood Luther. “This possibility is not to to be entirely excluded.
Nevertheless Schlaginhaufen was quite certain, since he preferred conceal : what he had
heard, yet expressed it with \textit{cl.}
Luther, as Kawerau emphasizes, was of the opinion
that the Spirit of God has a free hand everywhere, even on the
\textit{cloaca}.\footnote{G. Kawerau, \textit{Luther in katholischer Beleuchtung}, Leipsic, 1911, p. 60.}

Such was Luther’s experience in the tower, of which he later says
that for a long time he knew not what he was about, when from the
verse, “The just man lives by faith,” a light burst upon him which terminated one
period of his life. “Thereupon,” he says, “I went through.”\footnote
{\textit{Tischreden}, Weimar ed., V, no. 5518,}

So much is certain: Luther’s experience in the tower may claim to
be one of the most important and far-reaching events of his life. In
its essential features it does not permit of contradiction. The feeling
of joy which Luther tells us he experienced immediately after, is
quite comprehensible and does not provoke the least historical objection.
Psychologically it is not only possible, but characteristic of the
spirit which moved Luther. Of course, the Catholic ascetic will view
the sudden emotion of joy in quite a different light than Luther’s
admirers.

It is evident to any impartial observer that the new theological
doctrine of the certitude of salvation or, let us rather say, the certainty
of justification, was a deduction completely adapted to Luther’s
state of mind, as it soothed him in his sad personal struggle. He
erected it into an article of faith, to be believed by all. That one must
firmly believe that one is in the state of grace became a dogma of the
Lutheran faith.

In a similar manner Luther erected the personal experiences of his
own way of suffering into a general norm for all. Even at that time
he taught--and always adhered to this doctrine--that God leads those
whom He wishes to justify, through darkness and fears; that the
road of despondency \textit{per se} leads to salvation. On one occasion he
wrote that no man has a right to converse about divine things unless
he has experienced those things, and among those who have not he
classes the papists and the visionaries who deviated from his doctrine.

But Luther was not able to maintain himself in the certainty to attain
which cost him so much labor. In the sequel he often admitted,
sorrowfully, that this was not possible for him except at the cost
severe trials and ever new struggles.\footnote
{Grisar, \textit{Luther}, Vol. V, Ch, XXXII, especially no. 6.}
He instructed all that life is
nothing but a laborious contest for this ineffable good and that
assurance of grace depends on vigorous endeavors and daring defiance,
which, however, are not everyman’s business.
