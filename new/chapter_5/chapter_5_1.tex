\section{The Controversy on Indulgences}

In consequence of recent researches on the development of Luther,
far less significance is to be attached to the celebrated controversy
on indulgences which followed the theses of 1517, than tradition
has ascribed to it. The ninety-five theses nailed to the door of the
Wittenberg castle-church do not mark the commencement of the
Protestant Reformation. As we have heard Luther himself intimate,
the \textit{initium evangelii} is to be sought in the new theology of Wittenberg
and in the public movement which it created. The controversy
concerning indulgences simply caused the movement to assume universal proportions.
It placed the monk Luther upon the stage of
the world and offered him an opportunity of gradually unveiling
his revolutionary doctrine before all his contemporaries.

There was no room for indulgences in a system of grace and
justification which attacked the meritoriousness of good works and
the value of atonement.

Even before these ideas had fully matured (July 27, 1516), Luther
delivered a sermon in which he expressed himself correctly on the
Catholic doctrine of indulgences. Succeeding ages could have been
benefited by his instructions. He correctly emphasized that an indulgence
is not a remission of the guilt of sin, but “a remission of
the temporal punishment due to sin, which the penitent would have
to suffer, be it that it was imposed by the priest, be it that he had
to suffer for it in purgatory.” “In the gaining of a [plenary] indulgence,
therefore,” he says in conformity with the belief of the
age in which he lived, “one may not then and there feel sure of
salvation.” “Only those gain a plenary remission of punishment who
have become reconciled to God by true contrition and confession.”
At that time Luther still knew and appreciated the value of indulgences
for the dead. The application of these, he explains in the same
sermon, is made by way of intercession; hence a complete redemption of
souls in each instance is not to be assumed. The foundation
of indulgences he correctly states in these words: “They are the
merits of Christ and His Saints [\textit{i.e.}, they derive their efficacy from
this treasury of merits], and we must, therefore, esteem them with
all due reverence.” Whatever abuses may have crept in, he holds
that it is “most useful that indulgences should be offered and
gained.”\footnote
{Grisar, \textit{Luther}, Vol. I, pp. 324 sq. On the origin and the early development of indulgences
cf. the excellent work of N. Paulus, \textit{Geschichte des Ablasses im Mittelalter}, 3 vols.,
Paderborn, 1922--1923.}

The abuses indicated by Luther had reached a certain crisis in his
day. Since good works are requisite for the gaining of an indulgence,
and since it was customary at that time to require a small donation
to be made to some pious or useful purpose, to procure delivery of
the briefs of indulgence, indulgences were frequently made the
means of collecting money. Exaggerated recommendations and avaricious practices
combined to degrade them. The so-called \textit{quaestores},
who wandered about plying this trade, were the chief culprits.\footnote
{N. Paulus, \textit{op. cit.}, Vol. II, pp. 265 sqq., Vol. III, pp. 450 sqq., 471 sqq.}
But many ecclesiastical superiors were also guilty of having increased
the evil in the temple of the Lord by distributing indulgences with
all too temporal trimmings and worldly bustle.
The extent to which even the papal curia went, may be seen
in the case of the indulgences granted by Leo X, the proceeds of
which were intended for the construction of St. Peter’s basilica
at Rome. This indulgence provided Luther, who had already drifted
away from the Church, with an occasion for entering the lists against
indulgences as such, and not merely the abuse of them.

Bishop Albrecht of Brandenburg, who governed the dioceses of
Magdeburg and Halberstadt, a thoroughly worldly-minded ecclesiastic, had
succeeded in having himself elected archbishop of Mayence.
In order to unite these three bishoprics in one hand, he had to contribute
no less than 10,000 ducats to the Roman curia. In addition
to this, he was obliged to pay 14,000 ducats for the confirmation
of his appointment as archbishop of Mayence and for the pallium.
It was agreed that he might preach the indulgence for the construction
of St. Peter’s basilica throughout his extensive jurisdiction
in Germany, retaining one-half of the proceeds to reimburse
himself for the 10,000 ducats, which he had borrowed from the
Augsburg firm of the Fuggers, whilst the other half was to be devoted
to the erection of St. Peter’s at Rome. Albrecht kept a sharp
eye on the filling of the big indulgence chest which accompanied
the preachers and was placed under the supervision of the Fuggers.
It was a rather disedifying transaction. Even if it did not involve
simony, strictly speaking, it was nevertheless reprehensible, and can
be explained only as a result of the evil financial practices of the
time, which had taken root also in Rome, and of the activities of the
agents of Albrecht and an avaricious party of Florentine churchmen
at the curia.\footnote{Grisar, \textit{Luther}, Vol. 1, pp. 347 sqq.}

Only gradually did Luther become aware of these agreements. The
first motive of his intervention was supplied by his exasperation
at the new indulgence enterprise and at the existing abuses in general.
He personally witnessed an example of the general decline of the
system of indulgences. In the castle-church the Elector of Saxony,
Frederick “the Wise,” kept a casket of relics, partly genuine
and partly spurious, for which he succeeded in obtaining incredibly
rich indulgences from Rome. Like the Elector Albrecht of Brandenburg, Frederick
was a passionate collector of relics. Both were eager
to have each relic enriched with great indulgences, so as to attract
pious votaries and realize handsome profits at the annual exhibition.
Up to the year 1518, Frederick succeeded in obtaining for his
sacred casket in the castle-church of Wittenberg indulgences which
amounted, all told, to 127,799 years.

Princely interests played a nefarious role in connection with the
indulgence traffic of the Roman curia. Occasionally the rulers prohibited
the too frequent indulgence-preaching within their territories, because
they wished to prevent the flow to Rome of money
which they needed for their own countries, or its expenditure for
other purposes not agreeable to them. Thus the Elector Frederick
prohibited the promulgation of the Mayence indulgence in behalf
of St. Peter’s within the confines of his principality.

Elector Albrecht selected John Tetzel, a popular orator of the
Dominican Order, to preach the indulgence at Mayence. Tetzel was
not much of a theologian. His morals were beyond reproach, despite
the slanders to which he was subjected in the ensuing controversy. In his
sermons, which were attended by large numbers of
people, he adhered to the explicit directions of the ecclesiastical
authorities of Mayence, although he was unable to abstain from
rhetorical exaggerations. The directions of the Mayence authorities adequately
emphasized that an indulgence is a remission of punishment,
not of sin, and expressly required a contrite confession as a
condition. In one respect, however, the directions were defective. They
declared that an indulgence applicable to the dead became efficacious
upon the performance of the prescribed good work, regardless of
whether one was in the state of sanctifying grace or not. Some theologians
held this opinion and it was embodied in several other episcopal
instructions. The problems arising from the Church’s teaching on indulgences
had not yet all been clearly solved. The very nature of
indulgences had not yet been dogmatically defined. It was a matter of
practice, taught by the theologians; but its genuineness was warranted
by the ordinary teaching authority of the Church (\textit{magisterium ecclesiae
ordinarium}) .

Tetzel eagerly availed himself of the above-described, now abandoned,
opinion concerning indulgences for the departed. It cannot
be proved that he used the famous saw which has been attributed
to him: “As soon as money in the casket rings--The soul its flight
from Purgatory wings,” but in substance his words approximated
the proverb. Some critics looked with disfavor on Tetzel because
he often, \textit{e.g.}, at Annaberg, availed himself of the occasion of fairs
with their secular amusements to proclaim the papal indulgence.\footnote
{Cf. N. Paulus, \textit{Johann Tetzel, der Ablassprediger}, Mayence, 1899; Grisar, \textit{Luther}, Vol.
I, pp. 341 sq.; Vol. IV, pp. 84 sq.}

There is an unwarranted report to the effect that when Staupitz
had apprised him at Grimma of Tetzel’s conduct, Luther exclaimed:
“I shall put an abrupt stop to this, please God.” When Tetzel, in
the course of his preaching tour, had arrived at the confines of the
electorate and in the vicinity of Wittenberg, Luther decided that
the time for intervention had come. On November 1, the castlechurch
at Wittenberg celebrated its titular feast. The church was
dedicated to All Saints and was specially indulgenced for that day.
Many worshipers were sure to attend. On the eve of All Saints, Luther
caused a Latin placard containing ninety-five theses on the subject of
indulgences to be nailed to the door of the church, which was, at
same time, the university chapel.

The placard contained an invitation to a disputation. At first the
Latin placard did not attract much attention except among scholars.
But when Luther sent copies of it to the theologians of the neighboring
universities of Leipsic, Frankfort on the Oder, and Erfurt,
his theses began to attract attention. That they “spread throughout
Germany in fourteen days” is “an erroneous representation, based
on a later expression of Luther.”\footnote
{Paul Kalkoff, \textit{Luther und die Entscheidungsjahre der Reformation}, 1917, p. 22.}
It is in keeping with the fables
which have accumulated around the history of the theses. It is true
that many, including well-intentioned but shortsighted Catholics,
rejoiced that a courageous protest had been raised against the prevalent
abuses in connection with the preaching of indulgences. Under
the pressure of these abuses, the true meaning and import of the
theses were easily overlooked.

Luther’s placard was a challenge to a disputation designed to clarify a
set of theses which constituted a fundamental, though guarded,
attack on the Catholic doctrine of indulgences. The author had no
intention of abandoning them in a learned discussion. His theological
position would not permit of this. But he did not allow his novel
dogmatic teaching, which stood behind the 95 theses, to appear on
the surface. He maintained in his theses that indulgences were invalid
before God, but were to be regarded only as a remission of the
canonical penances imposed by the Church. He denied the doctrine
of the treasury of merits earned by Christ and the saints, which
constitutes the presupposition of indulgences. In addition to other
erroneous views he expresses false notions about the condition of the
departed. In defense of his attitude he seeks to place the absurdities
of the indulgence preachers in the forefront as the reason and the subject of
his theses. He goes so far as to say: “Let him who contradicts
the truth of the papal indulgences be anathema and accursed”; and:
“Bishops and priests are obligated to receive the commissioners of
the papal indulgences with all due reverence.” One sneering thesis
asks: “Why does not the pope build the basilica of St. Peter with his
own money, rather than with that of the poor, seeing that he is
wealthier today than the richest Croesus?” Towards the close he
clothes his own sharp objections is the artificial garb of a suggestion
to the effect that the objections of the laity against the pope and the
eleemosynary system ought to be clearly and thoroughly refuted,
adding that “if sermons were preached after the mind and intention
of the pope, these difficulties would be solved.” In this manner the
author of the theses thought he could, in a measure, safeguard his
position.

Not to omit Mysticism, the last theses enjoin the obligation of
striving not for the peace which indulgences seem to bring, but for
the cross. Not pax, pax, should be the watchword, but \textit{crux, crux}.
“Christians must follow their Leader through suffering, death, and
the pains of hell” (\textit{per poenas, mortes, infernosque}). This corresponds
with the idea, likewise expressed in the theses, that it is better voluntarily
to suffer the penalties of sin than to escape them by means of indulgences.
He also proclaimed (which was a general truth valid in
all ages), that a Christian’s entire life, according to the will of Christ,
should be one continuous atonement.

The celebrated 95 theses are not a candid or an honest document.
Neither are they a scientifically constructed or properly coordinated
whole. Least of all, are they the programme of a reformation, as they
are often represented to be.
The movement gradually assumed great dimensions. On the sixteenth
of January, 1518, the eve of the feast of the dedication of the
castle-church, Luther delivered in that church a sermon on indulgences
which was in conformity with his theses.\footnote
{N. Paulus in the \textit{Zeitschrift für kath. Theologie}, 1924, pp. 630 sqq.}

In a letter to Staupitz he laments in exaggerated language that
“godless, false, and heretical doctrines” were propounded with such
confidence in sermons on indulgences, that objectors were forthwith
declared worthy of the stake. He, on the contrary, had modestly
advanced his deviating opinions, which were “founded on the conviction
of all the doctors and the entire Church, that it is better to
make atonement than to seek for satisfaction by means of indulgences.”
Thereby he had invited the frightful wrath of the fanatical
representatives of papal authority.\footnote{\textit{Briefwechsel}, I, p. 198.}

We must not overlook the fact that some months before the publication of
Luther’s theses, Karlstadt had published 152 theses in conformity
with the new doctrine. It seems Luther did not wish to be
outstripped by his audacious friend. The controversy concerning indulgences,
moreover, afforded him an opportunity of assuming the
leadership of the Wittenberg movement in a popular field.

Soon after the posting of his theses, Luther wrote to the Archbishop
of Mayence and to Jerome Schultz (Scultetus), bishop of Brandenburg,
to whose jurisdiction Wittenberg was subject, in order to give an
account of the events as he saw them. Archbishop Albrecht was also
informed by the Dominicans, Tetzel and the brethren of his Order,
and, for the sake of his own indulgence, immediately brought the
matter before the supreme tribunal of the Church at Rome, by submitting
a copy of the 95 theses and those of the disputation of September
4, 1517. Thereupon, on February 3, 1518, an Augustinian,
Gabriel della Volta, was commissioned by Leo X, as representative
of the General of the Augustinians, to charge Luther’s
superiors with the task of severely dissuading him from his perverted
opinions, “lest a greater conflagration ensue as a result of negligence.”
It cannot be proved that the Pope originally styled the controversy an
empty “quarrel between monks.” Strict orders were issued
by Della Volta to Staupitz, who, however, was not inclined to
adopt thorough-going measures--an attitude which can easily be explained
in view of his previous relations with Luther. Luther confidently
wrote to him on March 31, 1518: “When God acts, no one can prevent Him;
when He rests, no one is able to awaken Him.”\footnote
{\textit{Briefwechsel}, I, p. 176.}
Della Volta meanwhile summoned him to appear before the imminent
chapter of his Order at Heidelberg, to give an account of himself.
There a district vicar was to be selected to succeed him, since his
three years’ term of office had expired.

On the other side Tetzel and the Dominicans were not satisfied
with a defense of their preaching. At Frankfort on the Oder, Tetzel
published a series of theses on the doctrine of indulgences, which
were couched in a moderate form and, generally speaking, correctly
reflected the position of the Church. They were composed by Conrad
Wimpina, a professor of that city, who afterwards became a literary
opponent of Luther. Maintaining his position, Luther replied in a
pointed sermon on indulgences and grace. Tetzel defended himself,
again in a moderate form, in a printed “representation,” in which
he stressed Luther’s violation of the papal authority. He published a
second series of theses, which, in turn, were followed by Luther’s
booklet entitled: “Freedom of a Sermon on Indulgences.”\footnote
{On the correspondence between Luther and Tetzel see Grisar, \textit{Luther}, Vol. IV, pp
372 sqq., where the calm and heavy publications of Tetzel are compared with Luther’s
first impetuous polemical broadsides.}
It was written in a more provocative tone than Luther had thus far used.

The first outsider to enter the lists was Dr. John Eck, who was
destined to achieve celebrity in his subsequent controversy with
Luther. He was a professor of the university of Ingolstadt, a quickwitted
humanist and theologian. He circulated “Obelisci,” \textit{i.e.}, annotations
to Luther’s theses in manuscript. Luther replied with
“Asterisci,” which were also originally circulated in manuscript
form.

As the time for the chapter at Heidelberg approached, April, 1518,
Luther undertook to safeguard his position. In the event of his refusal
to recant, he had to fear that he would be delivered up to the
ecclesiastical authorities--for such was the procedure of medieval
jurisprudence--and in the event of obstinacy would be confronted
with the severest ecclesiastical penalties, He procured from the
Elector Frederick of Saxony an order for his unmolested return to
Wittenberg. It was the first demonstration in behalf of Luther on the
part of that ruler, whose friendship was destined to increase with the
coming years.

The members of the chapter, or at least a majority of them, were
favorably inclined towards Luther and the result of their deliberations
was a verdict in favor of the defendant. It was a result entirely
contrary to the expectations of the Roman authorities of his
Order.\footnote{Grisar, \textit{Luther}, Vol. I, p. 334.}
He was even granted the privilege of arranging a great
disputation in the auditorium of the Augustinian monastery, which
was conducted by Leonard Beier, a Wittenberg master. University
professors and many guests attended. Beier and Luther argued against
free will and the ancient theology. One of the Wittenberg professors who
were in attendance interrupted the disputants when certain
strong declarations were made, exclaiming: “If the peasants could
hear this, they would stone you!” The Heidelberg chapter, so far as
can be inferred, did not treat the problem of indulgences. Luther,
now regarded as a courageous ornament of his Order, remained
unmolested. Among the students of theology at the university, several were
more or less won over by him. Some of them later on became
his helpers, such as John Brenz and Erhard Schnepf, and particularly
Martin Butzer (Bucer), a talented young Dominican endowed with a
very lively temperament.

On his homeward journey, Luther, who was delighted with the
issue, delivered a sermon in Dresden in the presence of Duke George
of Saxony and his court. He discoursed on the grace of Christ, eternal
salvation, and the conquest of fear before an angry God. The duke,
who was loyal to the Church, took great offense at these remarks.
Several others also were indignant. When Luther heard of their objections,
he disposed of them in these self-conscious words: These
babblers desire everything and can do nothing; they are “a serpent’s
brood,” “masked faces” whom I will ignore.\footnote{\textit{Ibid.}, p. 335.}

His arrogance increased because of his having escaped punishment,
because of the approval he met with, and because of the expected
protection of the Elector Frederick. At the same time his writings
and letters of those days reveal how he ever and anon calls up before
his mind the abuses actually existing within the Church, especially
the lucrative practices of the bishops and the Roman curia, in order
to encourage himself and excite his anger. Unfortunately, the abuses
furnished him with what he wanted. Oldecop, who was his pupil at
that time, thus describes Luther’s attacks which he continued at home,
on the indulgence traffic and the doctrine of indulgences itself:
“In his teaching against them, he exceeded all bounds, indulging in
every kind of rage and blasphemy.” He describes him on this occasion
as “naturally proud and presumptuous.” In a statement on indulgences and
grace, composed at the behest of members of the Heidelberg chapter, Luther
assured them that in his theses on indulgences
he had spoken only by way of disputation, to ascertain the truth.
In this way he constantly concealed his real opinion. However, in
the “Resolutions” which he published in connection with his theses
he expressed his attitude unequivocally. These Resolutions or “explanations”
were intended to elucidate, defend, and confirm the entire
series of theses. No dogmatic definition on indulgences having been
issued, he pretended that there was no binding doctrine on the subject
proposed by the \textit{magisterium ordinarium} of the Church. He
now proclaimed to the world his new doctrine on grace in a more
definite outline.\footnote{\textit{Ibid.}, pp. 335, 378 sq.}
He conceived the bold idea of dedicating his
“Resolutions” to Pope Leo, and of forwarding them to Rome through
Staupitz. He prefaced the work with an humble dedicatory epistle
addressed to the head of the Church. Couched in superlative phraseology,
it was designed to be an apologia of his conduct and an attack
on his opponents. True, he tersely says: “I cannot recant,” but
towards the end of the epistle he bursts forth with the assurance:
“Most Holy Father! I prostrate myself before thy feet, and offer
myself to thee with all that I am and possess. Do as thou wilt; give
life or death, call or recall, approve or disapprove; I will acknowledge
thy voice as the voice of Christ who reigneth and speaketh in thee.
If I have merited death, I shall not refuse to die.”\footnote
{Ibid., p. 335; \textit{Briefwechsel}, I, pp, 200 sq.}
How is this language to be explained? It constitutes one of the many riddles of
his psychology. It need not be taken as hypocrisy, but is, rather, a
reflection of the restless and profound struggles which buffeted him
about between loyalty to the Church and the new position which
he had assumed. It is possible that he wished to dispose the pope
favorably and he may also have intended to allay the alarm of his
many Catholic readers both at home and abroad. Nevertheless, a considerable
lack of spiritual equilibrium is plainly noticeable. When
his imagination is deeply roused, the ideas which agitate him at the
moment often assume most exaggerated forms, but later are in turn
easily displaced by contrary and equally vivid ideas. Concerning the
pangs of conscience which afflicted him at the beginning of his revolt
, he expressed himself thus on one occasion: “I was not happy
or confident concerning that undertaking.” -- “What my heart suffered
in the first and second year, and how I lay prostrate on the
ground, nay, almost despaired, they [my opponents] did not know,
who themselves afterwards attacked the pope with equal audacity.”
They were, he said, “ignorant of the cross and of Satan,” whereas
he “was compelled to go through terrible death-struggles and
temptations.”

In a remarkable passage of the “Resolutions” he describes these
phenomena in detail, though he is not aware that his qualms of conscience
are closely related to the neurotic precordial fear which he
frequently suffered.

Apropos of indulgences for the departed, he wishes to picture the spiritual
agonies of the souls in Purgatory, which were understood very well by such
as at one time or another had suffered similar pains, but of which the
indulgence-preachers had no conception. Then he proceeds in fantastic language:
That eminent doctor [Tauler] with whom the Scholastic theologians
are not familiar, speaks of such “dark nights of the soul”; and he himself is
acquainted with one so afflicted (\textit{i.e.}, himself; 2 Cor. 12:2).
% (2 Cor. XII, 2) -> 2 Cor. 12:2
The agonies are very brief, but so intense and infernal that no tongue can
express, no pen can describe, no uninitiate can believe them.
Were they to last but the tenth part of an hour, all of a man’s bones would
be reduced to ashes. “God, and simultaneously with Him, all creation,
appears horribly angry. There is no escape, no comfort, whether within or
without, only a hollow accusing voice.” The sufferer regards himself as a
reprobate, and does not even dare to say with the Psalmist: O Lord, rebuke
not me in Thy indignation. He believes that he is saved, but suffers eternal
punishment, and feels himself stretched on the cross with Christ, so that
all his bones are numbered. There is not a nook of the soul that is not filled
with bitter anguish, with terror, dread and sadness, accompanied by the
stifling sense that it is to last forever. In order to make a weak comparison:
when a bullet traces a line, every point in that line sustains the whole bullet,
but it does not compass the whole bullet. Thus the soul feels when that
deluge of eternity flows over it and drinks naught else but eternal pain;
but this pain does not abide; it passes away. It is an infernal torture, an
intolerable terror which excludes all consolation! Those who have experienced it must be believed.\footnote
{Grisar, \textit{Luther}, Vol. I, pp. 381 sq.; VI, 102; \textit{Werke}, Weimar ed., I, p. 5575 \textit{Opp. Lat.
Var.}, II, p. 180. }

This is the language of a sick man. Here Luther actually depicts
those phobias of traumatic neurosis which nervous persons experience
as a result of a terrible shock. We must regard them as after-effects
of the thunderbolt of Stotternheim. In his own opinion they were
that darkness of soul so familiar to mystics. In his case physical
fear was intimately associated with tortures of conscience, his internal
doubts and that abiding sense of fear, in which he imagined
God to be “horribly angry.” In their most aggravated form, they
were movements of precordial fear. Such psychopathic conditions
were not adequately known to the medical science of his day. It was
no pressure of circular “psychosis” which affected Luther in his
monastic years, as a popular Protestant biographer would have us
believe, who holds that monastic practices as such, when strictly and
conscientiously performed, ordinarily induce a certain degree of insanity.\footnote
{Adolph Hausrath; cfr. Grisar, \textit{Luther}, Vol. 1, p. 383.}

Insanity cannot be ascribed in any sense to Luther while
he was a monk. If one correctly understands his manifold testimonies,
he simply experienced the effects of extreme nervousness from early
youth to old age.

The “Resolutions” were followed by a tract on the “Force of Excommunication.”
It was inspired by anxiety about the condemnatory
verdict of the pope. In order to allay his own fears as well as the
fears of others, he wished to show that an unjust excommunication
does not separate one from the soul of the Church. To justify his
conduct he describes in lurid colors the abuses which attended the
all too frequent use of the power of excommunication by the bishops.
