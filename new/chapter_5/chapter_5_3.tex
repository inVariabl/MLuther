\section{The Leipsic Disputation of 1519 and the Papal Primacy}

After Luther’s trial at Augsburg, the controversy about indulgences
began to wane before other more important questions connected with
his revolt. Among these the primacy of the pope gradually began to
take precedence.

While the subject of indulgences was still burning, Cardinal Cajetan had
availed himself of his stay at Augsburg to compose a series
of scholarly treatises on the doctrine of indulgences.\footnote
{F. Lauchert, \textit{Die italienischen Gegner Luthers}, 1918, p. 141.}
He also drew
up a scheme for a doctrinal decree of Leo X on this question. This
highly important papal decree, which definitely determined the traditional
teaching of the Church, appeared on November 9, 1518.\footnote
{Kalkoff, \textit{Luther und die Entscheidungsjahre}, pp. 86--88.}
Cajetan, who was an excellent commentator of St. Thomas Aquinas,
also perceived the importance of the doctrine of the primacy in view
of the progress of the Lutheran controversy. He began the composition of
a scholarly work, On the Divine Institution of the Roman
Pontificate over the Universal Church, which appeared in 1521 and
was immediately reprinted at Cologne by Peter Quentell.\footnote{Lauchert, \textit{op. cit.}, pp. 142--155.}

John Eck, a theologian of Ingolstadt, shared Cajetan’s conviction
that the question of papal supremacy would become the real and decisive
battleground for the future. The Catholic cause is indebted to
his versatile and powerful pen for the comprehensive Latin work, \textit{Of
the Primacy of Peter, Three Books against Ludder}, which originally
appeared at Ingolstadt in 1520.\footnote{Wiedemann, \textit{Dr. Joh. Eck}, 1865, p. 517.}

In defending the spiritual supremacy of the successors of St. Peter
in Rome, both writers, the Italian and the German, appealed most
emphatically with the entire Catholic tradition to Christ’s words addressed
to the Apostle after the latter had solemnly professed his faith
in His divinity: “Thou art Peter, and upon this rock I will build my
church, and the gates of hell shall not prevail against it” (Matt. xvi,
18). They likewise quoted the immortal words of the Saviour, when,
after His resurrection, He confided His sheep to the pastoral care of
Peter on the shores of Lake Tiberias: ``Feed my lambs, feed my sheep''
(John xxi, 15 sqq.). By means of this concept of the Church, founded
by Christ as the visible spiritual kingdom of the faithful, they furthermore
demonstrated that, subject to its invisible Head, namely, the
Son of God, there must be a visible head to govern this spiritual society,
in order to bind its members in the necessary unity and to preserve
it from dissolution. They appealed to the history of the Church
from the time of its inception for illuminating proofs of the fact that
the successors of St. Peter in the see of Rome had always possessed the
supreme power in governing the Church, though a progressive
development in the use of that power must be acknowledged. They
emphasized that all centuries, up to their own, were replete with most
glorious achievements on the part of the papacy, though there had
been a few unworthy popes. Guided by God, the papacy had conferred
upon mankind the golden gifts of Christian civilization and culture,
resisted the arrogance of mighty rulers, protected the rights of the
lowly, and raised the heart of humanity to celestial things. Finally,
they maintained, it was a crime against the will of Christ, against the
foundation of the divine temple of the Church, against the duty of
gratitude towards the society of the faithful which had been
nurtured by the papacy, to attack the rights of the occupant
of the Holy
See.

John Eck, the most successful and most celebrated defend
papacy against Luther in Germany, was born in the town of Eck in
Swabia. His real name was John Mayr. In 1510 he became a professor
at the University of Ingolstadt and, at the same time, canon of the
cathedral of Eichstätt. Thoroughly conversant with humanism, this
learned theologian corresponded with a large number of the most
prominent men of the age. He was known for his cleverness in
scientific discussions, was well-read and endowed with a stupendous
memory. These gifts were supplemented by extraordinary physical
powers; he was a gigantic man who, when engaged in disputation,
was wont to dominate the situation with his stentorian voice.

In May, 1518, Karlstadt had published a series of theses against
Eck’s “Obelisci” at Wittenberg. Eck not only replied with a set of
other theses, but challenged him to a public disputation, to be conducted
at one of the great university cities, Rome, Paris, or Cologne.
Leipsic was finally selected, and Eck endeavored to obtain the consent
of Duke George, who was reputed to be a great patron of scholarly
pursuits. Prior to the meeting, however, Eck published twelve
theses, which were expressly, though somewhat covertly, directed
against the person of Luther and his doctrines. The final thesis dealt
with the Roman primacy. Luther, in the “Resolutions” which he had
appended to his theses on indulgences, had asserted (though without
denying the rights of the existing primacy) that he knew of no primacy
of Rome over the universal Church, at least not over the Oriental Church,
before the time of Gregory the Great, \textit{i.e.}, about the
year 600. Eck in his final thesis against Luther says: “We deny that
the Roman Church had no precedence over the other churches before
the age of Sylvester (died 335), and acknowledge him who occupies
the see of St. Peter as the successor of Peter and the universal vicegerent
of Christ.” Luther indignantly declared that he was the one
who had been challenged to the projected disputation. In this he was
not entirely wrong. Nevertheless he made up his mind to participate
in the Leipsic discussion. In the beginning of February, 1519, he
published twelve antitheses against Eck, and soon after, boldly added
a thirteenth against the authority of the pope. Although his most intimate
friends had hitherto cautioned him not to revolt against Rome,
“the thirteenth thesis appears entirely too audacious, nay, absolutely
untenable, even to more recent Protestant writers.”\footnote
{Köstlin-Kawerau, \textit{Martin Luther}, 5th ed., Berlin, 1903, Vol. I, p. 235.}
This thesis
declared that his opponents could base their proofs for the primacy only
on the “frosty decretals of the last four centuries.”

Luther expected to take part in the disputation and to surprise his
opponents with historical arguments. Accordingly, he delved into
history to discover proofs for the negation, which as far as he was
concerned, was irrefragable even without proofs. In his letter of that
period he repeatedly spoke of the hydra of the papacy, against which
it was his duty to launch an attack. His previous activities, he said,
now appeared to him as mere child's-play by comparison. ``The Lord
pulls me, and I follow Him not unwillingly.'' In this pseudo-mystical
frame of mind--led by the hand of God, as he imagined--he arrived
historically and Scripturally at the discovery that the pope was Anti-christ.\footnote
{Cf, Grisar, \textit{Luther}, Vol. III, pp. 142 sqq.}
He finds that the mysterious words in the second chapter of
Thessalonians, and those in the first Epistle of St. John (ii, 18) on the
advent of the Antichrist, are not applicable to a particular person, as
tradition would have it, but to the papacy as an institution, whose anti-Christian
nature, now that the end of the world was nigh, must be
exposed by him, the witness chosen of God.

This idea, which was to control his later life, soon struck deep roots
in him.

On December 11, 1518, he announced his “presentiment” to Wenceslaus
Link. In a letter of March 13, 1519, addressed to his friend and
helper at the Electoral court, Spalatin, he speaks more clearly: “In
connection with my disputation, I am reviewing the decretals of the
popes, and--I whisper it into your ear--am uncertain whether the
pope is himself Antichrist, or an apostle of Antichrist, so awfully is
Christ, \textit{i.e.}, the truth, crucified in the decretals.” He penned these
lines only two months after he had addressed his fawning letter of
submission to Pope Leo. Soon the mask drops from his face entirely.
Without awaiting the disputation, he publishes his conclusions in a
set of Latin “Resolutions” on the aforementioned thirteenth thesis, in
which he complains that no one wishes to acknowledge that Antichrist
“sitteth in the temple of God at Rome.” (2 Thess. ii, 4).

The day of the disputation was approaching, and as yet Luther had
not been invited. Duke George of Saxony was still opposed to his
taking part in it. Some of the bishops attempted to prevent the disputation,
because, no matter what its outcome might be, in their opinion
it would only serve to spread the innovations, and because the final
decision lay solely with the supreme ecclesiastical tribunal. Their
efforts, however, were futile.

Impelled by an intense desire to fight,
Luther accompanied his friend Karlstadt to Leipsic, where, at the
last moment, he was permitted to participate in the disputation.
He afterwards said that he entered the disputation under the aegis
of Karlstadt. The oratorical
contest began June 27, in the great aula of the Pleissenburg, in the
presence of the duke and his court, the professors of the university,
and many other scholars who had come from far and near.

Karlstadt had first to dispute with Eck. The disputation between
two men, mainly on free will, lasted up to July 2, inclusively.
Karlstadt showed himself inferior to Eck in versatility and knowledge.
He was small of stature and his voice was hoarse; he was often
timid. The audience became bored because his defective memory
compelled him to consult books to prove his assertions. The weather
was hot and quite a number of professors fell asleep.

The audience was aroused when, on July 4, Luther appeared at the
lecturer’s desk with a bouquet in his hand, which he, from time to
time, held to his nose, after the manner of one who pretends superiority.
His finger was adorned with a shining silver ring. For the rest he
wore his monastic habit. Mosellanus (Peter Schade of Bruttig), the
humanist, who was present, says that, seeing his medium-sized slender
frame, one was almost able to count Luther’s bones, a condition
resulting from worry, study, and labor. He also reports that Luther
spoke in a high, clear voice. Tradition has it that he distinguished
himself by an extraordinary adroitness in the use of Scriptural texts.
He did not measure up to the clarity and demonstrative force of
Eck, who, moreover, by his ready wit and acuteness in detecting
contradictions, defects and sudden transitions, showed up many a
weak point in Luther’s argument. This is proved by the report of
the proceedings drawn up by the notaries who were present.

The two disputants were supposed to discuss, in turn, the papacy,
indulgences, Purgatory, and other controversial topics. However, the
debate on the papacy consumed almost the entire time. Shrewdly
appraising the situation, Eck, on July 5, cited the ecumenical Council
of Constance, which had condemned Hus as 2 heretic for denying
the primacy. He did this in order to compel Luther to make a definite
profession of faith, Luther at first replied that he was certain that
among the condemned propositions of Hus there were many which
were quite Christian and evangelical, and which the universal Church
could not condemn. From this Eck at once drew the conclusion
that he (Luther) did not even recognize the ecumenical councils.
His opponent became startled and sought to retrace his steps, saying
that perhaps those decrees of the Council of Constance were not
genuine; for the rest, he contended, the word of God alone is infallible.
Then he modified this latter statement by saying that while
conciliar resolutions in matters of faith are binding, they may sometimes
be erroneous. Eck pinned him to his assertion that the Council
of Constance may have erred in the question of the primacy, and that
inexorably confronted him with all the inferences implied in that
assertion. Indignant at Luther, bluff Duke George, who
was loyal to the Church, exclaimed in a voice loud enough to be heard
throughout the great hall: “A plague on it!”

On July 14, Purgatory, indulgences, and penance formed the
subject of disputation between Eck and Karlstadt, but nothing
further was accomplished except that Eck clearly defined the position
of the Church, whilst Karlstadt denied the authority of the Church.
At the conclusion of the disputation, on July 15 (16) it was agreed
to submit the minutes to the universities of Erfurt and Paris, which,
however, also proved ineffective.

Luther was uneasy at the result of the controversy. After he had
returned home, he wrote to Spalatin that the Leipsic disputation
had commenced badly and ended badly, and that Eck and the men
of Leipsic were to blame, because they did not seek the truth, but
their own glory. He indemnified himself before the public by publishing,
towards the end of August, elucidations on the theses which
had been discussed at Leipsic. In these he proclaimed, even more
decisively than before his adhesion to his own assertions, and distorted
the position of his opponents.

Eck, on the other hand, triumphed, especially for the reason that
he had succeeded in exposing Luther as a heretic who wished to
destroy the authority of the councils and of the Church. He gained
the support of other Catholic writers, who espoused his cause and
that of the papacy which had been disparaged. Among those who
supported him was the priest Jerome Emser, formerly private
secretary to Duke George, a learned humanist and theologian,
who attacked Luther in a number of polemical writings, which elicited
violent replies.

The number of Luther’s friends and followers also increased in
consequence of the growing intensity of the battle.
It was of less importance that the Hussite opponents of the Council of Constance
in Bohemia complimented him on his attitude at Leipsic and his
subsequent writings. The Utraquists endeavored to form an alliance
with him, but their efforts did not result in any intimate, lasting
union. As a result of the terrible Hussite campaigns waged on German
soil, the Hussite faction had too bad a name in that country
to make it prudent for Luther to form an intimate alliance with
them at this juncture. The sympathy of the neo-German humanists,
which had been aroused by the Leipsic disputation, was of far greater
importance and promise for his cause. Crotus Rubeanus, a leader
of this group, wrote Luther from Italy, on October 16, 1519, reminding
him of their former association and adding that he had
extolled him at Rome as the father of his country, who was worthy
of a golden statue because he was the first to rise up in behalf of
the emancipation of God’s people from false opinions; for this
purpose he had been called by divine providence like another Paul
when a flash of lightning had prostrated him near Erfurt and driven
him into a monastery, a cause of “mourning to us, your companions.”\footnote
{Köstlin-Kawerau, \textit{Martin Luther}, Vol. I, p. 251.}

At this time he was also befriended by the Erfurt humanist and
jurist, Justus Jonas, subsequently his ally, who applied himself avidly
to the study of the new theology.

The most influential accession to the cause of Luther, however, was
the support of Melanchthon, who accompanied him to Leipsic and
whose enthusiasm for the light of Wittenberg was unbounded.

Philip Melanchthon (Schwarzerd), though but twenty-one years
of age, had achieved distinction as a philologist; at the recommendation
of Reuchlin he had left Tübingen in the summer of 1518 and
went to the University of Wittenberg to teach Greek and to carry
out his plan of issuing an edition of Aristotle in the original. His
acquaintance with Luther and the latter’s active influence attracted
the highly gifted young layman to theology, particularly in its
Lutheran form. Luther promptly detected the value which the
scholarly attainments and the amiability of the “weak little man”
would have for his cause. With his dominating nature he completely
captivated the pliant and susceptible youth. Even later, when Melanchthon
had opposed the doctrinal rigor and harsh conduct of
Luther, the pensive bookworm was unable to escape the overwhelming
influence of his master. In his antipathy toward Scholastics and
“sophists,” he at once launched upon the sea of Lutheran theology
with such impetuosity that he partly outdid Luther in his theses for
the theological baccalaureate which was conferred on September
9, 1519. His later achievements in behalf of Lutheranism, however,
consisted particularly in two things: first that his erudition and
formal training enabled him to cast Luther's ideas into a certain
systematic and academical form, and second, that he possessed a
certain skill, prudence, and flexibility which were necessary to insure
success in the public negotiations with the empire and with the
opponents of the new theology, gifts which Luther himself lacked.\footnote
{\textit{Briefwechsel}, II, pp. 204 sqq.}

In the first years of his acquaintance with Luther, Melanchthon
wrote to Spalatin: ``You know how carefully we must guard this
earthen vessel which contains so great a treasure \dots The earth
holds nothing more divine than him.''\footnote{Grisar, \textit{Luther}, Vol. IV, p. 269; Köstlin-Kawerau, op. cit., I, 442.}
He styles Luther “our Elias.”\footnote{Grisar, \textit{op. cit.}, III, p. 322.}
Luther appeared to him as one “destined by God” for his
work, “driven by the spirit of God.” “Leave him to the working
of his own spirit and resist not the will of God.”\footnote{Grisar, \textit{op. cit.}, III, 263; cfr. 322.}
Luther requited
him with exuberant eulogies. He declared that “almost everything”
about this youthful scholar was “superhuman.” “He excels me in
scholarship by his learning and the integrity of his life.”\footnote
{“\textit{Eruditione et integritate vitae}.” \textit{Op. cit.}, III, 321.}
Some
of the propositions which the theological “learning” of the philologian
was capable of inspiring, are set forth in the \textit{Loci Communes Rerum
Theologicarum}, published by Melanchthon in 1521, which will be
discussed in the sequel.

In connection with the polemical activity which Luther unfolded
in the year of the Leipsic disputation, we must here advert to his
pamphlet against Eck concerning the affair of the Franciscans of
Jüterbog. The latter had courageously preached against Luther’s
doctrines. Eck had seconded their efforts by means of printed theses.
Luther attacked the friars and Eck, their counselor, in a rude pamphlet
in which he styled them “vipers and a brood of vipers, ” and also for
the first time inveighed against confession, which, he alleged, was
not a divine institution,\footnote{Köstlin-Kawerau, \textit{Martin Luther}, Vol. I, pp. 254, 257.}
but introduced by a pope.

Jerome Dungersheim, professor of theology at Leipsic, who by
means of irenic and learned letters endeavored to persuade Luther to
abandon his course, received from him a private reply in which he
said: “We desire to have the Scriptures as our judge, whereas you
desire to judge the Scriptures.” He warned him not to abuse his
patience, since “countless wolves were tugging” at him already.\footnote{\textit{Op. cit.}, I, p. 258.}
He also reproved the Roman chamberlain Miltitz, when the latter
again appeared with conciliatory suggestions and endeavored to induce
him to go with him to Treves to let the Elector, Richard von
Greiffenklau, arbitrate the controversy. It was all the easier for him
to reject this proposal, since Miltitz had no papal approbation for
his plan, and since, moreover, the Elector of Saxony objected to the
journey to Treves on account of the dangers that beset it. For the
benefit of his high protector, Luther, in a letter written in December
to Spalatin, said: “I should wish to fall into their hands, so that
they could appease their fury, did I not fear for the Word and the
as yet small army of God.”\footnote{\textit{Briefwechsel}, II, p. 275.}

The Elector Frederick was very susceptible to the pious suggestions
of Luther, whom, however, for prudential reasons, he did not wish
to see. At his instigation, he diligently read the Bible. As the prince was
ill since the end of August, Luther composed a comprehensive and
consolatory treatise for him. It was written during the stress of his
polemical writings, many of which he prepared simultaneously
for the press. The work, an irenical and sententious treatise for all
sufferers in general, appeared in Latin and in German at the beginning
of 1520, and bore the title: “Tessaradekas” (the number fourteen).
Its fourteen motives for patience were intended to replace
the invocation of the Fourteen Holy Helpers. In this work the
productivity of his pen is evidenced in a marvellous manner; not-
withstanding the constant agitation in which he was steeped, this
work shows that he was endowed with ability to write in a tone of
sincere piety.

In addition to his polemical writings Luther composed many religious
works of a practical nature. His polemical productions however, outnumbered
the others. He denounced the distinguished Dominican
theologian Hoogstraten, of Cologne, who had appealed
against him to the pope, as “an illogical ass and 2 bloodthirsty enemy
of the truth.” In editing his lectures on the Epistle to the Galatians,
he sharpened his statements about the new Gospel by the use
of pointed denunciations, which stand side by side with reflexions savoring
of mysticism. He concluded this commentary with a violent tirade,
in the style of the Old Testament prophets, concerning the decadence
of the Church in his day. In the interval between his bitter invectives
against Emser at Dresden and other similar publications of a smaller
scope, appeared the printed beginnings of his larger interpretation
of the psalms (\textit{Operationes in Psalmos}) and his Latin postil, for
Advent. Incidentally he composed tracts on the Our Father the
Passion of Christ, preparation for death, usury, and other topics.
Prior to Luther no one had ever availed himself as extensively as he
did of the infant art of printing in the interest of a cause. Scarcely
anyone in succeeding ages attained to such an incessant activity in
the use of the press as Luther.

In addition there were many publications by others, either in

his defense or in opposition to him. Many of his sermons were copied
and printed either with or without his knowledge. Thus, two sermons
which had been carefully copied appeared simultaneously in print,
--one “On the Twofold Righteousness,” the other, a companion
piece, “On the State of Matrimony,” in a form which aroused lively
objections on account of the unheard-of frankness with which
that subject was treated. It is not possible to ascertain the extent to
which the printed text departed from Luther’s sermon. In consequence of
complaints that were made against it, he issued a revised
edition of it, in the introduction to which he says that there is “a
great difference between giving expression to something viva voce
and in dead letters.” In a letter to Lang at Erfurt he declined responsibility
for the first edition of his homily on matrimony, saying that
it was produced without his knowledge and caused him to feel disgraced.
In the revised edition he deals arbitrarily with the doctrine
and practice of the Church and expresses doubts about the validity
of clandestine marriages, which at that time were universally regarded
as valid. The sermon in its revised edition was extensively
circulated.

In the beginning of October, 1519, Luther reported to Staupitz
that he was satisfied with his success. Due to representations made to
Staupitz because of his favoring of Luther, he assumed a more reserved
attitude towards him. Archbishop Lang of Salzburg sought
to attract Staupitz to his episcopal city. In the above quoted letter
Luther complains: “You turn your back to me too much. As your
favorite child I am keenly hurt at this. I pray you, praise God also
in me, the sinner. I detest this very wicked life, I have a great fear
of death, I am devoid of faith, though richly endowed with other
gifts. However, I desire to serve Christ alone with my talents; He
knows it.”\footnote{\textit{Ibid.}, p. 184; October 3, 1519.}

Eck was a man of quite different character. Luther and Karlstadt
having sent their versions of the Leipsic disputation to the Elector
Frederick of Saxony, the latter forwarded them to Eck, who in
a lengthy publication frankly and honestly corrected the reports of
his opponents, showing “how they economized the truth in diverse
ways.”

Eck had to suffer much on account of the courageous stand he
had taken. Among those who inclined to Luther’s side Oecolampadius,
who subsequently became famous, wrote a sharp satire against him.
More bitter still was the contumely heaped on him in an anonymous
lampoon which bore the title: “The Planed Eck” (Eccius dedolatus),
supposed to have been written by Willibald Pirkheimer. As late as
1540, Eck, who had been persecuted throughout his life, wrote that his
traducers had depicted him in many forms, among others as a man who
had been “planed” and roasted.\footnote{Th. Wiedemann, \textit{Johann Eck}, p. 141.}
It was not as though Eck had not in
his private life furnished occasions for reproach; but in his defense
of the Church he permitted nothing to daunt him. Soon after the
disputation at Leipsic he ascended the pulpit of the magnificent
Gothic Church of Our Lady in Munich, the residential city of the
dukes of Bavaria, and raised his powerful voice against the Wittenberg
doctrines--the first to point out to Bavaria the ways of defending the
faith to which it subsequently adhered. He gradually completed his work
on the primacy of the pope, which had not yet
appeared in print at that time. The primacy of the pope and the
Roman Church in his opinion occupied the forefront in the controversy--so
much so that he desired nothing more ardently than a
final decision by the Apostolic See. He rejoiced very much, therefore,
when a brief of Leo X summoned him to Rome to report on
conditions in Germany. In the midst of winter, on January 18,
1520, he proceeded by way of Salzburg to the Eternal City, bringing with
him Luther’s German works, translated into Latin. On April
1 he presented to the Pope the manuscript of his own work on the
primacy. \footnote{\textit{Ibid.}, p. 150.}
