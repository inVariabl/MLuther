\section{Catholic Apologetics Against Luther}

The attack on the ancient Church summoned to her defense a
great host of writers, among whom were many illustrious minds.
The number of theologians, preachers, secular and regular clerics, and
Jaymen who wielded the pen in defense of the Church is surprisingly
great. Naturally not all their writings are valuable. The products
of the trained theologians transcend the works of mere preachers
or occasional writers. Many of these productions bear the stamp of
haste, the ephemeral polemics of the day, and the heat of battle; but
not a few possess permanent scientific and historical interest.
Unfortunately, it is a fact that this literature has not been sufficiently
noted, or at least not properly esteemed in its ensemble,\footnote
{In G. Wolfs \textit{Quellenkunde der deutschen Reformationsgeschichte} (II, ii, 1922, pp. 206
sqq.) the seventh section: “Die katholischen Gegner,” is unduly abbreviated.}
though, on the other hand, it labored too much under the influence
of contention and lacked unity and organization in its development.
Even to-day, it is often hardly possible to discover the most telling
achievements. The odiousness, moreover, with which the Protestant
partisans criticized these writings, lay as a heavy weight upon them.
In consequence of the disparagement with which these Catholic
authors were treated by the victoriously advancing reform movement,
it became customary to detect in them impotence of language
as contrasted with the fiery eloquence of a Luther, or intellectual
poverty in comparison with the bright flashes of his wit. It is true that
the recklessness of Luther’s language, by which he enthused the masses,
was lacking in the replies of those who defended the Catholic Church.
They do not possess the tempestuous force of the Wittenberg reformer.
In that tempest Luther alone was endowed with an epochal
linguistic talent far outdistancing in this respect even his own
followers.

In recent times 2 more appreciative estimate of the Catholic apologists
of the Reformation period has gained ground. The general
progress of objective historical research has contributed to this, as
well as numerous special studies, such as those of the scholarly Dr.
Nicholas Paulus. It has been aided particularly by the labors embodied
in the monumental Corpus Catholicorum, a collection of reprints
intended to unite all those old publications in a critical edition.

A more diligent study of these writings has revealed various excellences
possessed by many of the apologists of the old theology and
the rights of the Church. Versatility and brilliancy preponderate in
a surprising degree in the works of Jerome Emser, secretary to Duke
George of Saxony. Extensive reading and circumspection dominate
the writings of the cathedral canon, John Cochlaeus. Sincerity, directness, and acumen characterize John Eck, the professor of theology.
Dignity, nay, even a certain solemnity, are the marks of John Faber,

bishop of Vienna. Ingenious, though occasionally unrefined, humor is
discoverable in the writings of Thomas Murner, the popular Franciscan. Theological and religious austerity characterize the works of
the Dominican Jacob von Hochstraten. Erasmus, who originally
inclined towards Luther, in his subsequent controversial writings
shows great intellectual power and abounds in satire. These seven men
form a constellation of polemical writers around whom many other
able and distinguished defenders of the Church grouped themselves.
It was no easy resolve, but rather a certain risk for these men to
rise in opposition to the powerful Luther, or to resist the religious
revolt in general. These writers could expect no appreciation at the
hands of their opponents, but only derision and contempt.

This was particularly the case when Luther took notice of them. He
loaded them with opprobrium and often horribly distorted their teachings,
“I despise the opposition,” he says, “and regard them as downright fools.”
His adherents imitated his tactics.\footnote{Grisar, \textit{Luther}, Vol. III, p. 172.}
In the event that an opponent of his
achieved prominence, he was liable to be depicted as a beast in the cartoons
and pamphlets of the opposition. Thus Emser was represented as a goat,
C.Cochlaeus as an ass and a snail, Murner as a cat, Lemp as a dog. Luther and
his supporters accused the apologists of the Church that deep down in their
hearts they were convinced of the untruth of their own writings and the
justice of his cause; that they wrote as they did because they were friends of
the papacy or expected a liberal reward. If these accusations failed to dissuade
a courageous writer from his undertaking, he encountered difficulties in
procuring a publisher for his writings, as is evidenced, for example, by the case
of Cochlaeus, who made indefatigable efforts to maintain at least two efficient
Catholic printing presses. The reformed or religiously indifferent publishers
literally flooded the market. What they were after was profit. Their production
brought a good return in money, whereas Catholic books and pamphlets
were not acceptable to the colporteurs who traveled through the country
selling popular literature. Royalties as a rule were paid by neither party. The
Catholic apologists were confronted with the additional disadvantage of insufficient
material support on the part of the bishops, who were mostly remiss in
this respect, and by the absence of any stimulus for the work that
was so urgently needed.

If, despite these hindrances, a noteworthy and timely literature
was provided in defense of the Church, this fact was due to disinterested
zeal for the cause. Timely above all else were the recommendations made
by Catholic writers for a reform of the internal conditions of the Church.
Correctly sensing the need of the age, many
popular writers, whilst combating the new evangelical liberties, endeavored
at the same time to bring about a genuine reform of the
morals of clergy and laity. In reply to the criticisms of their opponents,
they unhesitatingly acknowledged the prevalence of abuses in
the Church. In many instances, effective notice of these evils was
taken only as a result of public criticism. Thus many Catholics admitted
that Lutheranism furnished the occasion for perfecting catechetical
instruction. For this reason the apologists of the time also exposed without
fear or favor the moral decadence which everywhere
accompanied the religious revolt. In forceful language they demonstrated
the fatal social consequences of the new gospel, especially on
the occasion of the Peasants’ War.

The defenders of the Church were benefited by the labors of the
theologians, which were often truly profound. They illustrated topics
which had hitherto been but inadequately treated, such as the Church
and her authority, the papacy as the center of unity, etc. The greatest
services in this regard were rendered by Cajetan de Vio. Other
Italians, such as Catharinus, as well as Germans, Frenchmen, and
Englishmen, followed his example. The foundations were laid for the
development of the theology of the Council of Trent and of the flourishing
post-Tridentine period \textit{in re} justification and grace. The apologists
also discussed Biblical questions more freely than before. This
change of tactics was necessitated by the Lutheran principle that the
Bible was the sole source of faith, and by the popularity of Luther’s
German translation of the Bible. As a consequence, several new translations
were made, such as those of Emser, Dietenberger and Dr. Eck.
Luther himself declared: “I have driven them to the Bible.”\footnote{Grisar, \textit{Luther}, Vol. VI, pp. 432 sqq.}

In general, the tone and style of these Catholic controversial writings
is moderate and convincing, free from the excesses of the opposition. Not
as though the indignation of the Catholic apologists did not
occasionally flare forth in their writings, as when, \textit{e.g.}, they saw how
ecclesiastical institutions and doctrines, with which they had been
acquainted since their youth, were subjected to monstrous distortion.
In their replies, however, they did not employ the rude style of
Luther, not even when, for instance, they, as cloistered religious, defended
their state of life against his vile book “On Monastic Vows,”
or when, as priests, they undertook to defend the most sacred thing in
their religion, the holy sacrifice of the Mass, against scandalous defamation.

Many are strikingly calm and conciliatory in their writings. This is
true, especially at the beginning of the controversy, of certain Franciscans
who had been well educated in the humanities. John Findling,
whose Hellenized name was Apobolymaeus, in 1521 published a
“Warning” to Luther, in which he addresses him as “dearest friend”
and refuses to characterize him as a heretic, although the papal condemnation
had already been issued. He challenges Luther’s divine
mission because of his unheard-of and hostile revolt--of course without
any prospect of influencing the reformer.\footnote{\textit{Ibid.}, Vol. 11, pp. 171 sq.}
Such works were inspired
by the laudable intention of not wishing to aggravate matters,
and also by a certain narrowness of view. The basic sentiment under
lying Humanism, which was very pronounced among the learned and
which particularly animated Erasmus, proved to be harmful.

Other apologists move more freely, for instance, the accomplished,
energetic Franciscan Caspar Schatzgeyer, a model of moderation combined
with correctness and vigor. He was the most prominent defender
of monastic life in southern Germany. At the time of his death,
which occurred in Munich, in 1527, he had composed more than
twenty works, most of which are excellent.

One of the grievances against Luther which pervades the works of
the Catholic apologists is his obstinate mendacity. They style him a
father of falsehood and a gross calumniator, sustaining this severe
indictment by many facts. Nearly all of them hold that the ex-monk
of Wittenberg is completely under the dominance of the devil, the
father of lies. Some of the most daring among them, in speaking of
the demoniacal traits in Luther’s character, insinuate that his is a
case of diabolical possession. It was their persistent belief that his
obstinacy and uncanny dexterity in inventing constantly new attacks
could not be explained except on the assumption that he was in league
with Satan. John Dietenberger, a learned Dominican, author of a
catechism and other works, calls Luther “the devil’s hired messenger”
and says that “here everything reeks of devils; nothing that the
devilish man writes can stand without the devil, who endevils all his
products.”\footnote{\textit{Ibid.}, Vol. IV, p. 355.}
The erudite and moderate Willibald Pirkheimer of
Nuremberg writes in a letter, in 1529: “Luther seems to have gone
quite mad, or to be agitated by some wicked demon.” Elsewhere this
author cites more than a dozen passages from varied contemporary
writings, which speak of a diabolic activity on the part of Luther.\footnote{\textit{Ibid.}, p. 353}
Luther himself gave occasion to the formation of such charges, among
other actions by such unintelligible performances as his alleged disputation
with the devil concerning the Mass. The general propensity
of the time to discover a special intervention of Satan in extraordinary
phenomena undoubtedly contributed to the formation of the
afore-mentioned accusations. Cochlaeus traces them to certain idiosyncrasies
of Luther shown when he was a young friar.

Elsewhere--first of all, so far as we know, in the writings of the
former Dominican, Peter Sylvius (who was, by the way a very ordinary
writer)--this contention of Satanic intervention is expanded into
the fiction, founded on fabulous narratives, that the devil himself
had begot Luther. This senseless babble was propagated by other
writers.
\footnote{\textit{Ibid.}, p. 356, 358.}
In general, distortions caused by the readily accepted false
legends about Luther, crept into the writings of lesser, nay, at times,
even of major authors. One of the most common of them concerns
his alleged drunkenness. This story plays its part both at home and
abroad. Thus, in Italy, the Dominican theologian Catharinus expatiates
at length on the inebriety of the religious reformer of the
North. Other authors, in view of the uncertain rumors which were
spread by the Anabaptists and other fanatics, write more objectively. The
learned Cardinal Cajetan, for instance, prefers to clarify
the questions at issue, without attacking the person or character of
his opponent, with whom he had become sufficiently acquainted at
Augsburg.

Only a few of those who discuss Luther’s character intimate that he
was abnormal in thought and sentiment. His nervous malady and its
influence upon his mental life were naturally hidden to the controversialists
. Men at that time were not interested in such observations.
The keenly penetrating mind of Erasmus, who was kept informed
by humanistic acquaintances of Luther, was ahead of his contemporaries
in this respect. In his controversial works, “Hyperaspistes”
(1526) and “Purgatio” (1534), and also in his letters, he calls attention
to this aspect of Luther’s nature,\footnote{\textit{Ibid.}, p. 353.}
though in his unrefined language
he at times goes too far. He states that Luther is mentally deranged in
various ways, imputes mental and emotional aberrations to
him (\textit{insanus, lymphaticus, non sobrius, febricitans, temulentus, sine
mente, delirus,} etc.). On one occasion he appraises him as follows: “In
writing thus, Luther, abandoned by the spirit, he is not himself active,
but there is active within him another spirit with his diatribes.”\footnote
{\textit{Ibid.}: ``\textit{Quis non videt, haec sine mente scribi, nec agere Lutherum, quum haec scribit,
sed agi spiritu quodam maledicentiae?}''}
Pirkheimer would not offer an opinion as to whether Luther was “demented
or actuated by an evil demon.” As early as 1524 John Clichtovaeus
describes the mental state of the ex-monk as “drunkenness
or demoniacal possession.” In 1522, the gentle Schatzgeyer is impelled
to use almost similar terms. The phenomenon was inexplicable to him
and therefore the admixture of falsehoods and exaggerations in the
delineation of Luther’s character is excusable.

The Catholics but rarely employed cartoons in combating Luther.
Their efforts in this direction are lacking in those captivating and
mordant elements which Luther did not shrink from applying in his
polemical cartoons. These are very vulgar in many respects, and it is
evident that the Catholic controversialists preferred to avoid such indecencies
on moral grounds. But they also lacked experienced artists
of the kind the reformers had in the person of Luther’s friend, Lucas
Cranach, and others who espoused their cause. Thus, while the well-known
effigy of \textit{Lutherus septiceps} (the seven-headed Luther) by
John Cochlaeus is based upon a sound idea, \textit{viz.}, to illustrate in
graphic fashion the contradictions of Luther and his vacillating attitudes,
the artistic representation is very defective and illustrates the
impossibility of tolerably representing a human being with seven heads.

In a rapid review of the most prominent protagonists and defenders of the
Catholic cause up to about 1520, the names of Eck, Cochlaeus,
and Faber must occupy a prominent place.

Dr. John Eck was such a prolific writer that he wrote thirteen
short treatises on. the religious questions of the day in 1518 and 1519.
He combated not only the doctrines of Luther, but subsequently also
those of Zwingli. On sundry journeys to Rome he acted as adviser
to the popes. As professor of theology at Ingolstadt he founded there
a veritable centre for the preservation of the faith. His incessant and
prolific literary activity was interrupted only by his pastoral labors. He
also achieved distinction as a powerful preacher. After his victory in
the disputation at Leipsic, he celebrated a great triumph in 1526 at
Baden (Switzerland) when he triumphantly defended the Catholic
teaching on the Eucharist in a disputation with the Zwinglians. With
a genuine mastery of the subject, he expounded the doctrine of papal
primacy in the first of his major works, which appeared in 1520. In
a second treatise, published in 1522, Dr. Eck set forth the Catholic
practice of penance and confession. He discussed other leading points
of the religious controversy in his writings on Purgatory (1523), on
the Sacrifice of the Mass (1526), and on the monastic vows (1527). In
1530--1531 he began the publication of his exposition of the Gospels,
originally in three parts, a work which achieved great practical results.
The most widely spread of his writings was his excellent and practical
“Enchiridion against the Lutherans,” a handy synopsis of all the
questions at issue, with a concise refutation of errors, accompanied by
citations from the Bible, the Church councils, and the Fathers. It was
an armory (\textit{armamentarium}, as he himself calls it) against the heretics.
No one combined such indefatigable activity with such practical
insight and such a forceful style as this Bavarian scholar, whom
Luther feared and endeavored to ridicule by applying to him epithets like
Dr. Geck (German for coxcomb) and Dr. Saueck (German for sow’s comber).\footnote
{Cfr. Grisar, \textit{Luther}, Vol. VI, Index s. v. “Eck.”}

John Cochlaeus, small of stature, but very active--styled the “puppet”
in Luther’s circle--was a native of Wendelstein near Schwabach
(whence the name Cochlaeus). As a humanist he had composed some
serviceable text-books.\footnote{\textit{Ibid.}, \textit{s.v.} “Cochlaeus.”}
While still a dean at the cathedral of Our
Lady at Frankfort on the Main, he was undecided what attitude to
take toward Luther, but after 1520 openly opposed him. In 1526 he
went to Mayence as a canon of Archbishop Albrecht. Upon the
demise of Jerome Emser, in 1528, he obtained the influential position
of secretary to Duke George at the court of Dresden, which he held
until the latter’s death, in 1539. The writings of this industrious and
self-sacrificing man number 202 distinct titles. They are noted for
their extensive learning and ready wit and, after the death of Eck,
advanced him to first place among the defenders of the ancient faith.
They are less conspicuous for theological depth. Luther made but one
reply to Cochlaeus, whose criticism proved very annoying to him,
and then chose to observe silence. The work to which Luther replied
was the first published by Cochlaeus. It bore the title, “De Gratia
Sacramentorum,” and appeared in 1522.\footnote
{Luther’s Reply was entitled: \textit{Adversus Armatum Virum Cochlaeum.}}
Other products of his pen
appeared in rapid succession, among them one in which the author,
inspired by patriotic motives, deplores the condition of Germany
caused by the religious controversy. His “Seven-headed Luther” bore
the sub-title: “Luther everywhere in contradiction with himself,”
and was published in Latin and German; it decisively influenced many
of those who still floundered in doubt.\footnote
{Grisar, \textit{Luther}, Vol. IV, pp. 380 sqq.}

John Faber, a native of Leutkirch in the Allgäu, was frequently
mistaken for the Dominican writer John Faber of Heilbronn and
for John Faber of Augsburg. He was a secular priest and originally
assumed an attitude towards the religious innovation which resembled
that of Erasmus. Subsequently, however, he initiated a great movement against
Lutheranism by his work “Against Certain New Doctrines of Martin Luther”
(1522) and his “Hammer against the Lutheran Heresy,” which appeared in
1524. He, too, occupied himself,
and that most effectively, with the contradictions in Luther’s writings,
to which he devoted his “Antilogies” of 1530. In the interim he composed
other works on the burning questions of the day. As vicar-general of the
bishop of Constance, Faber participated with his friend
Eck in the religious conference against Zwinglianism which was held
in Baden in 1§26. In 1527, Archduke Ferdinand sent him on important politico-ecclesiastical
missions to Spain and England. In 1528
this prince recalled him to Vienna, where he was to raise the religious
consciousness of the university and to oppose the spread of Lutheranism
in Austria. In 1530, at the urgent request of Clement VII, he
took over the vacant episcopal see of Vienna.\footnote
{Concerning Bishop John Faber, see Grisar, \textit{Luther}, Vol. VI, Index, \textit{s.v.}}

The Dominican John Faber, incidentally mentioned above, was
surnamed “Augustanus” (a native of Augsburg), because Augsburg
was his native city and for many years the scene of his activities. He
was an erudite scholar whose mentality closely resembled that of
Erasmus. Towards the close of the year 1520 he wrote a pamphlet
(“Ratschlag”) in which he judged Luther far too favorably. After
the appearance of the latter’s book on the “Babylonian Captivity,”
he decidedly changed his views, severed his connection with the Humanist
party, and combated the new theology in his sermons at Augsburg so courageously
that he was driven out of the city in 1525. He
died abroad in 1530, a victim of his incessant labors.\footnote
{Cfr. N. Paulus, \textit{Die deutschen Dominikaner im Kampfe gegen Luther}, p. 292. This
substantial and careful monograph also provides more particular information relative to
the following defenders of the faith.}

Faber of Augsburg had been vicar-general of the Dominican Congregation
of Upper Germany, which had seceded from the more
numerous body of “Observantines” of the same Order, who were
subject to the jurisdiction of separate provincials. This Congregation
of the Dominicans flourished side by side with the Saxon and
Upper German provinces of the same Order. All three of these great
bodies produced many learned and enthusiastic defenders of the
Catholic religion. The Order of St. Dominic, who had made the defense
of the faith a special object of his foundation, shared with the
Franciscans the leadership in the contest against Luther.

In the Saxon province of the Dominicans, two men achieved distinction by
raising their voices against the Lutheran innovation.
They were: John Mensing, a versatile theologian and author, preacher
at Magdeburg and Dessau, afterwards preacher and professor at
Frankfort on the Oder, and finally auxiliary bishop of Halberstadt;
and Peter Rauch, likewise an able protagonist of Catholicism in the
pulpit and by means of his pen, who died as auxiliary bishop of Bamberg.

The Upper German province of the Dominican Order, on its part,
was proud of Jacob Hochstraten of Brabant, professor, prior, and
inquisitor at Cologne. His first work against Luther, entitled, “Conversations
with St. Augustine,” which appeared in 1521--22, demonstrated that, in virtue
of his Scholastic training, he had a more correct and penetrating insight
into the errors of Luther than many
other contemporary Catholic scholars. His “Conversations” at the
same time reveal a comprehensive knowledge of the writings of St.
Augustine, the Doctor of the Church, whose utterances he contrasts
with the teachings of Luther. The “Conversations” were followed by
treatises on the veneration of the saints, Purgatory, Christian liberty,
and, finally, on justification and good works. They were composed
in a style which at times lacked due moderation. When Hochstraten
died, in 1527, the hatred of his enemies pursued him. They charged
that he had died amid tortures of conscience, having realized that he
had defended error--a calumny which was meted out to a large number of
Catholic apologists.

Conrad Köllin, a member of the same province, was a professor
at Cologne, celebrated for his knowledge of the writings of St.
Thomas of Aquin. Among other theses, he defended the doctrinal
infallibility of the pope, the indirect authority of the latter over temporal
matters, and the right of resistance to tyrannical rulers. In 1527
he published a ponderous and rather mordant refutation of Luther’s
doctrine of matrimony. Luther had denounced the Dominicans of the
University of Cologne as asses, dogs, and swine.

Ambrose Pelargus of Hesse, John Fabri of Heilbronn, Bartholomew
Kleindienst of Annaberg, John Dietenberger, a native of Frankfort
on the Main, and Michael Vehe of Biberach were members of
the same Dominican province. Dietenberger was the author of an
excellent catechism. He also translated the Bible and composed about
fifteen controversial works, which are noted for their learning and
acumen. They place their author in the first rank of the Dominican
champions of the faith. He labored chiefly in Frankfort on the Main,
in Treves, and in Koblenz. His works, composed exclusively in German, are
written in a plain and fluent style, whereas many other
controversialists of the time, less practiced in the use of their mother-tongue,
wrote ponderously in Latin.

Michael John Vehe labored in the service of Archbishop Albrecht
of Brandenburg. After the diet of Augsburg, he was appointed by
the archbishop a member of the “Neues Stift” of Halle, of which he
subsequently became provost. In 1531 he published an excellent treatise
on the “Reception of the Blessed Sacrament under One Species.”
It was composed in good German.

Next to the Dominicans, the Franciscans were represented by a
gallant host of apologists of the Church. At the close of the Middle
Ages, an energetic spirit of religious reform, sprung from the bosom
of this Order, had made its influence felt throughout Germany. Life
within the monasteries of the Poor Man of Assisi, as described by
John Eberlin, a Franciscan who had apostatized to Lutheranism, was
very edifying, characterized by penance, prayer, and zeal for souls.\footnote
{Grisar, \textit{Luther}, Vol. II, pp. 128 sq.}
Eberlin makes only one, and that a curious, complaint, namely, that
“the devil artfully uses their piety in order to corrupt humanity
with a false religion.” The example of good monastic discipline alone
was a defense against the religious innovation--a brilliant refutation
of the Lutheran attack upon Catholic morality. This practical defense was
seconded by the writings and sermons of excellent and
learned religious.

The Franciscan Augustine Alfeld, for example, entered the lists
at Leipsic, in the beginning of the religious controversy, with his
work “Super Apostolica Sede.” Owing to his clear insight, he at once
made the question of the primacy of the Roman see the central point
of controversy. His industrious and popular pen produced fifteen
Latin and German works, the style of the latter being superior to
that of the former, “remarkable alike for vigor and fervor.”\footnote
{Thus L. Lemmens, O. S. Fr., in his monograph \textit{Pater Aug. v. Alfeld} (1899), p. 99}
Caspar Schatzgeyer, provincial of the Upper German province of the Franciscan
Observantines, was another capable writer whose works enjoyed
even greater popularity. Thomas Murner, John Findling and
Conrad Kling were members of the same Order, as were also Nicholas Ferber
of Herborn, who labored with success in Hesse and
Cologne; John Wild (Ferus), a preacher and writer at Mayence, and
many other apologists and controversialists.

Of the Augustinian Order we shall only mention Bartholomew
Usingen (Arnoldi), Luther’s one-time teacher, and Konrad Triger.
The learned Nicholas Ellenbog of Ottobeuren was a member of the
Benedictine Order, and Paul Bachmann belonged to that of the Cistercians.

Certain converts who excelled in writing, among them Vitus Amerbach, Theobald
Billican, and later George Witzel, constituted a special and very remarkable phalanx of apologists.

A great number of secular priests, besides those already mentioned,
deserve recognition for their apologetic labors in the academic spheres
and in the pulpit. Let us but mention Jerome Dungersheim of Leipsic,
Ottmar Luscinius (Nachtigall) of Augsburg, and Konrad Wimpina, the soul
of the newly-founded university of Frankfort on the
Oder. An idea of the great number of German authors who wrote
prior to the commencement of the Council of Trent, may be formed
when it is recalled that the learned historian F. Falk published the
names of 105 such writers in the \textit{Katholik}, in 1891, and Nicholas
Paulus, writing in 1892 and 1893, supplemented this list by the addition
of 161 names of writers, without making any claim to completeness.\footnote
{\textit{Katholik}, 1891, Vol. 71, pp. 450 sqq. (Falk); 1892, Vol. 72, pp. 545 sqq.; 1893, Vol.
73, pp. 213 sqq. (Paulus).}
Who knows how many wavering souls were brought
back to the Church, or confirmed in the faith, through the efforts of
these apologists! That German Catholicism laid down its arms and
surrendered to Luther is an assertion which, at the present time, can
be attributed only to ignorance.

Non-German countries also furnished quite a number of defenders
of the faith, some of whom were very brilliant. Thus, in the Netherlands,
Jacob Latomus (Masson) of the University of Louvain was
very active in opposing Luther as early as 1518, and again in 1529,
when his work “De Primatu” appeared. Luther regarded him as
superior to all his other opponents, including Erasmus, who, he
said, in comparison with Latomus, could only croak. In France, Jodocus
Clichtovaeus, of the Paris Sorbonne, achieved celebrity by his
“Antilutherus” (1524) and his “Propugnaculum Ecclesiae adversus
Lutheranos (1526),” etc. In England the apologetical writings of
King Henry VIII and his famous chancellor, Sir Thomas More, as well
as those of John Fisher Bishop of Rochester, published in 1523, 1525,
etc., were preeminent. More and Fisher sealed their opposition to the
subsequent schism of Henry VIII with their blood and to-day are
honored by the Church as martyrs to the faith. The “Italian Opponents of
Luther,” according to Frederick Lauchert, who published a
monograph on this subject in 1912, comprised no less than sixty-six
scholars, beginning with Sylvester Prierias, Ambrosius Catharinus,
Thomas Cajetan, and Thomas Radinus, who rose in opposition to
Luther during the period preceding the Tridentine Council.
