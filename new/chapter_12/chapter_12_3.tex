\section{New Doctrinal and Controversial Writings}

Luther’s efforts to regulate doctrine and religious discipline are associated
with the publication of his two catechisms, which have attained great
significance. They originated in three series of catechetical
sermons preached in 1528. At that time it was a prescribed custom
at Wittenberg to preach week-day sermons on the catechism four
times a year, for two consecutive weeks.

“The Little Catechism, for children and simple folk,” first appeared
in the form of tables, designed to be affixed to various parts of the
home according to Catholic custom. Its arrangement, particularly
the sequence of Commandments, profession. of faith, and “Our
Father,” also followed the old tradition. A section on Baptism and
another on the Eucharist were appended. Its arrangement in the form
of questions and answers is a model of simplicity and clearness.
Though heterodox doctrines are emphasized in the Little Catechism,
polemical attacks upon the ancient Church are avoided. In the preface,
however, the bishops are blamed for the ignorance of Christian
doctrine existing among the common people, especially in the villages.\footnote{\textit{Ibid.}}

The Large Catechism was intended to instruct pastors and to enable
them to explain the smaller one in their sermons. It is not composed in
the form of questions and answers, and is rather prolix. Nevertheless,
certain characteristic doctrines, which were apt to give rise to dispute,
are omitted, as in the case of the smaller book. Thus the readers are not
informed of their right to pass judgment on the Bible and on matters
of faith, two things formerly so strongly emphasized by Luther. Its
pages are silent on the doctrine of original sin as the destroyer of every
inclination to good, on the enslaved will, and on predestination.

The two catechisms were very widely adopted and often reprinted.
Numerous Electoral regulations provided for their propagation and
use. In 1580 they were incorporated in the “Book of Concord” and
thereby raised to the rank of symbolical writings of the Lutheran
Church.

It cannot be denied that the desire not to be outdone by Luther
proved very beneficial to the development of Catholic catechetics.
The religious instruction of the people, it is true, had not been neglected.
To them Luther holds up his own example. Even in his manner of presentation
he completely followed the old method.\footnote{\textit{Ibid.}, pp. 489 sqq. }

But the
copiousness and systematic method of Catholic religious instruction
necessitated a certain gradation. It became necessary that it should be
transferred from the home, to which it had been previously entrusted,
to the school and the church. Then, too, the existing and quite estimable
number of elementary textbooks had to be multiplied. The
splendid work of Peter Canisius perfected the great amount of good
literature which was already in existence.

Luther believed himself justified in stating that, prior to his time, there
had been “no doctor on earth who had known the whole of the catechism,
that is, the Our Father, the Ten Commandments and the Creed, much less
understood them and taught them as now, God be praised, they are taught
and learned even by little children. In support of this I appeal to all their
books, those of the theologians as well as those of the lawyers. If even one
article of the catechism can be learnt aright from them, I shall allow myself
to be broken on the wheel or bled to death.”\footnote{\textit{Ibid.}, p. 485}
 He is better advised when he
recommends people to occupy themselves with the study of the catechism. He
makes his recommendations in a manner characteristic of himself. In the
preface to the Large Catechism he tells his preachers that whoever does not
heed and abide by the catechism, is to be classified with the “shameful
gluttons and belly-servers, who are better fitted to look after the pigs and
the hounds than to be pastors having the cure of souls.”\footnote{\textit{Ibid.}}
 Daily, every
morning and evening, he himself, though a doctor and preacher, recited “the

Ten Commandments, the Creed, the Our Father, etc., like a child.” The
preface exhorts all pastors “to practice well and always to occupy
themselves with” the catechism, and contains many practical ideas, in
proposing which Luther pleads with those “lazy bellies or presumptuous saints,
that, for God’s sake, they let themselves be persuaded.” His words have borne
fruit, and in many Protestant districts his catechisms have established and
sustained positive Christian beliefs.

Other minor writings support the Little Catechism. Thus, in 1526,
he wrote a little treatise on the ceremonies of Baptism, which supplied
the majority of Lutheran churches with the trinitarian formula and
the exorcism. In 1529, he composed a “Small Book on the Marriage
Ceremony for Simple Pastors,” which exercised a similar influence on
the marriage rite. According to the latter work, the nuptial ceremony
is a blessing joined with prayer, freely solicited by the bridal couple,
but not necessary for the validity of marriage.

At the head of his treatise “Von Ehesachen,”\footnote{Weimar ed., Vol. XXX, III, pp. 205 sqq.; Erl. ed., Vol. XXIII, pp. 93 sqq.}
 which he commenced
in 1529 and published in 1530, as well as in other writings,
Luther enunciates the doctrine, which he makes the basis and starting-point
of his instructions, that Matrimony is not a Sacrament, but
“a purely temporal matter, such as raiment and food, house and courtyard,
subject to secular authority.” The matrimonial problems with
which he had been overwhelmed, had crushed him. After the Catholic
laws on matrimony had been discarded, great confusion inevitably
ensued, and this disorder was augmented by the Peasants’ War and the
moral decadence which followed. “I have a lot of trouble with it,”
he writes to Spalatin; “I resist strongly, cry out and exclaim that such
matters should be left to the secular authorities.”\footnote{Similarly in a letter to Spalatin, January 7, 1527 (\textit{Briefwechsel}, VI, p. 6), in consequence
of the experience “that mankind could not be governed by the gospel.”}
In the treatise
mentioned, as elsewhere in his writings, he does not propose to give
any regulations pertaining to Matrimony, since temporal matters do
not pertain to the clergy and the Church has no right “to govern or to
enforce the law,” but her province is “to inform and console the
consciences.” Hence, he speaks only as one offering advice.

It is not necessary to demonstrate that his usual distinction between
the Church and the world, between the spiritual kingdom of Christ
and the external realm of the civil authority, between the forum of law
and the forum of conscience, was bound to involve him in vacillations
and contradictions, especially in the realm of Matrimony. An arbitrary
subjectivism all too frequently replaces objective law in his teaching.
Because this teaching--if it be possible to speak of uniform teaching
--sprang from the soil of his passionate attacks on the religious vows
and sacerdotal celibacy, it had a sensuous and dangerous aspect.\footnote
{Cf. S. Baranowski, \textit{Luthers Lehre von der Ehe}, Münster i. W., 1913, especially the
summaries at pp. 4 sqq. and 207 sqq.}

Nevertheless, he extols Matrimony as a dignified, yea, sublime institution
of divine provenience and delineates the natural side of conjugal
life in glowing phrases. He even boastfully claims that it was he who
“sang the praises” of the matrimonial state “in sermons, writings, and
examples,” whereas the papists did not recognize harlotry as a sin.\footnote
{\textit{Tischreden}, Weim. ed., Vol. IV, n. 5116.}
It is surprising to note how many Protestant writers repeat this self-praise
and at the same time overlook the tangible shortcomings inherent
in Luther’s theory of matrimony as well as the fatal consequences
of that theory when put into practice.

In his book “Von Ehesachen” Luther refrains from dealing with the
confused mass of questions which had arisen from the new concept
of Matrimony. For the jurists, even Schurf at Wittenberg, strictly defended
many of the traditional canonical regulations and interpretations against
his liberal notions. On the matrimonial impediments,
particularly those derived from consanguinity, as well as on divorce,
Luther expresses himself only in a cursory manner. He concedes
the right of divorce on the grounds of adultery and malevolent desertion.

One of the principal objects of this work is to combat clandestine
marriages contracted without parental consent. Matrimony, so he
teaches, is a public state of life and hence must be contracted publicly,
in the presence of witnesses and in the sight of the congregation.
Clandestine marriages contracted against the wishes of father and
mother or their representatives, are null and void. However, the
rights of parents should not be exaggerated. If parents wantonly prevent
their children from marrying, he says, the civil authority or, in
the event of the latter’s refusal, the pastor, with the assistance of
worthy friends, should permit and confirm the marriage. Luther was
confronted with severe conflicts with the secular authorities because
of his opposition to the constantly increasing secret marriages, or
secret betrothals, as he termed them. Because of his vexatious experiences
with matrimonial problems, his disgust increased to such an extent
that he subsequently wrote to Count Albrecht of Mansfeld:
“I have cast it [the matter of regulating marriages] from me and have
written to several persons that, in the name of all the devils, they
should do as they see fit.”\footnote{October 5, 1536; Erl. ed., Vol. LV, p. 147 (\textit{Briefwechsel}, XI, p. 90).}

Among other works written by Luther, mention should be made of.
his “Reply to the Libel of the King of England,” written in 1527,\footnote{Weimar ed., XVIII, pp. 26 sqq.; Erl. ed., Vol. XXX, pp. I sqq.}
--a crude rejoinder to the published reply of Henry VIII in which
the latter declined to cooperate with Luther, who had previously endeavored
to interest the King in his cause. In his rejoinder Luther says
that he “upholds his doctrine in defiance not only of princes and
kings, but of all the devils.” No less violent is his tract “On Secret and
Purloined Letters,”\footnote{Weimar ed., Vol. XXX, II, pp. 25 sqq.; Erl. ed.,, Vol. XXXI, pp. 1 sqq.}
 written in 1529 against Duke George of Saxony,
in connection with the affair of the alleged secret league of the Catholic
princes for the destruction of Lutheranism, discovered by Otto
von Pack. Luther had written a letter on this subject to his friend
Link, which had come into the possession of the Duke and was made
public by him, to the chagrin of the writer. In this letter, as well as
in the tract written to defend it, Luther asserts the existence of the
League, which had already at that time been shown to be a myth, and
maintained that George and his counselors were possessed by the devil,
and consequently he (Luther) was constrained to believe that they
harbored most wicked designs.\footnote{Grisar, \textit{Luther}, Vol. III, pp. 325 sq., V, P. 343.}

In 1527, Luther dedicated two treatises to the two “martyrs” of his
theological system, as he calls them. One of these is entitled, “Consolation
to the Christians of Halle on the Death of their Preacher,
George [Winkler]”; the other is a tract on “Leonard Kaiser, who was
Burnt in Bavaria for the Gospel.”\footnote{Weimar ed., Vol. XXIII, pp. 401 sqq.; Erl. ed., Vol. XXII, pp. 294 sqq. and Weim. ed.,
Vol. XXIII, p. 452; \textit{Briefwechsel}, VI, pp. 156 sqq.}
 Winkler had been attacked and
murdered in 1527 by unascertained culprits. Without proof, Luther
accuses the canons of Mayence of being accessories to the crime. The
second “martyr,” Leonard Kaiser (Käser), was a Bavarian ex-priest
who had studied at Wittenberg and was executed at Schirding after
being tried for heresy by Ernest of Bavaria, administrator of the bishop
of Passau.

In his printed message “To the Christians of Bremen”\footnote{Erl. ed., Vol. XXVI, 2nd ed., pp. 40 sqq. (\textit{Briefwechsel}, V, p. 112.).}
 Luther
glorified several other “martyrs”: two Augustinians of Brussels; the
Augustinian Henry of Zitphen (Henry Müller, died 1524); Caspar
Tauber, who was burnt at the stake in Vienna in 1524; a Lutheran
colporteur named George, who was executed in the same year at Ofen,
and an unknown individual of Prague. He pathetically exclaims that
“their blood would drown the papacy and its god, the devil.”\footnote
{Letter to Hausmann, November 17, 1524; Erl. ed., Vol. XXVI, 2nd ed., p. 403
(\textit{Briefwechsel}, V, p. 112).}

If several of his later utterances were taken seriously, many more
bloody victims of his party would have to be included among the
“martyrs” of the new faith. The number mentioned has been increased
by historical authorities; for the medieval laws against heretics continued
in operation. According to Riezler it is possible to demonstrate
that in Bavaria, which was accused of being especially cruel, relatively
few obstinate heretics were executed. In Württemberg, it appears,
capital punishment was more frequently inflicted upon apostate Catholics.
But it must be recalled that many of those who were executed
there and elsewhere were guilty of participation in the revolts connected
with the Peasants’ War as well as of other crimes which constituted one
of the grounds, if not the chief ground, for their execution. Others were
put to death because of their Anabaptist doctrines
and machinations, which were dangerous to the State; and for this
reason Lutheranism cannot claim them as confessors. A case in point is
the frequently cited one of Balthasar Hubmaier, who, having come
into prominence as an apostle of the revolutionary Anabaptist movement,
was burnt at the stake in 1527. In every religious conflict there the
have been deluded individuals who wrought up their errors to the very
point of dangerous madness and did not hesitate to risk their very
lives. The history of the severe laws formulated against heresy since
the early Middle Ages furnishes ample proof of this statement.

Among the polemical tracts which Luther issued against Catholic
worship, is his “Advice to a Dear Friend on Both Forms of the Sacrament
in Reply to the Mandate of the Bishop of Meissen,” published in
1528.\footnote{Weimar ed., Vol. XXVI, pp. 560 sqq.; Erl. ed., Vol. XXX, pp. 373 sqq.}
The exclusion of the laity from the chalice, decreed for practical
reasons by the Church, constituted one of the chief grounds of
attack against her. When, in 1528, the bishop of Meissen renewed his
regulations proscribing the use of the chalice for the laity, the Lutherans
availed themselves of these regulations in order to attack the
Church, and this afforded Luther an occasion to recapitulate his former
alleged proofs in support of the reception of the Sacrament under
both forms by the laity. The papal Church, thus he declared in opposition
to the bishop, sets the Word of God at naught in this respect,
thereby proving that she is the Church of the devil and the bride of
Satan.

The following two years saw a series of excellent Catholic replies
relative to the lay-chalice. The authors easily demonstrated that the
reception of the Eucharist under the form of bread alone does not
constitute a mutilation of the Sacrament, as Luther, appealing to the
alleged “clear, forceful words of Christ,” contended. They demonstrated
that the blood of Christ as well as His divinity were inseparably
united with His body. The Church in introducing the custom of
giving the host alone to the laity, because of the danger that the consecrated
wine be spilled or profaned, was within her rights, as she was
the administrator of the Sacraments for the glory of God. The demand for
the chalice was not prompted by zeal for divine worship,
but rather founded upon a spirit of opposition and provocation, since
the adherents of the new religion did not exhibit any great activity in
the reception of the Eucharist, and since, moreover, most of them did
not desire to receive the Eucharist under either form. Among those
who defended the traditional practice we find, for example, the Dominican
Michael Vehe of Halle, who, in a treatise, “On the Law of the
Reception of the Holy Sacrament,” composed in 1531, which was a
model of objectivity and calmness, demonstrated the right of Cardinal
Albrecht to prohibit the faithful at Halle to receive their Easter
communion under both species.\footnote{N. Paulus, \textit{Die Dominikaner im Kampf gegen Luther}, pp. 219 sq.}

By means of popular polemical pictures Luther endeavored to
arouse the passions of the people against the papal Church. In a pamphlet,
“A Vision of Friar Claus in Switzerland,”\footnote
{Weimar ed., Vol. XXVI, pp. 130 sqq.; Erl. ed., Vol. LXIII, pp. 260 sqq.
Cf. Grisar-Heege, \textit{Luthers Kampfbilder}, Vol. III (\textit{Lutherstudien},
n. 5), pp. 44--56, with illustrations,
especially p. 53.}
published in 1528,
he circulated an alleged vision of the papacy had by Blessed Nicholas
von der Flüe, which, he pretended, exhibited to the world Rome’s
“tyrannical, murderous, bloody dominion over body and soul.” He
sees in this utterly unhistorical representation a dismal head, crowned
with a tiara, whence six pointed swords are projected. In Luther’s
mystical interpretation, the triply serrated beard signifies the three
classes of men who adhere to the pope, “the sanctimonious, such as
monks, priests, and nuns; the scholars, such as jurists, theologians,
\textit{magistri}; the mighty of the earth, such as kings, princes, and lords.”
The pamphlet is a mental aberration, corresponding to its author’s
frame of mind.

Very bitter, even if droll, are the supplements which he appends
to his “New Intelligence of Leipsic” and “The Fable of the Lion and
the Ass.”\footnote
{Weimar ed., Vol. XXVI, pp. 539 sqq.; Vol. LIII, pp. 549 sqq.; Erl. ed., Vol. LXIV, pp.
324 sqq. Cf. Grisar-Heege, \textit{l.c.}, pp. 37--44, with illustrations, especially p. 43.}
Two authors of Leipsic, John Hasenberg and Joachim
von Heyden, had attacked Luther’s marriage to Catherine of Bora.
The above-mentioned document was intended to be Luther’s reply,
or, at any rate, a sort of reply. The former work is ornamented with
a square, in which the word ASINI is written with checkered letters
in such wise that, commencing with the center, the word “ass” can
be read forty times. The fate which, as he tells us, the writings of the
two Leipsic authors were to experience at his hands in his private
chamber cannot be described in decent language. The second of these
tracts depicts how the ass is made king of beasts; it is aimed at the pope,
whom the Leipsic professors honored, but whom Luther frequently
describes as “pope-ass.” An artistic illustration; furnished by Cranach,
which accompanies the work, presents two young asses, \textit{asini Lipsienses},
holding a large crown above the head of a braying, lean old ass, which
is followed by two young prancing asses, carrying halberds on their
shoulders. The text ironically ascribes to the pope-ass “the ability to
administer both the temporal and the spiritual government” and asserts
that there is “nothing about the entire ass which is not worthy
of royal and papal honors.”

Quite serious was Luther’s literary warfare against the Anabaptists.
His “Address to Two Pastors on Re-baptizing,” published in 1528,
was particularly characteristic.\footnote{Weimar ed., Vol. XXVI, pp. 144 sqq.; Erl. ed., Vol. XXVI, 2nd ed., pp. 281 sqq.}
 Contrary to the principal objection
of the re-baptizers, namely, that infants could not possibly have the
faith which Luther required as a condition for the efficacy of the
Sacrament, he obstinately asserts that children can and do have the
faith and denounces as a vain conceit the statement that they are as
yet devoid of reason. The constant practice of the Church supplies him
with a bulwark in support of Infant Baptism. The Church could not
have been permitted by God to remain in error for so long a time. He
speaks on this topic as if he did not accuse the preceding centuries of
the Christian era of the most glaring errors in other essential doctrines.
When he was in an agitated frame of mind, he took no account of such
contradictions.

The writings which were evoked by Luther’s controversy with
Zwingli occupy a special place among his literary productions.
