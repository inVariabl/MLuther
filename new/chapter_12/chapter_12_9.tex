\section{Further Spread of the Religious Revolt}

Unmindful of refutations, the new doctrine extended its conquests
under the influence of a false evangelical liberty and through the
forceful intervention of the secular authorities.

In the city of Braunschweig (Brunswick), Bugenhagen, in the
spring of 1528, endeavored to strengthen the Lutheran religion,
which had been introduced there by the magistrate. The magistrate
of Hamburg forthwith summoned him to the latter city to accomplish the
same object in the summer of 1529; after a period of intensive
organization, he returned to Wittenberg, where Luther, notwithstanding
his multifarious labors, had discharged the pastoral duties
in place of Bugenhagen. At the solicitation of the town-council of
Goslar, Amsdorf, while passing through Magdeburg, where he held
a position as preacher, went to Goslar in the same capacity. Luther’s
messengers had been originally banished from Lübeck, until the magistrate
of that town recalled them. In January, 1530, Luther exhorted
them to proceed with courage as well as caution.

In Hesse and electoral Saxony, no less than in Nuremberg, Ulm,
and Strasburg, Protestantism spread rapidly under the protection of
the defensive League formed by the princes and rulers after the protestation
of Spires. Ferdinand, the representative of the Emperor,
was authorized by a law of the Empire to avenge by force of arms the
manifest civil insubordination revealed in the progress of the new theology
and the League. But nothing was done. Luther was content to
issue a warning against the League, which had allied itself, in part,
with Zwinglian elements; he also characterized the rebellion as a
misfortune.\footnote
{Cf. Erl. ed., Vol. LIV, pp. 72 and 79 (\textit{Briefwechsel}, VII, pp. 101 and 110), on May
22 and at the end of May, 1529. But see \textit{infra} ch. XIV, no. 1; furthermore, Köstlin-
Kawerau, M. Luther, Vol. 11, p. 184, and Grisar, \textit{Luther}, Vol. III, pp. 44 sq., on Luther’s
vacillating attitude relative to armed resistance.}

In Pomerania, Lutheranism gradually gained ground under the
patronage of Duke Barnim XI, who had studied at Wittenberg
and remained in communication with Luther.

Paul Speratus, who, having been condemned to the stake at Iglau
because he had preached the new religion in Austria, was pardoned
and became preacher at the court of Duke Albrecht of Prussia at
Königsberg, and later (1530) so-called bishop of Pomesania, where
he indefatigably preached the new gospel and combatted the Anabaptists
and Schwenckfeldians.

In the electorate of Brandenburg, Joachim I, the elector, vigorously
excluded Lutheranism from his jurisdiction. His wife, Elizabeth, who
had been captivated by the new theology, secretly escaped from Berlin to
Torgau, and thence to the vicinity of Luther, who called her his
“Madam Godmother.” After the death of the vacillating Margrave
Casimir, in 1527, the Frankish-Brandenburgian territory was openly
Protestantized by his successor, George, who, residing in Franconia,
became one of the most active and influential Protestant princes.

Several times Luther communicated directly with the adherents
of the new theology in Livonia, which was subject to the Teutonic
Order. Thus, in 1523 and again 1524, he addressed letters to the
Christians of Riga. In 1525 he wrote an epistle “to the Livonians.”
In 1525, the inhabitants of Danzig requested him to send them a
preacher. The movement was impetuous, but was suppressed for the
time being by the King of Poland.

In July, 1527, Gustavus Vasa forcibly introduced the new religion
into the kingdom of Sweden; in doing so, he was “essentially influenced
by political motives.”\footnote{Thus Köstlin-Kawerau, \textit{M. Luther}, Vol. I, p. 625.}

In Denmark, Christian II, who at that time also governed Sweden,
had favored Lutheranism for political reasons, because he feared
influence of the Catholic clergy. Having been banished Germany,
he entered upon intimate relations with Luther. The latter, owing to
his ignorance of human nature and because he hoped for a change
in religion, supported him; however, Christian was not concerned
with religion, but solely with the recovery of his crown. His successor
in Denmark, King Frederick, was sincerely attached to the theological
innovation, which, however, triumphed only during the reign of the
despotic Christian III.

On the basis of certain premature reports concerning Italy, which
he had received from Gabriel Zwilling of Torgau, Luther wrote to
him on March 7, 1528: “I am delighted to hear that the Venetians
have accepted the Word of God.”\footnote{\textit{Briefwechsel}, VI, p. 222.}
In this instance, as in the case
of Christian II of Denmark, he deceived himself. For though the
writings of Luther had penetrated Venice, and Italy in general, there
was but a slight movement in favor of his cause. In Italy as well as in
Spain, the sharp-sighted Inquisition took precautions to prevent the
propagation of the new anti-ecclesiastical ideas.
