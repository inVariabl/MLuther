\section{Zwingli and the Controversy Regarding the Last Supper}

In the celebrated controversy on the Eucharist which arose between
Luther and the quick-witted Swiss reformer Zwingli, two
characteristic traits of Luther are made manifest. In the first place,
a more credulous tendency is discoverable in him, even to the extent of
accepting the miraculous and incomprehensible features
of the doctrine of the Eucharist, as contrasted with the subtlety to
which Zwingli and his partisans, Oecolampadius and Bucer, inclined.
Secondly, there is discernible that uncouth controversial method
which seeks to put down an opponent by every available means of
rhetoric, even if it be never so insulting, combined with most questionable
arguments. In the thought of mankind, the Last Supper is usually
enveloped by a divine peace. The Sacrament unites unto itself the souls
of men by the strongest bond of union ever bestowed upon mankind.
But now the holy table became the arena of a horrible quarrel which
was to last for a number of years and led to a desecration of the pearl
of Christian worship by the contending factions of the reformed religion.

The controversy was not confined to the two protagonists, Luther
and Zwingli, and their more intimate friends, but extended throughout
southern Germany, Switzerland, and the neighboring countries.

Zwingli, in virtue of his rationalistic conception of history, as set
forth in a letter written in 1524 to Alberus, in his “Commentary
on True and False Religion,” in which he discussed the words by
which Christ instituted the Holy Eucharist, had professed the following
proposition: “The Lord understands by bread and eating
nothing but the gospel and the faith \dots He absolutely does not
speak of any sacramental food.” The consecrated species, according
to his idea, are only a symbol of Christ, of His testament and His
grace; there is no Real Presence, which reason could not comprehend
and which appears contradictory, since the Body of Christ can be only
in one place, namely, in Heaven, at the right hand of His Father, and
not wherever the Lord’s Supper is celebrated.

The literary controversy with Zwingli was taken up on the
Lutheran side by Bugenhagen, whereupon Oecolampadius espoused the
cause of Zwingli by ‘his denial of the Real Presence. Luther had on a
former occasion defended his teaching on the Eucharist against
Karlstadt and the “fanatics” and now repeatedly expressed himself
against Zwingli, whose bold attacks on the Eucharist filled him with
indignation. He persistently classified Zwingli among the “fanatics,”
though Zwingli had openly declared himself opposed to the views
of Karlstadt. Hence, when Zwingli, in his ``\textit{Amica Exegesis}'' (1527),
renewed and intensified his opposition to the doctrine of the Eucharist
as formulated by the Wittenberg divines, Luther drew his sword
against his rival of Zurich by means of a violent polemical tract
entitled, “That the Words: “This is my Body’ are Still Firmly Established.”\footnote
{Weimar ed., Vol. XXIII, pp. 64 sqq.; Erl. ed., Vol. XXX, pp. 14 sqq.--On the history
of the controversy cf. W. Köhler, \textit{Zwingli und Luther; ibr Streit über das Abendmabl nach
seinen politischen und religiösen Bezichungen}, Vol. I, Leipsic, 1924.}
Zwingli replied in a tract which bore the scarcely less
vigorous title, “That These Words \dots Shall Retain Their Same
Ancient Significance Forever.”

Luther at last determined to put an end to the controversy by
means of a final tract, which he entitled, “Profession of the Lord’s
Supper” (1528).\footnote{Weimar ed., Vol, XXVI, pp. 261 sqq.; Erl. ed., Vol, XXX, pp. 151 sqq.}

This tract is not merely a summary and expansion of the reasons which he
had hitherto advanced against the “abominable heresy” of the “three leaders
of the Sacramentarians,” Karlstadt, Zwingli, and Oecolampadius, as well as
the Silesian Schwenckfeld; but it is at the same time a solemn profession of
faith, to which he is resolved to adhere with increased firmness, now that
its articles have been imperiled by the Sacramentarians. He analyzes these
articles, commencing with that “on the divine Majesty, on the Father, the
Son and the Holy Ghost.” The earnestness with which he discusses this
article is touching. In this part of his work he is also desirous of marking his
attitude towards the Catholic Church. He condemns her doctrine of free-will
and sin, represents the Mass as an abomination exceeding all other abominations,
rejects the invocation of the saints, and, relative to the Zwinglian
iconoclasts, asserts that, according to Holy Writ, sacred images are “very
useful;” finally, he contends that the Church consists in the communion or
assembly of all Christians without a hierarchy.

This so-called “Great Profession” made no impression upon the
opponents of Luther. Zwingli and Oecolampadius jointly published
a sharp reply. Luther, however, reverted only incidentally to the
great question of the Eucharist, until the memorable debate between
him and Zwingli at Marburg.

It cannot be denied that his presentation of the case against his
Swiss rival possesses certain merits in spite of obvious defects and the
injection of personalities. The necessity of adhering to the simple,
literal meaning of the words of institution is demonstrated by splendid
arguments and with convincing clarity. The obligation of believing
religious truths that transcend human reason is emphasized
in inspiring words. If, he says, Christ after His Resurrection passed
through closed doors, then His glorified body in the Eucharist is
not bound by the ordinary laws of nature, and it is simply a question
of submitting one’s intellect to the plain words of the Almighty.
He triumphantly appeals to the unequivocal belief of the Church
in the real presence, which she has held since the primitive period of
her existence. In these pages he is animated by the faith which he had
received in the days of his childhood and nurtured during his long
monastic career--a faith that resounds in the sonorous verses of the
\textit{Lauda Sion} of St. Thomas Aquinas: “\textit{Quod non capis, quod non vides,
animosa firmat fides praeter rerum ordinem.}” He pretends to have
gained a most intimate conviction of the real presence of Christ by
personal experience, nay, even to have been instructed in this truth
by angels.

A comparison between the two antagonists reveals the fact that
both are guilty of monstrous theological errors. Luther stands on
Biblical ground with only one foot as it were; his head is enveloped
by theological and philosophical clouds which completely obscure the
truth. As regards his demonstration from the Bible, he hesitates to
make a logical application of the literal interpretation. He rejects the
Catholic doctrine of transubstantiation and contends that the body of
Christ is simultaneously present with the bread. His arbitrariness is
manifested in his denial of the sacrificial character of the Eucharist as
maintained in the Bible.

In the theological and philosophical discussions of the past, this
sacred mystery, in as far as it was accessible to human reason, was
expressed in clear formulas. These Luther, because of his contempt for
Scholasticism, rejected as the product of “sophists.” As a result, he
founders in his attempt to escape the objections of Zwingli, as when,
\textit{e.g.}, he asserts that the body of Christ is at the throne of God, but
also participates in the divine omnipresence, so that it is found everywhere
throughout the universe (ubiquity), and, hence, also in the
Eucharist.\footnote{Erl. ed., Vol. XXX, p. 67.}
The important thing is, he says, that Christ, in virtue of a
special promise, desired His presence to be tangible to us somewhere,
and this precisely in the Eucharist. In Luther’s opinion this “special
promise” of Christ as regards the Eucharist resides in His intention
to strengthen our faith through the Sacrament and thus to mediate
our salvation. Nevertheless, he maintains that infidels actually receive
Christ when they receive the Eucharist.

Zwingli objected that, according to Luther’s doctrine, the Sacrament does
not constitute a thing \textit{sui gemeris}, since there are other
and even more effective means of confirming faith in the divine
promises, and therefore a superfluous miracle is postulated. Luther
can only meet this objection by asserting that in the Eucharist the
remission of sins, which is effected in a general way by the preaching
of the Gospel, is personally imputed to each individual. His main
object is to uphold his contention that the Sacraments do not
sanctify \textit{ex opere operato}, as taught by the Catholic Church.

Zwingli is equally arbitrary. It would be a most disagreeable task
to follow up this controversy in detail. Frequently one does not
even know to what extent both parties are sincere in their assertions,
so much are their assertions involved in contradiction. Thus
Luther does not approve of the expression that Christ suffered for
us only in His human mature, for the reason that this might cast
doubt upon the doctrine of the two natures in Christ. Julius
Köstlin, the Protestant biographer of Luther, says that, as regards
the controversy about the Eucharist, one is “justified in asking
whether Luther really understood what he maintained; especially
whether he had a definite and clear idea of what we call the distinction
between real and merely dynamic presence.”\footnote{Köstlin-Kawerau, \textit{M. Luther}, Vol. II, pp. 100 sq.}

The style in which this, the sublimest doctrine of the Christian
religion, is treated by the two leaders of the new theological systems
is anything but edifying. Ridicule, animosity, and misrepresentation
alternate with one another. Each is determined to remain master of the
situation and to capture the reader. How differently the same theme
is treated by the defenders of the Catholic doctrine of the Eucharist,
as, for instance, the scholarly Bishop Fisher of Rochester in his Latin
treatise, “On the Truth of the Real Presence” (Cologne, 1527).

Luther says that either his antagonists, above all Zwingli, or he himself
and his followers are servants of the devil; no other alternative exists. Satan,
he asserts, has crept into the Bible, which he (Luther) had once more
drawn from under the bench, and now produces rubbish and winks his
eyes, insinuating that Baptism, original sin, and Christ are of no consequence.
Zwingli’s objections are dictated by a “desperate black devil” and
his interpretations of Luther’s words are “impudent lies.”

The leaders of the “Sacramentarians,” he declares, are not persuaded of
the truth of their cause; in fact they cannot be, but must sincerely regret
(as he undertakes to demonstrate) that they began the quarrel. But the devil
is determined to be victorious.

“Whoever is willing to be warned, let him beware of Zwingli and avoid
his books as he would the poison of the infernal Satan; for that fellow is
thoroughly perverse and has lost Christ completely.”

In discussing Zwingli’s objection to Christ’s being seated at the right hand
of God, he mockingly speaks of the cope which Christ must wear in Heaven,
and asks whether all creatures are not there simultaneously with Him, “such
as lice and fleas, that infest the monk’s cowl.” His opponent replies with
undignified acerbity that he is not in need of the phantasmagorical heaven of
Luther, nor does he want his cowl or dog’s hood; Luther should take them
home and deck himself out in them.

Such were the depths of triviality to which these controversialists descended.
Nevertheless Zwingli assures his readers that he did not intend to
berate Luther with such “unrestrained language” as the latter employed to
insult him, but that he abstained from the use of such invectives as “fanatic,”
“devil,” “scoundrel,” ``heretic,'' “contumacious fellow,” “blockhead,” etc.
Indeed, his tone is more moderate in the beginning of his argument and
somehow betrays the cultured humanist. Eventually, however, he repays Luther
in his own coin and uses an uncouth, Swiss-German dialect.

In the interests of decency, especially of religious decency, it was
well that Luther made no further reply after his “Great Profession,”
but left the matter where it was by saying that “it was not the
proper thing for him to concern himself any longer with the doltish
replies and idiotic performances of his opponents.”\footnote
{\textit{Ibid.}, p. 102. Most of the passages cited above appear in Köstlin-Kawerau, \textit{op. cit.}}
