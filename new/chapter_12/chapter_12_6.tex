\section{Luther and Zwingli in the Castle of Marburg}

On April 22, 1529, the Landgrave of Hesse proposed to his friend
Zwingli that he and Luther, as well as the principal representatives
of the two parties in the controversy concerning the Eucharist, hold
a theological disputation. It was not zeal for religion which induced
him to suggest this conference, but the desire of realizing his political
ambitions, which, in comformity with Zwingli, were directed against
the Emperor and the house of Hapsburg. The exiled Duke Ulric of
Württemberg had won over the Landgrave to this hopeless idea of a
conference. The intrepid Zwingli favored it at once, without, however,
holding out any hope of union on his part. When the matter
was broached to Luther, he at first expressed himself against it.
Melanchthon also had his scruples. The Elector John at that time
was not disposed to have anything to do with the Sacramentarians, as
he feared the designs of the Swiss. Eventually, however, the urgent
invitations of the Landgrave were crowned with success. On September 29,
Zwingli and Oecolampadius, representing the Swiss, came to
Marburg, as guests of Philip of Hesse. Bucer, Hedio, and Jacob Sturm
represented Strasburg; Jonas, Cruciger, Myconius, and Menuis; besides
Luther and Melanchthon, represented Wittenberg.

At the first meeting, which was held on October 1, Luther And
Melanchthon engaged Oecolampadius and Zwingli, respectively, in
private conferences. But, apart from the burning question of the
Sacrament, the latter only succeeded in removing some suspicions of
the Wittenberg theologians against the doctrines of the Zurich reformer.

The second day witnessed a solemn public disputation between
Luther, on the one hand, and Zwingli and Oecolampadius on the
other. Luther had written on the table the words of institution in
large letters: “This is my Body.” In response to the sophistical arguments
of his opponents, he always repeated, after expounding the contrary thesis,
that he would not yield one tittle of these clear words.
Without yielding his position, Zwingli cordially demanded that they
come to a fraternal agreement.

After the conference had continued for three days, and a number
of futile meetings had been held, it was decided, upon motion of the
Landgrave, that at least a fraternal union, as desired by Zwingli,
should be established. But Luther refused to accept the hand that
was extended to him; for
to him the existing theological differences
about the Eucharist appeared prohibitive. When Zwingli pleaded
for Christian charity, Luther was prepared to grant the request,
provided it meant love of peace. He stuck to his oft-repeated declaration:
“You have a spirit which differs from ours,” perceiving and
fearing the frank rationalism of Zwingli, and believing it was this
spirit which induced Zwingli to regard the existing point of controversy
as not so important as to become an obstacle to the union
of both parties, and that eventually it would induce him to abandon
all religion.

The Fifteen Articles of Marburg, personally composed by Luther, were to
constitute a bulwark against the danger of infidelity, which he secretly
feared.\footnote{“\textit{Marburger Gesprich und Marburger Artikel}” Weimar ed., Vol. XXX, III, pp. 110
sqq.; Erl. ed., Vol. LXV, pp. 88 sqq.}
Contrary to his expectations, however, they were accepted by the
Swiss and Strasburg theologians. They expressed Luther’s position relative
to the Trinity, Christology, faith and justification, Baptism and private confession.
A further article, superscribed, “On the Sacrament of the Body and
Blood of Christ,” agrees with Zwingli in its negative aspect, namely, in its
opposition to Catholicism; for it repudiates the reception of the Eucharist
under one species and also the sacrifice of the Mass. This is followed by the
vague declaration that “the Sacrament is a Sacrament of the true body and
blood of Jesus Christ, and the spiritual reception of this same body and blood
is necessary for every Christian.” It states, furthermore, that “we all believe
and maintain the custom of receiving the Sacrament, as it has been given and
ordained by the Word of God, the Almighty, that the delicate consciences
be thereby moved unto faith and love by the Holy Ghost.” Thereupon the
existing dissension relative to the doctrine of the Eucharist is expressly
admitted and the necessity of charity (to a certain degree) is taught:
“Although we have not at this time settled the question whether the true body
and blood of Christ are corporeally in the bread and wine, nevertheless,
Christian charity, to the extent that every one’s conscience can tolerate,
should be mutually manifested by both factions, who should diligently
supplicate Almighty God to confirm us in the right understanding through His
spirit.”

Thus the opponents separated, having settled nothing. In the practical
field of the religious life the schism opened by the new theology
grew proportionately wider. Fortunately for Germany, one good
result came of this conference, namely, that the plan of an alliance
against the Emperor and the Empire, as fostered by Zwingli and the
Landgrave of Hesse, failed to materialize.

Although intimate union with the Swiss reformers was frustrated,
the Zwinglian doctrines continued to make progress in a portion
of Germany. In many parishes of the Swabian and Alemannic districts,
there arose a powerful Zwinglian faction. The Swiss doctrine and idea
of the Church, combined with the denudation of altars, the destruction
of sacred images, and certain political projects,
extended down the Rhine from Basle over Strasburg into the Netherlands.

In virtue of the attitude of Luther and his Elector, an alliance of
the Upper German Zwinglians with the Wittenberg reformers became impossible.
Nuremberg on the whole followed Luther, whereas
Ulm expressly joined the so-called “Burgrecht” of Zurich. The result
was dissension everywhere.

Luther looked in vain for another bridge to span the chasm. He
caused the so-called Articles of Schwabach to be proposed to the
towns of Upper Germany.\footnote
{Weimar ed., Vol. XXX, III, pp. 86 sqq.; Erl. ed., Vol. XXIV, 2nd ed., pp. 334 sqq.
Cfr. H. v. Schubert in the \textit{Zeitschrift für Kirchengeschichte}, XXIX (1918), pp. 342}
These articles had been drawn up by
the theologians of Wittenberg in July or August, and were more
or less opposed to the Zwinglians. On the sixteenth of October they
were rejected by Strasburg and Ulm at a congress held in Schwabach
and likewise at a convention held at Schmalkalden on November 29.
Landgrave Philip, whom Luther regarded with suspicion, oscillated
between the two great factions of the new religion, whilst Zwingli
hoped, through the mediation of the Landgrave, “to isolate Wittenberg
and thus finally to make it agreeable to his plans.”\footnote
{G. Kawerau, \textit{Reformation und Gegenreformation}, 3rd ed., p. 104.}
