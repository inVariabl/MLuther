\section{The Protest of 1529}

At the diet of Spires, which convened in April, 1529, under the
presidency of Archduke Ferdinand, the Catholic estates of the Empire
and their theological advisers appeared to be more united and
determined than before. They were prepared to put an end to the
abuse to which the recess of the diet of 1526 had been put by the
innovators during the last three years. For the flexible phraseology
of the resolution, which had been forced from that assembly, was
unscrupulously exploited by the reformers to advance their cause.

Hence, the Catholic majority of the estates at the new diet in
Spires succeeded in having the following modification adopted and
published by authority of the Emperor:

Those who adhered to the edict of Worms “shall and are required to
abide by the same until a future council”; in the case of the other estates,
all further “innovations” should be prevented, at least until the assembly of
the council, where it is impossible to abolish the new doctrine without
disturbance. In those parts where the new doctrine is upheld, preaching against
the Sacrament of the altar and placing obstacles to attendance at Mass were
specifically prohibited. Severe penalties were prescribed against the
Anabaptists and others who incited the people to rebellion. It was finally declared,
in order to safeguard religious peace, that no estate was to “protect
the subjects and relatives of one ‘faith’ against the authorities of the other”;
that the public peace declared by the diet of Worms was to continue; and,
if any one estate be “violently overrun” by another, it was the duty of the
“Kammergericht” (supreme court of judicature) to intervene.

There was nothing in these decrees which the reformers could
reasonably have represented as unfair, had they sincerely wished to
live in peaceful concord and in a temporary arrangement until the
assembly of the council which they themselves had desired. But they
harbored no such desire; on the contrary, they were eager to advance
farther and resolved to make a solemn protest against the legitimate
recess of the imperial diet.

This protest was made on the nineteenth of April, 1529, and from
it the entire party which made it derived the name of “Protestants.”
However, not all the princely estates which were favorably inclined
either to religious reform or to Lutheranism, concurred in this
hazardous protest. The Elector John of Saxony and the Landgrave
Philip of Hesse took the lead. They were joined by Margrave George
of Brandenburg-Ansbach, Duke Ernest of Brunswick-Lüneburg,
and Prince Wolfgang of Anhalt. Their following was considerably
increased when thirteen cities of Upper Germany, who were almost
exclusively in favor of Zwinglianism, joined their ranks. These cities
were Strasburg, Ulm, Constance, Lindau, Memmingen, Kempten,
Nördlingen, Heilbronn, Reutlingen, Isny, St. Gall, Weissenburg, and
Windesheim. Nuremberg also went over to their side. On April 22,
Electoral Saxony, Hesse, and the cities of Strasburg, Nuremberg, and
Ulm secretly formed a defensive alliance.

Whilst the Eucharistic controversy completely separated the
majority of the above-mentioned cities of Upper Germany from
the Lutheran faith, they nevertheless, in their inglorious procedure
against the authority and peace of the Empire, overlooked the religious
differences which prevailed in their own ranks and united against the
self-defense to which the Catholic Church was bound to resort.

The Protestants, moreover, frustrated the proposed movement
against the Turks. When the imperial diet convened, it was impressed in
the name of the absent Emperor with the necessity of
energetically repelling the danger of a Turkish invasion, which was
declared to be the most important subject before the assembly. The report
had reached Spires that the Turkish fleet was cruising along the
coast of Sicily, threatening the Occident. “It is an undeniable fact,”
says Wilhelm Walther, a Protestant authority, “that the [Protestant
estates] would not promise to render aid against the Turks, unless the
Catholic estates of the Empire arrived at some other conclusion concerning
the religious question than that under discussion, which [they
declared] it was impossible for them to accept.”\footnote{Grisar, \textit{Luther}, Vol. II, p. 383.}


Luther was naturally very much in favor of the idea that the
parties who espoused the new religion should issue a protest against
the resolution of the imperial diet. Two things, however, worried
him, and Melanchthon even more: namely, the formation of an
armed alliance of his followers in opposition to the Emperor; and the
approximation to Zwinglianism, which, though so far merely exterior, might
result in an intrinsic religious coalescence. Melanchthon
implored them not to break with Charles V, with Ferdinand, and
the “whole empire”; for thus far the Protestants had maintained
the semblance of not actually wishing to secede from the Empire,
nor even from the Church, but of desiring only a reform of the
same. For Luther, Zwinglianism, which he detested, constituted the
chief source of anxiety. In his worry concerning the new alliance,
he wrote to the Elector of Saxony on May 22 that the impetuous
temperament of the Landgrave [Philip of Hesse] would create havoc
in the Empire, that trust was to be placed in God, not in man, but
worst of all was the fact that, by uniting with the Zwinglians, the

Lutherans would incur the sins of those who antagonized the Sacrament
and were deliberate enemies of God.\footnote{Köstlin-Kawerau, \textit{M. Luther}, Vol. II, p. 120.}
The Elector bowed to
his representations and desisted from further attempts to bring about
an alliance.

With increased energy Landgrave Philip, on the other hand, pursued his
and Zwingl’s project of forming a close union of all the
adherents of the new religion in Germany and Switzerland, against
the Hapsburg power and against the Catholic Church. Before this
could be accomplished, however, it was necessary to effect some kind
of reconciliation between Luther and Philip. The latter was resolved
to go to impossible lengths, and it was decided that a personal conference
should take place between Luther and Zwingli.
