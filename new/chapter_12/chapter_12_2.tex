\section{Luther and the State Church of Saxony}

In the Saxon Electorate, which was his home, Luther urged the introduction
of visitations by commissioners to be nominated by the
secular ruler. This he did with increased zeal, the longer matters
were delayed by the prince, who proceeded in an arbitrary and autocratic
manner. The decadence of morals and discipline forced him
to take this stand.

He deplored “the ingratitude of the people for the holy Word of God.”
“They live like swine,” he says. “There is no longer any fear of God nor
discipline, because the jurisdiction of the Pope has ceased and everyone now
does as he pleases \dots If the elders do not want to, let them go to the devil
for all I care. But where youth is neglected and raised without discipline,
the authorities are to blame, and the land will be filled with dissolute
savages.” Thus Luther to the Elector John of Saxony, on November 22,
1526.\footnote{Erl. ed., Vol. LIII, pp. 386 sqq. (\textit{Briefwechsel}, V, p. 406).}

In respect to parishes and schools he adds: “Since religious compulsion is
at an end and all cloisters and foundations fall into the hands of the prince
as the supreme ruler, there arises the duty and inconvenience of regulating
such matters, to which no one else attends and in which no one can or should
be interested.” He contends that God has “summoned the prince to attend
to these matters in such an event.”

He demands, therefore, that the prince should entrust the visitation,
which had already begun in a tentative way, to four persons,
two of whom were to manage the property and stipends, while the
other two--they should go about the matter with “discretion”--
were to be entrusted with the regulation of doctrine and personnel.

The Elector’s instruction, which followed in 1527, prescribes the
course of the visitation and presupposes a direct right of the State
in spiritual matters,\footnote{Weimar ed., Vol. XXVI, pp. 195 sqq.; Erl. ed., Vol. XIII, pp. 1 sqq.}
formally setting up the territorial government
of the Church which was already in existence. In doing so, however,
the Elector to some extent exceeded the wishes of Luther, who, accordingly
, wrote a preface to the “electoral instruction” which in
some respects amounted to a restriction. Luther tries to preserve
a certain autonomy for the Church and to prevent the danger of a
ruinous dependence. In the preface which he added to the published
edict of the prince, he tries to justify the extensive measures which
the territorial lord was about to exercise in the sphere of religion
by the distressed condition of the Church and the fraternal duty
of a Christian ruler, whom he characterizes as an “emergency
bishop.” But his suggestions were disregarded and proved futile.
They are an ineffectual monument of his vacillation and embarrassment
in a question of fundamentals, namely, the authentic concept
of the Church. The voice of the court and its jurists, and the
facts created by the government were mightier than the vague and
impotent aspirations of Luther.

The ideas of the territorial government were, however, influenced
by an humble invitation extended to the Elector John by a friend
of Luther. In this invitation the Elector was styled a prince “of sacred
Christian lineage and descent” and exhorted to ameliorate the conditions
of the Church by becoming an example to other rulers and
following in “the heroic footsteps” of the pious Jewish king, Josaphat.
The religious functions of the kings of ancient Israel were held
up to him as worthy of emulation--an exhortation “afterwards frequently
repeated by the evangelicals.”\footnote{Köstlin-Kawerau, \textit{M. Luther}, Vol. II, p. 25.}
 There is no mention of the
prince’s functions being restricted by the distressed condition of the
Church, nor of fraternal duty; but the ruler, in virtue of his plenary
powers as prince, is competent to exercise authority over the churches.
In the name of the territorial ruler, Melanchthon supplemented
these general instructions by a set of additional “articles” for the
guidance of the visitators. They concerned the method of teaching
and set up a church discipline effective throughout the country. These
supplementary articles met with Luther’s approval and, after having
been examined and supplemented several times, by order of the Elector,
were formally promulgated.

Many individuals, both Lutherans and Catholics, were astonished
at the reactionary nature of these articles, which cautioned against
a declaration of the forgiveness of sins by faith alone, without previous
penance. The law which penalized sin was emphasized much
more forcefully than Luther had been wont to do. As a result,
Luther had to hear the objection: “We are crawling backwards
again.” Nevertheless, he stuck to his approbation of the articles of
visitation and remarked that they determined everything in the
simplest possible manner for the benefit of the mob; the objections
of the dissenters, he predicted, would soon cease.\footnote{\textit{Ibid.}, p. 31.}

In 1528, he himself composed a set of “Instructions for the Visitators,”
which was introduced into the territory of Duke Henry of
Saxony in 1538, and into the bishopric of Naumburg in 1545. Having become
quite disillusioned in consequence of his experiences,
Luther began to yield considerably in the matter of law and penance.
When his pupil, John Agricola, raised strong objections to the proposed
modifications, Luther opposed him.

On account of doctrinal differences, Agricola, at Eisleben, vehemently
opposed Melanchthon, with whom he was at personal enmity.
In general, he combated penance and, in part, the rules of a devout
life. A satire on the Wittenberg theologians classified him as an
Epicurean, “a discriminating voluptuary who knows how to choose
among pleasures.” His opposition to Luther at a subsequent period
caused the latter serious trouble in the so-called antinomian controversy.
At the same time, however, it gave him an occasion to
recede to an even greater extent from his original attitude toward law
and penance. “The first official act of the evangelical State Church
(the announcement of the visitation) thus became an occasion of
strife, yea, of charges of heresy within the most intimate circles of the
reformatory theologians of Wittenberg.”\footnote{\textit{Ibid.}}

The protocols of the visitators show the existing conditions among
the people, their preachers and new pastors, in the years 1527 to
1529, and later. It is not worth while to enter upon the melancholy
details. Luther’s summary complaint in a letter to Spalatin will suffice:
“Everywhere the congregations present a deplorable picture,
since the peasants neither learn, nor pray, nor do anything else but
abuse their freedom; they neither confess, nor go to communion, as
if they had completely cast off religion.”\footnote
{Middle of January(?), 1529; \textit{Briefwechsel}, VII, p. 45.}
 He adds: “Just as they
spurned the papal system, so now they condemn ours.” It is only fair
to remark that no such conditions existed under the papacy. The new
gospel of liberty and the Peasants’ War which sprung from it had
brought a return to barbarism. The vain excuse has recently been
put forth that the protocols of the Lutheran visitators exhibit a
state of decadence which “originated in the religious life of the Catholic
past.”\footnote{Köstlin-Kawerau, \textit{M. Luther}, Vol. II, p. 40.}

Though, as we have seen, the Catholic ages had their
shortcomings, it was nevertheless to be expected that a thoroughgoing
religious movement, such as the Reformation claimed to be,
should have produced a reform precisely at its inception. Indeed it
should have manifested decided signs of the spiritual spring which
Luther had persistently announced, especially when, in his extravagant
manner, he spoke of the breath of God which would renew all
things without violence.

An endeavor was made to increase the effectiveness of the visitations
by creating the permanent office of superintendent. It was another seal
affixed to the State Church. The office was established by the Elector
in his instructions governing visitations for 1527, and entrusted to the
pastors of the principal cities. It was their duty to supervise the belief,
teachings, and official functions of the clergy within their respective
jurisdictions and to report those who obstinately persisted in error to
the officials appointed by the prince, and through these to the territorial
lord. Where else was there authority that could inflict punishment?

In order to fortify the new religious system still more, the German
Mass was introduced for Sundays, to be uniform throughout Electoral
Saxony. It was an arrangement suggested by the territorial lord.
Luther had elaborated a set of hymns with the assistance of John
Walther, of the castle of Torgau, which was submitted to the Elector
towards the end of 1525. This Mass was prescribed for all the pastors
in the articles of visitation and was to be introduced with the least
possible disturbance. The article pertinent to this point cautiously
says: “It is not necessary to preach extensively to the laity about it.”
The deceptive resemblance to the Latin Mass was retained.

In all these ordinances there was not one word relative to that assembly
of “genuine and freely confessing Christians’ which Luther
had regarded as desirable. The ideal was interred in consequence of the
sad results of the visitation. In lieu thereof, the new church was overrun
by unspiritual members, whom Luther calls “pagans,” and developed more
and more into a compulsory organization. The Anabaptists
and other fanatics who rebelled against its doctrines, were
severely penalized. Luther even went so far as to demand the penalty
of decapitation for such heretics as were found guilty, not of insurrection
against the State, but of a fundamental deviation from his
doctrine.\footnote{Grisar, \textit{Luther}, Vol. VI, index s. v. “Intolerance” and “Heretics.”}

Walter Kohler, a leading Protestant historian of the Reformation, writes of
the tendency resulting from the regulations governing the visitations due to
Luther’s intervention: “Capital punishment for heresy was legitimized by
the Lutheran authorities \dots Freedom of conscience and of religion was
out of the question with Luther.” According to this writer, there is no doubt
that the trial of heretics by the Protestant churches was introduced by
Luther. When the preaching of the Word proved ineffectual, he appealed
to the secular authorities, with whom he was closely allied.\footnote
{For the resp. passages see Grisar, \textit{Luther}, Vol. VI, p. 266.}

The Protestant scholar, P. Wappler, who has made a special study
of the Protestant proceedings against heretics, and also of the Anabaptist
movement, shows by a number of concrete cases how, shortly after
the inception of the visitations, Luther decided in favor of the execution of
heretics, especially on account of the teachings of the Anabaptists. “The many
executions,” says Wappler, “of Anabaptists who are known not to have
been revolutionaries and who were put to death on the strength of the
declarations of the Wittenberg theologians, refute only too plainly all
attempts to deny the clear fact that Luther himself approved of the death
penalty against such as were merely heretics.”\footnote{\textit{Ibid.}, p. 267.}

The formula which Luther applied to representatives of the new
heresies was that they were to be remanded to “Master Hans,” \textit{i.e.}
handed over to the executioner. At Rheinhardsbrunn, for example,
six heretics were simultaneously remanded to Master Hans towards
the close of 1529 and the beginning of 1530, and were decapitated on
January 18, 1530. Of the inquisition of laymen which was provided
for in the electoral regulations of 1527, Wappler justly observes:
“The principles of evangelical freedom of belief and liberty of conscience,
which Luther had championed two years earlier, were most
shamefully repudiated by this lay inquisition;” and yet Luther said
never a word in protest.

An impartial and critical Catholic historian, Dr. Nicholas Paulus,
has filled 2 whole volume with convincing evidence of the intolerance
of the leaders of the religious schism, particularly Luther and Melanchthon,
which went even so far as to recommend capital punishment.\footnote{\textit{Protestantismus und Toleranz im 16. Jahrbundert}, 1911.}

Whilst obstinate denial of the faith was to be avenged thus severely,
--a measure which, of course, it was impossible to carry out universally
--an attempt was made to penalize also gross violations of the
moral law within the religious communities. The lack of the Catholic
ban was perceptible. The object of the regulations governing visitations
was to discipline notorious sinners by a refusal of the Lord’s supper.
Luther’s first aim was to have obstinate sinners declared pagans
in the eyes of the congregation, after a futile warning in the presence of
witnesses. Later he decided in favor of greater severity by adopting
a kind of excommunication. Various efforts on the part of others to
install elders for the supervision of morals were condemned to fail of
ure.\footnote{Köstlin-Kawerau, \textit{M. Luther}, Vol. II, p. 47.}
 The only measure that was finally retained was the office of
beadle, who, according to a phrase of Luther’s uttered in 1529, had
to hale those to church who despised the catechism, so that they might
learn the Ten Commandments, etc. Whoever scorned to learn the catechism,
he said, should be exiled by the civil authorities.\footnote{Grisar, \textit{Luther}, Vol. V, pp. 484 sq.}
