\section{Personal Experiences. Temptations}

On his return to Wittenberg from the Marburg Conference, Luther,
as he states in a letter to Link, was seized with grave spiritual afflictions
and temptations at Torgau.\footnote
{Letter of October 28, 1529, from Wittenberg; \textit{Briefwechsel}, VII, pp. 179 sq.}
 It is possible that the excitement experienced
at Marburg and the sermons he preached on his homeward
journey may have contributed to this state of soul. While he boasted
of his victory over Zwingli, he fully and painfully realized the latter’s
obstinacy and the dangers that threatened his own doctrine. In addition,
terrible accounts of Turkish aggressions tormented him. His
heart, which once before, at Wittenberg, had caused him trouble, again
became affected. He wrote to Link that he was scarcely able to reach
Wittenberg, since Satan’s angel had tormented him so that he despaired
of being able to return to his own.

At Wittenberg his suffering continued. Whilst blaming the Turks
to a great extent for his condition, he complained to his friend
Amsdorf that they (the Turks) had been sent by God to chastise
the blasphemous enemies of the gospel and to punish the people
for their intolerable ingratitude towards him. He said he sensed their
fury in the struggles of his soul, but with the aid of Christ hoped to
overcome their god, the devil.\footnote{On October 19; \textit{Ibid.}, p. 173.}
Only gradually a period of relative quiet assuaged him.

Previously, in July, 1527, Luther had survived an attack which
had brought him to the verge of death. At that time his mental
sufferings and self-reproaches were preceded and accompanied by
fainting spells. Bugenhagen, who attended him and also heard his
confession, wrote at the time that Luther’s internal sufferings were
comparable to the spiritual darkness “which is often mentioned in the
descriptions of the infernal torments of the soul that occur in the
Psalms”; they were more severe and dangerous than the mortal weakness
of the body. He adds that Luther had “been through a number
of such attacks,” although they were not all so severe. Luther himself
at the time said to his confidant Jonas, that those who observed
his exterior conduct were of the opinion that “he lay on a bed of
roses, though God knew how it stood with him.”\footnote{Cfr. Grisar, \textit{Luther}, Vol. V, pp. 333 sqq.}

In the language of Luther and his friends, those painful conditions
are termed spiritual temptations (\textit{tentationes spirituales}). We learn
more about their nature by his frank epistolary communications
during this period of severe internal struggles, through which he passed
at the close of the year 1529, after a temporary surcease in July.
These communications contain a vivid delineation of his despondent
mood and his theological fears--they are the most melancholy chords
of his entire life.

Undoubtedly his bitter experiences in connection with the visitation
of Electoral Saxony, the irritating discussions between himself and
Zwingli, the negotiations caused by the alleged papal conspiracy
discovered by Pack, the dangerously arbitrary conduct of Philip
of Hesse, the latter’s violation of the public peace and his hostilities
against Bamberg, Würzburg, and Mayence, were contributory causes.
Above all, the pathological condition of his nerves and his irregular
heart-action must be taken into consideration. When he was attacked
by spells of weakness and fear, his physical infirmities would
combine with the spiritual unrest caused by his reformatory efforts.

His fears centered around such questions as: Why have you disturbed
the peace of the Church? Are you sure that your cause is just? How
will you account for the ruin of so many souls? The bright memories
of his monastic days and of the happy hours he had spent within the
Catholic Church simply would not subside. His fears, therefore, did
not originate solely in his physical ailments; at least they were frequently
present without any concomitant sense of illness, as he protested
at times when they were most violent, that “as far as my physical
condition is concerned, I find myself tolerably well”; or, “I feel
well in body.”

“For more than the whole of last week,” he writes to Melanchthon
on August 2, 1527, “I was tossed about in death and hell, so that I
still tremble all over my body and am exhausted. Billows and
tempests of despair and blasphemy assailed me, and I had lost Christ
almost completely.”

He complains to his friends that he is suffering from the buffetings of the
prince of this world, the devil, who would be avenged on him; that reason
cannot comprehend how difficult it is to know that Christ is our justice;
and that “he seeks or thirsts after naught else than a merciful God.”
Consequently, he had not yet found Him with an abiding certitude, notwithstanding
the fact that when he apostatized, he based his entire doctrine and fight
against the Church and the pope.upon this supposed discovery.

“I am well in body; but as to how it stands with me in spirit I am not
certain \dots I seek only for a gracious Christ \dots Satan wants to prevent
me from writing [in behalf of the gospel], and to drag me down with him
to hell. May Christ tread him under foot. Amen!”

“My Katie is strong in faith,” he wrote at that time; but of himself he is
constrained to say: “I am scarcely able to breathe because of tempests and
despondency.” He laments that pope and Emperor, princes and bishops,
nay, the whole world assail him, including “Erasmus and the Sacramentarians
.” His very brethren torment him. In the words of St. Paul, he cries
aloud: “Combats without, fears within” (2 Cor. VII, 5). By citing his
favorite Biblical passages, he endeavored to fortify himself in his own doctrine;
but he felt that the “prince of demons,” who had risen against him,
was “armed to the teeth with Biblical quotations, so that his own knowledge
of Sacred Scripture vanished before him.”

With foolhardy temerity he nevertheless forces the habitual notions of
his own upon his conscience, lest he perish. “Christ, indeed, has become
weak” (in him); nevertheless, he would “believe with firmness (\textit{fortiter
credo}) that his work was pleasing to the Lord.”

What oppressed him most is the thought that Satan alone is
active in the attacks upon his conscience; that he (Satan) assumes
the form of Christ and decks himself out as an angel of light. Is it
not probable that all his spasmodic imaginings of Satan concealed
from him the just reproaches with which he accused himself? To
him the voice of his conscience is the voice of Satan.

But this is not the place to penetrate more deeply into Luther’s
dismal mental struggles during those months. It was the most tempestuous
period through which he had to pass. It approached its close
at the beginning of 1528, but there were painful after-effects.
“Blessed be my Christ,” he says amid a sight of relief, “blessed in the
midst of despair, death, and blasphemy \dots It is my glory to have
lived in the world in conformity with the will of Christ, forgetting
the very wicked life of the past.” The story of his sufferings reveals
the extent to which an impetuous will is capable of torturing the
soul. Scarcely another man ever commanded such titanic forces as
did Luther in his interior and exterior struggles.

An echo of his internal experiences is his famous hymn, “Ein’
feste Burg” (A safe stronghold our God is still), which he composed
in those days and which is still widely sung by his admirers, but
properly understood only by a few. In its ponderous verses, expressive
of the ardor of the battle which he at that time waged against the
pope and the devil, against the “ancient evil one,” he clung to the
Christ of his Gospel: “But for us fights the proper Man, whom God
Himself hath bidden \dots And were this world all devils o'er \dots they
cannot overpower us.”\footnote{See Grisar, \textit{Luther}, Vol. V, pp. 549 sqq., for the full text, and also Grisar, \textit{Luthers
Trutzlied “Ein’ feste Burg”} (\textit{Lutherstudien}, 1922, n. 4), PP. 14 sqq.}

Towards the end of January, 1528, he declared to a confidant quite
in his own fashion, relative to the Sacramentarians who annoyed him,
that he was determined, in order to get rid of his fears still more
effectively, “still further to provoke Satan, who was raging against
him with the utmost fury.”\footnote
{Grisar, \textit{Luther}, Vol. V, p. 338. Letter to John Hess in Breslau, January 27, 1528
(\textit{Briefwechsel}, VI, p. 199.)}

In the middle of the same year he told another friend that it is
always necessary, when temptations assail one, to exert oneself to the
utmost against the devil, who is plainly to be discerned; and that
“it is imperative to achieve salvation by blindly assuming as certain
that all thoughts to the contrary are mere devil’s treason.”\footnote{\textit{Ibid.}, pp. 338 sq.}

From time to time, nevertheless, his writings and addresses reecho
his lamentation of a “struggling conscience.” He hears how the devil
speaks through man: “It will not be easy for you to die.” Yet, as
time went on, his mental gymnastics increasingly overcame the reproaches
of reason and conscience, aided by the distractions of his
polemical life, the delirium of his successes, and the intoxicating
eulogy of his friends.\footnote{\textit{Ibid.}, pp. 338 sq.}

His courage found a more worthy cause to display itself when, in
mid-summer, a lingering disease, described as the plague (Pest), broke
out in Wittenberg and the rest of Germany. The university was
temporarily removed to Jena; many fled, but Luther and Bugenhagen,
the local pastor, remained to administer spiritual consolation
to the sick and the dying. The contagion also entered the former
monastery which was now his home. But he was not concerned
about his own life. “Christ is here,” he wrote to Spalatin, “so that we
may not be alone; and He will be victorious.”\footnote{On August 19, 1527 (\textit{Briefwechsel}, VI, p. 76.)}
 At that time he
composed the little treatise, “Whether one May Flee from Death,”\footnote{Grisar, \textit{Luther}, Vol. IV, p. 272.}

intended to inspire courage. Pastors and preachers, such is his exhortation,
ought to remain at their post, especially in such dire
trouble, when the flock is more than ever in need of spiritual help.

Luther exhibited the same courage during the epidemic of the so-called
“English sweat,” a fever which broke out in Wittenberg and
other cities in 1529. Again, in 1538 and 1539, he braved new outbreaks
of the plague at Wittenberg, regarding perseverance as a duty
imposed upon him by his office, which was watched by many with distrust.
“God usually protects the ministers of His Word,” he writes in
1538, “if one does not run in and out of the inns and lie in bed.”\footnote{\textit{Ibid.}, pp. 272 sq.}

Although many demands were made upon him, he willingly succored the suffering
and the poor, aiding them as generously as his
circumstances permitted. Thus he was able to say in a sermon to
the people of Wittenberg that “he himself was poor, but the joy
with which he utilized what had been given to him to satisfy his
needs exceeded that with which the wealthy among them enjoyed
their accumulated riches.” At the same time he censured the avarice
which he detected at Wittenberg. It was a theme to which he often
reverted. He was not accustomed to seek comfort in the pleasures
of the table. He loved simplicity in his domestic life no less than in
his manners, conversation, and intercourse with men. In this respect
he wished to be an example to those who were associated with him
in his work. When not afflicted with melancholy, his familiarity and
cordiality were a source of refreshment to his friends. He gladly displayed
his characteristic humor, occasionally even in dark hours, in
order to distract his mind. He speaks of this as a motive of his
jovial talks.\footnote{Grisar, \textit{Luther}, Vol. V, pp. 306 sqq.}

It is not true that the scene of his conviviality was a tavern where
he was wont to consort of an evening with his friends and pupils.
The account in question is a fabrication. As a matter of fact Luther
spent his evenings with his family, in the one-time monastery where,
with Catherine von Bora, he was usually surrounded by those who
were associated with him in his work, pupils or newcomers.\footnote
{H. Grisar, “\textit{Ein unterschobener Bericht},” etc., in \textit{Ebrengabe für Herzog Johann Georg
von Sachsen}, 1920, pp. 693—703.}

Nor is it true that he drank to excess.\footnote{Grisar, \textit{Luther}, Vol. III, pp. 294—318.}
 The so-called fanatics, the
Anabaptists, who were often strict in outward appearance, as well
as misinformed Catholic opponents, propagated unconfirmed rumors
to this effect. Some controversial writers discovered a pretext for
these accusations in certain misunderstood utterances of his. But
these critics overlooked the fact that their charges were based upon
jocose speeches or innocent quips by a man who was not always
cautious in his utterances. It is nowhere credibly reported that
Luther was drunk, even though there is evidence to show that he
imbibed rather freely, according to the prevailing German custom.
He was not exactly a model of abstemiousness, but he severely censured
the excesses of princes and courtiers. In theory he was undoubtedly
too compliant when he permitted a “good drink” (which
in those days meant a considerable quantity) in cases of depression
of spirit due to evil reports, worries, and heavy thoughts in general,
oppression owing to troubles and labor, temptations of the “devil”
resulting from sorrow and despondency. In his opinion, sleeplessness
and spiritual exhaustion alone were sufficient to justify a ``good
drink.''\footnote{\textit{Ibid.}, p. 312.}


Mathesius, his pupil and eulogist, who was in many respects his
mouthpiece, is even more indulgent. He says in one of his ``wedding
sermons'' that one must have “a certain amount of patience” with
those who sometimes, for a quite valid reason, “get a little tipsy”
or “kick over the traces.”\footnote{\textit{Ibid.}, p. 310.}


If Luther had been addicted to the use of wine and beer in an
excessive manner, he would not have been able to develop his marvelous
energy. A drunkard does not write books and pamphlets filled
with serious and thought-provoking ideas with the ease and facility with
which Luther composed his writings. Even the violent and indecorous
controversial tracts of the later period of his life are not saturated
with alcohol, as a Protestant writer in America has recently endeavored
to demonstrate; but they evince the spirit of an infernal hatred
which is to be adjudged pathological. The so-called “drunken doctor”
(\textit{doctor plenus}) must be obliterated from history. In passing
it should be remarked that this description of himself, which was
said to have been found in one of his letters, is based upon an incorrect
reading.\footnote{H. Grisar in \textit{Histor. Jahrbuch}, XXXIX (1919--20), pp. 496--500; cfr. \textit{Luther}, Vol. III,
pp. 316 sq.}

The existence of natural children of Luther, with which ignorant
polemicists of a former age frequently concerned themselves, is also
unhistorical. Erasmus says in one of his letters that Catherine von

Bora was confined a fortnight after her marriage with Luther. Subsequently
he retracted this false rumor.\footnote{Grisar, \textit{Luther}, Vol. II, pp. 187 sq.}

An alleged illegitimate child,
called Andrew, born at a later date, proved to be Luther’s nephew,
Andrew Kaufmann. The maid servant in Luther’s home, Rosina
Truchsess, turned out to be an immoral woman, but there was not
the least excuse for the gossip that Luther had sexual intercourse
with her prior to his dismissing her in a fit of anger. The adulter
infans (adulterine child) discovered in the controversial writings
of Aurifaber (1569) is merely a printer’s error for alter infans
(the other child), as correctly printed in the edition of 1568.\footnote{\textit{Op. cit.}, Vol. III, pp. 280 sq.}


History records that five children were born of Luther’s union with
Catherine von Bora: Hans, born June 7, 1526; Magdalen, born in
1529; Martin, born in 1531; Paul, born in 1533; and Margaret,
born in 1534.

In mid-summer, 1525, Luther secretly sheltered among his guests
in the Black Monastery, his former friend and Wittenberg associate,
Andrew Karlstadt, who had become his bitter enemy. This was a
pleasant trait of his character. After the unfortunate issue of the
Peasants’ War, in which Karlstadt was accused of having participated
in virtue of his sermons in Rothenburg ob der Tauber, he lacked the
necessities of life and now solicited Luther’s aid at any price, prepared
to suffer any kind of humiliation. He was willing to keep
silence with regard to his own special doctrines and to work for
a living, provided he was permitted to return to the Electorate of
Saxony. Luther was prepared to intercede for him with his sovereign,
Karlstadt came to visit him and secretly spent several weeks in the
former monastery. For a time not even Catherine was aware of
his presence. Only Luther’s servant Wolf Sieberger had been initiated
into the secret and daily brought food to Karlstadt. After Luther
had obtained from him a forced declaration concerning his teaching
on the Last Supper, he interceded with the Elector John, who gave
Karlstadt permission to remain in his Electorate.\footnote
{\textit{Ibid.}, Vol. III, p. 388. Barge, Karlstadt, Vol. II, pp. 369 sqq., treats extensively of this
incident and the sequel.}

After residing at various places, Karlstadt betook himself to
Kemberg, where he labored as a peasant and kept a small shop. Luther
published several new tracts against this backslider; whereupon he
evaded arrest in October, 1529, by fleeing from the Electorate of
Saxony to Holstein, where he joined the Anabaptist Melchior Hoffmann,
with whom he went to East Frisia. This vacillating man is
next found in Strasburg, then in Zurich with Zwingli, and finally
in Basle, where he joined the Zwinglians as a teacher in the theological
faculty, though still persevering in his peculiar opinions, and completely
at outs with Luther.
