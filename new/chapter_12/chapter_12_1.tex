\section{Charles v. Clement VI, and the Turks}

The first war between Charles V and Francis I of France ended in
the defeat of the French at Pavia, on February 24, 1525, and the capture
of their king. The imperial army, composed of Spaniards, Italians,
and the dreaded German lansquenets, won a decisive victory. The
treaty of peace concluded at Madrid between Charles and Francis on
January 14, 1526, was entirely too severe for France. The release of
the King was purchased at an exorbitant price.

Pope Clement VII, an astute politician, was of the opinion that the
treaty and the oath of King Francis were not binding because they
had been obtained by force. It has been frequently asserted that he
formally released the King from his oath; but the statement is uncertain.\footnote
{Pastor, \textit{Geschichte der Päpste}, Vol. IV, Part II, p. 208.}
Nevertheless the Pope, fearing the ascendancy of the Emperor
in Italy, and apprehensive of his own position in Rome, unfortunately
shaped his policies to favor Francis. This proved fatal to the
status of the Church in Germany. The action of the Emperor and the
Empire against the religious upheaval was paralyzed by the demands
made upon the latter in the war-like complications which had arisen,
especially in Italy. The so-called Holy League of Cognac, which had
been formed in opposition to the Emperor between certain Italian
States and France, strengthened by the accession of the Pope, led to
a profound schism between the supreme spiritual and the first temporal
authority in Christendom.

In the new conflict between the Franco-Italian and the imperial
forces, which lasted from 1526 to the “Ladies’ Peace” of Cambrai
(1529), Rome was stormed and fearfully sacked in 1527 by the mutinous
soldiers under Bourbon, the imperial field-marshal, and George
von Frundsberg, the commander of the ``Landsknechte.'' The capital
of Christendom, degraded by the morals of the Renaissance, suffered
a most abject decline. Clement VII, made a prisoner in the Castle of
Sant’ Angelo, was compelled to sign a humiliating capitulation and
lived in exile at Orvieto and Viterbo until he was able to return to
Rome, on October 6, 1528. The Emperor was exceedingly alarmed
at the capture and humiliation of Rome, which were contrary to his
intentions, and was told by Francesco Quiñiones, the intrepid General
of the Franciscan Order, that if he did not hasten to fulfill his obligations
toward the Pope, it would be impossible to call him emperor
henceforth; men would prefer to regard him as Luther’s captain,
since the Lutheran mercenaries had committed the most disgraceful
atrocities in his name and under his banner.\footnote{\textit{Ibid.}, p. 311.}

Luther, on his part, was jubilant at the course of events. When
apprised of the misfortune that befell the Eternal City, he wrote to
his confidant, pastor Hausmann of Zwickau: “Rome and the Pope
have been miserably devastated. Thus Christ governs, since the Emperor,
who, being in the service of the Pope, persecutes Luther, is
compelled to destroy the Pope on behalf of Luther. Thus Christ must
do everything for the sake of His own and against the enemy.”\footnote{On July 13, 1527; \textit{Briefwechsel}, VI, p. 69.}

In another letter, however, he says: “I would not like to see Rome
burned; for it would be too prodigious a sign.”\footnote{On November 11, 1527 to Jonas; \textit{Briefwechsel}, VI, p. 117.}

Melanchthon, the
humanist, was disturbed rather at the destruction of the ancient
classical sites.\footnote{Cf. \textit{Corp. Ref., Vol. I}, n. 445; XI, p. 130.}


In consequence of the conciliatory attitude of Charles V and the
overtures made by the Pope to the Emperor, a reconciliation was effected
between them. Charles V was crowned emperor at Bologna
by Clement VII on February 24, 1530. It was bruited about, however,
that he did not intend to allow the Papal States to attain to complete
sovereignty and independence.\footnote{Pastor, \textit{l.c.}, pp. 382 sq.}

Germany was a source of grievous injury to Catholicism as against
the reform movement. After the Emperor’s energetic stand at Worms
his failure to intervene in Germany was regretted. The loyal adherents
of the Church loudly clamored for his return. But they were disappointed
from year to year. In 1530, when the diet assembled at
Augsburg, the Emperor returned temporarily to the field of German
activity, which was very much desired by himself, as the country was
in a sorry religious plight. The complaint was heard that neither the
Emperor nor the Pope was properly informed about the condition of
the Church in Germany.

Besides the absence of the Emperor and the disturbances in Italy,
the events on the eastern.boundary of the Empire proved of great advantage
to Lutheranism. These events completely engaged the attention and the
strength of the imperial regent, Ferdinand of Austria.
The approaching danger of a Turkish invasion diverted the thoughts
of the princes who remained loyal to the Empire, from the religious
question.

During his captivity, the French king had appealed to Sultan
Soliman for aid. Since his victory over the Knights of St. John on the
Island of Rhodes, Soliman was consumed with a strong desire to resume
the ancient campaign of the Crescent against the West. He invaded Hungary
with an immense army and defeated King Louis, a
brother-in-law of Charles V, at Mohacz on the Danube. The King
succumbed in a morass on his flight (1526). His crown, together
with that of Bohemia, passed over to Ferdinand of Austria. The danger
to Germany remained, yea, became even more aggravated, since
Ferdinand’s rival in Hungary, John Zapolya of Transylvania, favored
the Turks. In order to protect Zapolya, Soliman renewed his attack
and besieged Vienna (1529), but was repulsed.

Luther for a long time maintained his unfavorable attitude
towards united action against the Turks, but finally perceived its
necessity.

The cause of his well-nigh inexplicable attitude of aloofness was
the prominent participation of the papacy in the Turkish war. By
virtue of its primacy, its ancient activities at the head of the Christian
family of nations, and its traditional efforts to check the expansion
of infidelity, the papacy was the natural leader in this movement.
Luther’s pseudo-mystical state of mind originally inclined him to regard
the Turks as a scourge of God which neither could nor ought to
be resisted, and to expect that that portion of Christendom which
suffered from this scourge would accept his gospel.\footnote{Cfr. his letter to Spalatin, December 21, 1518 (\textit{Briefwechsel}, I, p. 333).}
One of his
theses, which was formulated in opposition to the Turkish wars, was
condemned among other errors in the Bull “\textit{Exurge}” of Leo X in
1520.\footnote{Cfr. Grisar, Luther, Vol. 1II, p. 78.}
Luther naturally sustained this theses with all the more energy.
He accused the Pope of selfish and imperialistic designs because of
his demand for a crusade against the Turks.\footnote{\textit{Ibid.}, p. 79.}
In an impassioned
treatise, “Two Discordant Imperial Commandments” (1524), he
wrote: “We refuse to obey and to march against the Turks or to
contribute to this cause, since the Turks are ten times cleverer and
more devout than our princes.” Because the Catholic princes had rejected
his demands at the diet of Nuremberg, he delivered himself
thus in opposition to their resolution in favor of the crusade: “How
can such fools [the princes], who tempt and blaspheme God so
greatly, expect to be successful against the Turks?” The Emperor--
“a perishable bag of worms”--shamelessly constitutes himself, together
with the Pope, the supreme defender of the Christian religion,”
whereas “the divine power of the faith has no need of a protector.”\footnote{\textit{Ibid.}, p. 79.}

He delights in repeating his assurance that “the government of the
Pope is ten times worse than that of the Turk \dots If ever the Turks
were to be exterminated, it would be necessary to begin with the
Pope.”\footnote{\textit{Ibid.}, p. 79.}

Thus he availed himself of the extreme need of Christendom to
agitate against Rome and to promote the interests of his own cause.

When, finally, in 1529, Vienna was threatened and cries of alarm
rang through Europe, he changed his tone. In his little book “On the
War against the Turks” he now demanded protection against the
Turks and asked that they be proceeded against as robbers and destroyers;
but there was to be no crusade such as had been undertaken
against the infidels in the foolish days of old.\footnote{\textit{Ibid.}, p. 79.}

The edict of Worms had been renewed at Augsburg, where it was
further resolved that arrangements be made for a “free general council”
to be assembled at some accessible place in Germany. The diet
of Spires, in 1526, was forced to wage an even greater battle with
the partisans of the new religion who had increased their forces in
the interim. Its efforts were but partially successful. The diet, it is
true, received a declaration from the Emperor attesting his firm
resolve to act. He left no doubt that he meant to uphold the edict
of Worms and its demands upon the cities and princes. Nevertheless
Electoral Saxony and Hesse boldly led the other friends of Lutheranism
in their resistance. If the edict of Worms were upheld, the
partisans of the reform movement now threatened to refuse their
assistance in the war against the Turks and the necessary contributions
for the support of the imperial government. This placed
Ferdinand of Austria and the Catholic party in a quandary. Finally
the regulation of affairs was once more deferred until the convocation
of a general council, which was definitely expected. The recess
stated in rather doubtful language that, pursuant to the edict of
Worms, the estates had “unanimously agreed, in matters pertaining
thereto, so to live with their subjects, to govern, and to conduct
themselves, as each expects and trusts to be held accountable for to
God and his imperial majesty.”

Was this a legal recognition of the new system of territorial
churches?

It has been so affirmed, but without any warrant. What is true is
that the elastic statement was thus interpreted at an early date.

The declaration, however, “does not imply what was inferred from it; the
right of reformation could be deduced neither from its wording, nor from
its origin, nor from its spirit.”\footnote{A. Kluckhohn in the \textit{Histor. Zeitschrift}, Vol. LVI, p. 217.}
 “One can scarcely say that this formula
granted a formal right to the evangelicals to secede from the ecclesiastical
communion and to institute a reformation on their own responsibility.”\footnote{W. Friedensburg, \textit{Der Reichstag zu Speier 1526}, p. 482. Cfr. Janssen-Pastor, Vol, III,
pp. 53 sq.; also Köstlin-Kawerau, II, p. 27.}

The wish of the Emperor, referred to in the formula, was quite clear. Then,
too, the diet of Spires only intended to promulgate a temporary norm of peace
until the assembly of a general council. True, the edict of Worms was not
formally renewed, but the German estates could hold themselves “accountable
to the imperial majesty” only for conduct in conformity with that edict.
Luther himself during the first three years never interpreted the proposition
in question as a legal foundation justifying the formation of national
churches. It must be admitted, however, that the expression was ambiguous,
and can be explained only as a sorry expedient in the situation created by
the opposition. It is more excusable in the light of the diet’s concomitant
appeal to a general council; for this appeal the diet presupposed, “not the
dissolution, but rather the acknowledgment of ecclesiastical jurisdiction.”\footnote{Janssen-Pastor, \textit{l.c.}}

The misuse to which the recess of the diet of Spires was subjected,
promoted the development of territorial churches of the new religion
which were in process of formation. A fine opportunity was furnished
by the delay in the execution of the edict of Worms. Philip
of Hesse had a synod held at Homberg in 1526, under the presidency
of a Frenchman, the apostate Franciscan, Francis Lambert of Avignon.
This body drafted a radically new ecclesiastical regimen, based upon a
purely synodical constitution. Mainly on account of Luther’s opposition,
however, this regimen was never enforced. On the contrary, the
landgrave himself assumed the government of the Church and ruled
it as supreme territorial bishop. The monasteries were suppressed,
sacred images in the churches and shrines abolished, and the ritual
ruthlessly altered.

Philip’s personal interest in religion was so feeble that, after his
change of religion, he partook of the Eucharist but once in fifteen
years and lived persistently in adultery and public vice. According
to his own confession he did not observe conjugal fidelity towards
his wife Christina for even three weeks. As early as 1526 he harbored
the idea of taking a second wife during the lifetime of his
first spouse--a design which he executed on March 4, 1540, with
the sanction of the Wittenberg reformers, as will be set forth in the
sequel.
