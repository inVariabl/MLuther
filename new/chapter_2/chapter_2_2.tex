\section{Studies and First Experience as Teacher}

After his first Mass, Martin began the study of theology. John
Nathin, a celebrated teacher of his Order, was prefect of theological
studies in the Erfurt monastery. But Martin was not introduced
into positive and speculative theology in an orderly fashion. After
about a year and a half spent in the study of Gabriel Biel’s treatise
on the “Sentences” and other Nominalistic writers, his superiors
in the autumn of 1508 transferred him to the Augustinian monastery
of Wittenberg, where he was ordered to lecture at once on the Nicomachean
Ethics while he continued his theological studies.

At Wittenberg Luther came into closer personal contact with
Staupitz, the vicar of his Order, with whom he probably had had
converse already at Erfurt. At his instigation, he was promoted to
the baccalaureate in Sacred Scripture at the university, on March 9,
Luther had made the reading of the Bible his specialty; it appealed
to him more than Scholasticism and methodical scientific
works. For this reason, and because of his talents, Staupitz kept an
eye on him, in order eventually to make him his successor in the
academic chair of Biblical science, which had been entrusted to the
Augustinians, but to which he could not do full justice on account
of his official journeys. The Biblical baccalaureate was a preliminary
step for Luther. It obliged him to explain certain parts of Sacred Scripture
to his academic audiences. Later he was appointed to the
office of “\textit{sententiarius},” which entitled him to deliver university
lectures on the celebrated Book of Sentences of Peter Lombard.
He advanced to this dignity in the autumn of 1509. It terminated
his first sojourn at Wittenberg. The Order sent him back to Erfurt,
where a \textit{sententiarius} was needed. Due to these labors, his own
further training must have been considerably neglected.

Luther’s interior life during the days spent at Erfurt and Wittenberg
was constantly furrowed by deep anxiety. The terrors from
which he suffered at the beginning of his monastic life, would not
desert him. He continued to worry about the sternness of the divine
Judge, the remission of his sins, and the problem of predestination
by an alleged and inscrutable divine decree. In part these terrors
were caused by his psychological condition, which, when later symptoms
are taken into consideration, seems to have been a kind of precordial
fear; in part, they were the product of melancholy thoughts
which he harbored and which reacted upon his physical condition.
On one occasion, while attending divine service in the choir of the
monks, he fell prostrate to the floor and was racked by convulsions,
as the Gospel of the demoniac was being sung, and screamed aloud:
“It is not I! It is not I!” (meaning that he was not the man possessed).\footnote{Grisar, \textit{Luther}, I, p. 17.}
No mention of epilepsy is made in his subsequent history.
The many later references made by him to his mental sufferings during
this period, lack precision. They may pertain to his sojourn at
Wittenberg or to his first residence at Erfurt, or to both.

He says that his life in the cloister was always sad. When he
discussed his sins with Staupitz and raised all kinds of imaginary objections,
the latter told him to dismiss the specter of his “puppet
sins.” His thoughts, which were replete with fantasies, appeared
unintelligible to others; few knew how to console him as well as
Master Bartholomew (Usingen), whom he styled the “best paraclete
and comforter” in the Erfurt monastery.\footnote{\textit{Ibid.}, 1, p. 10,}
Once Staupitz told him:
“Master Martin, I do not understand that.” On another occasion
Luther was deeply impressed as Staupitz admonished him when
he was affrightened at the idea of predestination: “Why torment
yourself with such thoughts and broodings? Look at the wounds of
Christ and His blood shed for you! There you will see your predestination
to Heaven shining forth to your comfort.”\footnote{\textit{Ibid.}, p. 11.}
Yet, in many
passages of his later writings and addresses Luther says of his monastic
life: “My heart trembles and flutters, when I meditate on how
God may be merciful to me. Often have I been frightened at the
name of Jesus, and, when I looked upon Him as He hung upon the
cross, He was as lightning to me.” He was often compelled to say:,
``I wish there were no God.'' Never, so he says with exaggeration
had he been able to recite a prayer properly. He had lived in great
tortures and at times so sensed the terrors of God’s judgment “that
his hair stood on end.” He became startled when death or the future
life was discussed in the monastery.\footnote{\textit{Ibid.}, III, p. 109.}
According to his representations,
it was principally his good friend Staupitz who prevented him
from being “drowned,” as he puts it, in the fear of predestination.
But, are not many polemical admixtures recognizable in these
portrayals of his depressed and melancholy state of mind in
the monastery, from which he alleged he was forced to flee?
When Luther left Wittenberg, he had neglected to deliver the
necessary introductory lecture as \textit{sententiarius}. As a consequence
the scrupulous theologians of Erfurt did not want to let him lecture
on the Books of Sentences there; they may also have been unfavorably
disposed toward him for other reasons. However, in the end he
was permitted to lecture.

The notes on Peter Lombard which he penned in those days reveal
an active mind, but at the same time an adverse and extremely surprising
self-conscious mannerism of formulating judgments. He
sneers at the drolleries of contemporary theologians, at “the rotten
rules of the logicians,” at the masks worn by the “philosophers,”
at “the rancid philosopher Aristotle.” For the latter he showed a decided
aversion. At the very beginning of his career he styles him a
comedian whom he would unmask. The Middle Ages had appreciated
Aristotle quite differently. But Luther showed a contentious and an
audacious spirit already at Wittenberg. Mathesius, his eulogist, says
of him: “Our Frater Martinus there applied himself to the study of
Sacred Scripture, and commenced to disputed in the university against
the sophistry which was everywhere in vogue at that time. And since
all schools, monasteries, and pulpits appealed to the ‘master of
sublime thought’ (Peter Lombard), besides Thomas of Aquin, Scotus,
and Albertus, in support of the foundation of Christianity, our
Frater Martinus began to dispute against their principles, at which
good people were highly amazed even at that time.”

The Erfurt professors were probably conspicuous among the “good
people” who opposed Luther. It is not incredible that, as he relates
afterwards, the Bible, which served the fiery combatant as a means
for his boastfulness, may have been withheld from him for a while.
In order to understand his beloved Bible properly, Luther began to
study Greek under the direction of John Lang, a fellow-member
of his Order, who had a humanistic training and shared his opinions,
Luther’s studious spirit also impelled him to take up certain writings
of St. Augustine, the founder of his Order and a doctor of the
Church. We have marginal notes made by him in 1509 on certain
treatises of Augustine. But owing to his lack of leisure and his preconceived
notions he was not able to fathom their depth. Augustine’s
teachings on grace, free will, and justification, on natural good works
and acts meritorious for Heaven, really remained a sealed book for
him all his life. In vain he appealed to particular passages to support
his own peculiar opinions.

The town of Erfurt was hardly aware of Luther’s residence at the
highly esteemed Augustinian monastery. Luther himself is silent for
a long time concerning the storms and struggles which the town
experienced. It is only afterwards that he mentions Erfurt, and then
with a feeling of resentment. In January, 1510, the ancient city
council of Erfurt was violently deposed by a popular democratic
party. The Saxon Elector opposed both the insurgent workers and
the rights of the Archbishop of Mayence, who ruled the town. The
spacious “old college” in which the university lectures were delivered,
was destroyed on August 4, in a riot between the students and the
municipal Landsknecht. It was the “frantic year,” as it is called in the
annals of the city. During this uprising Luther lectured in perfect
peace in the quiet halls of the Augustinian monastery.

In this same year, 1510, a grave controversy broke out in the
Observantine Augustinian congregation. It was occasioned by the
vicar, John Staupitz, who jeopardized the canonical and disciplinary
autonomy of the Observantine monasteries entrusted to his care. He
intended to affiliate them with the monasteries of the German Augustinian
province, who were non-Observantines. The consolidation of
the province, which had hitherto been directed by separate provincials,
with the monasteries of his own jurisdiction would have greatly
extended his authority. He counted upon the support of the General
of the Order and increased vigor in the life of the German
monasteries, although no noticeable decline had been manifested by
them.

The monks of Erfurt and of six other monasteries of the Observantine
congregation judged otherwise. They considered their observance
jeopardized by the influence of the communities which had affiliated
with them and insisted upon the privileges of their congregation,
which was protected in virtue of papal legislation against the arbitrary
interference of the General. In the Franciscan Order, Brother
Louis of Anhalt, whom Luther met at Magdeburg, had effectually
defended the Observantine monasteries of St. Francis in Germany,
whose constitutions enjoyed papal sanction, in the interests of the
stricter life, against Aegidius Delfini, the General of his Order.\footnote
{Cf. Lemmens, \textit{Franziskanerbriefe}, pp. 20 sqq.}
In this posture of affairs, Luther assumed the role of eloquent spokesman
against Staupitz and, in behalf of the insurgent monasteries,
was sent to Halle in company with the theologian John Nathin,
where Adolph von Anhalt, the provost of Magdeburg cathedral, sojourned.
Both appealed for assistance to the provost. In order to
assure themselves of success, the monasteries decided to send Luther
to the headquarters of the order at Rome and to the papal curia.
This was the occasion of Luther’s journey to Rome, an event destined
to become highly significant in his life.
