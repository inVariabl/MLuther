\section{In Wittenberg}

The Augustinians of Wittenberg did not participate in the attack
upon Staupitz in connection with the Observantine controversy.
The party of the vicar was in control there. What attitude did Luther
assume towards him? When he re-entered Germany, his views about
this internal question of discipline were different from those which
impelled him to visit Italy. He became an opponent of the so-called
Observantines and espoused the party of Staupitz. What caused this
striking change has never been fully cleared up. Perhaps the opposition
which he encountered at Rome influenced him. Possibly his
transfer from Erfurt to Wittenberg had something to do with his
altered attitude. It is also possible that Staupitz himself influenced
him decisively. Cochlaeus, his subsequent opponent, who at that time
was in touch with the brethren of the Order and had learned from
them some things about Luther, drastically expresses the change in
his conduct thus: “He has apostatized to his friend Staupitz.” For
the rest, his change of attitude need not cause too much surprise in
view of the sanguine temperament of the young monk. It is also
permitted to inquire whether the consolidation contemplated by
Staupitz, did not possess some merit. A uniform government of all
the Augustinian monasteries in Germany under an energetic general
and active provincials, according to the general rule of the order,
was in itself a rather desirable thing.

The internal conflict was settled in May of the following year at
a chapter of the Augustinian congregations held in Cologne. The
settlement was effected as a result of the conciliatory policy of Staupitz,
who had previously brought about a certain union of the seven
convents at Jena, in July, 1511. The proposed consolidation of these
congregations with the Saxon province, \textit{i.e.}, with the non-reformed
German Augustinians, was to be abandoned--a proposition with
which the general now agreed. The Cologne chapter was held without
the participation of the “Province of Saxony.” This fact alone
would indicate a certain retreat on the part of Staupitz, even if it
was but a temporary one. In the meantime the opposition within
the congregation, once having manifested itself, continued to smoulder.
There were friends of the Observance, and, as it appears, some
enthusiasts, who exhibited a strict compliance with the statutes. On
the other hand there were enemies of the Observance, who complained
of unkindnesses and calumnies on the part of their opponents. In the
congested atmosphere of the monasteries the conflict grew more and
more acute.

At Wittenberg, Luther soon became the passionate spokesman of
the opponents of the Observantines, who were by far the more
numerous party. He had participated in the chapter at Cologne
(1512), as his works testify. On the return journey from Cologne,
he visited the valley of Ehrenbreitstein near Koblenz, where a monastery
of his Order was situated at Mühltal.\footnote{
    H. Grisar, \textit{``Luther zu Koln und Koblenz,''} in the
    jubilee number of the \textit{Koblenzer Volkszeitung}, February, 1922.}
Paltz, a celebrated
Augustinian and a native of Erfurt, had retired to this monastery
some time previously, having become dissatisfied with his position
as theological teacher at the “studium generale” of his Order at Erfurt.
It is possible that Luther, while at Cologne, had been proposed
for the doctorate in theology at Wittenberg. According to his later
story he raised strenuous objections to the doctorate when Staupitz
subsequently discussed this matter with him in more definite terms
at Wittenberg. His objections were based principally on the state
of his health. In spite of this, however, the superiors would not alter
their decision.

In Wittenberg various tasks diverted him from the preparation for
the doctorate. Thus he was obliged to preach in the smaller monastery
church. He was also made subprior in the monastery. On
October 4th, he obtained the academic title of licentiate in theology.
A few days later, on October 9th, we find him at Leipsic, where
he writes out a receipt for fifty guldens, which the Saxon Elector,
Frederick, had assigned to him out of the local exchequer to pay
the expenses of his pending promotion.\footnote{\textit{Briefwechsel}, I, p. 9.}
Staupitz declared to the
Elector that the office of Biblical lecturer, which he himself had occupied
at Wittenberg, was henceforth to be entrusted permanently
to Luther.\footnote{Scheel, II, pp. 311 and 431.}
According to the terms of their endowment these lectures
were assigned to the Augustinian monastery. After passing the
required examination, Luther was promoted to the doctorate, on
October 19, 1512, in the castle-church at Wittenberg. The ceremony
was held under the direction of the university professor Andrew
Bodenstein of Karlstadt, with whom Luther in after years lived in
strained relations on account of the controversies which arose over
the new doctrines.
