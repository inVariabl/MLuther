\section{The Journey to Rome}

After having spent a considerable part of his life in the narrow
confines of the monastery and in academic halls, the journey to Rome
was bound to bring the active mind of young Luther in contact
for the first time with the great ecclesiastical world. He was to receive
an abundance of new ideas. He was destined also to become
aware of the religious and moral abuses which had been accumulating
on all sides, but particularly in the Rome of the Renaissance. In the
late autumn of 1510 he commenced his journey on foot, in accordance
with a custom of his Order, accompanied by a fellow member.
They traveled through Bavaria and over the mountains to Tyrol,
thence, from Innsbruck over the Brenner Pass to Lombardy and
beyond. Following the usual road of the pilgrims, they crossed Central
Italy and proceeded via Viterbo and Lake Bolsena to the Eternal
City. Whenever practicable, he called upon the numerous monasteries
along the road and enjoyed their hospitality. The hardships of the
winter season, just previous to the close of the year, were probably
not small. With reinvigorated energy he ascended the heights of
Mario, whence he obtained his first sight of Rome. Near a chapel
he knelt down and recited the customary prayers of pilgrims, in
greeting the sacred walls and plains, the home of innumerable saints
and martyrs. In Rome he took up his abode with his fellow-monks.

To his great disappointment, his mission in behalf of the Observantine
faction proved futile. He was advised that he would have to
obtain a letter from the legitimate superior of his order in Germany
(\textit{i.e.}, Staupitz) in order to gain admission to the papal curia. He
had no such a letter, and the General of the Augustinians as well
as his procurator, who were not in favor of his mission, refused to
intervene in his behalf. The efforts of the seven monasteries were
unwelcome to them. Again, he must have poignantly felt another
failure, namely, the refusal of the papal authorities to grant a petition
which his passionate fondness for study had inspired. In some cases,
religious had been given permission to devote certain years to study
at the universities, outside of their monasteries and without appearing
in the habit of their order. Luther’s request to have this extraordinary
privilege extended to the German Augustinians was declined
because he had no recommendation of his superiors. The report of his
pupil Oldecop on this subject is trustworthy, since during his stay at
Rome, some time after this incident, Oldecop made inquiries concerning
this matter. Luther compensated himself by studying Hebrew
with a German Jew at Rome. He also made it his business to visit
all the sacred places in Rome, and to become acquainted with the
religious monuments of the city. He hunted up, he says, all the
churches and crypts. The traditions of the various places edified
him; he did not balk even at the false ones. Only at the so-called
Stairs of Pilate, in the vicinity of the Lateran palace, which he
climbed on his knees in accordance with custom, the question arose
in his mind whether the tremendous indulgences connected with
these steps were indeed genuine.\footnote{Grisar, \textit{Luther}, VI, 496.}
This doubt, however, was not the
germ of his subsequent doctrine of justification without good works,
as has been asserted, but was occasioned by the uncritical \textit{Mirabilia
Vrbis Romae}--a guidebook for pilgrims which was in circulation at
that time. No trace of so-called reformatory ideas can be detected in
Luther either at the time of his pilgrimage to Rome or for some
considerable time thereafter.

He was, however, deeply depressed by what he saw of the decline
of morality in Rome, including the higher and the lower clergy.
Especially what he heard concerning the person and court of the
recently deceased Borgia Pope, Alexander VI, his relatives, certain
cardinals, the pomp and worldliness of Julius II (Giuliano della
Rovere), the then reigning warrior-like successor of Alexander VI,
sank deep into the soul of the receptive northerner. These recollections
were violently revived during his subsequent contest with
Rome and furnished him with weapons against the Roman Anti-Christ,
whose true character he fancied to have discovered in another
manner. He appears, while at Rome, to have come in contact with
German and Italian residents who collected reproaches against the
morals of the curia in an odious and at times frivolously exaggerated
form and apparently took less note of the prevailing good traits
in the life of the city and the supreme government of the Church.
The same holds good in regard to his entire journey through Italy.
An honorable exception were the great hospitals he visited, with
their ample equipment and the charity which they dispensed. The
exemplary care of the sick and of poor pilgrims exhibited at Florence
elicited favorable comments from him later on.

It appears that the splendid edifices and the grand works of this
period, inspired by the joyousness of creative art, at Rome as well as
along the whole way of his journey, failed to attract his attention.
Even in his advanced years he relished no taste for the creations of art.
As a pilgrim to Rome, he lacked the proper enlightenment to appreciate
these matters.

When Luther, after his apostasy, described himself as having been
the most pious monk at Rome, who said Mass so solemnly and slowly
that several other priests finished saying Mass at the same time, and
when he maintains that, inspired by the great Roman indulgences
applicable to the souls of the departed, he in his pious zeal wished
that his parents had already departed this life, so that he might gain
these indulgences in their behalf--we have a series of grotesque exaggerations,
suggested partly by his native humor, partly by exaggerated
criticism of Roman conditions. We know that he did not say
Mass regularly while in Rome. According to a later declaration
of his, he desired to make a general confession extending over
his whole life, but found the clergy in Rome insufficiently instructed
for this purpose. Whatever he says about conditions or his own
monastic virtues must be received with a large grain of salt.

At all events, it is certain that his visit in the center of Catholic
Christianity did not shake his devotion to the Church, nor his submission
to the papal authority, nor his loyalty to the monastic state,
though the subsequent crisis was accelerated thereby.

His stay in the city on the Tiber lasted about four weeks. Luther
did not get to see Pope Julius, who, on account of threatening war,
had betaken himself to Upper Italy. Luther did not return to Germany
by way of Lombardy and across the Tyrolean Alps, but, due
to the danger of war, made a detour over Nice and the Avignon
country, up the valley of the Rhone towards Switzerland and thence
to Bavaria. Some traces of this journey have been preserved.\footnote{
H, Grisar, \textit{Lutheranalekten}, I (``Zu Luthers Romfahrt; Neues über den Reiseweg'')
in the \textit{Histor. Jahrbuch der Görresgesellschaft}, Vol. XXXIX (1919--1920), pp. 487 sqq.}
Nor did the pilgrim return to Erfurt, but, in compliance with the orders
of his superiors, went to Wittenberg to teach, a choice which probably
conformed with his own desire.
