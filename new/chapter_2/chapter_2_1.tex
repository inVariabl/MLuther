\section{Novitiate, Profession, First Mass}
The completion of a year’s novitiate was the first obligation incumbent
on the new monk. During this probationary period he was
not permitted to study. Prayer, pious reading, labor and penances,
service in the choir, and mastering the rules and life of the Order
occupied his time. For this purpose the novices were assigned to the
direction of an elder monk. Luther was placed under an experienced
novice-master whom he praised in later years
as a wise and sympathetic religious. The master of novices explained
to him the statutes of the Order, which John Staupitz, at that time
superior of the entire congregation, had composed in 1504
on the basis of the old constitutions, adapted with wise discretion to
the needs of the age. They were detailed and precise, but tolerated many
dispensations in the monastic observance. The master of novices also saw to it that
young novice entrusted to his care diligently read the Bible.

The statutes enjoined upon all the duty of
“reading the Bible with fervor, to hear it read with devotion, and to learn it with assiduity.”
To hold that the Sacred Scriptures were not in the hands of the faithful, even of the pious, in the days
when Luther was a youth, is a wide-spread error. In the case of Luther himself, who afterwards
rendered this statement current, there was not a day “on which the Word of the Scriptures was not
perceived abundantly by ear and intellect. It came to be a permanent companion,
a monitor and comforter, a judge and a benefactor.”\footnote{Scheel, \textit{M. Luther}, 11, 2 ed., p. 2.}
From that day a pronounced inclination towards the Bible began to take hold of him.

Besides the Bible, the young novice joyfully saturated his mind
with the writings of St. Bernard
and St. Bonaventure, those profound and sympathetic teachers of
the Middle Ages. His spiritual director understood how to comfort and guide the novice, who at
times showed a lack of courage. Long afterwards Luther remembered
how the good man spoke to him of the remission of sins and occasionally
called his attention to the fact that the article of faith on the
\textit{remissio peccatorum} must be set up against all scruples.

Brother Martin willingly adapted himself to the discipline of the
well-regulated monastery. He learned to love his new abode, was
determined to become a good monk, and resolved to devote his
energies to the salvation of souls.

At the commencement of his novitiate, he received sad reports
from Mansfeld, where his family resided. The pestilence invaded the
little town and carried off two of his brothers. His heart, on the
other hand, was gladdened at the news of his father’s belated consent
to his entrance upon a monastic career. It was difficult for the father to relinquish the hopes he had placed in a secular career for his
highly promising son. Perhaps the affliction with which his home had
been visited moderated his attitude.

When the probationary year had terminated and the hour for the
taking of the solemn vows had arrived, Brother Martin, advancing
to the altar of the Augustinian church at Erfurt, unperturbed by
misgivings and with joy in his heart, made profession of the vows
that bind forever. The vows were couched in the usual form:
``I, Brother Martin, make profession and vow obedience before Almighty God and the ever Blessed Virgin Mary, and before you,
Brother Winand, local prior, in the name and place of the prior general of the Order of Hermits of the holy Bishop Augustine, and his
lawful successors, to live without property and in chastity according
to the rule of the same Blessed Augustine to the end of my life.''
No act of his life, no promise ever made by him, took place after such
mature deliberation and with such a complete knowledge of the circumstances and obligations,
as this oblation of himself to the divine
Majesty in the bosom of the universal Church by means of the triple
vow of poverty, chastity, and obedience. The act was witnessed by
the community of those who were to be henceforth more closely
united to him.

Luther was now a full-fledged member of the German Congregation of the Hermits of St. Augustine,
who, as a Congregation of Observantines, were subordinated to John Staupitz, the vicar or
representative of the general of the Order who resided in Rome.
Besides the monasteries of the congregation of the Observantines, there
existed in Germany numerous other Augustinian monasteries which
had not introduced the Observance. They constituted the so-called
Provincia Saxoniae, and extended over central and upper Germany.
According to the general administration of the Order, they were
under the jurisdiction of a provincial. Both, Observantines and non-Observantines,
were classified under the common canonical character of mendicant friars,
with this sole exception that the Observantines had their own peculiar exercises which were conducted in the
spirit of the enthusiastic founder and father of their Order, Andrew
Proles, the predecessor of Staupitz (died in 1503 at Kulmbach).

Luther was not exempt from the task of begging alms. Despite
the fact that he possessed the master’s degree, he was obliged to assume
this task like other humiliating exercises of the Order. Some
years afterwards, however, in view of his academic degree and of
the studies he was destined to pursue, he was absolved from the obligation of the
“\textit{saccum per naccum},” as begging with a sack about the
neck was humorously termed in the monastery.

As the day of his ordination to the priesthood was approaching,
Luther read the thoughtful and edifying treatise on the holy sacrifice
of the Mass by Gabriel Biel--but as he assures us after his defection from the ancient Church, he did so with a bleeding heart.
His disposition inclined him to view with terror the thought of the
sublimity of the sacred function no less than the idea of intimate
union with God through the sacrifice of the body and blood of
Christ. In the second semester of 1506, the preliminary orders of the
subdiaconate and diaconate were conferred upon him. These events
were followed by his ordination to the priesthood, probably on April 3, 1507.
He received holy orders in the magnificent dome of Erfurt,
at the hands of the suffragan bishop, John Bonemilch von Lasphe.
His soul now highly stimulated, he prepared himself for the celebration
of his first holy Mass. The extant letters in which he extended an
invitation to his various acquaintances to be present on the greatest
day of his life--for it was celebrated with great solemnity--reveal
his profound earnestness and lively realization of his own unworthiness.
The style of these letters is invested with a certain pathos, be it
in consequence of the humanism he formerly cultivation of his natural disposition.

While he said his first Mass at the altar of the Augustinian church,
the thought of the proximity and magnitude of almighty God caused
him to be seized with such fright that he would fain have interrupted
the holy Sacrifice and hastened away from the altar, had not
the assistant priest held him back. The reports which have come to
us from his own lips, as well as those contained in the copy of his
lectures on Genesis, are too definite as to permit the possibility of a
doubt concerning the abnormal event.\footnote{
See the citations from the sources in Scheel II, pp. 345 sq. I am unable to regard as
sound the objections variously raised against Luther’s account in the \textit{Table Talks} and the
statements contained in his commentary on Genesis. Cf. Grisar, \textit{Luther}, I, p. 15; VI, pp.
99 sqq., 195 sqq.
}
Afterwards he said that he
always said Mass with a shudder, aye, “with great horror.”\footnote{3 Cf. Grisar, \textit{op cit.}, VI, 197.}
His father, accompanied by no less than twenty riders, arrived for
the celebration on horseback, defraying his own expenses. At the
festive banquet, Martin desired to persuade his father to give a
new and open approval to his entrance into the monastery, since
his previous consent had not been whole-hearted. Therefore, Martin
praised the “pleasant and quiet” life of the monastery and the “divine
nature” of his chosen state of life. But when he mentioned the heavenly
call on the occasion of the storm at Stotternheim, his father
became angry and exclaimed: “Would to God that it was not a
hallucination of the devil!” He was a choleric man and his patience
was exhausted. The select company which surrounded him did not
restrain him from giving vent to his displeasure. He even remarked,
though without justification, whether the son had forgotten that
children owe complete obedience to their parents as regards entrance
into the cloister, that the fourth commandment was above the notion
which induced him to select the monastic state, etc. It must have
been an unpleasant scene when the monks, who were seated next to
them, tried to defend the monastic life and their promising confrère.
Thereat the father expressed himself in these acrid words: “I
would prefer to be somewhere else rather than to be here, eating
and drinking.” To such an extent his irascible temperament led him
to forget the requirements of the festive occasion. In course of time,
however, the old man became reconciled. When Luther, fourteen
years later, was in open conflict with his monastic order as a result
of the publication of his treatise against the monastic vow, he justified
the conduct of his father in the preface of the dedicatory letter
which he addressed to him by citing the latter’s statement relative
to the obedience due to parents.\footnote{\textit{Werke}, Weimar ed., VIII, pp. 573 sq.}
It seems never to have disturbed
him previously. But in the aforementioned treatise (1521) he
assured his readers that the words which his father uttered on the
occasion of his first Mass made a deep impression on him, “as if God
Himself had spoken them.”
