\section{The Schmalkaldic Articles. Repudiation of the Proposed General Council by the Schmalkaldians}

In view of the prevailing conditions, in particular the attitude of
France, the prospects of summoning an ecumenical council, which
had been the object of Vergerio’s negotiations with Luther, were
rather unfavorable. Nevertheless, Paul III, who was intent upon reform
and the defense of Catholicism, adhered to the project and
fixed the date of the council for May 23, 1537. Mantua was designated
as the city where the council was to assemble. For the purpose
of definitively winning over the Protestants to the idea of an ecumenical
council, the Pope ordered Van der Vorst to visit Germany
as his legate. The Emperor, who had intended to convoke a national
council, for a while contemplated making concessions to and peaceful
covenants with the estates who adhered to the new religion, in order
to gain their assistance in his campaign against the Turks. But when
it had been announced that the great convention of the Schmalkaldic
League was to be held on February 9, 1537, Charles V sent his councilor
Held to persuade the Protestants to participate in the council
which they had so often demanded. Vergerio, the papal nuncio, also
hastened to Schmalkalden.

Meanwhile the strength of the League had increased. The agreement
was renewed for a period of ten years. Accompanied by a large retinue,
the Protestant princes and the representatives of the cities entered the
small town of Schmalkalden. They were accompanied by a
large number of theologians--larger than any that had yet appeared
at a similar assembly. It was intended that the convention should be
not only extraordinarily solemn, but also decisive. Elector John Frederick
of Saxony, an enthusiastic follower of Luther, would have preferred if
the latter had issued a summons for a council of his own,
which, in his opinion, was to be a “general and free Christian council,”
to which Catholic representatives were to be invited. But Luther
and the theologians, as well as the jurists, persuaded him to abandon
this all too daring plan. They pointed out the great discord which
would probably manifest itself among the Protestants and might result
in profound schisms in their own ranks. Only Bucer and a few
others still adhered to the proposal of convoking a general Protestant
synod.

Within a few days after assembling, the Schmalkaldic League formulated
a declaration in which they decisively declined the invitation
to attend the council called by Paul III. They declared that the Pope
and his party did not intend to renounce their errors, that papists
were not competent to pass judgment on the new religion, and that
the selection of an Italian city as the seat of the council was objectionable.
They contended moreover that the political situation rendered
a general council impossible, and that the religious peace of
Nuremberg must be preserved and extended to all members who had
recently joined the League. The papal nuncio was treated with provocative
disrespect by John Frederick. The delegates commissioned
Melanchthon to prepare severe declarations against the papacy for
adoption by the assembly and its theologians. Luther, on his part, was
prepared to favor the acceptance of the papal invitation, but only
under conditions to which the papacy could not agree. He desired to
preserve the appearance of being conciliatory on account of the advantage
which would accrue to his party from this attitude.
Luther was still at Wittenberg when, in anticipation of the approaching
assembly of the League, he was compelled by an order of
the Elector to draft the so-called Schmalkaldic Articles.

The Elector desired, on the one hand, a clear and definite compilation of
the doctrines and practical requirements which were to be adhered to under
all circumstances in opposition to the Roman Church and, on the other, a
list of articles which were debatable. The summary was to be submitted to
the Saxon theologians for their signature, and then proposed at Schmalkalden.
At the head of the agenda, as outlined by Luther, were the sublime
articles on the Divine Majesty which the partisans of the papacy did not
dispute. These were followed by the articles which were rejected by the
Catholics as absolutely unacceptable. The first of these was the article which
set forth the Lutheran doctrine of justification by faith alone (\textit{sola fide}).
“It is not allowed cither to deviate from, or to surrender this article,” wrote
Luther, “even though heaven and earth should fall. Everything is founded
upon this article, which we teach and by which we live in defiance of the
pope, the devil, and the world.”

The second article was a condemnation of the Sacrifice of the Mass,
which was denounced as a “dragon’s tail” that has produced much
filth and vermin,” namely, Purgatory, pilgrimages, confraternities,
relics, indulgences, and invocation of the saints. These points are instanced
without methodical order and set forth with a torrent of invectives. A
third article, in similar language, demands the disestablishment of pious
foundations and monasteries and the repudiation of
the divine prerogatives of the papacy. The last section mentions the
debatable points concerning sin, the law, penance, the Sacraments,
and the marriage of priests. On these points there was to be no surrender,
but it was expected that the opponents might be forced to
make concessions concerning them.\footnote
{Weimar ed., Vol. L, pp. 192 sqq.; Erl. ed., Vol. XXV, 2 ed., pp. 163 sqq. Cfr. Grisar,
\textit{Luther}, Vol. III, p. 430; Vol. IV, pp. 525 sq.; Vol. VI, p. 310. Of the Sacrament of the
Altar it is asserted: “We hold that bread and wine in the Last Supper are the true body
and blood of Christ, and that they are communicated to and received not only by pious, but
also by wicked Christians,”}

Luther took these articles with him when, accompanied by Melanchthon and
Bugenhagen, he started out for Schmalkalden on January 31. The document
was the cause of much dissension among the
theologians. They quarreled particularly about the severity with
which Luther expressed his belief in the real presence of Christ in the
Eucharist. Ambrose Blaurer in particular declared, in opposition to
Amsdorf and Osiander who defended Luther, that the conciliatory
formula of the Wittenberg Concord had been violated in these articles.
Melanchthon, cautiously and with reserve as was his wont,
agreed with Blaurer. As other points also threatened to bring about a
rupture, and Luther himself was taken ill, Melanchthon succeeded,
through the mediation of the Landgrave of Hesse, in having Luther’s
Schmalkaldic Articles entirely withdrawn. No doubt the confused
document with its exaggerations and disputable points was repugnant
to the taste of this fastidious scholar, who insisted that the Augsburg
Confession and the Concord of 1536 constituted an adequate profession
of faith for the assembly of Schmalkalden. Melanchthon was now
summoned by the estates to come forward with a declaration against
the pope and the primacy of the Apostolic See. It was to be the final
breach of the German Protestants with the Church of Rome. Under
pressure of the highly exasperated delegates, and during the excitement
caused by the illness of Luther, Melanchthon’s fickle pen imparted a very
odious form to his tracts “On the Power and Primacy
of the Pope” and “On the Power and Jurisdiction of the Bishops.”\footnote{Grisar, \textit{Luther}, Vol. III, pp. 438 sq.}

There is no longer any recognition of episcopal jurisdiction, even of
the purely “human” jurisdiction which he had formerly proposed.

Thenceforth Luther’s spirit asserted itself more and more in Melanchthon
and produced a notable change of attitude in him toward
the Catholic Church. Thus when, prior to the Schmalkadic War, he
issued a new edition of Luther’s “Warning to his Dear Germans”
against the “papistical bloodhounds,” as they are styled in this work,
he accompanied it by a preface which contained unheard-of attacks
upon everything Catholic.

At Schmalkalden, his writings against the pope and the bishops
were subscribed to by thirty-two of the theologians and preachers
there present and accepted by the convention. When, at a later date,
the formulas of Concord were drawn up (in 1580) Melanchthon’s
above-mentioned tracts were incorporated among the “Symbolical
Books” of Lutheranism.\footnote{\textit{Ibid.}, pp. 440 sq.}
