\section{Luther’s Fellow-Combatants}

After 1530, the friends of Luther, particularly at Wittenberg,
made every effort to promote his cause.

Philip Melanchthon, while devoting his energies mainly to his
humanistic studies, at the same time actively intervened in the theological
controversies up to the time of his death. Because of the
success of his labors in behalf of the new creed, it can truthfully
be said that he “created Evangelical theology” and “established the
Protestant ecclesiastical system.”\footnote{Thus Gustav Krüger; see Grisar, \textit{Luther}, Vol. III. pp. 349 sq.}
 But it is equally true that in course
of time he changed his teaching in many points and deviated widely
from Luther. His \textit{Confessio Variata} shows a different complexion
from the original Augsburg Confession. Commencing with the
edition of 1535, his \textit{Loci Communes}, or “Outlines,” differ considerably
from the earlier editions. As early as 1532, his Commentary
on St. Paul’s Epistle to the Romans contained a different theology.
“He was no longer the interpreter of Luther’s ideas,” says Frederick
Loofs, one of the most respected Protestant historians of dogma.\footnote{Grisar, \textit{ibid.}}
How different is the attitude of the Catholic apologists with their
uniformly consistent doctrine! Even though liberty of action prevails
among them, and they differ amongst each other in making
unessential concessions, they occupy firm common ground in all
dogmatic questions.

Melanchthon at first disapproved of Luther’s denial of free-will
and abandoned the doctrine of unconditional predestination. Subsequently,
he also opposed Luther’s exaggerated estimate of the
doctrine of justification by faith alone and his low valuation of good
works. He gave a more tolerable form to his master’s views on
penance and fear as a motive of contrition. In later years he even
applied the epithet ``blasphemous'' to the principal thesis of Luther’s
chief work on the “Enslaved Will.” Relative to the doctrine of the
Lord’s Supper, a deep chasm separated Melanchthon from Luther,
who was always more inclined to favor Zwinglianism.\footnote
{On Melanchthon’s doctrinal deviations from Luther see Grisar, \textit{Luther}, Vol. III, pp.
346 sqq.; Vol. V, pp. 252 sqq.}
Luther
was aware of these differences of opinion in matters of doctrine,
but nevertheless adhered to Melanchthon; for he could and would
not dispense with his talents and reputation. Melanchthon on
his part carefully avoided whatever might have led to an open rupture.

No matter how far he was prepared to go in his attempts at reconciliation,
Melanchthon never denied his Protestant sympathies.
Because of his fundamental deviations from Lutheranism, however,
he was violently assailed by his Wittenberg colleagues. Thus Conrad
Cordatus, Luther’s table companion, passionately attacked him in
1536. Luther found an excuse for Melanchthon, but, displeased with
his philosophical ideas, said: “I shall have to chop off the head of philosophy,
and may God help me do it; for so it must be.”\footnote{Grisar, \textit{op. cit.}, Vol. III, p. 371.}
 Jacob
Schenk, an eloquent and aggressive Lutheran pastor, in 1537, accused
Melanchthon of making treasonable concessions to the Catholics. The
Elector was drawn into the controversy and privately consulted
Luther with reference to it. Luther again expressed his regard for
Melanchthon and deprecated his “being driven from the University”
of Wittenberg.\footnote{\textit{Ibid.}}
 But events soon conspired to induce Melanchthon,
under pressure of his adversaries and broken down by his silent conflict
with Luther, seriously to contemplate abandoning Wittenberg.\footnote{\textit{Ibid.}, p. 370.}

He remained, however, for he was unable to form any firm resolutions.
“Let us cover our wounds,” he afterwards wrote to Bucer, “and
exhort others to do the same.”\footnote{\textit{Ibid.}, p. 377.}
The Catholic spokesmen increasingly
revealed his inconstancy and weakness of character, which was the
cause of his suffering. Mercilessly they censured his pliancy, which
approached perfidy. Cochlaeus warned the humanist Andrew Cricius,
bishop of Plozk, against the connection which Melanchthon endeavored to
form with him. His admonition was based on the opinion
which he had formed of Melanchthon from personal observation at
Augsburg, in 1530. “Take care lest he cheat you with his deceitful
cunning, for, like the Sirens, he gains a hearing by sweet and honeyed
words \dots He seduces [men’s hearts] with dishonest words.”\footnote{Grisar, \textit{op. cit.}, Vol. V, p. 267.}

Luther on one occasion aptly styled his friend “the Erasmian intermediary.”\footnote{\textit{Op. cit.}, Vol. III, pp. 343 sqq.}

Melanchthon was so deeply immersed in his humanistic views,
which he shared with the much admired Erasmus, that his theology,
which he had studied only \textit{en passant}, was affected by his rationalistic
and immature philosophy. Although far removed from the Catholic
truth, he nevertheless contended that he fundamentally agreed with
the religious position of Erasmus.\footnote{\textit{Op. cit.}, Vol. V, p. 268.}
 His concepts of faith, its foundation
and postulates, were rather shallow. It was deplorable that
this philologian, who lacked profound theological learning, was able
to wield so much influence in the sphere of theology. “When barely
eighteen years of age,” says John Faber, bishop of Vienna, “he began
to teach the simple and by his soft speeches has disturbed the whole
Church beyond measure.”\footnote{\textit{Ibid.}, p. 267.}
The Catholic apologists soon discovered
the shallowness of his theology and philosophy. He delights in speaking
of the “celestial academy,” where men sit in the \textit{schola} of the
apostles, prophets, etc. He bedecks revelation in a vesture of classicism.
But, not content with style, he alters the content of religion
for the sake of sophistry or expediency or to promote his irenic endeavors.
In brief, he is dominated by a desire to reduce all things to
a humanistic level.

It was his supreme desire to pursue his humanistic studies in peace
and tranquillity. The princes, as “theocrats,” he held, should establish
such a state for himself and the faithful. He placed all religious
authority in their hands. In this he was aided by his ideas of classical
antiquity. He was of the opinion that the growing corruption
could be overcome only by the civil authority in religious matters.\footnote{\textit{Op. cit.}, Vol. V, p. 584; Vol. VI, p. 673.}
His ability to describe the decadence of the age approached that of
the convert Wicel.\footnote
{\textit{Op. cit.}, Vol. V, pp. 178 sq. On Melanchthon’s demand for a council composed of
followers of the new religion, \textit{ibid.}, Vol. V, pp. 169 sqq.}

Did Melanchthon counsel his mother to remain a Catholic? A
report which made its first appearance in 1605, has him say to her:
“The new religion seems more acceptable, but the old one is safer”
(\textit{Haec plausibilior, illa securior}). According to this account he did
not desire to see his mother disturbed in her faith--an attitude quite
conformable with his character. One may say with the author of the
article “Melanchthon” in the Protestant “Encyclopädie für Theologie”:
“The story is at least not improbable, even if it cannot be
demonstrated as true.”\footnote{Grisar, \textit{Luther}, Vol. V, pp. 270 sqq.}


An entirely different type was the ex-priest John Bugenhagen, a
Pomeranian, pastor of Wittenberg, and Luther’s right bower in the
propagation of Protestantism in northern Germany. Köstlin characterizes
him as a man “endowed with great and sturdy natural powers
of mind and body.”\footnote{Grisar, op. cit., Vol. III, pp. 404 sqq.}
 Indefatigable self-sacrifice and tireless industry
were characteristics of this robust and stern man. While not a great
theologian, he was gifted with an unusual talent for organizing, as
his ecclesiastical ordinances show. In the preface to Bugenhagen’s
published Commentary on the Psalms, Luther says: “I venture to
assert that Pomeranus is the first person on earth to give an explanation
of the book of Psalms.” This eulogy, however, appears “strange”
to Albrecht, the Protestant editor of the preface, who observes:
“Luther had no clear perception of the defects of Bugenhagen’s exegetical
method.”\footnote{\textit{Ibid.}}
 Luther freely unbosomed himself to Bugenhagen
and acknowledged that he often derived great consolation from a
single word that came out of his mouth. When his friend labored in
distant parts, Luther felt his absence keenly. He classified Bugenhagen
with those who were able to offer “strong limbs” to the temptations
of the devil; of such, he said, “there must be some \textit{in ecclesia} who are
well able to bear the brunt of the devil’s blows.”\footnote{\textit{Tischreden}, Weimar ed., Vol. II, n. 1307.}
He rejoiced that
his associate heartily despised the ring of the Catholic apologist. One
of Bugenhagen’s statements against the apologists of the ancient
Church ran as follows: “Dear Lord Jesus Christ, arise with Thy holy
angels and thrust down into the abyss of hell the diabolical murder
and blasphemy of Antichrist!”\footnote{Grisar, \textit{Luther}, Vol. III, p. 412.}
 Luther’s opponents in his own camp
were likewise an abomination to Bugenhagen, and once when Luther
complained of Karlstadt, Grickel, and Jeckel (\textit{i.e.}, Agricola and
Schenk), Bugenhagen interrupted him and proposed this radical
remedy: “Doctor, we should do what is commanded in Deuteronomy
(13:5 sqq.), where Moses says they should be put to death.” And
Luther acquiesced.\footnote{\textit{Ibid.}, p. 409.}
 Pomeranus was blunt and superstitious. When,
on one occasion, the devil crawled into his churn and spoilt the butter,
he proceeded to insult his satanic majesty by easing himself in the
churn. Luther praised this act as most effectual.\footnote
{\textit{Tischreden}, Weimar ed., Vol. III, n. 3491; Grisar, \textit{op. cit.}, Vol. III, p. 230,}

Nicholas von Amsdorf, superintendent of Magdeburg, proved to
be a second Luther. Because of his fidelity to the new evangel, based
upon a certain mystic disposition, he impressed many favorably.
After Luther’s death he published a book entitled, “That the Proposition
‘Good Works’ are harmful to Salvation is a Sound Christian
One.”\footnote{Grisar, \textit{op. cit.}, Vol. VI, p. 409.}
 Luther styled him “a born theologian.”

Other less famous friends of Luther were: John Brenz, co-founder
of Protestantism in Swabia; George Burkhardt, surnamed Spalatin,
promoter of Lutheranism at the court of the Elector Frederick, and,
after the latter’s demise, Lutheran pastor at Altenburg; Nicholas
Hausmann, pastor at Zwickau; Wenceslaus Link of Nuremberg;
John Lang of Erfurt, etc. Because of their activity in distant parts of
the country, Luther often revealed his inmost soul to them in his
epistolary correspondence. Of Brenz he says: “Amongst all the theologians
of our day there is not one who knows how to interpret and
handle Holy Scripture like Brenz.”\footnote{\textit{Op. cit.}, Vol. I, p. 405.}

The second of the above-mentioned associates of Luther, Spalatin,
was actively engaged in historical research. In practice he was
a model of intolerance, particularly in the Protestantizing of
Meissen. Nevertheless, when, on one occasion, he visited his native
Catholic city of Spalt, he delivered himself of this advice: “Stick
to your own form of divine service.”\footnote{\textit{Op. cit.}, Vol. III, p. 285,}
He presented the congregation
of Spalt with a picture of Our Lady, which had once belonged
to the castle-church at Wittenberg; this image is venerated at Spalt
even at the present day. In the same city he founded a yearly Mass for
his deceased parents. In his later days Spalatin was much disquieted
by melancholy and temptations to despair. Luther endeavored to
comfort him; but he counseled him in vain “to find consolation even
against his own conscience.”\footnote{\textit{Op. cit.}, Vol. III, p. 285; Vol. V, p. 330.}

Of the more intimate friends of Luther, Justus Jonas remained longest with him
at Wittenberg.\footnote{Grisar, \textit{Luther}, Vol. II1, pp. 413 sqq.} He was
a lover and master of sociability, and, when Luther was depressed by
melancholy, willingly complied with Kate’s summons to the “Black Monastery” to
entertain him with his agreeable conversation. He was an able humanist and
versed in jurisprudence. His Latin translations of Luther’s works were highly
praised. His original productions were less numerous, but, being a courageous
fighter, he earned the respect of his friends for his various publications on
the religious question. He calumniously attacked Catholic apologists, such as
Faber and Wicel. Besides Melanchthon, Bugenhagen, and Cruciger, Jonas was one
of the most circumspect participants in the transactions and legal opinions
that issued from Wittenberg. Luther, who was wont to eulogize the talents of
his friends, said that Jonas had all the gifts of a good orator, “save that he
cleared his throat too often.”\footnote{\textit{Tischreden}, Weimar ed., Vol.
II, n. 2580.} As provost of the castle-church of Wittenberg Jonas had an
income, though it was never adequate for his needs. He was dean of the
theological faculty from 1523 to 1533, and took part in all the important
actions of Lutheranism , such as the Marburg Conference, the diet of Augsburg,
the visitations in electoral Saxony after 1528, and the introduction of
Protestantism into the duchy of Saxony. In 1541 he founded and subsequently
directed the affairs of the Lutheran Church in the city of Halle, which up to
that time had been the residence of Cardinal Albrecht of Mayence.\footnote{Cfr.
Grisar, \textit{Luther}, Vol. V, pp. 165 sq.} When qualms of conscience and
theological doubts assailed Jonas, Luther had to be at hand to encourage him.
On one occasion he sent Jonas the consoling words with which he was wont to
find comfort in similar circumstances.\footnote{\textit{Tischreden}, Weimar
ed., Vol. IV, n. 4852; Grisar, \textit{op. cit.}, Vol. III, pp. 414 sq.} On
another occasion, Jonas expressed the opinion, approved by Luther, that since a
man could not comprehend the articles, it was sufficient to begin with a mere
assent.\footnote{\textit{Tischreden}, Weimar ed., Vol. V, n. 5562.} “Yes,” said
Luther, “if a man could but believe it.”\footnote{\textit{Tischreden}, Vol. IV,
n. 4864.}

When Jonas railed at the infidelity of the country people around
Wittenberg, Luther admitted that he knew only one peasant in all
the villages who seriously instructed his household in the Word of
God and the Catechism. “The others,” he said, “are all going to the
devil.”\footnote{\textit{Ibid.}, II, n, 2622b; Grisar, \textit{op. cit.}, Vol. III, p. 415.}
 In consequence of “spiritual temptations” (G. Kawerau)
which he suffered after the Schmalkaldic Wars, Jonas developed a
severe mental disorder similar to the \textit{morbus melancholicus} of
Spalatin. It is said that his death (1555) was happier than his life.\footnote{\textit{Ibid.}, p. 416.}

It is remarkable with what frequency the contemporary documents
mention this disease as occurring within the Protestant fold, especially
in the later years of life. Melancholia may almost be considered as
the chief malady of the age of the Reformation.\footnote
{Concerning the following, see Grisar, \textit{Luther}, Vol. III, p. 416; Vol. IV, pp. 218 sqq.}
Nicholas Paulus
has latterly again called attention to this peculiar phenomenon, which
had been previously noted by others. He supports his contention with
a mass of documentary evidence.\footnote{Cfr. Grisar, \textit{op. cit.}, Vol. IV, p. 225, n. 3.}
Among other things he mentions
that Jerome Baumgirtner of Nuremberg, Luke Osiander, and Zachary
Rivander speak of healthy people everywhere suffering from
fear, lack of consolation, and mental strain; that the number of
suicides increased in so frightful a manner as to cause one’s hair to
stand on end; and that they believed it was a sign forecasting the
approach of doomsday. Jerome Weller, whom Luther endeavored to
console in various ways, Nicholas Hausmann, his intimate intellectual
associate, Simon Musaeus, who wrote two treatises against the
“melancholy devil,” Nicholas Selnecker, the editor of Luther’s Table
Talks, Wolfgang Capito, the celebrated spokesman of the new religion
at Strasburg, and Joachim Camerarius, an intimate friend of
Melanchthon, who in a letter to Luther expresses his despair because
of the moral decadence of the age, were all affected by this disease
of chronic religious melancholy, not to speak of a number of less
famous preachers, scholars, and authors who professed the new religion.

When the preacher Nicholas Beyer narrated in the presence of
Luther how the devil had tempted him to stab himself, Luther consoled
him by confessing that he had been assailed by similar temptations, though
we have no evidence that he was ever seriously tempted
to commit suicide. Mathesius, Luther’s pupil and eulogist, “could not
bear the sight of a knife in the last year of his life because it enticed
him to commit suicide” (G. Loesche). The Nuremberg preacher
George Besler, a victim of melancholy induced by the religious conditions
of the time, committed suicide with a “hog-spear” during
Luther’s lifetime.

Antonius Musa, pastor of Rochlitz, confided to Luther that he
was depressed in his mind because he could not believe what he
preached to others. Thereupon Luther replied as follows, according
to Mathesius: “Praise and thanks be to God that this also happens to
others. I fancied it was true only in my case.” Mathesius adds: “Musa
never forgot this consolation all his life.” He says that Musa himself
told him this story.\footnote{Grisar, \textit{op. cit.}, Vol. V, p. 364.}
 The same eulogist of Luther writes: “There are
many who lead a languishing existence and despair; there is no longer
any joy or courage among men.” A peculiar kind of literature became
popular at that time, consisting of consolatory exhortations for
those afflicted with melancholy. A Hamburg preacher, J. Magdeburgius,
wrote: “The need of consolation was never felt more keenly
than at present.” Amsdorf lamented that many who were afflicted
by melancholia returned to Catholicism, because “they were at
their wit’s end” on account of the doctrinal dissensions of Protestantism.

One of those thus tormented was Luther’s table companion, John
Schlaginhaufen. His suffering was increased by a profound sense of
guilt. The interviews with Luther, which he reports in writing, are
a vivid reflection of the prevalent malady of that age. Schlaginhaufen
was disinclined to believe Luther when the latter maintained that
Satan alone could cause such dread melancholia, but Luther insisted.
“The devil,” he said; “feels his kingdom is coming to an end, hence the
fuss he makes.” The troubled man, however, grew more gloomy,
because he could “not distinguish between the law and the Gospel.”
Luther consoled him by saying that he himself and the Apostle Paul
had “never been able to get that far,” namely, to make a proper distinction
between the law and the Gospel. Finally, Luther resorted to
his authority and said: “I have God’s authority and commission to
speak to you and to comfort you.”\footnote
{Grisar, \textit{op. cit.}, Vol. IV, pp. 226 sq.; \textit{Tischreden}, Weimar ed., Vol. II, n. 1263, 1289,
1492, 1557.}
