\section{Luther’s Illness at Schmalkalden. New Polemics}

The illness with which Luther was seized at Schmalkalden was a
violent attack of gallstone, an old trouble which had become greatly
aggravated. He was no longer able to participate in the conferences
of the convention. In fact, people began to fear for his life. Although
he was suffering intense pains and thinking of death, he would not
allow even a thought of reconciliation to arise in his soul. On the
contrary, he prayed as follows: “O God, Thou knowest that I have
taught Thy Word faithfully and zealously \dots I die in hatred of
the pope.”\footnote
{\textit{Tischreden}, Weimar ed., Vol. VI, n. 6974; Grisar, \textit{Luther}, Vol. III, p. 435.}
Once, when racked with pains, he said to a chamberlain
of his Elector that his death would be a source of joy to the pope, but
the latter’s elation would not be of long duration; for the truth of
the epitaph which he (Luther) had prepared, would remain. The
tenor of this epitaph was that his death would be the death of the
pope: “\textit{Pestis eram vivus, moriens ero mors tua, papa.}” This horrible
hexameter, it is true, is not inscribed on his tomb at Wittenberg,
but since about 1572, it appears upon a huge memorial tablet with
his effigy which had originally been destined for Wittenberg, but was
transferred to Jena.\footnote
{Grisar, \textit{Luther}, Vol. III, p. 436; Vol. V, p. 102; Vol. VI, pp. 370, 377, 389, 394,
395 sq.}

He did not want to die at Schmalkalden, but in the company of his
friends at Wittenberg; for he did not wish the papal nuncio, the
“legate of the devil,” to enjoy the satisfaction of seeing him die in his
immediate vicinity. Accordingly, on February 26, he was conveyed
to a coach that was to carry him home. A multitude surrounded him
as he was about to depart. He made the sign of the cross over them
and said: “May the Lord fill you with His blessing and with hatred
of the pope.” Such was “his last will and testament,” according to an
expression of Mathesius in his eleventh sermon on Luther,” who adds:
``He [Luther] bequeathed to his friends, the preachers, \textit{odium in
papam}.''\footnote
{Mathesius, \textit{Historien} (1566), p. 130; Köstlin-Kawerau, \textit{M. Luther}, Vol. II, p. 389;
Grisar, Luther, Vol. III, p. 435.}

On the very next day after his departure Luther experienced an
improvement in his health. Having arrived at Gotha, after an exhausting
journey, he made a will in writing--his so-called First Testament
--in which he expressed the wish that he might live until Pentecost, in
order to attack the Roman beast with even greater vigor
than he had done before. In this testament he assures the princes that
they need not worry over the spoliation of church property. “They
do not rob like some others do; indeed, I see how, with these goods,
they provide for the welfare of religion.”\footnote{Grisar, \textit{op. cit.}, Vol. III, p. 437.}
It was more of an attempted
easing of his own conscience than a statement in conformity
with the truth. According to the reports of his friends, he went to
confession at Gotha, as was his wont, and received absolution from
Bugenhagen.

He arrived safely at Wittenberg on March 14, his health having
greatly improved.

When, during the ensuing period, his strength and ambition
flagged, he stimulated himself by resorting to a remedy which always
proved effective. He filled himself with hatred of the pope. “Then
my mind is completely refreshed,” he says in his \textit{Table Talks}; “the
spirit is quickened and all temptations flee.”\footnote{\textit{Tischreden}, Weimar ed., Vol. II, no. 2410.}
Certain phenomena of
his inner life can hardly be judged by ordinary standards. The idea
of the devil being at work in the papacy distorts all his thoughts. In
the case of abnormal phenomena, among which we must reckon his
imprecatory prayer, puzzling psychological problems constantly recur.
He is not the victim of fixed ideas; for free-will and accountability
are clearly operative in his case; but his guilt is diminished when
the psychopathic condition which oppressed him since his youth and
the monastic period is taken into consideration.\footnote
{See the chapter: “The Darker Sides of Luther’s Inner Life,” Grisar, Luther, Vol. VI,
pp. 99--186.}
Moreover, many
pages of the works and letters which he composed at Wittenberg betray
this nervous condition, which was accompanied by heart disease
and precordial dread.

In 1538 he published his Schmalkaldic Articles, which the convention of
1537 had suppressed, intensifying many of their polemical
acerbities.\footnote{See \textit{supra}, n. 1.}
He represented these articles in this work as a document
which had been approved by the convention, saying they were
“adopted unanimously acknowledged by our party,” in order that
“they might be “publicly submitted and introduced as our profession
of faith” before a truly free council of the church. This assertion was
false and it cannot be established how Luther came to make it. Can it
be assumed that he had no reliable information with respect to the
actual proceedings of the convention of Schmalkalden?\footnote{Grisar, \textit{Luther}, Vol. III, p. 440, note 2.}


In the same year (1538) Luther published a revision of his “Instruction
of the Visitators to the Parsons,” in which, besides some good
exhortations, he directed the parsons to “condemn vehemently the
papacy and its adherents.”\footnote
{Weimar ed., Vol. XXVI, pp. 195 sqq.; Erl. ed., Vol. XXIII, p. 57. Cfr. Grisar, \textit{Luther},
Vol. III, p. 438.}

The larger work of the succeeding year, “On the Councils and the
Churches,”\footnote
{Weimar ed., Vol. L, pp. 509 sqq.; Erl. ed., Vol. XXV, 2 ed., pp. 278 sqq. Cfr. Grisar,
\textit{Luther}, Vol. V, pp. 377 sqq.; 106 sq.}
was the execution of a proposal made by Amsdorf, who
had suggested that he once and for all thoroughly repulse the Erasmians
and all papists who appealed to the Church and proclaimed
her right of rendering final decisions. Luther said that people constantly
clamored for “the Church, the Church, the Church,” in order
to destroy his gospel. With an impetuous diligence he read the history
of ancient councils and other ecclesiastical documents in order to find
material to deny the authority of the Church. The tone of this work,
which is written in a self-conscious, provocative, and abusive style,
constitutes a psychological problem, despite its somewhat scholarly
form. “Whoever teaches differently [than we], though he be an
angel from heaven,” he says, “let him be anathema” (Gal. 1:8). “We
desire to be the pope’s masters and to trample him under foot,” etc.
The pope must “side with us” in the proposed council. “Emperors and
kings ought to co-operate in this matter and coerce the pope into
compliance.” Such statements were apt to enlighten certain blind
men in Germany who still good-naturedly believed that peace could
be brought about by way of negotiations and religious colloquies.
These so-called expectants believed that they could keep the Lutheran
question in abeyance by means of a few concessions, until the ecumenical
council convened.

Cardinal Albrecht of Mayence seems to have held this opinion.
His immoral private life blinded his intellect and rendered his character
weak. Luther was enraged at him because he had thus far declined
to join the reformers. He employed the affair of a certain
Anton Schonitz to vent his resentment against Albrecht in a violent
pamphlet, entitled: “Against Cardinal Albrecht, Archbishop at Magdeburg
,”\footnote
{Weimar ed., Vol. L, pp. 395 sqq.; Erl. ed., Vol. XXXII, pp. 14 sqq. Cfr. Grisar, \textit{Luther},
Vol. V, pp. 106 sq.}
which he caused to be printed against the express wish of
his Elector, who did not desire to see his colleague insulted. After the
book appeared, Luther had to promise the Elector not to publish anything
of a personal matter without the previous censorship of the
electoral curia. The incident furnished the weak Cardinal Albrecht
with an opportunity of seeing how little hope there was of effecting a
conciliation with the innovators. We may add that a few years later
a change came over the Cardinal. A new spirit animated Mayence and
its archiepiscopal court, stimulated by the activity of Giovanni Morone,
the papal legate, and of Bl. Peter Faber, a companion of St.
Ignatius, who came to that city in 1541. The spiritual exercises conducted
by Faber influenced the worldly-minded Cardinal and induced him to become
a defender of the Church and to lead a better
life until his death, which occurred in 1545.\footnote
{Cfr. Grisar, \textit{Luther} (German original), Vol. III, pp. 1025 sq. (omitted in the English
translation by E. M. (Lamond).}

Duke Henry of Braunschweig-Wolfenbüttel was one of the bitterest
opponents of the religious revolt. He was a personal enemy of
John Frederick of Saxony and Landgrave Philip of Hesse. He was
accused of many deeds of violence and led an immoral life. He was
also an author and wrote a vigorous attack upon the Protestant
princes and the new Church at New Year’s, 1541. With impetuous
haste, Luther, though afflicted with violent ear-ache, replied to him
in a pamphlet entitled, “Against the Clown” (\textit{Wider Hans
Worst})\footnote{Weimar ed., Vol. LI, pp. 469 sqq.; Erl. ed., Vol. XXVI, 2 ed.,
pp. 19 sqq.}

The contents of this work are in accord with its contemptuous title.
This uncouth lout, Luther says, is at the same time a disgraceful liar
in his attacks upon the alleged evangelical heretics. In this work as
well as in the one “On the Councils” Luther proposed to show where
the true Church was. It is not with the papists, who lack twelve essential
parts; the true and ancient Christian Church is rather on his
(Luther’s) side. The devil’s harlot is an epithet which he applies to
the papal Church, while Duke Henry, the loutish clown, is characterized
as an incendiary and a dastard, who is forced to hear evil reports
because of his immoral conduct.\footnote{Grisar, \textit{Luther}, Vol. IV, p. 64.}
 Luther wrote to Melanchthon that
he marveled at himself because he had observed such moderation in
the composition of this book.\footnote
{April 12, 1541; see \textit{Briefwechsel}, XIII, p. 300, on his book \textit{Contra istum diabolum Mezentium.}
Mezentius was a notorious tyrant.}

In 1537, Luther became involved in an exciting feud with antinomianism.
John Agricola of Eisleben, afterwards of Wittenberg, a
former friend of Luther and one of his most renowned theologians,
passionately declaimed against the law of Christian morality. He contended
that the law did not effect true penance, but death and damnation. He
wanted conversion and penance to be the product of love.
For a considerable number of years, Luther had been wont to concede
greater effectiveness to the law and the fear of punishment than he
had granted in the early part of his career. Now the unsparing attacks
of Agricola violently aroused him, especially since that writer
quoted former statements of his own. He condemned the doctrine of
Agricola as antinomianism, \textit{i.e.}, perversion of the law.\footnote
{Grisar, \textit{Luther}, Vol. V, pp. 15--25.}
On December 18, he delivered a discourse against the antinomian theses, which,
however, Agricola refused to acknowledge as his own. The controversy
aroused wide-spread attention. Luther’s friends, among them
Amsdorf, bitterly complained that the pupils pretended to be wiser
than their master. Luther arranged a second disputation for January
12, 1538, to justify his former position. This was followed by a third,
on September 13, which proved to be an extraordinarily lengthy argument
against the new “spiritual blusterers” and “conscious hypocrites.”\footnote{\textit{Disputationes}, ed. Drews, Disp. I, pp. 246 sqq.; II, pp. 334 sqq.; III, pp. 419 sqq.}

Luther’s work, “Against the Antinomians,” published in the
beginning of 1539, sealed the embittered conflict with Agricola and
the numerous adherents whom the latter had attracted.\footnote{Weimar ed., Vol. L, pp. 468 sqq.; Erl. ed., Vol. XXXII, pp. 1 sqq.}
 Meanwhile
the founder of antinomianism had timidly retreated from the field of
battle. Luther nevertheless printed things about him which must have
hurt him keenly. In March, 1540, Agricola brought suit against Luther
before the Saxon Elector, to whom he wrote that he had been
trodden under foot for well-nigh three years and had slunk along
at Luther’s heels like a wretched cur.\footnote{Grisar, \textit{Luther}, Vol. V, p. 21.}

As a final solution, Agricola left Wittenberg about the middle of
August, 1540, and betook himself to Berlin, where a position as court-preacher
was offered to him by the Elector Joachim II of Brandenburg, who had been
converted to the new evangel.
