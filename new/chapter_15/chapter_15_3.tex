\section{Further Violent Measures}

In 1535, Joachim II succeeded his father, Joachim I, who had faithfully
adhered to the ancient Church, as Elector of Brandenburg. Like
his mother, Elizabeth, a friend of Luther, Joachim II had favored the
new religion before his accession to the throne, but only in secret,
because he had solemnly sworn to his father that he would remain
true to the Catholic faith. In 1540 Joachim was persuaded by the
Landgrave of Hesse to issue a ritual in his own competency as territorial
bishop, in which he effected a forceful reorganization of the
electorate in conformity with the new religion. The clergymen who
resisted were exiled, the monasteries were suppressed, the property of
the Church as well as the metallic treasures of art which adorned the
churches were confiscated to the crown, whence they passed into the
hands of the “silver squires” and found their way into the insatiable
mint of the country.\footnote{Janssen-Pastor, III, pp. 479 sqq.}
The prodigality of the Elector, his buildings
and mistresses, caused him immeasurable expenses. According to the
testimony of contemporaries, the country was ruined in consequence
of his misgovernment. Relative to divine service, Joachim II avoided
all striking changes so carefully “that the bulk of the nation, the poor
people of the countryside, did not realize what had actually happened.”\footnote{J. G. Droysen, quoted \textit{ibid.}, p. 481.}

The Elector deceptively declared that he had not introduced
any new doctrine, but had merely abolished prevailing abuses.
Agricola, the pliant court-preacher, faithfully assisted him. The Latin
“Mass” was celebrated with churchly vestments; the host and the
chalice were elevated; many feast-days were retained; meat was prohibited
during the forty days of Lent; solemn processions were held
as of old; and the clergy, vested in white gowns, carried the viaticum
to the sick. Joachim declared that he did not wish to be bound by
the ordinances of the Church of Wittenberg in these matters.

Luther, who regarded the activities of Agricola with distrust, was
in the habit of characterizing him as a comedian. He approved of the
new ritual only in part, and demanded that the people should comply
with it only on condition that the pure gospel be preached. To Buchholzer,
a preacher who felt uneasy about the retention of the clerical
vestments, he wrote: “In God’s name, walk about [in the procession]
with a silver or golden cross and a cape or robe of velvet, silk or
linen.” If the Elector were not satisfied with the clergy’s wearing one
robe, let them put on three; if one procession did not suffice, let them
hold seven, like Josue at Jericho, and let His Electoral Highness leap
and dance like David before the ark of the Lord.\footnote
{December 4, 1539; \textit{Briefwechsel}, XII, p. 317; Janssen-Pastor, \textit{l.c.}, p. 482; Grisar,
\textit{Luther}, Vol. V, p. 313.}
As time went on,
it was but natural that these temperamental differences produced a
certain opposition between Wittenberg and Berlin.

Joachim II was encouraged in his opposition to the faith of his
forbears by the almost simultaneous and sudden turn in the religious
situation which took place within the duchy of Saxony. Duke George,
the noble, valiant and faithful defender of the ancient Church and
of the Emperor, passed away on April 17, 1539, without a surviving
son. His brother Henry, who succeeded him, precipitously destroyed
the Catholic status of the duchy which George had sedulously nurtured
since Luther commenced his public career. Luther had always
hoped for the death of Duke George. The judgment of God, so he
said in 1522, would inevitably overtake him.\footnote{J. G. Droysen, quoted \textit{ibid.}, p. 481.}
After the Duke had
died a Christian death, Luther predicted that his race would perish.\footnote{\textit{Tischreden}, Weimar ed., Vol. IV, n. 4623; Grisar, \textit{l.c.}}

Like Joachim II, the new duke had favored Lutheranism before he
succeeded to the throne. As the arbitrary ruler of the Church in his
duchy, he commenced his reign by ruthless measures against the
Catholics. Luther came temporarily from Wittenberg in order to aid
him by his sermons and counsel. Melanchthon, Jonas, and Cruciger
associated themselves with him for the purpose. In July, 1539, scarcely
four months after the demise of his predecessor, Duke Henry, following
the example of the Elector of Saxony, decreed a so-called
evangelical visitation, as the most practical method of Protestantizing
his people. The decree was executed by Luther’s preachers.

The Catholic clergy were forcibly removed and replaced by apostate
priests and monks, nay, sometimes even by ordinary laborers, who, though
devoid of all education, pushed themselves to the fore by their fluency of
speech and a hastily acquired stock of Biblical quotations.

Luther was not pleased with the conditions which speedily developed at
court and among the nobility and the people. His letters reveal a gloomy
picture. At the court of the aged and feeble prince he sees nothing but
“arrogance and the desire of amassing wealth,” coupled with an “inordinate
repugnance to promoting the cause of God.”\footnote{Grisar, \textit{op. cit.}, Vol. IV, p. 194.}
 In his depressed mood he
believes that the scandals of the court are “ten times worse” than the scandal
caused by the bigamous union of the Landgrave of Hesse, styles the
courtiers and nobles “the harpies of the land,” and says that they will end
by “eating themselves up by their own avarice.” Despite their continuous
appropriation of the property of the Church, he charges them with allowing
the preachers to starve. He advises a pastor, who was to have been
chosen visitator, as follows: “Even should you get nothing for the visitation,
still you must hold it as well as you can, comfort souls to the best of your
power, and, in any case, expel the poisonous papists.”\footnote{\textit{Ibid.}, p. 195.}
 Thus, Luther’s idea
of advancing the kingdom of God is bound up with the harshest and most
unfair methods and he extols the introduction of the new religion into
the duchy of Saxony as a wonderful work of God for the salvation of souls.

The religious apostasy made progress also in the North German
jurisdiction of Albrecht of Mayence, namely, in the archbishopric of
Magdeburg and the bishopric of Halberstadt. In 1541, Justus Jonas
introduced the new religion into his native city of Halle.

As early as 1533, Protestantism made great gains in the duchy of
Jülich-Cleve, in Anhalt-Köthen, and in Mecklenburg. In March,
1534, Anhalt was completely Protestantized, on which occasion Luther
sent congratulations and best wishes to the ruler of that city. In July
of the same year, the city and district of Augsburg adopted the new
religion. In 1539, the archbishopric of Riga in Livonia was brought
under Protestant control.

In 1534, dukes Barnim and Philip of Pomerania forced their subjects to
embrace the new evangel, despite the resistance offered by
the nobility and the prelates. Bugenhagen, who was a Pomeranian,
aided the rulers by his unflinching energy and talent for organization.
In order to strengthen the new religion, Duke Philip married a sister
of the Saxon Elector. During the marriage ceremony, which Luther
solemnized according to his new rite, the wedding ring happened to
fall to the floor. For a moment Luther was nonplussed, but then exclaimed:
“Do you hear, devil, this wedding does not concern you; you
will labor in vain.”\footnote{Köstlin-Kawerau, \textit{M. Luther}, Vol. II, pp. 290 sq.}

Bugenhagen was actively engaged in the promotion of Lutheranism,
not only in Pomerania, but also in Braunschweig, Hamburg, and
Lübeck. From 1537 to 1539 he labored in the service of King Christian
III, who introduced the new religion with extremely violent
measures in Denmark.

On February 4, 1538, Bugenhagen joyfully reported to Luther from
Copenhagen that “the Mass was now prohibited throughout the entire
country;” that the mendicant friars had been exiled as “sedition-mongers”
and “blasphemers” because they refused to accept the offers of the king;
that all the canons had been ordered to attend the Lutheran communion on
festivals, and that every effort would be put forth to subject the four
thousand parishes to the new evangel.\footnote{Grisar, \textit{Luther}, Vol. III, p. 413.}
 The tyrannical ruler caused all the
bishops within his territory to be incarcerated. According to an account of
the superintendent of Zealand, who had come to Denmark from Wittenberg
in the company of Bugenhagen, some of the monks were hanged.

The King was solemnly crowned by Bugenhagen on August 12, 1537.
“Everything proceeds favorably,” Luther wrote to Bucer in Denmark. “God
is working through Pomeranus. He crowned the king and queen like a true
bishop.”\footnote{\textit{Ibid.}}

In few countries were the external ritualistic forms so little disturbed as
in Denmark under the calculating influence of Bugenhagen. Even at the
present time, the number of Catholic practices, commencing with the celebration
of high Mass to the ringing of the angelus bells, is amazingly great
in Denmark, Norway, and the duchies formerly united to the Danish crown.

The Protestant ministers, when celebrating “Mass,” still frequently vest
themselves in a chasuble made of red silken velvet, which is worn over an
alb of white linen. They also perform the elevation of bread and wine after
the so-called consecration, which is pronounced in the middle of the altar.

In Sweden also Catholic ritualistic solemnities were retained for a
long time. In his career of royal hierarch Gustavus Wasa, who had
Protestantized that country as early as 1527, continued to rule in
disregard of all the liberties of the people. He maintained friendly
relations with Luther, from whom he procured a tutor for his son
Eric in the person of George Normann, a native of Pomerania, who
came to Sweden fully empowered to supervise the bishops and the
clergy. The impetuous spokesmen of the new religion in Denmark
spread rumors to the effect that King Gustavus was not sufficiently in
earnest about the new gospel. Gustavus pleaded with Luther to protect
him from such reports. In 1539, Luther wrote a testimonial certifying
that “his piety was marvelously extolled above that of other
princes,” that he was imbued by God with a loftier spirit not only for
the cause of religion, but also for the cultivation of the sciences. He
exhorted the King to establish schools throughout his kingdom, particularly
in connection with the cathedral churches; for this was the
principal obligation of a pious prince.\footnote{\textit{Briefwechsel}, XII, p. 132; April 18, 1539.}
 He had in mind schools that
were founded upon his gospel and labored efficaciously to promote
the same--such as he himself advocated for Germany.
