\section{Literary Defenders of the Catholic Cause After 1530}

The literary defense of the Catholic cause proceeded with unabated
vigor, in spite of the great difficulties which the Catholic writers encountered.

It was not encouraging for authors who wrote in defense of the
Church or of the outraged rights of Catholics, to realize that they
were exposed to the vulgar invectives to which Luther and his disciples
resorted in their replies, or that the products of their industry
could be published only amid the greatest difficulties and at the cost
of severe sacrifices, because there was no adequate Catholic press. The
bishops continued to withhold their support. Papal subsidies were
shared only by a few, who succeeded in presenting their petitions to
Rome through powerful intercessors. Many talented apologists were
driven from the monasteries or ecclesiastical positions in the course of
the religious upheaval and, deprived of material support, endeavored
in vain to wield their pen in the service of the faith. Had there been
a number of periodicals at the disposal of talented Catholic writers, as
is the case today; had there been available an organized Catholic
daily press as a means of reaching the masses, the position of the
Church would have been quite different.

The necessity of moral reform, among Protestants as well as
Catholics, was greater, however, than the need of scientific or popular
instruction; for the new freedom promoted moral decadence in a
very high degree. Catholic writers complained that their efforts were
largely offset by the rejection of the precepts of the Church and by
the unheard-of compulsion exercised by the courts and magistrates of
many cities, and against which no remedy could be found. As a result,
some ecclesiastics, who might have been able to wield a mighty influence
in the literary sphere, dedicated themselves to preaching and
practical action. Others were deterred from literary work by the inconsistency
and fatuousness of the claims made by the Protestants,
who asserted one thing today and denied it tommorow and demanded
recognition in one place for what they rejected in another.

Among the books which exercised a powerful influence before and
after 1530 were the earlier and later literary productions of Eck,
Cochlaeus, and Faber.

Dr. John Eck, of the University of Ingolstadt, was called “the
Achilles of the Catholic party” by Cardinal Pole.\footnote{Janssen-Pastor, \textit{op. cit.}, Vol. VII, p. 593.}
 His practically arranged
“Manual against the Lutherans” (\textit{Enchiridion}) was in general
use among Catholics and, up to 1600, went through some fifty editions.
In addition to his sermons on the Sacraments, his principal
achievement consisted in his commentaries on the Gospels for Sundays
and feast-days. Intended for the clergy, they evidenced their author’s
intimate acquaintance with the errors of the day. No less than
seventeen editions of the Latin version of these sermons, which comprised
several volumes, appeared up to the year 1579. In 1530, Eck
began to reissue at Augsburg his writings against Luther, the first
installment being entitled \textit{Prima Pars Operum contra Ludderum}.\footnote{Wiedmann, Johann Eck, Ratisbon, 1865, p. 586.}

They were followed by a long series of new works, among which were
treatises on Purgatory and the Mass, dissertations against Zwinglianism
and against the errors of the Jews, memoranda composed for
princes and religious conferences, and commentaries on various books
of the Bible. Eck displayed incredible energy up to the time of his
death, in 1543. This humble priest never coveted ecclesiastical dignities.
When offered honorary canonries, as was frequently the case, he
invariably declined them, saying: “I desire to remain a schoolmaster
as long as I live.”\footnote{Janssen-Pastor, \textit{op. cit.}, Vol. VII, p. 592.}
 Courageously he bore the slanders which were
heaped upon him by the Lutheran party as well as the derision to
which he was subjected.

He felt more keenly the studied silence with which his enemies met
his arguments. In his Apology of 1542, he addresses his opponent
Bucer thus: “Listen, you apostate; does not Eck quote the words of
Sacred Scripture and the Fathers? Why do you not reply to his writings
on the primacy of Peter, on penance, the Mass, Purgatory, or to
his many homilies and other writings? \dots Do you believe you were
right,” he asks, “when you said at the beginning of the controversy
that Eck would be unable to advance any other authorities than his
Scotus, Ockham, Thomas, etc.?”\footnote{From Eck’s \textit{Apologia}; cfr. Wiedemann, \textit{op. cit.}, p. 275.}
 As a matter of fact, Eck’s
scholarly use of Sacred Scripture and the Church Fathers constituted
one of the principal merits of his controversial method. Consistency
and fortitude were characteristic of the activity of this man, who
also made a striking impression by his athletic appearance. “At the
religious conference in Ratisbon, in 1541, the superficiality of the
friends of the Interim gave way to the lucidity of his principles and
his solidity.”\footnote{Thus Janssen-Pastor, VII, p. 587.}
 His vivacious temperament and blunt honesty, coupled
with a fine sense of humor, doubtless inspired many a harsh passage in
his writings which it would have been better to omit. But it was a
boisterous and turbulent arena to which he was summoned by his
vocation.\footnote{On certain blemishes in his private life cf. Janssen-Pastor, VII, p. 592, n. 4.}

John Cochlaeus in his literary activity revealed not so much a
profound theologian as an ever ready and eloquent controversialist.
Hardly a year passed but that this man, who was small of stature,
participated in the controversies of the day, which he conducted with
great versatility. Descriptions of the age in which he lived, exhortations,
admonitions, and at times violent personal attacks fill the books
of this active controversialist after 1530 as well as before that time.
When Eck died, Cochlaeus took over and vigorously prosecuted the
work of the latter. The high-minded Duke George of Saxony, in whose
service he labored, supported him in every possible way. When, in
1539, George was succeeded by his brother Henry, who favored Lutheranism,
Cochlaeus saw his labors suddenly interrupted; his publisher, Nicholas
Wolrab, of Dresden, was thrown into prison; books in
defense of the Church by Witzel and Nausea, which Wolrab had in
press at the time, were cast into the water. Only with difficulty
Cochlaeus succeeded in inducing a relative of his to open a printshop
for Catholic books in Mayence. The printer, Francis Behan,
succeeded in establishing the foremost printing establishment for
Catholic works in Germany, which flourished at Mayence together
with that of Cologne, the second most important center for Catholic
publications.

“For twenty years,” Cochlaeus wrote in 1540, “there was nothing
more disadvantageous for us Catholic authors, in contrast with the
heretics, than the great unreliability of our publishers \dots The publishers
were almost all Lutherans, and we were able to obtain their
services only at a great outlay of money.”

He instances the sad experiences of Eck, Nausea, Mensing, and
others, with whom he had attended the religious conference at
Worms.\footnote{Janssen-Pastor, \textit{l. c.}, p. 566.}
 His own material condition was improved by a canonicate
at Breslau. In 1548 and 1549 he lived at Mayence. He died at
Breslau in 1552, exhausted by his labors. The works which he wrote
after the diet of Augsburg (1530) embrace an excellent treatise on
the saints, various publications on the question of holding an ecumenical
council, an effective and thorough reply to Bullinger, “On
the Authority of the Canonical Books and the Church,” which is
ranked among the best of his books, and his pointed “Philippica”
against Philip Melanchthon, in which he attacks, among other things,
the “serpentine artifice and hypocrisy” of that innovator.\footnote{\textit{Ibid.} The treatise in Cochlaeus’ works on the veneration of the saints (1534) is
actually the product of Arnoldus Vesaliensis.}
Cochlaeus
deserves special credit for his Latin “History of the Acts and Writings
of Luther” (Commentaria, etc.), which first appeared at Mayence in 1549,
and was frequently reprinted. It embraces the entire
controversial period and depicts the course of the great religious upheaval
as his keen eye observed it. The story is copiously supported by
citations from his own works and those of the unfortunate author of
the schism. The work proved to be a mine of information for later
Catholic writers.

John Faber, formerly vicar-general of Constance, became bishop
of Vienna in 1530, through the influence of Ferdinand, and as such
continued his very successful activity against Lutheranism by means
of the spoken and written word, especially by advising the princes.
In 1535 he wrote in defense of the Mass and the priesthood against
Luther. In the following year he wrote on faith and good works.
There is extant an instructive memorandum intended for the religious
conferences, addressed by him to the Catholic leaders. He was esteemed
by his fellow-Catholics for his learning and wisdom, and for
the purity of his morals—which fact did not prevent Justus Jonas,
in a pamphlet composed at the instigation of Luther, to characterize
Faber as a “patron of harlots,” because he combated the marriage
of priests. He could afford to ignore all such insults. He died at Vienna
in 1541.\footnote{Janssen-Pastor, \textit{op. cit.}, VII, pp. 580 sqq.}

Faber was succeeded in the episcopal see of Vienna by his friend
Frederick Grau, called Nausea, another energetic and gifted apologist
who opposed the heretical deluge. Grau was a man of excellent
culture and thoroughly trained in the sciences of language and jurisprudence,
no less than in theology. Originally employed as secretary
by the papal legate Campeggio, he afterwards functioned as a preacher
and writer in Mayence. His sermons are noted for their correct interpretation
of Sacred Scripture. Owing in part to his lack of means, he
was unable to publish his excellent catechism before 1543. In his work
on the council, he favored the granting of communion under both
forms, thinking that the Protestants could be won over by this concession.
He likewise urged the pope to abolish the law of sacerdotal
celibacy for the sake of removing scandal. He participated in the
Council of Trent, and died in that city in 1552.\footnote{\textit{Ibid.}, pp. 582 sqq.}

Of the large number of other defenders of the Catholic Church
and her doctrines we will mention only a few of the more prominent.
A man of very striking characteristics was George Witzel (Wicel), a
priest who traveled much and was extremely active. He died at Mayence
in 1573. Influenced by the writings of Erasmus, he embraced
Lutheranism and married, but after having acquired a more intimate
knowledge of the true aims of Luther and seeing the moral decline
which followed the new religion, he returned to the Catholic Church
and at once, in 1532, published an excellent treatise on good works,
followed by a book on justification, another on the Church, and an
apologia of himself. During his varied career he composed nearly a
hundred works, all characterized by combativeness and learning. No
one scourged the conditions within the bosom of the Lutheran Church
so effectively as Wicel; few experienced such adversities on this
account as he, so that--as he himself laments--``I am scarcely safe
anywhere, even in my own home, and I cannot travel without exposing myself
to the greatest danger.''\footnote{\textit{Ibid.}, p. 570.}
 From 1533 to 1538 he was
pastor of the few Catholics remaining in the town of Eisleben. During
this period, he was compelled, as he himself says, “to live in the
midst of wolves.” As protégé of Duke George of Saxony, he lived in
better circumstances for a short time at Dresden. At Fulda, where he
stayed with Abbot John, life was made intolerable for him by persecutions.
In Mayence he was assailed by the Lutherans because he defended the imperial
interim of 1548, which was repugnant to them.
This conciliatory interim, which was designed to end the schism and
hence made undue concessions to the Protestants, was in harmony
with Wicel’s ideal to win over the opposition by means of conciliatory
measures. He wished to stand above the disputants of both parties. Without
abandoning Catholic dogma, as he understood it, he
wished to prepare the way for a reconciliation, which, however,
proved ineffectual and was, in part, impossible. In this respect his
Erasmian training was a hindrance to him. He even censured the
theologians at Trent because they refused to adopt his peculiar methods
for the re-establishment of peace.

The lively discussions which this obstinate man had to carry on
with his fellow-Catholics were evidence of the fact that the latter
carefully weighed the idea of religious peace. If the idea itself was
regarded as impracticable, this was not due to a blind refusal of
conciliation. Wicel himself was forced to realize the extent of the
injury from which the Church was likely to suffer in consequence of
such ill-advised concessions, when, imbued with the best of intentions,
he enthusiastically participated in the new ecclesiastical regime introduced
by Joachim II of Brandenburg, which in the end Protestantized that country.

Another Catholic spokesman who deserves to be mentioned is the
Augustinian eremite, John Hoffmeister. He began to unfold a splendid literary
activity, commencing with 1538, when he wrote his
“Dialogues” and a refutation of the Schmalkaldic Articles. He continued
his efforts even after he had been made vicar-general of his
Order for all Germany. He, too, was animated by the delusive hope
of conciliating the Protestants.

The Franciscans furnished many renowned and learned defenders
of the ancient faith, \textit{e.g.}, Augustine von Alfeld (died about 1533),
Nicholas Herborn (died 1535), Conrad Kling (died 1556), and the
excellent pulpit orator John Wild of Mayence (died 1554).\footnote{On Wild see Janssen-Pastor, \textit{op. cit.}, Vol. VII, pp. 546--550.}
 Caspar
Schatzgeyer, a Minorite, was the model of them all in gentleness and
the noble style of his popular writings. Henry Helmesius, John Heller,
John of Deventer, Francis Polygranus were other Franciscans who
defended the Catholic cause.

The most celebrated Dominican authors were: Michael Vehe (died
1539), who produced one of the first German hymnals; Bartholomew
Kleindienst (died 1560), who, among other literary compositions,
addressed a “Right Catholic Admonition” to “his dear Germans,” in
imitation of the title of one of Luther’s books; John Dietenberger
(died 1537), the author of a number of small popular pamphlets, a
refutation of the Augsburg Confession, and an excellent catechism;
and John Mensing, who actively opposed Protestantism until his
death (about 1541), unhindered by the high offices which he held,
among which was that of auxiliary bishop of Halberstadt.\footnote
{On Mensing see Grisar, \textit{Luther}, Vol. I, p. 79; Vol. III, p. 195; Vol. IV, pp. 121, 160,
303, 385; Vol. VI, pp. 276, 391, 409 sq., 482 sq.}
The University
of Frankfort on the Oder honored the memory of his temporary professorship
there. Conrad Wimpina, a theologian of that university, did not long continue
his labors in refutation of the Augsburg
Confession which he had commenced in Augsburg, but died in 1531,
and left behind him, among other works, a brief but excellent history
of the religious sects in the \textit{Anacepbalacosis Sectarum}.
As in former times, so also now, prominent men outside of Germany
opposed the prevalent heresy. A splendid figure was the learned
Stanislaus Hosius, leader of the Polish episcopate. He became bishop
of Ermland in 1551, and later on a cardinal. To the select circle of
his friends belonged Frederick Staphylus, who had studied at Wittenberg
as a Protestant, became a convert to Catholicism in 1552, and
composed an “Epitome of the Doctrine of Luther” which became
famous. Italy and other countries, especially the Netherlands and
France, likewise produced eminent antagonists of Lutheranism before
as well as after 1530. Ambrosius Catharinus, a native of Siena, continued
his literary activity for ten years in France. On account of his
criticism of Cardinal Cajetan, he was out of harmony with his Italian
confréres. In Italy, not only courageous members of the monastic
Orders, such as the Franciscan Giovanni Delfino, but also men who
had been elevated to the cardinalate, like Jacopo Sadoleto, Marino
Grimani, and Gasparo Contarini, contributed by their writings to the
defense of the Catholic religion.
