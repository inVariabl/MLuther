\section{Belligerent and Pacific Movements in the Empire}

In 1537, a more intimate union of the Catholic princes against the
Schmalkaldic League, which was threatening war, became a necessity.
After many efforts on his part, the imperial ambassador, Held,
succeeded in March, 1538, in drafting a plan for a “defensive league”
at Spires. It was adopted at Nuremberg on the tenth of June. Emperor
Charles and King Ferdinand headed the League, whose other
members were: Bavaria, Duke George of Saxony, Dukes Henry and
Eric of Braunschweig, and the Elector Albrecht of Mayence (for
Magdeburg and Halberstadt). Owing to the fact that other Catholic
princes kept aloof, this so-called Holy League did not attain to the
importance which might have been expected.

The Schmalkaldic League soon afterwards sustained a disadvantage,
due to the armistice signed at Nice (June 17, 1538) between the
Emperor and France, in consequence of which the League lost all
hope of obtaining the aid of France, which it had sought. Emperor
Charles, on his part, needed all his forces against the Turks. This confirmed
him in his project of friendly negotiations with the Protestants.
But the Schmalkaldic League prepared for war. Landgrave Philip
labored unceasingly to bring it about.

Luther effectually supported the political programme of the
Schmalkaldians in a memorandum which decisively advocated armed
resistance, though he regretted the war and would have preferred to
see Germany invaded by “a pestilence” rather than ravaged by bloody
strife. In conjunction with Jonas, Bucer, and Melanchthon, at the end
of January, 1539, he drew up a formal opinion, wherein he indicated
to his Elector that the imperial constitution as well as the natural
law permitted princes to engage in aggressive war in defense of the
menaced gospel and the ecclesiastical possessions which it had acquired.
In the event that the Emperor would have recourse to arms,
he said, his status would have to be regarded as that of a mercenary
in the service of the pope, or as that of a highway robber, for there
was no difference between a common assassin and the Emperor, especially
since he tried to force his subjects to commit blasphemy and
idolatry.\footnote{Köstlin-Kawerau, \textit{M. Luther}, Vol. II, pp. 401 sq.}

In February, 1539, the leaders of the Schmalkaldic League, whilst
in a similar frame of mind, convoked an assembly which was to meet
in Frankfort on the Main.\footnote{Köstlin-Kawerau, \textit{M. Luther}, Vol. II, pp. 401 sq.}
 It was attended by delegates of the Emperor
and of King Ferdinand. At this convention, Saxony and Hesse
declared themselves in favor of aggressive war, “in order to get the
better of the enemy.” However, they were confronted with opposition in
the assembly. It was asserted that the programme adopted by
the Catholic League at Nuremberg expressly declared that peace must
be preserved. Although France promised to aid the Protestants, the
war, which was expected to break out at any moment, was avoided
for the present. The so-called peace of Frankfort was brought about,
chiefly because it appeared that there was no leader competent to conduct
the war, Philip of Hesse being severely ill. As frequently in the
past, so now, Philip was stricken with an attack of syphilis, which
he had contracted by his dissolute life. He left Frankfort on April 17,
and hastened to Giessen for medical treatment.

The Frankfort peace provided an arrangement which redounded
to the advantage of the new religion. The peace was to endure fifteen
months, with temporary suspension of all lawsuits pending in
the “Kammergericht” (the supreme court of judicature). At the
same time it obliged the signers to participate in a religious conference
soon to be held for the sake of effecting a “Christian and laudable
union.” These proposals were combated by the Catholics. Conrad
Braun, a jurist of the imperial supreme court of Spires, maintained
in his writings that the reference to a religious conference was a violation
of the rights of the proposed Church council. He held that the
use of force against sedition-mongers and despoilers of the Church
was perfectly proper.\footnote{G. Kawerau, \textit{Reformation und Gegenreformation}, p. 135; cfr. Janssen-Pastor, \textit{op. cit.},
III, p. 447.}
 These views were favorably received by many
ecclesiastical authorities. But where was there any prospect of the successful
application of violent measures under the then prevailing circumstances?
The strength of the Schmalkaldic League was increased
by the very fact that it gained time through the constant extension of
the tolerance which was granted to it.

The ecumenical council convoked by Paul III could not take place
at Mantua, as planned. It was at first deferred and then summoned
to convene at Vicenza, on May 1, 1538. On account of untoward
circumstances, it had again to be postponed, until it was finally opened
at Trent, November, 1542, at the urging of the Emperor. On July 6,
1543, it had to be adjourned because the war between the Emperor
and France prevented many bishops from attending.
An unlucky star also governed the contemplated religious conference. The
Emperor ordered it to be held at Hagenau, in June,
1540, but it miscarried, because most of the Protestant theologians
departed in consequence of a dissension that had arisen among them
relative to certain preliminary questions. The conference was resumed
at Worms in the fall. Its deliberations were presided over by the imperial
chancellor Granvella. Each side had appointed eleven delegates
as spokesmen, among the Catholics so appointed being Eck, Cochlaeus,
and John Gropper, whilst the Protestants selected Melanchthon,
Bucer, and Calvin. The Augsburg Confession was presented by Melanchthon
as one of the bases for discussion. It was not the original
text, however, but the so-called Confessio Variata, which had been
altered and published by Melanchthon in 1540. The alterations were
important. In treating the doctrine of the Eucharist, Melanchthon
had met the wishes of the Swiss theologians. In respect of justification,
he had attenuated the Lutheran position, eliminated the doctrine of
strict imputation and assumed a certain righteousness in man which
was imputed to him by God. As regards good works and the observance of
the law, “actual changes, or at least attenuations of a
dogmatic nature” had likewise been made.”\footnote{The phrase in quotes is Theodore Kolde’s. Cfr. Grisar, \textit{Luther}, Vol, III, pp. 440 sq.}
 To all these changes
Luther raised no objections, whereas Dr. Eck during the conference
at once charged his opponent, Melanchthon, with arbitrarily changing the
basic document; he did not, however, terminate the negotiations,
which were soon after transferred by the Emperor to Ratisbon,
where the diet was then in session.

Eck was convinced that the conference was bound to prove futile
because the question at issue was loyalty to the Catholic Church or
positive rejection of her teaching authority. For this reason, he also
found fault with the attenuation of Catholic dogmas, especially that
of justification, attempted by Gropper. Gropper and Julius Pflugk,
being the most moderate representatives of the older religion, differed
from the other Catholic theologians in some respects. Gropper participated
with Bucer in drawing up certain compromise articles which
were proposed for discussion. The whole movement was finally frustrated
by the justified objection of Rome to the proposed formula on
justification, in which human co-operation and merit were omitted;
and, on the other hand, by Luther’s declaration that the articles of
compromise were “impossible proposals” which neither party could
accept.\footnote{\textit{Briefwechsel}, XIII, p. 341; cfr., pp. 267 sq.}

Although an agreement was reached as to some other non-essential
points, the plans for reunion were regarded as shattered on May 22.
Cardinal Gasparo Contarini had vainly tried to help matters by his
personal participation at the conference as papal legate. Under the
influence of the Emperor and of his own fond expectations, he went
rather far in accepting the Lutheran idea of justification, at least in
certain expressions. After the close of the conference he expounded
his views in a much discussed “Letter on Justification” (\textit{Epistola de
Justificatione}). Despite the many attacks directed against this letter,
Pope Paul III continued favorably disposed towards the Cardinal.\footnote
{The \textit{Epistola}, newly edited, with a critical introduction, in \textit{Corpus Catholicorum}, Vol.
VII (1923) by F. Hünermann (\textit{G, Contarini, Gegenreformatorische Schriften}).}
At the close of the diet of Ratisbon, during which the opposition
between the two parties became constantly more acute, the Nuremberg peace
pact of 1532, but also the strict decrees of Augsburg were
renewed, subject, however, to a declaration (which was not accepted
by the Catholic estates) that the ecclesiastical property usurped by
the Protestants be protected and that the application of the Augsburg
decree be restricted to religious matters. All this was to be in force up
to the assembly of the proposed ecumenical council or a new diet.
The Catholic cause unexpectedly profited by the weakening of the
League. Philip of Hesse, its mainstay, began to vacillate in consequence
of an event that was creditable neither to himself nor to the
Protestant party. The consequences of the bigamous marriage which
he contracted (to be discussed later) affected the political affairs of
the Empire. When the matter became known publicly, he was threatened with
severe penalties under the laws of the Empire. In order
to evade them, he resolved, in 1541, to make terms with the Emperor.
His abandonment of the military League of Schmalkalden was an
irretrievable loss to that organization, which now began to decline.
