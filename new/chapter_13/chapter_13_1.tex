\section{The Diet of Augsburg}

Emperor Charles V was finally able to set out on his journey to
Germany, for which the German Catholics had ardently longed. The
treaty of Barcelona with the pope, and the treaty of Cambrai with
Francis, king of France, had opened the way for him.

On January 21, 1530, shortly before his coronation as emperor,
Charles published at Bologna the convocation of the imperial diet at
Augsburg, in which he himself intended to take an active part. It was
his wish that the diet should remove—peaceably, if possible—the
grounds for the religious controversy which filled him with anxiety.
It was the intention of this zealous Catholic ruler, sincerely to adopt
the ways of kindness and to effect an arrangement by peaceful
methods. For this reason the convocation adopted a conciliatory tone
and assured the Protestants that they would be given 2 hearing.

The Emperor’s brother, Ferdinand, journeyed to the Brenner in
Tyrol, to meet Charles as the latter was coming from Italy. At Gries,
on the northern declivity of the mountain, there is 2 monumental
inscription marking the spot where the brothers embraced each other.
Charles was depressed by Ferdinand’s report of the existing conditions.
Nevertheless, on June 15, the high-minded Emperor hopefully entered
the city of Augsburg.

This ancient free city on the Lech, a flourishing center of art and
commerce, still retained its venerable towered walls with moats and
gates. Inside were the homes of the wealthy and comfortably situated
patricians, and lofty, antique buildings, conspicuous among them the
palace of the bishop and the splendid town-hall, where the sessions of
the diet were to be held. Both have since been either replaced by other
structures or completely remodeled. The banking house of the Fuggers,
which had been established ten years before, still exists, a vivid
reminder of the great commercial firm which once dominated international
trade. The Fuggers remained loyal to the Catholic Church
when the new religion established itself in the city.

Bishop Stadion of Augsburg, assisted by the Emperor and the
Catholic estates who arrived for the diet, presided over the Corpus
Christi procession which wended its way through the streets of the
city, and was celebrated with a splendor never before witnessed. The
Protestants ostentatiously kept aloof. As a consequence, the Emperor
prohibited the preaching of Protestant sermons.

When the sessions of the diet were opened, on June 20, Luther’s
partisans succeeded in inducing the estates to deliberate on the religious
question before devising ways and means of combating the
Turkish menace, which Charles wished to be considered first of all.
It was their plan to submit an extensive statement justifying their
attitude on the religious question. Above all, it was their intention to
procure protection for themselves and publicly to advance their efforts
at propaganda. The so-called Augsburg Confession constituted a
means to this end. The lengthy document had been written by the
prudent and pliant Melanchthon, who at that time was very timorous,
and had the approval of Luther, who during the sessions of the diet
lived in the castle of Coburg, which was situated not far from Augsburg.
The Confession was so drawn up as to speak not in the name of
Luther or the theologians, but in that of the rulers who had adopted
the new creed and by whom it was submitted. Originally, therefore,
it was a confession of faith by the princes. Afterwards it became a
symbolical document, i.e., the official statement of the Lutheran
creed. By means of this instrument, the princes intended to indicate
through Melanchthon, their spokesman, the kind of religion they had
thus far suffered to be preached within their territories. Melanchthon,
whilst engaged in the composition of this document, had also intended
it to serve as a refutation of a work of Dr. Eck, who had caused an
exhaustive theological indictment of the new religious system, consisting
of 404 articles, to be submitted to the Emperor before his arrival in
Augsburg.

In order to procure a favorable decision on the part of the diet, the
author of the “Confession” tried to show that in reality there were no
great differences between the two camps. He proposes certain essentially
Lutheran doctrines, but veils them in clever formulas devised
to show that they coincide with what the Catholic Church had always held.
The question which, according to him, is of prime importance,
is about abuses which in the general opinion of men ought to be
abolished. In fact, the first official edition of the “Confession,” printed
in 1530, contained the deceptive declaration (which was subsequently
altered) that the impugned doctrines meant no deviation from the
Scriptures or the teaching of the Roman Catholic Church, in as far
as that teaching could be ascertained from Catholic authors.

The Emperor reluctantly consented to have the “Confession”
publicly read in the presence of the estates. It was so read on June 25,
not, however, during a regular session of the diet in the town-hall,
but in a smaller Gothic hall of the episcopal palace. The twenty-eight
articles were read in a stentorian voice by Baier, the Saxon chancellor,
who designedly read the German version of the text so distinctly
that it was audible through the open windows by those who lingered
in the courtyard without.

On closer inspection, the Catholic theologians were compelled to
marvel at the ingenuity with which a road to a pseudo-union with the
ancient Church had been kept open. They noted the absence of any
declaration relative to the pope, whom the Lutherans had come to
regard as Antichrist. The declaration was silent about the universal
priesthood of all the faithful in place of the clergy, the incapacity
of the human will to do good, and absolute predestination, the very
pillars of the doctrinal system of Lutheranism. The antitheses between
the two religions on the subject of indulgences and Purgatory were
likewise hushed up, and the differences in the veneration of the saints
had also vanished.\footnote
{\textit{Corpus Ref.}, XXVI, p. 290. Luther also maintains: ``\textit{Audita nostrorum confessione
primum communis vox et sententis ommnium fuit, nos nihil docere contra ullum fidei
articulum mneque contra scripturam \dots }'' Letter to Joh. Brismann, November 7, 1530;
\textit{Briefwechsel}, VIII, p. 311.}


Hence, honest candor, the preliminary condition of reunion, was
missing.

Luther himself censured the omission of some of his doctrines.
However, he did not wish to disavow the action of Melanchthon, his
indispensable, industrious, and respected mediator. He averred that
“he could not step as softly and quietly as he” (Melanchthon)\footnote
{Letter of May 15, 1530, to the Elector John of Saxony; Erl. ed., Vol. LIV, p. 145
(\textit{Briefwechsel}, VII, p. 335). In a letter to Jonas, July 21 (\textit{Briefwechsel}, VIII, p. 133),
Luther also says that the Confession conceals important doctrines.}
and regarded himself as incompetent to deliberate in such an assembly.

By order of the Emperor, Catholic theologians at once undertook
to compose a refutation of the “Confession,” in order to expose its
errors as well as its vagueness and its omissions. In addition to Eck,
Faber, and Cochlaeus, were some of the other Catholic apologists
whom we have heretofore mentioned: Usingen, Dietenberger, Wimpina, etc.
‘The opposition was officially asked whether they had any
other articles they wished to defend besides those contained in the
“Confession” which they had submitted. They replied evasively. The
tone of the hastily composed Catholic “Confutatio” appeared too offensive
to the Emperor and his advisers. It was revised and, after it had
been cast into the form of an answer given by the Emperor, was read
aloud on August 3 in the same hall in which the Protestants had been
permitted to submit their “Confession.”

The Emperor now ordered the Protestants to return to the pale of
the Christian communion, which they had deserted, lest he be compelled
to proceed against them in his capacity of “guardian and protector of
the Church,” which was his bounden duty as emperor. At
that time Charles was actually inclined to resort to military force,
but after October 30, in virtue of the representations of the Catholic
estates, he became somewhat reconciled to the idea of a general
council, not, however, until the time for waging a successful war had
passed.\footnote
{Cf. E. W. Mayer, \textit{Forschungen zur Politik Karls V. während des Augsburger Reichstags (Archiv für Reformationsgeschichte,} 1916, pp. 40 sqq.).}
The papal legate, Campeggio, was in favor of the strictest possible
execution of the edict of Worms.

In a state of painful anxiety, Melanchthon approached Campeggio
with proposals suggested by the delusive hope of coming to a mutually
satisfactory agreement. While he shuddered at the thought of an open
break, he did not wish to yield in principle, although many of the
Catholic leaders hoped for his conversion on account of his conciliatory
addresses. In the subsequent negotiations he became more and
more a pitiable figure.\footnote
{Grisar, \textit{Melanchtons ritselbafte Nachgiebigkeit auf dem Augsburger Reichstag (Histor.
Jahrbuch,} XLI [1921], pp. 257--267; \textit{Luther-Analekten} VI). Cf. Grisar, \textit{Luther}, Vol. II,
pp. 383 sqq.}

His depressed condition of mind is the only
thing that helps him over the charge of conscious deception. Many
friends of the Lutheran cause were opposed to him and to any kind of
approach between the two parties. Landgrave Philip of Hesse, to
signify his protest, left Augsburg precipitately.

The negotiations which the Emperor had authorized between seven
representatives of each faction proved fruitless. In vain did the
Catholic spokesmen, subject to papal approval, offer to have the lay-chalice
introduced in the Protestant districts, or to tolerate the marriage of
priests until the assembly of a general council. Every effort to
restore peace failed in consequence of the inflexible attitude of Luther,
who issued frequent letters from the castle of Coburg. Melanchthon
indicated his willingness to have the jurisdiction of the bishops restored,
but it was an insidious and ineffectual offer, because of the
underlying presupposition that the bishops would have to give free
scope to the new “gospel.”\footnote
{Wilhelm Walther, \textit{Für Luther}, p. 434: “Melanchthon was only too ready to acquiesce
in equivocal formulas and to make concessions which in truth could not be harmonized
with the ‘reservation that nothing may be conceded which contradicts the Gospel’; a reservation
which was constantly repeated.” The Protestant historian A. Berger (\textit{Luther
in kulturgeschichtlicher Darstellung} (1889), Vol. II, I; pp. 226 sq.) notes the weak attitude
of Melanchthon and says that, “objectively considered,” it was “a betrayal of the
Protestant conscience.”}
A smaller commission thereupon undertook
to effect an understanding. Its Catholic members were: Eck,
Cochlaeus and Wimpina, but their efforts were futile.

In the meantime Melanchthon’s tireless pen produced an “Apologia
Confessionis Augustanae,” which was directed against the Catholic
“Confutatio.” His party, however, did not succeed in having this
“Apologia” publicly read. Upon his homeward return, the author
privately published a Latin edition of it. The “Apologia,” like the
“Confession,” was soon regarded by Protestants as a symbol of their
faith.

Meantime the number of estates who declared their adherence to
the Augsburg “Confession” constantly increased. The original signers
were: Elector John of Saxony, Margrave George of Ansbach, Duke
Ernest of Braunschweig-Lüneberg, Landgrave Philip of Hesse, and
Prince Wolfgang of Anhalt. In addition to these names, the Latin
copy of the “Augsburg Confession” contained those of John Frederick, heir
to the throne of Electoral Saxony, and Duke Francis of
Braunschweig-Lüneberg. Nuremberg and Reutlingen were the only
cities to subscribe. Four cities which professed Zwinglianism, namely,
Strasburg, Constance, Memmingen, and Lindau, submitted a separate profession
of faith, composed by Bucer and Capito; it was
called “Confessio Tetrapolitana.” Other cities of Upper Germany,
though favoring the Reformation, kept aloof. In the course of the
deliberations at the diet of Augsburg a better understanding was
effected between the Lutherans and the Upper Germans with respect
to the Augsburg Confession, although Article X of the Confession
was supposed to be directed against Zwingli. Bucer was a smooth
politician and knew how to surmount the difficulties arising from that
document. After several of the cities represented in the diet had accepted
the Confession, Strasburg also declared its adherence at a conference
which was held towards the end of December, 1530, at
Schmalkalden. Thus everything conspired towards the creation of the
fateful League named after that city.

The Protestant leaders at the diet of Augsburg used the new
evangel as the basis of a political alliance designed to divide Germany.
Before his departure, the Landgrave of Hesse threateningly declared
that if he had to die for the faith, certain leaders of the opposition
would die with him.

After some delay, due partly to the Turkish menace and partly to
his own scruples, Emperor Charles issued a decree prohibiting all theological
innovations. The Protestants were ordered to accept the articles
upon which no agreement had been reached, by the fifteenth of April
of the next year, at the very latest. They vociferously objected to this
and at the same time refused to consent to the required intervention
against the German Zwinglians and the danger to the Empire caused
by them and by Zwingli at Zurich. Nevertheless, the Emperor, in his
\textit{Reichstagsabschied} of November 19, renewed the edict of Worms
with its severe measures, but at the same time referred the litigants to
the coming ecumenical council, which was expected within a year.\footnote
{Cf. Janssen-Pastor, III, pp. 251 sqq.; Grisar, \textit{Luther}, Vol. II, p. 384.}
Both the renewal of the edict of Worms as well as the Emperor’s
reference to the expected convocation of a general council proved
ineffectual. The edict could not be enforced because of the united
front of the opposition, and the council was postponed by Pope
Clement VII because of the fear that schisms would develop among
the faithful, because of the expectation of small benefit to those who
had separated from the Church, and, still more, because of the political
difficulties in the way of holding a council.

Thus the diet of Augsburg, which had been hailed with such great
expectations by the Catholics, due principally to the obstinate attitude
of the Protestants, in a certain measure furthered the unfortunate
schism. On December 12, Luther gloatingly reminded his elector that
the schemes of men “always turn out differently than expected, so that
one must say: I surely did not intend that. Pope and Emperor did not
succeed at Augsburg as they expected; nor shall they succeed henceforth.”
He imagines that his party is sustained by God and will
“remain with God.”\footnote{Erl. ed., Vol. LIV, p. 201 (\textit{Briefwechsel}, Vol. VIII, p. 331).}

Nothing illustrates Luther’s way of thinking and proceeding
more graphically than a close scrutiny of his behavior during his
sojourn in the lonely castle of Coburg at the time of the diet of Augsburg.
