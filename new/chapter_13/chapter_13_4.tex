\section{Luther’s Writings During His Sojourn at Coburg and the Following Months}

Some of the works produced by Luther’s tireless pen during this
period have already been mentioned.

Among the others, which are to be classed with them, his treatise
on Purgatory should be noted.\footnote
{“\textit{Widerruf vom Fegfeuer},” Weimar ed., Vol. XXX, ii, pp. 367 sqq.; Exl. ed., Vol.
XXXI, pp. 184 sqq.}
It was written by Luther to supply
the absence of any reference to this subject in the Augsburg Confession,
and to disclose all the disgraceful lies and atrocities of this papistical
doctrine which, he asserted, had been introduced allegedly for
the sake of filthy lucre.

Simultaneously with this work he published a tract in German and
Latin, entitled, “Some Articles which Martin Luther will uphold
against the entire School of Satan.”\footnote
{Weimar ed., Vol. XXX, ii, pp. 420 sqq.; Erl. ed., Vol. XXXI, pp. 121 sqq.}
It comprises forty theses, which
had been occasioned by the negotiations at Augsburg. Of these, no
less than ten are an attempt to demonstrate the astounding proposition
that the marriage of priests is to be regarded as a Christian institution
even according to the papists, and that those who inveigh
against it, therefore, deserve to be branded as “public assassins, robbers,
traitors, liars and miscreants.”

The book “On the Keys,” to which Luther devoted himself immediately
after, and which he rewrote twice, was his most important
work on the power of the Church.\footnote{Weimar ed., Vol. XXX, ii, pp. 435 sqq. Erl. ed., Vol. XXXI, pp. 126 sqq.}
He teaches here that sins are
not remitted by the Church in virtue of an (imaginary) power of the
keys, but by the word of grace entrusted to the congregation, of
which each individual avails himself in faith. If sins are to be retained
instead of forgiven, the congregation must cooperate; it must be the
“co-judge”; and hence, for the sake of discipline, sinners must be
properly denounced. It was an exaggerated and impracticable demand,
as he himself experienced in his several attempts to introduce
the ban.\footnote
{Köstlin-Kawerau, \textit{M. Luther}, I, p. 223: “Luther adheres to this, although he knew
how difficult it was to establish ecclesiastical discipline according to this principle in the
new evangelical churches.”}

The “Epistle of M. Luther on Translation and on the Intercession
of the Saints”\footnote{Weimar ed., Vol. XXX, ii, pp. 632 sqq.; Erl, ed., Vol.}
in its first and longer section is a defense of the principles
followed by him in translating the Bible. Among other things,
he undertakes to formulate an extensive justification of his arbitrary
insertion of the word “alone” in Rom. 3:28: “For we account a
man to be justified by faith ALONE, without the works of the law.”
The Catholics severely criticized him for inserting the word “alone”
in the interests of his doctrine. It was intended to strengthen the Lutheran
position, though it must be admitted that the legitimate meaning
of the text, as preserved by tradition, is not exactly incompatible
with the word “alone.” Luther insists most vigorously upon his interpolation.
“I will not have either the pope-ass or a mule for judge.”

“I would not answer such asses, nor reply to their vain, monotonous
babbling about the word \textit{sola}, otherwise than to say: Luther will
have it so and says, he is a doctor superior to all other doctors in all
popedom. Thus shall it be.” With an appearance of humor and a
high sense of superiority he repeats that, if there be any “papist who
would make himself obnoxious on account of the word sola,” he
“should be told straightway that Dr. Martin Luther will have it so.
\dots \textit{Sic volo, sic iubeo, sit pro ratione voluntas}.” He would also “rail
and boast for once against the blockheads, as St. Paul did against his
crazy saints” (2 Cor. 11:21 sqq.), etc.\footnote
{For further details see Grisar, \textit{Luther}, Vol. V, pp. 515 sqq.}
Only an abnormal person
with a deep-seated grudge could write in this manner. Apart from
this, the treatise here under review contains many good suggestions, in
particular concerning the task which he proposed to himself of faithfully
reproducing, in conformity with the genius of the German
language, the ideas of the sacred writer rather than their material
words.

The second part, which attacks the Catholic doctrine of the invocation
and veneration of saints as a “shameless lie of the pope-ass,”
constitutes but a loose appendix to this queer “Epistle.” Luther incidentally
admits that “it has been immeasurably painful” to him to
have “torn” himself away from the saints. He is well aware that the
veneration of the saints is an ancient heirloom “of all Christendom.”

Another literary product of his sojourn at Coburg Castle bears the
title, “Admonition relative to the Sacrament of the Body and Blood
of Christ.” Besides piously exhorting the evangelicals, it attacks the
doctrine of the Holy Sacrifice of the Mass as upheld by the Catholics
at Augsburg.\footnote{Weimar ed., Vol. XXX, ii, pp. 595 sqq.; Erl. ed., Vol. XXIII, pp. 162 sqq.}

By means of his work, “That Children Should be Urged to Attend
School,”\footnote{Weimar ed., Vol. XXX, ii, pp. 517 sqq.; Erl. ed., Vol. XVII, 2 ed., p. 377 sqq. Cf. my
article on “Luther” in the \textit{Pädagogisches Lexikon} of Roloff.}
Luther designed to remove a drawback which vexed him
very much in connection with the appointment of pastors.

As early as 1524 he had discussed this matter in his “Appeal to the Aldermen
of all Cities.” In consequence of the religious contentions and the
social revolution, the schools had deteriorated very much. He now laments
and fears that eventually there will be no fit candidates available for the
pastoral office, as a consequence of which there will be but one pastor to
every ten villages. The decline of the schools would likewise prove dangerous
to the secular offices. The proposals which he develops for the education of
youth are good; but here again he treats the Catholic schools of the past
with flagrant injustice. According to Frederick Paulsen, author of a “Geschichte
des Gelehrten Unterrichts,” he regards the “entire basis of artistic
education,” as given before his time, as “the work of the devil.”\footnote{Grisar, \textit{Luther}, Vol. VI, p. 21.}

Education, he claims, ought to be founded exclusively on the Gospel.

The civil authorities are systematically invited in this work to exert
pressure upon parents who are remiss in the discharge of their educational duties.
This function, moreover, should be exercised by the civil authorities in the
interests of procuring suitable candidates for the public offices, “when they
see a lad who displays ability.” Luther does not advocate universal compulsory
education on the part of the State. “It is unfair,” Gustav Kawerau
truly says, “to represent Luther as the harbinger of universal compulsory
education.”\footnote{\textit{Ibid.}, p. 8.}
Neither is there any justification for the assumption that
enthusiasm for the humanities and the advance of science and education in
themselves constituted the starting-point of this treatise. “The religious
viewpoints alone are the decisive ones,” remarks Julius Böhmer, a Protestant
author. Another Protestant, F. M. Schiele, says Luther was concerned with
devising a remedy for the “collapse of an educational system which had
flourished throughout Germany”--a collapse “brought about by the preaching
of Wittenberg.” The damage could be remedied only with great difficulty
and very slowly in the course of subsequent years.

Schiele holds that the statement that “Luther’s reformation gave a general
stimulus to the schools and to education generally,” must “melt away
into nothing.”\footnote{\textit{Ibid.}, pp. 20, 26 sq.}

Whilst various other writings of Luther may be passed over, there
is one more work of his which is deserving of mention, as it reveals
a more pleasing aspect of the man. It is his German edition of the
fables of Æsop, intended for the use of school children. This work
was intended, on the one hand, to furnish a diversion from serious
thoughts; on the other, Luther sincerely desired by his edition of
Æsop to provide the young with “the finest possible precept, admonition,
and instruction” adapted to their “external life in the world.”
The adaptation was couched in classical language and the indecent
admixtures of former editions were omitted. Luther intended to make
it a “jovial and lovable, and withal a respectable and decent Æsop.”
The projected edition was never completed. Only parts of it are
available.\footnote{\textit{Werke}, Weimar ed., Vol. L, pp. 440 sqq.; Erl. ed., Vol. LXIV, pp. 349 sqq.}

They are valuable on account of the suitable German
proverbs which the editor has inserted. In general, his works abound
in proverbs, of which he made a collection in 1535 or 1536.\footnote{Weimar ed., Vol. LI, pp. 645 sqq.}

Two controversial works of Luther remain to be mentioned as belonging
in a certain sense to his Coburg productions. Both were directed against
the diet of Augsburg and were issued soon after Luther’s return to Wittenberg,
whilst he was still in an agitated frame
of mind and filled with the thoughts of his sojourn at the castle of
Coburg. They are entitled: “Warning of Doctor Martin Luther to
his Dear Germans,” and “Gloss on the Pretended Imperial Edict.”

The “Warning”\footnote{Weimar ed., Vol. XXX, iii, pp. 276 sqq:; Erl. ed., Vol. XXV, 2 ed., pp. 1 sqq.}
is directed above all else against the use of force
on the part of the Empire and the Emperor, which he believed to be
impending. Casting the most vulgar and insulting aspersions upon
the Catholic members of the diet, he advises his “dear Germans” not
to come to the aid of the papists in the event of war or insurrection.
Necessity demands, he says, that resistance be offered to every violent
attack.\footnote{On the “Warning” cf. Grisar, \textit{Luther}, Vol. II, pp. 388 sq., 391 sq. Vol. III, pp. sq.,
442 sq.; Vol. IV, p. 316.}
The suggestive force of this impassioned work was calculated
to inflame the minds of the masses, who had embraced the new
theology, with a determination to offer stern resistance. This book
was read aloud to the mob in public squares and markets and from it
the people learned that if Dr. Martin Luther would be executed, a
large number of bishops, priests, and monks would go with him.
Luther here spoke to the masses as “the Prophet of the Germans,”
claiming that it was necessary for him to adopt this title against the
papists and asses.

In the “Gloss”\footnote{Weimar ed., Vol. XXX, iii, pp. 331 5993 Erl. ed., Vol.
XXV, 2nd ed., pp. 49 sqq.}
he proclaims with the Psalmist (91:13): “In
the name and calling of God I shall walk upon the lion and the asp
and trample under foot the young lion and dragon, and this ”shall
be commenced during my life and accomplished after my death. His
self-consciousness rises to a dizzy height against the “insipid cattle
and filthy swine” who would conceal the pretended imperial edict,
which is denounced as an invalid, unjust, and surreptitious decree.
All were warned to leave untouched his principal dogma of justification
by faith alone. “Thus I, Doctor Martin Luther, most unworthy
evangelist of our Lord Jesus Christ, declare that the Roman emperor,
the Turkish emperor, the Tartar emperor, the pope, all cardinals,
bishops, priests, princes, lords, the whole world and all the devils shall
leave this article stand; and in addition, they shall have the flames of
hell about their heads and no reward. This be my, Dr. Luther’s, inspiration
from the Holy Ghost.” The Catholic leaders saw in this
declaration an inspiration from an entirely different source. The
quixotic exclamations in which Luther indulged at that time almost
approach the borderline of insanity. It is less difficult to understand
why Luther should invoke the nationalistic sentiments of his “dear
Germans,” for he wished to incite them against their alien oppressors,
especially against “the principal rogue, Pope Clement, and his
servant, the legate Campegius.” In both of the works here under
consideration he repeats the most revolting lies about the Augsburg
diet; as, for instance, when he asserts that it was evident at Augsburg,
and many admitted it, that he was in the right and that the
Catholic Church was steeped in errors, but tyrannical obstinacy had
triumphed.

Luther was most furious against Duke George of Saxony, the
protagonist of the Catholic cause, in the months following the diet.
On Easter, 1531, appeared his diatribe “Against the Assassin of Dresden,”
which is a monument of hatred against a noble prince who
remained loyal to the Emperor.\footnote
{Weimar ed., Vol. XXX, iii, pp. 446 sqq.; Erl. ed., Vol. XXV, 2 ed., pp. 108 sqq.}
 In his published reply to Luther’s
“Warning to his Dear Germans,” George had defended the diet, the
empire, and Catholicism, and represented Luther as a rebel. This forceful
reply was published anonymously and is lost, except for a few
lines which have been preserved by Cochlaeus. Another reply directed
against Luther’s “Gloss” was published by Francis Arnoldi
under the title: “Reply to the Booklet Launched by Martin Luther
against the Imperial Recess.”\footnote{Erl. ed., Vol. XXV, 2 ed., pp. ?? sqq.}
Its author was a pastor in Cöllen
near Meissen, who was well acquainted with Duke George. Arnoldi’s
“Reply” most probably embodied some ideas suggested by the Duke.
Luther, in his libel “Against the Assassin of Dresden,” endeavored
to defend himself particularly against the charge of sedition, which
Duke George and others made against him. The word “assassin” in
the title signifies “calumniator.” But Luther is not satisfied with
defending himself; he once more attacks the “bloodhounds” of the
opposition and announces that he will continue his attacks in perpetuity.
He says he had humiliated himself sufficiently, nay, too often,
and it would now be his boast that he would bubble over with invectives
and imprecations against the papists. At the close he admits
that he is unable to pray without cursing. He could not utter the petition:
“Hallowed be Thy Name,” without adding the words: “Accursed, damned, disgraced
shall be the name of the papists and of all
who blaspheme Thy Name.” “Verily,” he says, “I pray thus every
day.” And he believes that God hears his prayers; for even now He has
miraculously caused “this terrible diet to come to naught.” “In spite
of all, however, I maintain a kindly, friendly, peaceable, and Christian
heart towards everybody; even my greatest enemies know this.”

In reply Arnoldi published an answer “To the Libel,” etc., which
was again inspired by Duke George, who had been so grievously insulted
by Luther.\footnote{\textit{Ibid.}, pp. 129 sqq.}
 Like the first work which bore Arnoldi’s name,
this one, too, is composed in a very blunt style. It was the Duke’s desire
that free vent be given to his sentiments of indignation and that
satisfaction be rendered to the maltreated Catholics by way of a severe
attack upon their opponent. On account of the religious revolt, the
Duke had suffered much in his duchy, despite sincere efforts to abolish
the prevailing abuses. The monasteries and the clergy were profoundly
shaken by the religious revolt, and his people were being corrupted.
Luther had only himself to blame if the Duke and Arnoldi, animated
by love of the Church, the Emperor and the Empire, and convinced
that they were standing before an abyss, to a certain extent imitated
his offensive language by using such epithets as bloodhounds, whore-mongers,
etc. The historian cannot shirk the unpleasant duty of quoting
some passages from these violent replies. There is first of all the
quotation which Cochlaeus has preserved from the pamphlet entirely
composed by Duke George.\footnote
{Cochlaeus, \textit{De actis}, etc. (1565), folio 211b; Erl. ed., Vol. XXV, 2 ed., p. 89.}
