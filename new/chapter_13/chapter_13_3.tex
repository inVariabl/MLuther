\section{The “Proviso of the Gospel”}

Luther learned with satisfaction that the Augsburg Confession had
been read in the presence of the estates of the Empire. But he did not
share Melanchthon’s expectation that it would lead to some sort of
reunion. His opinion was that rejection was the only thing to be expected
from the “obduracy” of his enemies. He would “not allow
himself to be discouraged, no matter what the course of events” at
Augsburg might be, he declared to Melanchthon.\footnote{On June 29; \textit{Briefwechsel}, VIII, p. 43.}

In addition to the latter, Jonas, Agricola, Spalatin, Brenz, and
others were active in promoting Luther’s cause. It was to them and to
Melanchthon that he wrote: “If we fall, Christ, the Ruler of the
world, falls with us.”\footnote{On June 30; \textit{ibid.}, p. 51.}
 The Emperor, though well-intentioned, is
unable to prevail against so many devils. Should he, however, “take a
stand against the plain Scriptures or the Word of God,” his decision
cannot bind us.\footnote{Cf. Köstlin-Kawerau, \textit{M. Luther}, Vol. II, pp. 216 sq., 225.}
 There can be no question of restoring the property
of the Church. It would be an advantage for his partisans to demand
a council, since the demand could not be satisfied.

At the beginning of the negotiations proposed by Melanchthon,
which at first concerned external matters only, Luther declared himself
in favor of declining all concessions contrary to the Gospel, and
demanded courage and perseverance on the part of his representatives.
He would prefer--thus he wrote on July 15--that they should depart for
home. “Ever and anon homeward, always homeward,” is his
watch-word.\footnote{Letter to Jonas, etc.; \textit{Briefwechsel}, VI, p. 113.}
 His letters to the Elector John of Saxony also demonstrate
his negative attitude.

Under date of August 26, he writes a curious letter to Melanchthon. He
encourages his pusillanimous friend, whom he always treated
with great indulgence, in his ambiguous proposals: “I am certain that
you will be unable to commit aught, except at the utmost a personal
offense against me, so that we shall be charged with perfidy and vacillation.
But what will the consequence be? Matters may easily be
remedied by the steadfastness and the truth of our cause. True, I do
not wish that it should so happen; but speak in such wise that, if it
should happen, despondency do not ensue. For, once we shall have
attained peace and escaped violence, we shall easily make amends for
our tricks (lies) and failings, because God’s mercy rules over us. ‘Do
manfully, and let your heart take courage, all ye who wait for the
Lord’ (Psalm 27:14).” Later editions have omitted the word
“lies (\textit{mendacia}) which appeared in brackets between “tricks” and
“failings.” The textual tradition, however, renders it probable that
the deleted word appeared in the original, which is lost. But even if it
had not appeared there, Luther’s mind is sufficiently expressed by the
words “tricks” and “failings” (\textit{doli et lapsus}).\footnote
{On August 28; Briefwechsel, VIII, p. 235. For more details on this letter, see my
article on the same in the \textit{Stimmen aus Maria-Laach}, 1913, No. 3, pp. 286 sqq.}

His strictures grow more severe in course of the following month, especially
when, on September 20, he receives reports from the Nuremberg representatives
at the Augsburg diet, bitterly complaining of Melanchthon’s obsequiousness. “I
am actually bursting with anger and indignation,” he wrote to Jonas on this
occasion. “I beseech to cut the matter short and come back home.” “They have
our Confession and the gospel \dots If war is to come, then let it come. We
have done and prayed enough. The Lord has given them over to us as a holocaust
‘to reward them according to their works’ [2 Tim. 4:14]; us, His people, He
will save from the fiery furnace of Babylon \dots What I have written for you
is meant for all.”\footnote{Grisar, \textit{Luther}, Vol. II, p. 391.}

Writing to his friend Link, he expresses the hope that no definite
concessions will be made to the opposition; Christ will transform
all offers “into a lance by which to play a deceptive game with the
opponents who intend to play us false; Christ is preparing their destruction
in the Red Sea.”\footnote{Köstlin-Kawerau, \textit{M. Luther}, Vol. II, p. 237.}

In a more considerate tone he pleads with Melanchthon, who is burdened
with so many cares, to furnish him with more accurate information; for
he fears that he will be made a victim of violence and
deception.

In order to console Spengler, his informant from Nuremberg, who
had indulged in laments, he wrote to him: “Though Christ may
appear to be somewhat weak, this does not mean that He is pushed out
of His seat \dots In the proviso concerning the Gospel, there are embodied
snares (\textit{insidiae}) other than those which our adversaries can
employ against us.”\footnote{Grisar, \textit{Luther}, Vol. II, p. 385.}
Hence, in the last analysis, the proviso concerning
the Gospel and its secret snares (\textit{insidiae}) was expected to save
everything. This means: No agreement may be regarded as valid or
binding if it runs counter to the new gospel, even though such concessions are made.

In the meantime events at Augsburg followed the course we have
already described.

The greatest sensation was produced by Melanchthon’s concession
to recognize the jurisdiction of the bishops under certain conditions.

In treating of this proposal, Luther, on September 23, writes to his
confidant, Nicholas Hausmann, to the effect that the main condition
for the recognition of episcopal jurisdiction was this, that the bishops
“were to attend to the teaching of the Gospel”; and he adds in all
seriousness that nothing had been done in this direction and hence
his enemies had conjured up their own destruction.\footnote{\textit{Briefwechsel}, VIII, p. 269; Grisar, \textit{Luther}, Vol. II, pp. 387 sq.}
 He speaks as if
the concession was not a mere pretense.

Still more characteristic is Luther’s excuse after the close of the
diet, addressed to Landgrave Philip of Hesse in response to the latter’s
complaint. Here he frankly admits the true nature of the proposed recognition
of episcopal jurisdiction: It was not at all to be
feared that this proposal would be accepted; moreover, it never could
have been accepted; but, he avers, it served “to raise our repute still
further” (\textit{i.e.}, to capture public opinion). The offer would have been
a mistake only if it had been adopted. Philip, therefore, ought to be
satisfied; in his next work, he (Luther) proposes to discuss at length
the unfairness of his opponents.\footnote{\textit{Briefwechsel}, VIII, p. 295; Grisar, \textit{Luther}, Vol. II, p. 388.}

In this work, entitled “Warning,”
he actually boasts of the conciliatory attitude of his partisans at Augsburg.
Nevertheless, all peace overtures were lost upon those obstinate
men. “Our offers, our prayers, our cries for peace” were all wasted.\footnote{\textit{Ibid.}, pp. 388--389.}


The real nature of the “proviso of the Gospel” is revealed only if
due consideration is given to all these texts.

Towards the end of his sojourn at the castle of Coburg, Luther was
visited by Martin Bucer and John Frederick, the son and future
successor of the Elector of Saxony. Both found him in comparatively
good health. His exterior appearance had changed, due to a long
beard which he wore until his return to Wittenberg. Bucer’s object
was to effect an approach between his party, which sympathized with
Zwingli, and Luther, relative to the controversy on the Eucharist.

He by his artful diplomacy succeeded in impressing Luther
favorably by means of a vague formula on the Real Presence. After
the termination of the diet, Luther probably hoped to resist the Emperor
with a more numerous and more compact following. Prince
John Frederick, eager to show his respects to Luther, presented him
with a precious signet-ring bearing the latter’s “escutcheon,”--a
heart overlayed with a cross in the midst of a rose. Luther at once
found a mystical interpretation for this symbol, by referring it to
his doctrine and position.

With a certain resignation he discussed with these and other callers
the unfavorable decision of the diet. In reality, and as a matter of
course, he did not expect and could not have expected any other.
In his letters he now entrusted everything to Providence.”
His letters and writings at this period contain pious and beautiful sentiments
and abound in phrases calculated to console himself and his friends.

Some historians love to extol the excellence of the prayers which he
composed during his solitude. Among others they refer to a collection
of exhortations which he compiled at that time. It is, in reality, a
treasury of elevating thoughts, taken from Holy Writ to arouse confidence
in God.\footnote
{Weimar ed., Vol. XXX, ii, pp. 700 sqq.; Erl. ed., Vol. XXIII, pp. 154 sq. Cf. Haussleiter
in the \textit{Neue Kirchliche Zeitschrift}, 1917, Pp. 149 sqq.}
It is apparent at once, however, that all the texts
have been selected to serve as a defense and confirmation of the personal
standpoint which Luther assumed in his contentions. The same
is true of most of his prayers. They are designed to corroborate his
presumptive right. Every true prayer ought to contain, above all
else, a petition to know and bow before the will of God, even as related
to the whole conception of life. In the prayers which Luther
composed such willingness is hardly detectable. He will not concede
the possibility that another course besides the one which he has entered
upon may be the right one.

This observation is applicable also to the frequently cited prayer which Vitus
Dietrich is supposed to have heard from Luther’s lips, and which culminates in
the words addressed to God: “Thou hast power to extirpate the persecutors of
Thy children; if Thou dost not do-it, the danger is Thine. What we have done,
we had to do.”\footnote{Cfr. Grisar, \textit{Luther}, original German ed., Vol.
III, p. 998.}

Such is not the spirit of resignation as expressed in the \textit{fiat voluntas
tua}, the basis and crown of all prayer; but it is a command addressed
to God to do the bidding of the supplicant. Dietrich, who was an
enthusiastic disciple, also tells us that Luther, while sojourning in the
castle of Coburg, devoted at least three hours daily to prayer. It is
not unlikely that many an hour may have been spent by him in sighing for
relief, especially when he was unfit for work and in periods
of protracted sickness and spiritual affliction. Moreover, his work of
translation undoubtedly offered him many opportunities of meditating on
the Psalms and other Biblical texts. Hence, it is probable that
his customary prayer may have often been protracted. But it is difficult
to believe that Luther devoted himself regularly to prayers for
more than three hours daily. The strenuous literary work which he
performed demanded a most diligent use of time.
