\section{Luther in the Castle of Coburg}

The ancient castle of Coburg rises above the city of the same name
on the south side of the Thuringian Forest, in the midst of attractive
rows of hills. A road, rising gently at first, but soon growing steeper,
leads up to the castle, which once upon a time was rightly styled
“Frankish Crown.” Even to-day the visitor beholds in it the best type
of the massive, crown-like castle of the Middle Ages, simply yet magnificently
constructed.

During the diet of Augsburg, Luther resided in the topmost portion
of the edifice, reserved for princes. His suite is still extant. As he
walked upon the detached so-called High Bastion on the side of the
courtyard, he had an unobstructed and charming view of the splendid
landscape in the direction of the city where the fate of his theological
system appeared to be at issue. As an outlaw, Luther could not enter
Augsburg; therefore the Elector of Saxony, who owned this castle,
had assigned it to him for a residence, whence he could easily correspond
with his representatives at the diet.

This lonely and almost unoccupied castle became a “Sinai” and a
“hermitage” to him. In his first correspondence “from the kingdom
of the jackdaws” he humorously describes the antics of the birds about
his lofty room, comparing them with the garrulous and agitated assembly
at Augsburg. Good humor, he says, was necessary to “repulse the heavy
thoughts which rushed in upon him, so far as they
permitted themselves to be banished.”\footnote
{Köstlin-Kawerau, \textit{M. Luther}, Vol. II, p. 196,}

The thoughts which occupied his mind during well-nigh the entire
time of his sojourn at the castle of Coburg, and which often tortured
him, not only concerned the diet; he was afflicted by deep anxiety
regarding the Turkish menace to Germany, no less than by interior
“temptations” against his teaching.

Luther arrived at Coburg on April 15, 1530, and remained in the
castle up to October 5. Aside from the inevitable visitors, his only
company consisted of a young pupil, Master Vitus Dietrich, and a
nephew, Cyriac Kaufmann, a student at Wittenberg.

The first task which he undertook to discharge was the composition
of a violent pamphlet by which he intended to intimidate the clerical
members of the diet. It appeared under the title: “A Warning to the
Clergy assembled at the Diet of Augsburg.”\footnote
{Weimar ed., Vol. XXX, II, pp. 268 sqq.; Erl. ed., Vol. XXIV, 2nd ed., p. 356.}
With passionate exaggeration
he reproaches them on account of their immoral lives, the
abuses in the government of the Church, and eulogizes the Reformation.
He threatens them with revolution if they obstruct his gospel.
In terrifying words he calls out to the assembly: “Alive, I am your
plague; dying, I am your death; for God has instigated me against you.
I must be unto you, as Osee says (XIII, 7 sq.), a bear and a lion in the
way of Assur. Ye shall have no peace before my name, until you
amend your ways or perish.”

Then he worked on his translation of the Bible, especially Jeremias
and Ezechiel. Whilst engaged in the study of Jeremias, his soul was
overcome by a profoundly mystical mood. He was captivated by the
prophecy concerning Gog and Magog, whose names pertained to the
most remote and barbarous period of paganism, and were connected
with the destruction within the kingdom of God. Luther interpreted
the prophecy as signifying the devastating incursion of the Turks. He
published this interpretation in a special work immediately after the
publication of his “Warning to the Clergy.”\footnote
{Weimar ed., Vol. XXX, II, pp. 223 sqq.; Erl. ed., XLI, pp. 220 sq. In this little tract,
Luther, under the influence of a strong illusion, claims that the Turks had special designs
against him and his little band of followers, but Christ, according to the prophecy, will
destroy both pope and Turk, “with His splendid advent, which we daily expect.”}
From the Psalter, with
which he also occupied himself, he next selected for publication, under
the title, “The Beautiful Confitemini,” his interpretation of Psalm 117
(Vulgate), which he held in great esteem.\footnote
{Weimar ed., Vol. XXXI, I, pp. 65 sqq.; Erl. ed., Vol. XLI, pp. sqq.}
It is the Psalm which the
Breviary prescribes for recitation on Sunday, beginning with “\textit{Confitemini
Domino quoniam bonus \dots Dicat nunc Israel},” etc. Luther
was wont to apply this Psalm to his dangers and his confidence in
salvation, the latter especially because of the passage: “I shall not die,
but live: and shall declare the works of the Lord.” In order to relieve
himself in his physical and mental sufferings he inscribed these words
with musical notations on the wall of his room at Coburg castle,
where they were seen by the physician Ratzeberger twenty years
afterwards.

Later, his anxiety concerning the Augsburg diet once more set his
pen in motion. He published an “Open Letter” to Archbishop Albrecht of
Mayence, who, as yet, was not sufficiently pliable to suit
Luther, but took a conciliatory attitude.\footnote
{Weimar ed., Vol. XXX, ii, p. 397; Erl. ed., Vol. LIV, pp. 159 sqq. (\textit{Briefwechsel},
VIII, pp. 84 sqq.).}
In terms far milder than
those of his “Warning,” he demands that, since it was impossible for
them to unite, the rival religious parties be unmolested in their respective
professions of faith.

In consequence of physical and mental ailments, his literary labors
became more and more difficult. His afflictions were partly a result of
nervous over-excitement, and partly an effect of the hasty and impassioned
labors which he performed. He had not been well even
before he left Wittenberg. Beginning with the end of January, 1529,
his melancholia was aggravated at times by violent spells of dizziness
and a ringing noise in the head. On January 1, 1530, he said in a sermon
at Wittenberg that he would not ascend the pulpit any more
because of his disgust at the indifference of the people towards the
Word of God. According to a remark of the editor of the sermon in
the Weimar Edition, this declaration admits that “the only possible
explanation of this step is a pathological one.”\footnote
{Grisar, \textit{Luther}, Vol. VI, p. 168.}
In May he found it
impossible to work for weeks at a time on account of buzzing sensations
which he described as “thunder in the head,” and a tendency to
swoon.\footnote{\textit{Ibid.}, pp. 99 sqq.}

He assigned the cause of his afflictions to the devil, who enlivened
Luther’s imagination with peculiar images during his sojourn at Coburg.
The ex-monk firmly believed in the Satanic apparitions and
effects which were reported to him at that time. Thus he declared
that he had seen a large host of mysterious spirits, who, coming from
Cologne, caused themselves to be carried across the river at Spires and
marched towards Augsburg to attend the diet. “They were evil spirits,
devils in disguise.” Melanchthon regarded them as omens of a “terrible
revolution,” and his son-in-law, George Sabinus, described the
apparitions in poetical form. Luther afterwards defied the wrath of
these spirits by exclaiming: “Let them have their way--those spectre-monks
of Spires!”\footnote{Cfr. Grisar, \textit{Luther}, Vol. II, p. 387; Vol, VI, p. 209.}

Luther avidly accepted the report of Bugenhagen, who wrote from Lübeck
at the time that the devil had testified for the new gospel through the medium of a maiden who was
possessed by him. “The cunning demon,” he wrote, “designs prodigies.”\footnote
{\textit{Ibid.}, Vol. III, pp. 410 sq.}

Concerning himself, he complains in a letter to Melanchthon (May
12, 1530) that when he was alone (Dietrich and Kaufmann being
absent) the devil sent “his messenger” to him and so overpowered him
with gloomy thoughts that he was driven out of his room and forced
to seek other companions. “I can hardly await the day,” he adds in a
characteristic phrase, “when we shall see the great power of this
spirit and, as it were, his almost divine majesty.”\footnote
{Briefwechsel, VII, p. 332: “\textit{Habuit Satan legationem suam apud me}.” He is eager to
see his \textit{plane divina majestas}.}

At Coburg he saw the devil in a phantastic visual illusion. About
nine o’clock, on the evening of a rainy day in June, as he stood at his
window and looked out over the little forest near by, as Vitus Dietrich
bears witness, he saw “a fiery, flaming serpent, which, after twisting
and writhing about, dropped from the roof of the nearest tower
down into the wood. He at once called me and wanted to show me
the ghost (\textit{spectrum}), as I stood by his shoulder. But suddenly he saw
it disappear. Shortly after, we both saw the apparition again. It had,
however, altered its shape, and now looked more like a great flaming
star lying in the field, so that we were able to distinguish it plainly,
even though the weather was rainy.”\footnote{Grisar, \textit{Luther}, Vol. VI, p. 130.}

In his fright, Luther regarded
the apparition as the devil. It may have been one of the inmates of the
castle passing by with a torch or a brightly shining lantern which cast
a reflection on the roof, the woods and field. Whoever visits the place
will at once perceive that this is a plausible explanation. Luther, however,
was so sure he had seen the devil that he mentioned it in the following
year to those who were present to aid him in the revision of his
German translation of the Psalms. He said: “I saw my devil flying
over the wood at Coburg,” adding that Psalm 18 (Vulg. 17), verse
15, which they were just then discussing, speaks of a \textit{materia ignita}.\footnote{\textit{Ibid.}}


His morbid fancy was followed by an unusually violent buzzing in
the head and an increased tendency to faintness in the succeeding
night--symptoms which indicated that his nervousness had reached
a crisis. To young Dietrich this was but a new proof that all the ailments
of his master were caused by the devil who had just appeared
to them. It is not surprising that a blotch of ink on the wall of the
room which Luther occupied in this castle was later attributed to Old
Nick, just as the legendary one in the Wartburg.\footnote{\textit{Ibid.}, Vol. II, p. 96.}

Outside of this
case, we know of no other manifestation of the evil spirit to Luther.

Luther himself tells us many details of the spiritual “temptations”
to which he was subject at this time. He compares his soul, assailed
by temptations, to a land dried up by heat and wind and thirsting for
water.\footnote{\textit{Ibid.}, Vol. II, p. 390, and Vol. V, p. 346.}

He says that he is far stronger in his public controversies than
in these personal struggles.\footnote{\textit{Ibid.}, Vol. II, p. 390.}

To Melanchthon he writes that he would
rather endure this torture of the body than “that hangman of the
spirit who \dots will never stop until he has gobbled me up.”\footnote{\textit{Ibid.}, Vol. V, p. 347.}

After
his return to Wittenberg, he recalled these spiritual struggles with
horror. He was but forty-seven when he wrote to Amsdorf: “I now
am really beginning to feel the weight of my years, and my powers
are going. The angel of Satan [2 Cor. 7:7] has indeed dealt hardly
with me.”\footnote{\textit{Ibid.}}

On another occasion he said to Dietrich at the Coburg,
if he were to die (he had already selected a place for his grave),
and his body were cut open, his heart would be found all shriveled
up “in consequence of my distress and sadness of spirit.”\footnote{\textit{Ibid.}, Vol. V, p. 348.}


These well-attested spiritual agonies of the ex-monk, which were
naturally accompanied by qualms of conscience, stand in striking
contrast to the narratives of most Protestant biographers, who laud
the spiritual repose, the interior joy, and unflinching faith of Luther
in the days which he spent in the castle of Coburg. It is true that when
storms assailed him he constantly sought comfort in the idea that his
restlessness was attributable to the devil and that he finally overcame
his scruples with increased defiance.

At that time he advised one of his pupils, Jerome Weller, how to conduct
himself when assailed by “temptations.” The latter was tormented by great
fear as to the forgiveness of his sins and the spiritual condition of his soul.\footnote
{In July (?), 1530; \textit{Briefwechsel}, VIII, pp. 159 sqq. Cf. the letter to the same, dated
August 15, 1530; \textit{ibid.}, p. 188.}
Luther assures him that he also had such temptations, which were caused by
the devil, who insidiously persecutes us on account of our belief and trust in
Christ. Hence, when tempted “to despair and blaspheme,” one should disregard
the temptation as much as possible. Avoid being alone, he advises him;
jest with my wife, imbibe somewhat more freely. Such temptations are useful.
By means of them he himself had become “a great doctor.” Moreover,
Weller should not fear on account of minor moral infractions. When thus
afflicted, “some kind of sin should be committed,” in order to manifest one’s
hatred and contempt for the devil. If the devil tempt me, he should “know
that I acknowledge no sin and hold myself guiltless. The Ten Commandments,
with which the devil afflicts and tortures us so much,” ought to be
removed from our sight and our mind. Satan is simply to be referred to our
Saviour, the Son of God.\footnote
{“\textit{Nonnunquam largius bibendum, ludendum, nugandum, atque adeo peccatum aliquod
faciendum in odium et contemptum diaboli. Utinam possem aliquid insigne peccati designare
modo ad eludendum diabolum, ut intelligeret, me nullum peccatum agnoscere ac me
nullius peccati mihi esse conscium.}”}

In this strange letter Luther also recalls his monastic days. It is possible
that, in the lonely life which he led in the castle of Coburg, his monastic
past may have impressed itself upon him more forcibly in contrast with his
present career; just as, during his seclusion at the Wartburg, he was similarly
impressed with the significance of his monastic vows. In his letter to young
Weller, he refers to the fearful and terrible thoughts (\textit{horrificae et terrificae
cogitationes}) with which he was tortured while a monk.

He persuaded himself more and more that the feeling of depression
which he had experienced in the monastery was entirely a result of
his observance of the Catholic doctrine of virtue and merit. He now
held that a doctrine which makes piety dependent upon meritorious
works, instead of on faith alone, was unable to give peace, but could
only engender misery and fear in the soul. It was only after he had
discovered his new Gospel that the way of interior peace opened to
him.

This is Luther’s legendary version of his monastic life, an interpretation
of his youthful experience made in after years. It is a
weapon which after his sojourn at the castle of Coburg he began to
use with predilection in his fight upon the ancient Church.\footnote
{For additional details see Grisar, \textit{Luther}, Vol. VI, pp. 187 sqq.}

Formerly he had hardly made this charge; but now he claims to have been
a pious monk, “one of the best,” according to the Catholic ideal--a monk
who languished unto death in the performance of the works of papism, with
its fasting, vigils, freezing, etc.; and if ever anyone entered Heaven by such
“monkish” practices, he, too, had been determined thus to get there.\footnote{Köstlin-Kawerau, \textit{M. Luther}, Vol. II, p. 305.}

Hence, even if he was driven to despair in consequence of it, he was well acquainted
by personal experience with the “over-sweetened infernal poison cake” and
the untenableness of the Catholic doctrine of good works, which is sure to
make all men as unhappy as he had been in the monastery.

In making these charges he fails to take into consideration that the unhappy
state in which he found himself after his apostasy was not a result of
the doctrine and practice of the Catholic Church, but rather a product of
his own over-wrought and sickly condition; that a contributory cause was
his willfulness, as opposed to the discreet spirit of the rule and the direction
of his religious superiors; and that thousands have attained to the greatest
interior happiness by the conscientious observance of the Evangelical
Counsels and the performance of good works.

The false notion referred to crops out in the writings which he issued
from the castle of Coburg. Thus he says: “If a conscience is intent upon its
works and builds on them, it is erected upon loose sand; it is ever slipping
and sliding away; it must ever be seeking for works, for one and then for
another, and ever more and more, until at last even the dead are clad in
monks’ cowls, the better to reach Heaven.” However, by means of his new
doctrine, he had prevented this calamity.\footnote{Grisar, \textit{Luther}, Vol. VI, p. 230.}

The legend of the emancipated
holy monk Martin is utilized after his return to Wittenberg in the sermons
which he commenced to preach in 1530 on chapters VI to VIII of St. John’s
Gospel, where he says that he had “mortified and tortured” himself like
others, nay, even more than they, and accomplished thereby only this, that
while “one of the best” of monks, he was in despair and so far removed from
the faith that he “would have been ashamed to assert that Christ was the
Redeemer.” The papacy did not want a Redeemer, but wished to achieve redemption
by means of its works.

Improbable though it was, this legend of Luther’s monastic experience
became increasingly prominent up to the close of his life,
when it grew still more pronounced, and imposed itself upon countless
thousands. It is still widely believed to-day.
Besides the tribulations which filled the soul of Luther during his
abode in the castle of Coburg, the death of his aged father depressed
him greatly. Hans Luther departed this life on May 29, “strong in
his faith in Christ,” as Martin learned.\footnote{Köstlin-Kawerau, \textit{M. Luther}, Vol. II, p. 209.}
 The news of his father’s
illness having been communicated to him, he addressed a consolatory
letter to him from Wittenberg in the middle of February. Gladly, he
writes, would he have visited him, had the journey not been fraught
with such grave peril to himself; “peasants and princes” were opposed
to him, and he did not dare “to tempt God by exposing himself to
danger.”\footnote{February 15, 1530; Erl. ed., Vol. LIV, p. 130 (\textit{Briefwechsel}, VII, p. 230).}
 Yet it was only a matter of a short journey within the
territory of the Elector. The words quoted testify to the isolation in
which this once popular man now found himself. After the Peasants’
War his popularity had waned. Many of the lower classes regarded
him as their oppressor, while the upper classes were largely at war
with him because they had enriched themselves by robbing the
Church.\footnote{On Luther’s declining popularity, cf. Grisar, \textit{Luther}, Vol. VI, pp. 75 sqq.}
But more of this anon.
