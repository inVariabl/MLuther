\section{The Militant Spirit of Luther in Word and Picture}

During the first half of the forties, the thought of death frequently
engaged the mind of Luther. His apprehension was most strongly expressed
on November 10, 1545, When he celebrated his sixty-first
birthday in the circle of his friends, Melanchthon, Bugenhagen, Cruciger,
Major, Paul Eber, and some others, who surrounded him at
the festive board. He said then that he would not live to see Easter
and spoke sorrowfully of the future apostasy of certain brethren,
which would inflict a blow upon the Gospel greater than any of which
the papists were capable, who, for the most part, were rude, ignorant
Epicureans. He spoke of his impending death in a similar vein, though
less definitely, in his declining years.

And yet these years furnish an exemplification of the incessant
activity of this man, who was afflicted with melancholia. His literary
productions, some of which have already been noted above, must be
considered first.

With his impetuous pen he commenced a vigorous onslaught on
the Jews and the Jewish religion. In the first years of his public
career he had spoken of them in a different tone. At that time, as we
have seen above, he harbored the seductive idea that they might be
converted to the new Evangel. His treatise of 1523 was inspired by
this visionary idea.\footnote{Grisar, \textit{Luther}, Vol. V, pp. 411 sq.}
The acceptance of his Gospel on the part of the
Jews was to be a divine seal of approval. But his expectations came
to naught. As a result, indignation at their infidelity, their blasphemies
against Christianity, and their oppression of non-Jews in
Germany, took hold of him. With the sanction of Luther, John
Frederick expelled the Jews from electoral Saxony in 1536, whilst
King Ferdinand granted them an asylum in his territories.
Luther first proceeded against the infidelity and proselytizing zeal
of the Jews in his “Epistle against the Sabbatarians,” published in
1538. The Jews had succeeded in introducing the observance of the
Sabbath and other rites in some localities, even among Christians.
Then, commencing with 1542, he launched a violent attack, intended
to annihilate the hostile Jewry. In the year mentioned he published
his tract “On the Jews and their Lies.”\footnote
{Weimar ed., Vol. LIII, pp. 412 sqq.; Erl. ed., Vol. XXXII, pp. 99 sqq.; cfr. Grisar, \textit{op.
cit.}, Vol. V, pp. 402 sqq.}
Immediately afterwards he
completed his tract, “On Shem Hamphoras and the Generation of
Christ”--a work which exhibits greater forcefulness, but overflows
with attacks and is saturated with vulgarities.\footnote
{Weimar ed., Vol. LIII, pp. 573 sqq.; Erl. ed., Vol. XXXII, pp. 275 sqq.; cfr. Grisar,
\textit{op. cit.}, Vol. V, pp. 404--407.}
Shem Hamphoras (or
“peculiar name”), according to Luther, was a Kabbalistic formula of the
Jews, supposed to be endowed with great power, by means of which Jesus
was alleged to have wrought His miracles. In 1543 followed
another literary attack upon the Jews, which bore the title: “On the
Last Words of David.”\footnote{Erl. ed., Vol. XXXVII, pp. 1 sqq.}
An ardent zeal for outraged Christianity is reflected
in these productions, which bear evidence of an agitated frame of
mind. Luther’s zeal for truth and justice, however, does not improve
matters because that work abounds in extravagant appeals to inflict
violence upon a race struck with religious blindness and weighed
down by injustice.

Once more he raised his voice against that persecuted race in his last
sermon at Eisleben, on February 14, 1546:\footnote{Grisar, \textit{op. cit.}, Vol. VI, p. 374.}
“You rulers,” he said,
“ought not to tolerate, but to expel them.” By indirection it was a
summons to rise against the Jews, who were favored in the neighboring
country of Mansfeld, but had become notorious by their fraudulent
and usurious practices.

Usury was an evil which Luther also attacked in other vigorous
pamphlets. Having written two sermons against it as early as 1519
and 1520, and having condemned it anew in his tract, “On Commerce
and Usury,” written in 1524,\footnote{Grisar, \textit{op. cit.}, Vol. VI, pp. 86 sqq.}
he once more returned to this
favorite topic in 1540, when he wrote his “Appeal to Pastors to
Preach against Usury.”\footnote
{Weimar ed., Vol. LI, pp. 331 sqq.; Erl. ed., Vol. XXIII, pp. 282 sqq.; Köstlin-Kawerau,
\textit{Martin Luther}, Vol. II, p. 432; Grisar, \textit{l.c.}}
Although he manifested an exaggerated
zeal against the abuses connected with money loans, he revealed no
insight into the commercial and trade relations whose development
had then practically begun, and which appeared to justify a fair rate
of interest on loaned capital. He condemns interest-taking outright,
\footnote{Thus Köstlin, \textit{l.c.}}
and makes but one exception, by granting that the aged and widows
and orphans might, if necessary, exact interest on loans in order to
secure a livelihood.

An “Exhortation to Prayer against the Turks,” which he wrote in
1541, is superior in tone and contents to Luther’s previous pamphlets
against the Turks.\footnote
{Weimar ed., Vol. LI, pp. 585 sqq.; Erl. ed., Vol. XXXII, pp. 74 sqq.; Köstlin-Kawerau,
\textit{Martin Luther}, Vol. II, p. 563. Cfr. Grisar, \textit{op. cit.}, Vol. V, pp. 167, 417 sqq.}
The elector had urged the people to pray against
the Saracen menace, which constantly grew more threatening. It is
the will of God, Luther declares, that Christians should beseech the
Lord for aid in a penitential spirit and sincere faith. The Turk and the
pope are on the decline and Judgment Day will soon console the faithful.
We have here a repetition of the thoughts on which he delighted
to dwell in view of his approaching death. His “Computation of the
Years of the Word” (\textit{Supputatio Annorum Mundi}), which is in line
with this tendency of his mind, appeared in the same year and was
re-edited with alterations in 1545.\footnote{Weimar ed., Vol. LIII, pp. 151 sqq.}
His dreams of Antichrist and the
end of the world form the subject-matter of his work on “The
Twelfth Chapter of Daniel,” which was published at the same time.\footnote{Erlangen ed., Vol. XLI, pp. 294 sqq.}

This series was followed by a preface to Ezechiel, as well as two
works, not devoid of merit, which were intended to combat the
Koran, and his booklet, “Consolation for Wives who have not Fared
well in Bearing Children,” a practical work designed to meet the
spiritual wants of the nation. In 1543 began the publication of his
lectures on Genesis, based upon notes made by his hearers under the
editorship of Vitus Dietrich. Luther himself at this time, was engaged
in the interpretation of the Messianic prophecies of Isaias, a work
which was published only after his death.

His “Brief Profession of Faith in the Blessed Sacrament” (1544)\footnote{\textit{Ibid.}, Vol. XXII, PP. 396 sqq.}
was the fruit of the deeply felt need of once more settling accounts
with those who opposed his teaching of Christ’s real presence in the
Eucharist. “I, who am about to die,” he says, “wish to take with me
before the judgment seat of my Lord this testimony, that, in compliance
with God’s command in Tit. 3:10, I have earnestly condemned and avoided
the fanatics and adversaries of the Sacrament,
Karlstadt, Zwingli, Oecolampadius, Stenkefeld (\textit{i.e.}, Schwenckfeld),
and their disciples at Zurich, or wherever they may be.” He answers
their objections with the question, whether, by the same token, they
are not compelled to deny belief in the humanity and divinity of
Christ. “We must,” he says, “either believe everything, wholly and
entirely, or nothing.” For God is omnipotent. The adversaries, with
their “infernal hearts and lying mouths” are not even deserving of
prayer.

This angry tract was occasioned by the new movement of the Swiss
reformers against belief in the Real Presence and by the above-mentioned
so-called “Reformation of Cologne,” composed by Melanchthon and Bucer,
which, to the chagrin of Luther, spoke in
Bucer’s sense of a purely spiritual communion with the body and blood
of Christ in the Last Supper.\footnote{Köstlin-Kawerau, \textit{Martin Luther}, Vol. II, p. 581.}

Melanchthon, who had collaborated in
the “Reformation of Cologne,” was worried about himself and his
fate. It was said at Wittenberg that Luther was about to propose a
formula to which all would be compelled to subscribe. When this
formula (“Brief Creed”) appeared, there was rejoicing because it
contained no reference to Melanchthon and Bucer. Luther had suppressed
his chagrin and did not care to cast suspicion upon these men
in public. All the more ruthlessly, however, did this tract sever his
relations with the Swiss innovators, which, notwithstanding the theological
controversies, had gradually become more tolerable.

The Swiss reformers soon issued energetic counter-declarations. Bullinger,
above all, entered the lists against Luther with his “True Creed” of the Zurich
theologians,\footnote{Cfr, Grisar, \textit{Luther}, Vol. IV, pp. 325 sq.; Vol.
V, p. 409.} in which he states that the abusive language of the Wittenbergers
would not be reciprocated by him. He severely censured the violent and indecent
effusions of an aged and otherwise highly respected man, and especially the
autocratic manner of his decisions. The theological reasons which Bullinger
advances for the Eucharistic beliefs of Zwingli’s disciples are not very
impressive.

Whilst Melanchthon was still engaged in the composition of his
cautious “Wittenberg Reformation,” which was intended to be a
programme in opposition to the Council of Trent, which was then
commencing, Luther summoned all his available strength to deliver
a new blow against the papacy, for his hatred was not yet quenched.
This fresh outburst was contained in a work, the first part of which
bore the title, “Against the Papacy at Rome founded by the Devil”
(1545).\footnote{Erl. ed., Vol. XXVI, ii, pp. 131 sqq.}
 He attacks the papacy along the entire line and frequently
in a furious fashion, because it refused to succumb to his assaults,
nay, even dared to gain new vigor at the Council. Luther held that
the papacy originated in hell and was sustained by infernal powers.
This is drastically illustrated by a picture on the title page of this
pamphlet,\footnote{Grisar and Heege, \textit{Luthers Kampfbilder}, n. 4 (\textit{Lutherstudien}, V), p 20.}
 which represents the pope seated on his throne in the
widely distended and terrible jaws of hell, and borne upward by ropes
drawn by devils. Whilst adoring the prince of hell, who flees before
him, he is crowned with a tiara which tapers into a point composed of
human excrements.

We may be permitted to omit quotations from this horrible pamphlet,
which contains repetitions of former ideas, but clothed in forms
which seem to force an irrevocable decision concerning the mental
state of its author, which, as is known, frequently obtruded itself
upon the reader of his former writings.\footnote
{Passages from this pamphlet quoted in Grisar, Luther, Vol. V, pp. 381 sqq., 421 sqq.;
Vol. III, p. 151.}

Luther arrived at the unfortunate resolution of publishing this
pamphlet with illustrations.

He had already increased the number of his previously published polemical
illustrations by some which were calculated to arouse the brutal passions of
the masses.\footnote{Cf. Grisar und Heege, \textit{Luthers Kampfbilder}, n. 2
and 3.} One pamphlet depicted the pope as Satan.\footnote{\textit{Ibid.}, n. 4,
plate 2, with text, pp. 67 sqq.} A frightful, savage and nude giant, with an
immense tail, wears the triple crown and is adorned with the ears of an ass. In
his right hand he holds the trunk of a tree resembling a club; in his left,
which is extended in a threatening manner, he holds a large, broken key. Amid
fire and smoke, this ``pope-ass'' expectorates worms and filth, like the dragon
in the Apocalypse (c. 17). The wings of a bat, serrated after the manner of
flames, can be seen on his back. Beneath him, the fires of hell burst forth. A
devil wearing a cardinal’s hat and seated at the right on a papal Bull, devours
a bishop, and allows his excrements to drop upon the papal seal.

One series of controversial illustrations furnished by Luther is entitled,
“Illustrations of the Papacy.” It was published in 1545 and
was intended to illustrate, as it were, his “Papacy founded by Satan.”
It contains the following caricatures:\footnote{\textit{Kampfbilder}, n. 4.}
“The Pope-Ass of Rome, a
monster found in the Tiber in 1496”; the ascent of the pope from
hell; the mockery of the ban by two rogues with exposed backs and
emitting blasts of wind against the pope. Again, there are illustrations
deriding the papal government by the most crude defilement of
the papal arms, depicting a wretch discharging his faeces into them
whilst two others are getting ready to follow his example. Another
illustration depicts the great keys on the papal arms as master-keys in
the hands of thieves. The fifth illustration shows the manner in which
the papacy rewards the emperors—the fictitious decapitation of Conradine
by the hand of a pope. Sixth, there is “the reward of the most
Satanic pope and his cardinals,” represented by the death of the
pope on the gallows in the company of two cardinals and a priest.

The seventh and eighth cartoons, which appear on one page, represent
the pope riding a sow and offering to the world steaming human excrements
with his blessing, which was designed to be an insult to the
proposed Council; the other illustration (intended to be a mockery
of the pope’s biblical exegesis) depicts him in the capacity of an ass
performing on the bag-pipe. The last and ninth cartoon is designed to
suggest that the pope was born of a nude she-devil, a scene vulgar
beyond all description.

A wood-cut, which depicts Pope Alexander III placing his foot on
the neck of Emperor Barbarossa, as well as the pope-ass above described,
do not belong to this series. The two repulsive illustrations of
the origin of Antichrist, \textit{i.e.}, the papacy, and the origin of the monastic
life, were free supplements, for which Luther seems to be partially responsible.
The entire collection has become extremely rare,
owing probably to the outraged sensibilities of those who were offended
by them. In recent times, these cartoons have been resubmitted to the
public in the interests of history, but not by partisans of Luther.\footnote{\textit{Ibid.}, plate 3 and text, pp. 92 sqq.}

Luther’s active participation in the “Illustrations of the Papacy”
has been placed beyond question by recent research. He even assisted
the “artist” with his crayon, besides contributing the ideas
and the crude verses that accompanied the cartoons.\footnote
{\textit{Ibid.}, pp. 73 sqq., 86, p. 89, p. 91 and the testimony of Christoph Walther, Aurifaber,
and Amsdorf. Luther had some proficiency in drawing; cf. \textit{Kampfbilder}, n. 3, pp. 59 sqq.}
His name is
attached to the illustrations of the series, as well as to the cartoon of
the pope-devil. The drawings themselves were without exception the
product of his confidant, Lucas Cranach, an artist who had previously
achieved fame by his fine religious paintings.

Hence, it is historically untenable if Protestant authors hold Cranach
solely responsible for the disgraceful cartoons of the papacy and
ascribe only the text to Luther. These illustrations are his spiritual
property in the fullest sense of the word, and Luther himself described
them as his last will and testament to the German nation.\footnote{\textit{Kampfbilder}, n. 4, p. 86.}

He expressed the wish that these cartoons might enter every home
and workshop, and thus bear effective witness against the papacy.
Luther did not even shrink from designing a cartoon which represented
the death of the pope on the gallows, clearly intended to provoke deeds
of violence.\footnote{\textit{Ibid.}, plate 1 in actual size, and text, p. 26.}

This cartoon deserves special mention because of its sanguinary
and inciting character. The pope is distinguishable as the then reigning
pontiff, Paul III. His tongue has
been pulled out of his throat, and an executioner is engaged in nailing
it to the gallows, as had already been done to three others who had
been hanged thereon, namely, Cardinal Albrecht of Mayence, Cardinal Otto
Truchsess of Augsburg (?), and the priest Cochlaeus (?).
Four devils convey the souls of the executed felons to hell. Despite
his exhortations to the contrary, there are numerous passages in the
writings of Luther which incite, or are apt to incite, to sanguinary
deeds of violence against the clergy and the monks.\footnote
{\textit{Ibid.}, pp. 137--139, where twelve passages are quoted. Cf. the cartoon published by
Luther in 1538, which picture Paul III as Judas. (\textit{Kampfbilder}, n. 4, p. 4.)}
Of course, it
cannot be inferred from such expressions of passionate rage that
Luther was actually prepared to endorse the assassination of ecclesiastical
dignitaries, or personally to take a hand in them. It is quite patent,
however, that bloody results were apt to follow the dissemination
of cartoons such as those described, especially the last brutal
gallows scene, in conjunction with Luther’s sanguinary cries for violence.
