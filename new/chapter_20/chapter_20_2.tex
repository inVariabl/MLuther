\section{In the Midst of Ruins}

Many contemporary writings contain doleful lamentations, not
only on the part of Protestants, but also on that of Catholics, concerning
the decay of the internal life of the Church brought about
by the Protestant Reformation. Thus when, in 1549, John Cochlaeus
printed the preface to his great work, “On the Work and Writings
of Luther,” he deplored the wounds of the Church in melancholy
terms.\footnote{\textit{Commentaria de Actis}, etc., part I, Cfr. Grisar, \textit{Luther}, Vol. IV, pp. 365 sq.}
 He was grieved, above all, by the unhappy lot of the souls
that had been severed from the Church, the fountain of salvation.
“The bond of charity and concord which unites Christian people,”
he says, “has been loosened, discipline undermined, reverence for God
destroyed, wholesome fear extinguished, and obedience cast aside. In
lieu of these prevail sinfulness and a freedom that is alien to God.”
This courageous defender of the Church died at Breslau in the year
1552.

John Wild, who died two years later, was a distinguished cathedral preacher of
Mayence and a Franciscan Observantine. He paints an impressive picture of the
moral decadence which had set in and proved so detrimental to Catholic
religious life.\footnote {Cf. N. Paulus, \textit{Johann Wild (Dritte
Vercinsschrift der Görresgesellschaft}, 1893), p. 15; Grisar, \textit{Luther},
Vol. IV, pp. 366 sq.} “Alas,” he exclaims, “all fear of God is driven out of
the hearts of men by dint of sermons lacking all sense of modesty and urging
faith alone \dots The other, namely, good works, has been trodden in the mire.
The result is that we are now for the most part Christians merely in name, but
so far as works are concerned, more depraved and wicked than even Jews or
Turks. The cause of the very grievous sufferings of the Church,” he says, is
“that her children have been and are so lamentably led astray that they refuse
any longer to acknowledge their own mother.” With depressed feelings Wild views
his German fatherland torn by factions and become a byword to its neighbors.
“Everybody wants a bit of us.” People say: “Ha, these are the haughty Germans
who \dots have a finger in every war; now they are going to set to on each
other.”

Duke William of Bavaria, another opponent of the Reformation,
presented the diet of Ratisbon (1541) with a memorial in which
he bitterly complained of the “great injury and corruption” into
which his fellow-Catholics were being led by the Protestantizers.
“Contrary to the commandment of God, in defiance of law and Christian
usages \dots churches and monasteries are seized by force \dots religious
foundations and property are torn from them mercilessly.”

And yet “the Catholics have no dearer wish than order and justice.”\footnote{For these and the following two passages see Grisar, \textit{op. cit.}, Vol. IV, pp. 367 sq.}

Once more the religious and judicial grievances of the Catholics were
prominently brought forward at the diet of Worms (1545). It was
maintained that the opponents of the faith were suppressing everything;
yet they complained about being suppressed themselves. “What
the Protestants call proclaiming the Word of God, is for the most
part, as they themselves complain, mere slander and abuse of the
pope.” “Nothing constant any more,” says the Cologne Doctor Carl
von der Plassen, “discipline, loyalty, and respectability have vanished.”
He is pained to observe the evil reaction of the religious innovation
on the lives of Catholics. “What misery results from want
of clergy and schools even in the lands which have remained Catholic!”

Besides the alarming increase in the number of violations of sacerdotal
celibacy, already mentioned above, the decrease in the reception
of the sacraments and attendance at divine worship in the Catholic
parts of Germany was especially lamentable. Thus Peter Canisius
complains in his letters about Ingolstadt that communions in that
Catholic city had practically ceased. He writes in 1550 that, although
two bells are rung to summon the people to Mass, which is said in a
church located in the center of the city, “still, we cannot induce two
persons to attend Mass, even if we would pay them for coming.” The
law of fasting had become practically obsolete in the academy known
as the Georgianum.\footnote{Braunsberger, \textit{Canisius}, 2nd and 3rd ed., 1921, pp. 44--47.}
 In Austria the state of Catholicism was scarcely
less disheartening. Esteem for the clergy had profoundly declined.
Almost in no wise did the shepherds measure up to the tasks of their
vocation. Monastic discipline had deserted most of the cloisters.
Ignorance and barbarity prevailed among the masses. The University of
Vienna had deteriorated very appreciably. In the course of twenty
years this institution, which had formerly yielded such abundant
fruits for the Church, scarcely produced one student a year for the
priesthood.\footnote{\textit{Ibid.}, pp. 51 sq.}

Luther always tried hard to propagate his doctrines in Austria.
Among others, he endeavored to attract the Moravian Brethren to
his cause. Notwithstanding important variations of doctrine, he
treated the senior of the Brethren, who came to Wittenberg in 1540
and again in 1542, with great distinction. He exhorted the Brethren
in writing to persevere with him to the end in unity of doctrine and
spirit, since he expected soon to depart this life.\footnote{Köstlin-Kawerau, \textit{Martin Luther}, Vol. II, p. 579.}
Wittenberg did
not adopt any part of the ecclesiastical discipline which the Moravian
Brethren maintained. Nor was there ever any spiritual connection
between them. Later on, in 1772, the congregation of Moravian
Brethren (Herrnhuter) in Saxony resulted from a rapprochement
between Lutheranism and the Brethren.

In Transylvania, also, efforts were made, towards the end of Luther’s
life, to effect a closer union with Wittenberg and an increased
separation from Rome. In 1542, the preacher William Farel dispatched a
letter from Metz requesting a union of the followers of the
new religion with the Schmalkaldic League. About this time overtures
were begun in Vienna for the sake of obtaining Luther’s advice and
direction. They were, however, devoid of notable results. Italy was
preserved from religious subversion by the watchfulness of the Inquisition.
For the rest, the weakness of the new religion was too little
known abroad.

The extent to which authority, spiritual as well as temporal, was shaken in
consequence of Luther’s rebellion against the ancient faith, became clearly
manifest in Germany. Preachers who were highly esteemed by Luther were treated
with contempt and derision by their own followers. Wenceslaus Link, once the
honored successor of Staupitz as vicar of the Augustinians, was derisively
hailed in Nuremberg as “Pfaff” by the magistrates whilst carrying on his
Protestant activities in that town. Luther consoled him by writing: “The civil
authorities have ever been, and always will be, enemies of the Church.” “Our
respected domestic opponents,” he continues, “are dangerous to us, according to
the prophecy: ‘After the revelation of Antichrist, there will come men who say,
there is no God!”\footnote{Grisar, \textit{Luther}, Vol. V, p. 325.} “Each will
desire to be his own rabbi,” he says on another occasion, “whence the greatest
devastation will ensue.”\footnote{\textit{Tischreden}, Weimar ed., Vol. III, n.
3900.}

Under Link as vicar-general, the Augustinian congregation began
to decline. The three succeeding vicars witnessed its complete
ruin. The Saxon province of the Order also fell a victim to the religious
innovation. Its own members promoted subversion and confusion.\footnote{Cfr. Kolde, \textit{Die deutsche Augustinerkongregation}, Gotha, 1879, last chapter.}
Melanchthon, and still more Bucer, continued to cherish the
fervent hope that, during the prevalence of this state of confusion,
“our princes and estates will bring about a council or some kind of
harmony in doctrine and worship, lest everybody proceed on his own
responsibility” (Melanchthon). That a council would only be productive
of greater disunion, Luther perceived more clearly than the
others. The reason was because there was no sanction, and this was
the permanent cause of the ruination.

“In God’s kingdom, in which He rules through the Gospel,” Luther taught
as a fundamental truth of his theology, “there is no going
to law, nor have we anything to do with law, but everything is
summed up in forgiveness, remission and bestowing, and there is no
anger or punishment, nothing but benevolence and service of our
neighbor.”\footnote{Grisar, \textit{op. cit.}, Vol. V, p. 565.}
 As a consequence, doctrines and ethical precepts were
invalidated, as there was no authority to enforce them. Luther’s system
is altogether devoid of an authoritative foundation, such as the
Catholic Church possesses in her constitution; everything is “opinion
and advice,” as he himself avers.\footnote{\textit{Ibid.}, p. 566.}
 He is neither able nor does he desire
to lay down precepts. Since, however, he cannot afford to do without
some force that binds men, he appeals to civil authority, to the State,
which must be Lutheran, and to which he even ascribes the right of
deciding doctrinal controversies, provided only that the decision is
“in conformity with Scripture.” In this manner--as the Protestant
theologian, Christopher Ernest Luthardt, is compelled to acknowledge
in his \textit{Ethik Luthers}--Luther arrived at propositions which are “at
variance with his fundamental views,” and at suppositions concerning the
secular authority “which he decisively denies where he lays
down principles.”\footnote{\textit{Ibid.}, p. 567.}

Nor did Luther have an authoritative argument by which he might
have convinced those of his opponents who denounced the Blessed
Trinity, as happened first in Protestant Nuremberg, if the civil government
did not proceed against them. He and Melanchthon refused
to reply to Campanus, who denied the Trinity, lest they advertise his
opinions.

Antinomianism, for which Luther had persecuted Agricola, continued its
course beneath the ruins, being promoted by the eloquent
and active Jacob Schenck, who was for a time court-preacher at Weimar.
In a sermon which he delivered at Eisenach in 1540, Luther discovered
a confirmation of his suspicion that Schenk’s “opposition to
the law” furnished the common people with an occasion of moral
laxity. Schenk was called to the electoral court of Joachim of Brandenburg,
as assistant to the antinomian Agricola, and it is claimed
that he starved himself to death in a spell of melancholia.\footnote{Köstlin-Kawerau, \textit{Martin Luther}, Vol. II, p. 467.}
 When
Agricola again visited Wittenberg, in company with his wife and
daughter, in 1545, the old differences manifested themselves anew;
despite the fact that he brought with him a letter of recommendation from
his elector. Luther refused to see the “arrogant and impious fellow,”
as he had branded him on a former occasion, though he
received his wife and daughter. After Luther’s death, Agricola, bowing
to the situation that existed at the Brandenburg court, posed as a
defender of genuine Lutheranism against the “Philippists,” \textit{i.e.}, the
adherents of Melanchthon. The solemn religious services which he
conducted in honor of the Reformation in the court-chapel at Berlin,
in 1563, were a sort of triumphal assertion of what pretended to be
orthodox Lutheranism. “Thus the man whom Luther had proscribed,
contributed to the triumph of rigorous Lutheranism.”\footnote
{Kawerau in the \textit{Realenzyklopädie für Theologie}, etc., Vol. 1, 3rd ed., p. 253.}
Agricola died during an epidemic in 1566.

Another characteristic feature of the spreading theological ruin
was Agricola’s impassioned opposition to Melanchthon’s revision of
the Augsburg Confession, the so-called \textit{Confessio Variata} of 1540.
Melanchthon, as is well known, underwent a process of theological
development which took him farther and farther away from Luther.
He interpreted essential portions of the Augsburg Confession, which
he himself had composed in 1530, so that Agricola spoke of the “Variata”
as a “falsified” Confession and availed himself of Melanchthon’s
arbitrary changes as an argument in his indictment of “Philippism.”
Melanchthon, on his part, stated at the religious conference of Augsburg
that the only changes he had made were certain modifications
of language with a view to greater clearness of doctrine.\footnote{Grisar, \textit{Luther}, \textit{l.c.}}
 The doctrines
of justification, good works, and penance, however, had actually been
altered in accordance with the ideas which Melanchthon
had proposed in 1535 in his \textit{Loci Theologici}, and which approximated
the Catholic teaching. The propositions on the Last Supper reveal
concessions to the Swiss reformers, who denied the real presence of
Christ in the Eucharist. “That there was question of actual changes,
ought never to have been denied,” writes Theodore Kolde, a Protestant
authority on Luther.\footnote{\textit{Symbolische Bücher}, ed. by H. T. Müller, 10th ed., Introduction, p. XXV.}
Luther himself never publicly rejected
the “Confessio Variata.” He did not wish to provoke an open breach
with his learned and indispensable ally. But after his death Melanchthon
experienced the ill-will of the theologians of the New Gospel.
How bitterly he felt it may be gathered from the fact that, shortly
before his demise (April 19, 1560), he wrote with his own hand that
among the reasons why he did not fear death so much was this: “You
will be delivered from all trouble and the fury of the theologians.”\footnote{Cfr. Grisar, \textit{op. cit.}, Vol. V, p. 263.}

A compliant attitude towards dogma, similar to that of Luther
towards Melanchthon’s “Confessio Variata,” is discovered in the position
taken by both these men toward the Articles of Agreement
elaborated by the English Protestants in 1536, proposing a union between
Anglicanism and Lutheranism. At that time it was a question
of winning over an important country.\footnote{G. Mentz; cfr. Grisar, \textit{op. cit.}, Vol. V, p. 260.}
 After the attempt had
failed, the German reformers were rewarded by hostility on the part
of the new Anglican Church toward Lutheranism. Luther himself
declared that he was glad to be rid of the “blasphemer” (Henry
VIII).\footnote{Köstlin-Kawerau, \textit{Martin Luther}, Vol. II, p. 400.}
 He was very much depressed, however, when the King (in
1540) executed Luther’s friend Barnes, who had played the role of
mediator for years between Wittenberg and Henry VIII. Barnes was
put to death as a heretic because he promulgated Luther’s doctrine
of justification.

A strong rival of Luther’s ecclesiastical polity appeared during the
declining years of his life in Calvinism, which deviated widely from
the Wittenberg school. John Calvin, who had established his politico-religious
rule at Geneva, in 1541, began by opposing Luther’s assertion of the real
presence of Christ in the Eucharist. The Genevan
innovator flatly denied this doctrine and described the intentionally
vague formulas of Melanchthon and Bucer as “mere vaporing,” intended to
deceive their opponents. Also with regard to predestination,
Calvin discarded the hypocrisies of the Lutheran position by
asserting that it is not free will which governs men’s efforts to save
their souls, but the irresistible providence of God in a deterministic
sense. Calvin was a most pronounced and consistent champion of
unconditional predestination. From the practical standpoint, it was
important that he rejected Luther’s fundamental principle of the
separation of the spiritual and temporal kingdoms and in its place set
up a theocracy at Geneva, where his theology permeated every fibre
of public life and he himself governed with a reckless absolutism.
Subsequently, this type of political religion was adopted to a greater
or less degree by the Calvinistic churches in other countries. Notwithstanding
the differences between their respective doctrines, however, Luther and
Calvin mutually eulogized each other. Calvin was
treated with consideration by Luther and, in his turn, acknowledged
the influence which Luther had exercised upon him. Had Luther
lived longer, the two reformers would no doubt have become embroiled in
violent altercations.\footnote{On the relations between Calvin and Luther see Grisar, \textit{Luther}, Vol. V, pp. 399 sqq.}

In view of Zwinglianism and Calvinism, Protestants often speak
of a complete and free evolution of Protestantism. It would be more
proper to speak of a multiplication of ruins, which the spirit of innovation
wrought within the domain of dogma.

Continuing the discussion of German Lutheranism—when we turn
from the sphere of dogma to that of practical affairs, we discover
that the greatest damage during Luther’s declining years was done to
ecclesiastical property. Luther was fully aware of the fact that the
confiscation of the temporal possessions of the Church would constitute
a powerful stimulus for the civil governments to open their territories
to the new ecclesiastical régime. With sentiments of self-satisfaction
he refers his elector to the “considerable wealth, which
increases daily.”\footnote{\textit{Ibid.}, Vol. III, p. 35.}
 Whatever the territorial ruler did not appropriate,
was confiscated by the magistrates of the different municipalities. These
seized pre-eminently the minor benefices which, as a rule,
depended either upon them or upon prominent families. The emoluments,
so Luther and the rulers frequently asserted, were intended
for the maintenance of schools, preachers, and parishes. Still, there
are numerous complaints made by Luther and his followers that such
was not the case, or that the “harpies” among the nobility interfered,
in order to enrich themselves. Who, in surveying that age, can calculate
the immense sums derived from the confiscation of bishoprics,
clerical benefices, and monasteries, which were forever alienated
from the spiritual or educational purposes for which they had been
given, including the foundations of Christian charity which dispensed
help with a lavish hand? Even if the former use of these
properties was not always in conformity with the pious intentions
of their founders; even if the revenues from these ancient endowments
were allowed to accumulate excessively, when contrasted with
the possessions of the bourgeoisie, a fact which gave rise to many
complaints and altercations—yet the spoliation, perpetrated by ineffable
acts of violence, was assuredly not the proper solution of the
existing problem. It merely caused ruin and destruction.\footnote
{On the fate of the Church property, cfr. Grisar, \textit{Luther, passim} (see index).}

As these ruins accumulated, the theologians of Wittenberg gave up
the idea of regulative intervention. When, in 1544, the magistrate
of a certain city requested Luther to advise him, according to Sacred
Scripture, on the question of the confiscation of ecclesiastical property,
he replied: “This matter does not concern us theologians. Such
things must be decided by the lawyers.” It was a formal evasion of
questions which the theologians themselves had raised.\footnote{Grisar, \textit{op. cit.}, Vol. V, pp. 206 sq.}
They saw with their own eyes and acknowledged the curse which always follows
the spoliation of the Church. Thus Paul Eber, Luther’s Wittenberg
friend, speaks of the penury which was visited upon the devotees
of the Lutheran Church in consequence of the spoliation, and predicts
that the future will reveal even more clearly how the confiscated Church
property will react upon its beneficiaries, who “so
greatly warmed and fattened themselves by means of these spiritual
possessions.”\footnote{\textit{Ibid.}, Vol. IV, pp. 59 sqq.}

A particularly sad chapter in the history of the dissipation of the
property of the Church is furnished by the destruction of numerous
works of ecclesiastical art that adorned the churches. In Lutheran
Germany this destruction was not as great as in the Zwinglian parts
of Switzerland and of southern and western Germany, where a veritable
mania developed against images, altars, and other objects of
sacred art. Thus the city of Nuremberg, for instance, owed the preservation
of many precious art treasures to the indulgent attitude of the
populace and its civic spirit. Still, Lutheran communities also became
the scene of much destruction. There is extant a catalogue of Blasius
Kneusel, which lists the objects of ecclesiastical art destroyed at Meissen.
He enumerates fifty-one objects of great value which had been
robbed by spoliators—amongst them a golden cross “of the weight
of 1,300 gulden, heavy with precious stones,” a diamond cross worth
16,000 gulden, several golden crosses adorned with precious stones
and pearls, a gold plate appraised at a thousand gulden, a large bust
of St. Benno, made of precious metals, weighing more than 36
pounds, which had been purchased with the charitable gifts of the
members of the parish of Meissen. From time to time, even now,
treasures of religious art are discovered in hiding places which were
purloined at that time. In its avarice, this barbaric age did not hesitate
to consign to the melting-pot the most precious monstrances, chalices,
and patens, excusing itself on the strength of the commonplace
Lutheran charge of “idolatry.” Luther’s hostility to pictorial representations
became fatal to art, even though he moderated his expressions on this
subject as time went on. His unchanging attitude was
that the religious images would gradually disappear if his doctrine
prevailed. The creation of a religious image or statue was no longer
counted as a good work.\footnote
{On Meissen, \textit{op. cit.}, Vol. V, pp. 203, 169; on destructive activities in Erfurt, \textit{ibid.},
pp. 213 sqq.; on Luther’s attitude toward the veneration of images in general, cfr. \textit{ibid.},
pp. 207 sqq.}

The decline of artistic development in
Germany, which had justified the highest expectations at the close of
the Middle Ages, was brought on by Luther’s work.

Moreover, a perceptible retrogression in the care of the poor resulted
from the destruction of ecclesiastical revenues. The practice
of Christian charity sustained a severe blow. The assertion that
good works were of no value was bound to weaken the spirit of charity,
so splendidly manifested towards the close of the Middle Ages
by the foundation of hospitals and other charitable institutions under
ecclesiastical auspices. Luther intended to substitute for them the
so-called community poor-boxes and a more intensive care of the
poor on the part of the civil authorities. But these boxes were successfully
operated only in a few places. Luther’s failure at Leisnig and
elsewhere produced a deterrent effect. Luther, moreover, wished to see
begging completely prohibited. His movement was directed against
the mendicant Orders of the Catholic Church, but it produced no
far-reaching social results.\footnote{\textit{Op. cit.}, Vol. VI, pp. 50 sqq.}

Complaints that the moral sense, which in the last analysis must
sustain all charitable endeavors, was becoming extinguished, were
multiplied by Luther and his partisans. Under the papacy, he says,
people had been eager to make sacrifices for the poor, but now they
had grown cold.\footnote{\textit{Ibid.}, pp. 54 sq.}
 In his opinion the society of true Christians,
planned by him, was bound to cultivate the spirit of sacrifice in the
era of the new Gospel; but, he says, “would that we had nations and
individuals who sincerely desired to be Christians!” In one of his
sermons he exclaims: “Woe unto you peasants, burghers, and members of
the nobility, who appropriate all things unto yourselves, who
scrape and hoard, and yet desire to be good evangelicals!” It was a
proof that he was “unsophisticated,” as a modern sociologist mildly
puts it, that Luther ascribed to the “faith” which he preached the
sole power of overcoming the obstacles to charity by means of the
community poor-box.\footnote{Feuchtwanger, quoted \textit{ibid.}, pp. 56 sq.}

Adolph Harnack concedes that, “where Lutheranism was in the
ascendant, social care of the poor was soon reduced to a worse plight
than ever before.”\footnote{\textit{Reden und Aufsitze}, Vol. II (1904), p. 52.}
 The lack of resourcefulness of the Protestant
system of poor-relief continued for a long time.

Protestant authors, by way of contrast, have referred to the excellent
systems of poor-relief that flourished in the cities of South Germany at the
time of the religious schism. But these institutions were a heritage of the
fifteenth century, that is, of the Catholic Middle Ages. They owed as little
to Lutheranism as the excellent institutions and arrangements for the poor
which existed in the Catholic Netherlands, such as, for instance, at Ypres,
in 1525. It was Catholic idealism, humanism, and the rising civic spirit of
the municipalities as they attained to independence, which created those
praiseworthy institutions. Protestantism, on the other hand, even after
Bugenhagen had improved the parish treasuries in virtue of his superior genius
for organization, as a rule attained only to inadequate governmental regulations
of the system of poor-relief with a tinge of religious influence. These
“poor-chests” were occasionally described by envious parties as “clerical and
usury funds,” which does not, however, prove that in certain localities, such
as Hesse and Strasburg, they did not really benefit the poor, especially when
administered by men of truly Christian charity.

William Liese, the most recent Catholic writer on the history of charity,
correctly observes that “in practice the olden Catholic ideals and motives
continued to operate,” but “it can scarcely be affirmed” that there was a
Protestant impetus in the interest of poor-relief, or a growth of charity in
early Protestantism, whilst, on the other hand, “new principles are wont to
make their influence felt most clearly in the beginning,” and in the history of
Christianity it was “precisely the primitive age” that “produced the noblest
fruits of charity.”\footnote
{\textit{Geschichte der Caritas}, Vol. T (1922), pp. 255 sq. Liese reports the findings of Feuchtwanger,
Püschel, Otto Winckelmann (1922), and others, and concludes: “If we review
the recent vivid discussion of the subject, we find substantial agreement on the following
points: (1) The Reformation has not promoted, but rather injured charity; (2) it
has given a powerful impetus to governmental poor-relief, as is revealed by the multitude
of municipal ordinances passed from 1520 to 1530.”}

Relative to the schools, also, the aging Luther failed to see about
him that revival for which he had appealed in various writings; on
the contrary, here, too, he observed increasing dissolution and decline.
“Now that it is a question of founding true schools,” he laments,
“every purse is closed with iron chains, and no one is able to give.”
This deplorable state of affairs made him beg of God a happy death,
so that he might not live to witness Germany’s punishment.\footnote
{Grisar, \textit{Luther}, Vol. VI, p. 53. On the decline of schools, \textit{ibid.}, pp. 22 sqq., and
Janssen-Pastor, Vol. VII, 14th ed. (1904), pp. 5 sqq., 81 sqq.}
Even more frequent and persistent are the complaints of Melanchthon, who
was a professional educator, on the failure of his endeavors in this
sphere. In consequence of the decadence of Christian schools, he once
wrote, we shall yet become pagans.\footnote{Grisar, \textit{ibid.}}
In Catholic districts a similar
deplorable retrogression of the school system ensued in consequence
of the religious controversies. At first, indeed, Protestantism was able
to retrieve itself in virtue of the support given it by those princes
who were intent upon procuring recruits for their bureaucratic system
and favored the general education of the people.\footnote{Grisar, \textit{ibid.}}
That Luther
is the founder of the “\textit{Volksschule}” is as unfounded a claim as that
he is to be regarded as the author of poor-relief and a promoter of
charity on a grand scale. These claims are but the extreme expressions
of a Luther cult which has no basis in history. His claims as a champion
of culture are equally baseless. Beginning with the Peasants’
War he spoke and wrote rabidly against the peasants and the mob
and continued to do this to the end of his life. He had adopted the
maxim: “It does not do to pipe too much to the mob, or it will too
readily lose its head.”\footnote{\textit{Op. cit.}, Vol. V, p. 577, n. 1.}
 In his speeches he frequently works himself
into a veritable rage against the mob, calling it “Master Omnes,”
the “many-headed monster,” etc. As a Protestant author, Feuchtwanger,
says, Luther is not far removed from the politico-social ideas
of Machiavelli, who counsels rulers to keep a tight rein on the masses.\footnote{\textit{Op. cit.}, Vol. VI, p. 57.}

Gradually he began to claim absolute authority. “If compulsion and
the law of the strong arm still ruled,” he says, “as in the past, so that
if a man dared to grumble, he got a box on the ear--things would
fare better; otherwise it is all of no use.”\footnote{\textit{Op. cit.}, Vol. VI, p. 74.}
“Christ does not wish to
abolish serfdom,” he says in another passage on the oppressed condition
of the peasants, whose lot was constantly growing worse. “What cares
He how the lords or princes rule [in temporal matters]?” In his sermons
on the first book of Moses, he actually represents serfdom as a
relatively desirable state. “If society is to endure, \dots it will be
necessary to re-establish it.”\footnote{\textit{Ibid.}}
Possibly these declarations were but
the outbursts of a transient mood; yet they betray sufficiently the
sentiments which he harbored toward the lower classes.

One truly interested in the advancement of civilization should be
intent upon the preservation of ancient popular usages, especially
those whose cultural worth resides in the maintenance of the spiritual and
particularly the religious life. In many respects Luther proved himself
an enemy of the popular customs of the Middle Ages, because he
suspected hidden idolatry in those quite indifferent customs of which
the people had become fond and which were rooted in primitive
ages. He passionately declaimed not only against the abuses which
were connected with them; but, although he was himself descended
from the common people, he blindly combated popular usages which
were characteristic and educational.

A proof of this, among others, is contained in the memorandum
which he presented to the Elector of Saxony during the diet of Augsburg,
in 1530. This memorandum was connected with his object of
promoting a scheme which he had devised for the purification of the
Church.\footnote{\textit{Briefwechsel}, Vol, VII, pp. 256 sqq. (March, 1530.)}
 The lengthy list of “abuses” was intended, according to
his own words, to make known “the great, nay, atrocious injuries
which were inflicted upon souls and consciences.”
It impresses one as strange that he includes the custom of St. John’s fire,
which has remained popular even to the present day. The people build a
bonfire in midsummer, on June 24. It was taken over from paganism,
but divested of its heathen accompaniments. Similarly he condemns the use
of St. John’s wine, which was taken on December 27, the feast of the
Apostle John. He does not spare either the innocent celebration of St.
Martin’s eve, with its old custom of children bearing lights, or the ancient German
funeral banquets, which were primarily arranged for the benefit of
mourners who had come from afar.

Luther condemned the very popular semi-dramatic plays, inherited from
the devout and childlike Middle Ages. They were presented on the high
feast-days of the Church with a view to elevate the minds of the faithful,
and required only a little refurbishing here and there to be entirely
acceptable. Such were the cherished Christmas plays enacted at the manger
of the Christ Child; the Palm Sunday procession with the figure of Christ
riding on an ass; the solemn processional veneration of the Holy Cross
during the last days of Holy Week; the customary touching celebration of the
Resurrection, symbolized by the elevation of the Cross above the tomb;
the dramatic representation of the Ascension by means of a statue rising on
high; and the coming of the Holy Ghost by means of an ornamented dove.
These were all extremely ancient popular usages, which had taken deep root
and imparted ardor to the religious life of the people. It goes without saying
that a general war of extermination was declared on all specifically Catholic
customs.

How profoundly the religious life of the people was affected by these
changes may be seen from the fact that the above-mentioned memorial proposed
to abolish all confraternities with their religious demonstrations, which
were so frequently inspiring, all pilgrimages and processions, the customary
blessing of the fields by devout processions with the Cross, and the elevating
public solemnities in commemoration of the departed on All Souls’. Furthermore
the use of bells, candles, candlesticks, banners and the vestments worn
at divine service.

It was also proposed to discontinue the custom of carrying biers into
church. The offering of the pence during divine services was to be proscribed.
The veiling of images during the season of Lent, as well as the
hanging up of the so-called black cloth that covered the altar during Lent,
were to be discontinued. All these practices were dear to the people. Fasting,
the recitation of the Divine Office, the solemn rites of consecration, the
ceremony of the washing of the feet on Maundy Thursday, the use of Holy
Water, and the celebration of the Roman Jubilee were to be inhibited.
Further study of Luther’s writings shows that an endless number of other
deep-seated religious customs, which reflected the active participation of the
people in the life of the Church, were condemned to extinction.

Divine worship and the religious life of the people necessarily languished
after the forcible abolition of these popular customs. The
radical innovations of Lutheranism, so foreign to human feeling, produced
ruins where formerly the seeds of civilization had been strewn
in abundance, even though they were frequently in need of better
care in order to blossom forth vigorously. The aging Luther did not
sense this very noticeable decline of cultural life, but imagined that
he had taught the people to worship God in the spirit, whilst even
among his own followers complaints were rife that he failed to do
adequate justice to human nature, which, in the final analysis, is a
composite of body and soul.

On the other hand, he admitted the existence of other and, in some
respects, even greater ruins. They are touched upon here only in
passing, since most of them have been treated of in a previous part
of this work.

He saw domestic life undermined as a consequence of his arbitrary
loosening of the conjugal tie. His parsons importuned him on this
point with endless letters. Poignantly and frequently he sensed the
decline of the liberty of the Church resulting from the intervention
of civil authority. The position of his jurists, who partly endeavored
to observe the old canon law and partly favored the religious innovations,
became impossible. Ecclesiastical regulations and consistories
but too frequently proved inadequate aids, until they assumed the
character of administrative governmental measures. Luther witnessed
a certain decline in the power of the Schmalkaldic League after the
Landgrave of Hesse had drawn closer to the Emperor. He heard the
coming uproar of the religious war and trembled for its issue, knowing
scarcely any consolation but the day of judgment. The empire
itself, its unity and power, and especially the authority of the Emperor,
were weakened to their very foundation. That his work was
one of the causes of the unhappy condition of the empire was a
thought which he had to bear to his grave.

The ruins which Luther saw round about him, did not, however,
prevent him from asserting his claims. He did not live to see the decisive
defeat of the Schmalkaldians at Mühlberg.
