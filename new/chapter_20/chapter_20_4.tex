\section{Luther’s Death}

Despite his weakened condition, Luther, at the beginning of October,
1545, undertook a journey to Mansfeld. Count Albert of
Mansfeld, a protagonist of Lutheranism, had solicited his aid in the
adjustment of some lawsuits in which he was involved with his
brother Gebhard and his nephew on account of certain revenues
from mines and various legal claims. This first journey to Mansfeld
was followed by a second, on Christmas. Luther requested Melanchthon to
accompany him, which the latter did grumblingly, as he
disliked to listen to the quarrels of contentious people. When Melanchthon
took sick, Luther returned with him to Wittenberg. The litigation
at Mansfeld induced him to set out, in January, 1546, for his
native city of Eisleben, which was also to be the city of his death.

Undoubtedly, these frequent journeys were inspired, at least in part,
by that unrest which so often leads men to change their habitation at
the approach of death, and also by that discontent with Wittenberg
which had driven him to visit Zeitz, Merseburg, and Leipsic but a
short time previously.

On the twenty-third of January, in spite of the rigors of the winter,
Luther once more left Wittenberg, accompanied by his sons, their
tutor and his amanuensis, John Aurifaber, the future editor of the
Table Talks. At Halle he was compelled to spend three days with
Jonas on account of floating ice in the Saale river. The devil, he wrote,
dwells in the water, but he was resolved not to get drowned to give
the pope and his myrmidons cause for delight.\footnote{\textit{Ibid.}, p. 372.}
 In a sermon delivered
in Halle he poured forth his anger against popery and demanded the
expulsion of the “lousy monks” who still remained in that locality.
“You ought to drive the imbecile, sorry creatures out of town!” On
January 28, he and his escort, which now included Jonas, resumed
their journey across the swollen Saale. Whilst riding in his carriage in
the vicinity of Eisleben, the bitter wind caused him to experience difficulty
in breathing, vertigo, and great debility. “The devil always
plays me this trick,” so he consoled himself, “when I have something
great on hand.” Arrived at his destination, he jocosely wrote to Catherine
that the Jews who lived at Rissdorf had raised the cold wind
against him and attempted to turn his brain to ice.

The litigant counts and their counselors were already present in
Eisleben. They assigned the house of the town-clerk, which still
stands, to Luther, Jonas, and Aurifaber, and liberally provided for
their sustenance. Luther extols the Naumburg beer, says that his
three sons had gone on to Jena; two of them returned to the narrow
quarters at Eisleben, but spent a great deal of their time in Mansfeld,
which lay close by. Luther’s friend, Coelius, who was courtpreacher
in Mansfeld, was also in the house. Luther entertained his
friends every evening in a room on the ground-floor beneath the quarters
which he occupied; he enjoyed his meals, drank heartily, and exhibited
a jovial disposition. When Catherine again expressed fears
about his health, he replied in a jocose vein, giving her an account
of all that her anxious thoughts had brought upon him: the fire that
broke out just in front of his door, had almost burnt him up; the
plaster that fell down from the ceiling of his room had nearly killed
him. “I fear, if you don’t put your fears to rest, the earth will finally
open and swallow us up.”\footnote{\textit{Ibid.}, pp. 373 sq.}

While he sojourned at Eisleben, Luther preached four sermons,
which severely taxed his strength. In one of them he attacked the
Jews, contending that, if they did not become Christians, the rulers
were obliged to expel them. He upbraided the Jews of Mansfeld not
only for hating the Christians, but also for cheating them and practicing
usury--which charge was not without justification. In assailing
the Jews he was well aware that the Countess of Mansfeld was regarded
as their protectress. Twice Luther partook of the Lord’s
Supper while at Eisleben, after having “absolution” administered to
him, as is reported. On the occasion of his second communion, he
ordained two priests, in conformity with what he alleged was apostolic practice.

He was annoyed to find that the negotiations designed to effect a
reconciliation between the counts proceeded in an extremely tedious
manner. He blamed the devil for the stubborn resistance that was
offered to the proposals of mediation put forth by the experts and by
himself. All the devils, he said, had convened at Eisleben to cast
mockery upon his efforts in this sorry affair. He writes that he was
prepared to rush in upon the disputants like a “hobgoblin” and “to
grease the wagon with his anger.” In the excess of his excitement
he experienced the above-mentioned hallucination of the devil seated
on a fountain--a scene which, as Coelius writes, caused him to shed
tears because of the malicious indecency which Satan exhibited
towards his person. In the end a satisfactory settlement was reached.
On February 14 he wrote to his “dear and amiable wife”’: “God has
shown great favor here; for the rulers have practically settled all
their differences with the aid of their councilors.” He announced at
the same time: “We expect to return home this week, if it so please
God.”

On February 16, sickness and death were the topics of a very lively
discussion during supper, according to the notations of Aurifaber.
During these discourses, Luther said: “When I shall arrive at home
in Wittenberg, I shall lay me in a coffin and offer the maggots an
obese doctor to feed on.”\footnote{\textit{Tischreden}, Weimar ed., Vol. VI, n. 6975.}
 As yet, he did not anticipate that death
would overtake him at Eisleben.

Worry over the religious situation to some extent diverted Luther’s thoughts
from the fate which confronted him. He learned how
strictly the Emperor insisted upon submission to the Council which
had already begun; how fruitless were the proceedings of the religious
conference at Ratisbon; how, after the failure of all attempts at reunion,
the empire was preparing for the oft-threatened war against
the Schmalkaldians. Other disquieting reports were brought by Prince
Wolfgang von Anhalt and Count Henry von Schwarzburg, two
friends of Luther, who had come to Eisleben to act as arbitrators for
the counts of Mansfeld. “The Emperor is unalterably opposed to us,”
sighed Luther; “he reveals now what he has hitherto concealed.” His
solicitude for his elector inspired him to utter these words: “God save
our gracious Lord; he is due for a struggle!” On one occasion he absented
himself from his fellow-boarders, as they were at supper, exhorting them
to “pray for the realization of the designs of God, that
it may go well with the affairs of the Church; the Council of Trent
displays a great deal of indignation.”

At the same time, his spirit was wrapped in gloom because of the
violent opposition that developed on the part of the Swiss and other
Sacramentarians to his doctrine of the Last Supper. These renewed
attacks had been called forth by his recently published “Brief Confession.”
His displeasure with them was undoubtedly heightened because of the
further fact that he reread those of his works in which
he had vented his anger during the controversy on the Eucharist,
and which were included in the volume of his German works just then
in course of publication. Hence, in his sermons at Eisleben, he paid
his respects to the Sacramentarians and in forceful words warned
his hearers against the arbitrary interpretation of Sacred Scripture
by “that prostitute, human reason.” His last notes, written on Feb.
16, 1546, appear to be a reference to the misuse of the Bible by the
Sacramentarians. “No one,” he says, “has sufficiently tasted the Sacred
Scriptures who has not governed churches with the prophets for a
hundred years.” He claims, moreover, that one must have been a
shepherd or a farmer for five years in order to understand the Buccolics
and Georgics of Vergil; and, in order to appreciate Cicero’s letters,
it was necessary for one to have spent twenty-five years in the
service of an important state.\footnote{\textit{Briefwechsel}, Vol. XVII, p. 60.}
 Shortly before he had inscribed in the
postil of a friend the following words, based upon John VIII, s1,
which sound like a presentiment of death: “If any man keep my
word, he shall not see death forever.” In this connection he also
wrote: “Blessed in the Word is he who believes and goes to sleep and
dies over it.”\footnote{Köstlin-Kawerau, \textit{Martin Luther}, Vol. II, p. 620,}
 This was a favorite thought with him. Frequently
during these days he also quoted the text: “God so loved the world,”
etc., which he undertook to explain in a lengthy address to his friends.

The first indication that his heart trouble was growing worse was
noticed on the 17th of February, when he grew restless and said,
among other things: “Here at Eisleben I was baptized. What if I
were to remain here?” In the evening he was seized with oppression
on the chest, a pain he had experienced in former ailments. He felt
relieved when rubbed with hot cloths and went down for supper with
his companions, with whom he ate and drank copiously in his usual
jovial mood, recounting anecdotes and participating in the serious discussions
which were carried on. Among other things the mutual recognition
of friends in Heaven was debated. In Heaven, he said, we shall
recognize each other in virtue of the illuminating spirit of God, who
caused Adam immediately to recognize Eve as flesh of his flesh, whom
He had built from a rib taken from his own body whilst he was
asleep. He also discussed his own death and the devil. He said he had
lived to be sixty-three years of age in order that he might witness
all the malice, faithlessness, and misery which was caused by the devil
in the world. The human race is like sheep being led to slaughter.\footnote{Grisar, \textit{Luther}, Vol. IV, pp. 376 sq.}

Shortly afterwards he repaired to his room. It was still early in the
evening; he recited his prayers at the open window, as was his custom,
and then retired for the night.

While at prayer, a new severe attack of heart oppression came on.
His friends again hurried to his aid, tried to give him relief by rubbing
him with hot cloths. He got an hour’s sleep on a sofa in his
room, after Count Albrecht, who had been summoned, and his relatives offered
him various remedies. He refused to have the doctors
called, as he did not think there was any danger. Having told his
friends, who in the meantime had come to see him, to retire, he
withdrew to his chamber. Jonas and a servant, Rudfeld by name,
had a couch in the same room, which was small and lacking in ventilation.
The couch which Luther occupied may still be found in the
self-same crowded room. He slept in his bed only from ten or eleven
o’clock until about one in the morning, when he got up and awakened Jonas,
saying to him: “Oh, my God, how ill I feel!” Aided by
Jonas and the servant he dragged himself into the sitting-room,
saying he would probably die at Eisleben after all, and repeating the
prayer: “Into thy hands I commend my spirit!” As he lay outstretched on
the couch, the constriction of his heart became unbearable.

The inmates of the house, the counts and the princes, who appeared
at intervals to express their solicitude and sympathy, were kept informed
of the condition of the patient. Two physicians, one a doctor
and the other a master of medicine, were hurriedly summoned. Before
they appeared, however, the malady had completely overcome the
patient. They found him lying on the couch unconscious and with
no perceptible pulse. After a brief interval, however, Luther recovered
consciousness, and, bathed in the cold sweat of death, was heard
to exclaim: “My God, I am so ill and anxious; I am going.” He also
recited some prayers, in which he expressed his thanks to God for
having revealed to him His Son Jesus Christ, in whom he believed,
whilst the hateful pope had blasphemed Him.\footnote{\textit{Ibid.}, p. 378.}
Thrice he repeated
the familiar verse: “God so loved the world,” etc. In vain Count Albrecht
and his relatives offered strengthening or refreshing draughts
to the patient. As he again lay practically unconscious, Jonas and
Coelius, in order to obtain a confession from him in the presence of
the attendants, shouted into his ear the question whether he remained
steadfast in the faith in Christ and His doctrine which he had
preached; to which they caught the reply: “Yes.” That was his last
word. He did not recall his life’s partner at Wittenberg, nor did he
mention his children. It seems the stroke had stupefied him and
blotted out the memory of those dear to him.

About three o’clock in the morning Luther drew a deep breath and
departed this life to return his soul into the hands of the eternal
Judge. It was a frosty morning (February 18, 1546), and the earth
was still veiled in darkness.

During this last crisis, or soon after, John Landau, an apothecary
of Eisleben, was sent for with the request to attempt to restore animation
by the application of a clyster. Landau was a convert to the
Catholic Church and a nephew of the famous controversialist Wicel.
He came at once, but, after examining Luther, who had already departed
this life, declared: “He is dead; of what use can an injection
be?” The physicians, however, insisted upon a try with the instrument,
so that the patient might again come to himself if there still
was life in him. When the apothecary inserted the nozzle he noticed
some flatulency given off into the ball of the syringe. The two physicians
disputed together as to the cause of death. The doctor said
it was a fit of apoplexy, for the mouth was drawn down and the
whole of the right side discolored. The master, on the other hand,
maintained that so holy a man could not have been stricken down
by the hand of God and that it was rather the result of a suffocating
catarrh and that death was due to choking. Neither knew anything of
Luther’s chronic disease affecting the arteries of the heart.\footnote
{Cf. the opinion of Dr. Guido Jochner in the Innsbruck \textit{Zeitschrift für kathol. Theologie},
Vol. 45 (1921), pp. 486 sqq., and also that of Dr. Tscharnak in Janssen-Pastor, Vol.
IIT (18 ed.), p. 6o1. Of material importance is Jonas’s letter in Kawerau’s edition of his
correspondence, Vol. II, pp. 182 sq., where he says Luther suffered from heart disease in
the year before his death. In Ioh. Manlii Libellus Medicus (Bisle, 1563), pp. 24 sq., it
is stated that kapdiaxn ie., heart disease, was the cause of Luther’s death. Manlius also says:
“\textit{Paulo ante mortem mibi scripsit, se eo morbo rursum tentatum esse}.” Melanchthon wrote
to V. Theodorus (Corp. Ref., VI, p. 68): “\textit{Non apoplexia, non asthmate extinctus est}.”
But the real cause of Luther’s death he did not know cither.}
The true
cause of his death was neither apoplexy nor catarrh.
After Luther’s death, all the distinguished guests assembled in his
chamber. Jonas, who sat at the head of the bed, wrung his hands
and wept aloud. He assured the others that Luther had been more
cheerful on the previous evening than for many a day. “Oh, God
Almighty, God Almighty!” he exclaimed. The apothecary was bidden to administer
a thorough rubbing to the nose, mouth, forehead,
and left side of the corpse with a costly scented fluid which the counts
had brought. The guests still expected to see signs of life and remarked
that on several former occasions Luther had lain for a long
time motionless and was thought to be dead, for instance, at Schmalkalden,
in 1537, when he was tormented with gall-stone. The apothecary soon
ascertained that rigor mortis had set in. Jonas then suggested
that a detailed report be at once dispatched by courier to the Elector
of Saxony. About four o’clock in the morning, he composed a comprehensive
account of the event, assisted by Coelius and Aurifaber.

In the meantime the corpse, still lying on the couch, was re-arranged
so as to enable the expected visitors to obtain a better view of it.
After sunrise Luther’s friends sent for a painter to draw the features
of the departed. The picture, which represents Luther lying on the
couch, was unsatisfactory and a second picture, based in part on the
first, was made the following day. The painter of the second picture
was Lucas Fortenagel of Halle; when he went to work, Luther already
lay in his coffin. This portrait is preserved in the university library of
Leipsic, which also preserves a less perfect representation depicting
Luther’s head as resting on a pillow. It is probable that this latter
picture is the one drawn immediately after his demise.

If a Catholic opponent of Luther, familiar with his life and deeds,
a man noble-minded and sympathetic of heart, had entered that room
in the morning after the reformer’s death, what would have been his
thoughts? Above all else he would have implored God to be merciful
to the soul of the departed man, thus complying with the teaching of
Him who commands men to love even their worst enemies. Then,
there would have flashed before his mind’s eye the monstrous and
embittered attacks launched by Luther upon that sacred institution,
the indestructible Church established by Jesus Christ at the price of
His blood and founded upon Peter and his successors. In spirit, he
would have beheld the deep wounds inflicted upon that Church by
this man, so remarkably endowed with eloquence, will-power, and
energy. How many thousands of souls redeemed by Christ, he would
have said to himself, have been torn from the Saviour’s living body by
this man, without any fault of their own, and frequently without
their knowledge, bequeathing their misfortune to posterity. But yielding
to mercy, he would also have recalled the fateful enthusiasm of
the dead reformer for his own cause, and that profound and serious
self-delusion which domineered his ardent temperament with ever
increasing force since the inception of his contest with Rome. Did not
Luther, thus the spectator might have soliloquized, eventually find
himself in a state of true mental obsession, though, of course, of his
own volition and which, at least in its inception, had been caused by
himself? Was it an obsession which allowed him to see naught else but
his supposed vocation as the promulgator of a new and true Gospel,
directed against Antichrist and the demoniac forces, just as he imagined
the imminent dissolution of the world and the advent of the
Great Judge? Did this delusion, in the evening of his life, incapacitate
him for receiving even one ray of light?

If our hypothetical friend, thus absorbed in reflection at the bier of
Luther, had been granted an insight into the mental evolution of the deceased,
\textit{i.e.}, into his psychological condition since he left the parental
roof, his frightening experiences at entering the monastery, as well as
the state of despondency and the constant struggles caused by his disease,
he would have felt all the more inclined to pronounce a charitable
judgment on the dead reformer. Was Luther a great man? he might
have asked himself, as he left the chamber of death impressed by these
reflections. There could be no other answer than this: If he is to be
called great, his greatness is negative. As our observer later in life
recalled the stirring scene in Luther’s death chamber, he might have
entertained the hope that the misguided reformer would be saved.
Janssen, the great Catholic historian who penetrated so deeply into
the inwardness of the Reformation period, used to recommend to
converts who sought his guidance to pray for the repose of Luther’s
soul.\footnote{Cited by Pastor toward the close of his biography of Janssen.}
 God alone searches the hearts and reins of men. Human understanding is too limited.

The account we have given of the circumstances of Luther’s demise
is based, first, on the report made the same day by Jonas to the
Elector; secondly, on letters written by other eye-witnesses either on
the day of Luther’s death or immediately afterwards; thirdly, on the
account of the Catholic apothecary Landau, on the funeral orations,
and especially on the “Historia” of Luther’s death composed at Wittenberg
by Jonas, Coelius, and Aurifaber and apparently published
about the middle of March.\footnote{The “Historia” in Walch’s edition of Luther’s Works, Vol. XXI, pp. 280 sqq. Landau’s
account first appeared in Cochlaeus, \textit{Ex Compendio Actorum M. Lutheri} (Moguntiae, 1548;
cf. Grisar, Luther, Vol. VI, p. 379, n. 2). All the accounts of Luther’s death were more
recently collected by Jak. Strieder, Bericht diber Luthers letzte Lebensstunden (1912; cfr.
the same writer’s article, \textit{Luthers letzte Stunden}, in the Histor. Vierteljahsschrift, 1912,
No. 3), and Christoph Schubart, \textit{Die Berichte}, etc., (1917), which contains 2 more detailed
account based on all the letters. The best exposition according to the sources is supplied
by N. Paulus, \textit{Luthers Lebensende} (1898), whose conclusions have not been affected by the
less important sources which have since come to light.}
While it cannot be denied that the
letter of Jonas and the “Historia” contain palpable exaggerations
concerning the pious aphorisms and prayers of Luther--expressions
of devotion of which he was hardly capable in consequence of his
repeated lapses into unconsciousness,\footnote
{In the \textit{Lutherkalender} for 1911, p. 93, A. Spaeth concedes that the first letter of Jonas
and the “Historia,” may have been inspired by a desire to represent Luther’s death in as
favorable and edifying a manner as possible.}
there is, however, no adequate
warrant for impugning the substantial credibility of this and other
accounts, as has been done in recent times. In view of certain accounts
that originated in foreign countries and were written for polemical purposes,
it has been asserted that Luther was found dead in bed at daybreak,
and that, accordingly, all the occurrences reported of the night
of his death are fictions invented for the purpose of concealing
the disagreeable facts in the case. But such a colossal deception was
an impossibility because of the large number and the rank of those
who knew the facts at first hand, including several women. The calm
and detailed account which Landau, the Catholic apothecary, published
in 1548, absolutely forbids the acceptance of the above-noted
arbitrary theory of deception. Moreover, a falsification of facts, such as
is here supposed, would most assuredly have assumed a different form.
It would not have failed to mention that Luther spoke affectionately
of his Catherine, and to describe a touching scene in which the dying
father bade farewell to his children.

The fable of Luther’s alleged suicide, which some writers (notably
P. Majunke) have exploited in recent years, is based on an apocryphal
letter, attributed to an alleged servant of Luther, whose name is not
mentioned. It was circulated about twenty years after Luther’s death
among his opponents, particularly in foreign countries. The story of
the unknown servant was mentioned for the first time in a book
which the Italian Oratorian, Thomas Bozius, published at Rome in
1591. The Franciscan, Henry Sedulius, was the first to print the text
of the letter in a book published at Antwerp in 1606. In this letter,
the servant is quoted as stating that he discovered “our Master Martin
suspended from his bed, wretchedly strangulated.”\footnote
{In the \textit{Allg. Deutsche Biographie}, Vol. LII (1900), pp. 156 sq., we read that the fable
of Luther’s suicide is no longer defended by any serious Catholic, nay, that Catholics have
been among its foremost opponents.}

The fable belongs to a category of inventions, quite common at the
time, devised for the purpose of imputing a disgraceful death to
an opponent, especially if he happened to be an ecclesiastic. Many
prominent men were made to die in despair and impenitent, or to terminate
their lives by suicide.\footnote{Cf. Grisar, \textit{Luther}, I, 303 sq.}
Luther himself was notorious for this
form of fabrication, and readily placed credence in reports of this
kind.

Strange, too, are the amplifications made by certain authors regarding
the legends of Luther’s decease. It is claimed that he had “his nun”
with him on the fatal night of his death; yea, that Catherine Bora
strangled him during a quarrel. Others allege that the devil either carried
him off alive or murdered him.

The above-noted accounts of Luther’s death are not surprising
in view of the mass of false statements made about the reformer in
succeeding ages by short-sighted and uncritical Catholic authors, who
were horrified at the way in which he ravished the Church. Thus it
was alleged that he inwardly abandoned all his doctrines in his old
age; that he contemplated a return to the papacy, without, however,
being converted; that he said to Catherine on one occasion, as she
admired the starry firmament, that heaven was not for them. It was
asserted that he had three children apart from those born of his marriage;
that he indulged in “orgies” with escaped nuns; that he began
his fight upon the Church in order that he might be able to marry
while yet a monk; that at a later date he advised people in writing to
pray for many wives and few children; that he was the author of the
saying: “Who loves not woman, wine, and song, remains a fool his
whole life long.”\footnote
{These slanders are incidentally repudiated in our text; see also our larger work, \textit{Luther},
Vol. III, pp. 280 sqq.; Vol. V, p. 372. On the charge of inebriety, cfr. Vol. III, pp. 294
sqq.}

Rumors were circulated especially about his inebriety and habitual
excesses at table, which have already been mentioned, in connection
with which certain misconstrued jokes were reproduced. It is claimed
that he indulged excessively in eating and drinking on the eve of
his death. He was described as extremely corpulent, a characteristic
supposed to be verified by his own previously adduced phrase of
the “obese doctor.” He was rather stoutish, as the portraits of his
corpse reveal; but this was only after he had reached middle age.
Such exaggerations as that contained in the celebrated verse of
Gothe’s “Faust” are to be rejected.

His inveterate opposition to the pope, which he reaffirmed shortly
before his death at Eisleben, was embellished by a very questionable
flourish of his friend Ratzeberger, who was not even in Eisleben at
the time. He informs us that Luther, ag he partook of his last meal,
wrote the following celebrated verse on the wall with a piece of
chalk: “\textit{Pestis eram vivus, moriens ero mors tua, papa}” (In life, O
Pope, I was thy plague, in dying I shall be thy death). The silence of
other sources, particularly that of the panegyrics, where Luther’s
previous use of this verse is mentioned, renders Ratzeberger’s account
rather incredible.\footnote{Grisar, \textit{op. cit.}, Vol. V, p. 1025 Vol. VI, pp. 377-394.}
 The so-called death-mask of Luther, preserved
at Halle, is also the product of an erroneous Protestant tradition.
According to the investigations of Frederick Loofs, professor of theology
at Halle, it originated in the eighteenth century.\footnote{Loofs in \textit{Religiose Kunst} (1918), No. 1, pp, 2-15.}
 There was
a natural desire to have authentic memorials of the famous man.
Likewise most of the objects exhibited at the present time as having
supposedly been left behind by the deceased, are insufficiently attested.

Catholic controversialists distorted his obsequies\footnote{Grisar, \textit{op. cit.}, Vol. VI, pp. 394 sqq.}
 They alleged
that when the funeral procession arrived at Wittenberg, the coffin was
found empty. According to others, the hearse had to be abandoned on
the road to Wittenberg because of the horrible stench emanating from

the corpse. A number of rooks circling in the air about the corpse at
Halle were later made out to have been devils, “who came to attend
the burial of their prophet.” Persons who were possessed by the devil
remained unmolested at that time, since all the devils were taking part
in the funeral, and so forth. These tales merely prove how greatly the
Catholics had been horrified at Luther’s conduct.

Having waded through the legends occasioned by the death of
Luther, we must now attend to his obsequies. The body was enclosed
in a tin coffin at Eisleben. After Jonas and Coelius had delivered memorial
addresses there, the remains were conveyed to Halle, on February 20, thence
to Wittenberg, on the morning of February 22. At
the Elster Gate--the scene of the famous burning of the Bull of Excommunication
--the coffin was received by the university, the towncouncil, and the
burghers, and escorted to the Schlosskirche, where, by
order of the Elector, Luther’s mortal remains were to find their last
resting-place. On Feb. 22, they were interred in front of the pulpit,
where they still rest at the present day.\footnote
{An investigation made February 14, 1892, revealed the presence of Luther’s remains
in the Schlosskirche at Wittenberg. Hence, they were not removed, as was charged, after
the entry of the victorious imperial troops in the Schmalkaldic War.}
It is worthy of note that the
day of Luther’s interment was the Feast of the Chair of St. Peter, or,
as formerly known in the Catholic Church, the Feast of the Institution
of the Papal Primacy.

When the procession that escorted the corpse arrived at the castlechurch,
Bugenhagen delivered a funeral oration. This was followed
by a eulogy pronounced by Melanchthon. All the addresses delivered
on this occasion, including those of Jonas and Coelius, previously referred
to, have been preserved in print.\footnote
{Cf. Grisar, \textit{Luther}, Vol. VI, pp. 387 sqq., where excerpts are given.}
In the mind of Bugenhagen,
Luther was “without doubt the angel of which the Apocalypse speaks
in Chapter XIV: ‘And I saw an angel flying through the midst of
heaven, who had an eternal gospel to preach!’ ” God the Father, according
to Bugenhagen, “revealed” the \textit{evangelium aeternum}, the
great mystery, through Luther, “the divinely appointed reformer of
the Church.” Melanchthon, in his funeral oration, similarly extolled
the deceased as one of a long line of divine tools starting in Old
Testament times, as a man taught by God and exercised in severe
spiritual combats of a friendly nature, not at all passionate or quarrelsome,
and only inclined to be violent when such medicine was required
by the ailments of the age. He said Luther was endowed with
all the gifts of God enumerated by the Apostle Paul in his Epistle to
the Philippians (IV, 8), where he says: “Whatsoever things are true,
whatsoever modest, whatsoever just, whatsoever lovely, whatsoever
of good fame.” Now, he concluded, he has gone to join the company
of the prophets in Heaven.

No more impressive contrast to these eulogies can be conceived than
the hymns of praise chanted by the Church on this very day in honor
of Blessed Peter, and of his successor in the Apostolic See: “\textit{Tu es
Petrus}”--thou art Peter, the holder of the see, against whom the
gates of hell shall not prevail.

In a bulletin in which he announced the death of Luther to the
students of Wittenberg University, Melanchthon said: “Alas, the
chariot of Israel and the driver thereof have departed (4 Kings, II,
12), who has ruled the Church in this old age of the world. Human
wisdom has not discovered the doctrine of the remission of sins and
fiduciary trust in the Son of God, but God has revealed it through
this man.”

In Coelius’ address at Eisleben, Luther was represented as appearing
a “true Elias and Jeremias,” a “John the Baptist or an apostle before
the great day of the Lord.” Jonas in his sermon prophesied that now
all papists and monks would “turn to dust and perish,” as Luther himself
had frequently predicted as a consequence of his death; thus the
death of the prophet would exercise a peculiar influence on the godless
and impenitent; yea, within two years the deluded papists would
be overtaken by a “dreadful punishment.”

In harmony with these effusions medals were struck bearing Luther’s celebrated
verse, “\textit{Moriens ero mors tua, papa}.” Epitaphs appeared both in verse and
in prose, particularly at Wittenberg, replete
with the most exaggerated praise. A noteworthy leaflet of this character,
appearing anonymously, was probably the product of Paul
Eber.\footnote
{O. Clemen, \textit{Gedichte auf Luthers Tod}, in the \textit{Jahrbuch der Luthergesellschaft} (1919),
pp. 59 sqq.; the same, \textit{Flugschrifien aus den ersten Jabren der Reformation}, Leipsic, 1907;
cfr. \textit{Zeitschrift für katholische Theologie} (1922), pp. 137 sqq.}

The shop of Cranach flooded Protestant Germany with portraits
of Luther, which were of questionable worth. The defiant,
coarse-grained nature of the man is strongly emphasized in these
representations, which, though they by no means completely corresponded
with each other, form the basis of the typical portrait of Luther
which came into use later on. A cloud of spoken and written encomiums,
uttered in the style of funeral sermons, overcast the memory of Luther,
fascinated the impressionable masses and prevented
thousands from obtaining a true insight into the facts of the case and
the real character of the man.
