\section{After Luther’s Death}

The glorification of Luther by his biographers deserves a special
treatment on account of its after-effects.

Among the earliest biographies, or rather the attempts at such,
which were destined to establish the fame of Luther, mention must
be made of the biographical sketch which Melanchthon published
at the beginning of the second volume of Luther’s Latin works, which
appeared in 1546. The writer either closed his eyes to the defects of
Luther’s character, or excused them. In his admiration of Luther’s
greatness, he completely forgot the pique which he suffered at his
hands.

John Mathesius, an enthusiastic disciple of Luther, but of no pronounced
talent, while pastor of Joachimstal in Bohemia, delivered a
series of sermons on Luther which were crammed with historical and
unhistorical assertions--“pious panegyrics,” as they are properly styled
by the Protestant historian William Maurenbrecher. In his eagerness
to edify his hearers, Mathesius disregarded the facts. His sermons appeared
in print at Nuremberg, in 1566, twenty years after Luther’s
death, under the title, “Historien von des ehrwirden in Gott seligen
thewren Manns Gottes Doctoris M. Lutheri Anfang, Lehr, Leben und
Sterben.” Due to their popular style, they have enjoyed a wide circulation
up to the present time.

In the same year, John Aurifaber, one of the witnesses of Luther’s
death, published at Eisleben his “Colloquia oder Tischreden” of Luther,
to which we have already adverted. The preface, addressed to the
“imperial cities of Strasburg, Augsburg, Ulm, Nuremberg,” etc.,
eulogizes Luther as “the venerable and highly enlightened Moses of
the Germans.” The contents of this work, partly entertaining and
partly instructive, display a popular and rather blunt style and wielded
an incredibly powerful influence on the masses and thereby confirmed
the domination of Luther over many minds. Others, notably Stangwald (1571)
and Selnecker (1577), were inspired by Aurifaber’s
success to issue similar publications.\footnote{On these and other biographers of Luther see Grisar, \textit{op. cit.}, Vol. VI, pp. 388 sqq.}

Cyriacus Spangenberg, a fanatical admirer of Luther, composed a
book entitled, “Theander Lutherus,” which was divided into sermons.
The principal title is followed by a high-sounding, lengthy subtitle,
which enables one to infer the tendency and worth of the whole
work. In this subtitle Luther is called “the esteemed man of God,”
“a prophet,” “an apostle,” and “an evangelist,” “the third Elias,”
“the second Paul,” and “the true John,” “a most excellent theologian,”
“the angel of the Apocalypse,” etc. These sermons, delivered
at Mansfeld in 1562, first appeared separately, but were afterwards
published in a collective edition (1589). They did not, however,
attain to the popularity of Mathesius’ “Historien.”\footnote{\textit{Ibid.}, p. 391.}

Flacius Illyricus, a professor at the University of Wittenberg, and
Nicholas Amsdorf, were two of the most enthusiastic champions of
Luther, who achieved eminence not by writing biographies of their
hero, but by eulogizing and battling for him. Thus Flacius, in his
book on “The Marks of the True Religion” (Magdeburg, 1549),
treated of the mark of sanctity, which he discovered not only in the
doctrines of his deified master, but also in his extraordinarily pious
life, abounding in examples of Christian virtue. Amsdorf, likewise, describes
Luther as a saint, the equal of St. Paul. He maintains
that Luther was “raised by a special grace and given to the German
nation.” He strongly stresses his German nationality. By laying emphasis
on Luther’s patriotism, efforts were put forth at a very early
date to persuade the “German nation” of its obligation to honor its
great leader. German nationality, German customs, and German patriotism
were made to serve as effective levers to raise the figure of
Luther to a high pedestal.

However, the authors just mentioned, like many later eulogists of
Luther, cannot avoid painful references to the serious schisms and
counter-currents of the time. Many theologians and preachers differ
in their teaching from the so-called orthodox or Gnesio-Lutherans,
without ceasing to extol Luther. Amsdorf complains about these
“pretenders to wisdom,” Flacius Illyricus fulminates against them as
“apostates.” Many were persecuted by the inflamed theologians of
the new religion. Mathesius is alarmed at seeing “all sorts of impure
and poisoned water” penetrating the “pipes of Wittenberg,” through
which the waters of life are dispensed.

What Luther had frequently foretold came to pass after his death.
The profound theological controversies that agitated the schools and
churches soon entailed the intervention of the civil governments.
The names used to describe the controversies (such as Osiandric,
Majorite, Adiaphoristic, and Synergistic disputes) are reminiscent of
movements that were as replete with theological contrasts as with
passions and hatred.\footnote{\textit{Ibid.}, pp. 408 sqq.}
How the Gnesio-Lutherans, and particularly
Flacius and his followers, were singled out for attack, may be seen
from a cannon in the fortress of Coburg, cast at that time, in which
the favorite court-preacher is portrayed in the act of seizing an adherent
of Flacius by the throat and strangling him. In electoral Saxony, the
classic land of the Lutheran Reformation, Cryptocalvinism,
so-called, gained tlie ascendancy under the Elector August, who became
ruler in 1553. The representatives of this movement published
a \textit{Corpus Doctrinae Philippicum}, extracted from the writings of Melanchthon,
which deviated from the teaching of Luther. Although
protected by the civil authorities, Melanchthon and the “Philippists,”
as his followers were called, suffered much from the persecution of the
Lutheran theologians.

More tranquil times dawned only after “orthodox” Lutheranism had established
its rule. The extravagant praises of Luther
resounded everywhere during this period; but soon the Age of Enlightenment
came and effected a considerable decline of Luther’s influence among scholars
and the educated laity. The contradictory
nature of his doctrines and their defects were more widely recognized
and conclusions drawn from his premises which, while they were not
illogical, would have been very unwelcome to Luther himself. Some
even dared to criticize publicly the character and private life of the
founder of Protestantism. Finally, the rise of the historico-critical
method threatened to impair the esteem harbored for his doctrines
and person. About the time of the centenary celebration of Luther’s
birth (1883), however, a reaction favorable to his reputation set in
among influential Protestants in Germany. This was due to various
circumstances, not the least of which was aggravated opposition to
a newly ascendant Catholicism. In conformity with modern ideas,
Luther was now hailed as a champion of liberty and civilization, a
guide to a new spiritual future, as well as the representative of the
national ideals and customs of Germany.\footnote
{Probably only a few individuals, however, regarded criticism as prohibited to such a
degree as the author of a prominent jubilee book for 1917, who wrote: “After four hundred
years, we do not feel justified in criticizing the shade of this great and singular man.”
But he admits that he “likes the Luther of the diet of Worms better than the Luther of
the year 1545.”}

In the World War he was
to be the hero of unadulterated and triumphant German tradition and
inspiration; but the defeat of the Central Powers disappointed the
audacious hopes of Lutheranism.\footnote
{Cf. H. Grisar, \textit{Der deutsche Luther im Weltkrieg und in der Gegenwart}, Augsburg,
1924.}

In the meantime, especially since the revolution, the religion of
Luther has, in many respects, assumed the rdle of a so-called “German
religion” without dogmas. The Protestant churches, honoring his
name as a symbol of their title, are actively engaged in securing their
future under a new form, the former system of national churches
having ceased to exist. Some laudably endeavor to preserve for Protestantism
the positive Christian elements which Luther retained. In
general, however, the religious Luther is relegated to the background.
Though his admirers ought to consider him primarily as a religious
innovator, they abandon, with striking unanimity, the religious phase,
and, instead, celebrate Luther as a champion of modern culture. This
became evident in 1917, at the time of the fourth centenary of the
Reformation, and during the celebrations commemorative of that
event in the ensuing years, such as that at Worms.\footnote
{Grisar, \textit{Lutherstudien}, n. 1: “\textit{Luther zu Worms und die jüngsten drei Jahrbundertfeste
der Reformation},” Freiburg, 1921.}
A closer inspection
of the voluminous scientific and popular literature of this period, and
of the flood of published addresses delivered at the larger
festival assemblies, elicit amazement at the thoroughness with which
the historic Luther has been obliterated. His teaching is discarded as
unimportant, and the highest aims of his life are tacitly treated as
antiquated and obsolete. Public attention is directed to the excellence
of his German style, the literary skill shown in his translation of the
Bible, the popular appeal of his hymns, and, naturally, his alleged
genuine “Germanism.”\footnote
{Grisar, \textit{Die Literatur des Lutherjubiläums} 1917, \textit{ein Bild des beutigem Protestantismus},
in the \textit{Zeitschrift für kathol. Theologie}, Vol. XLII (1918), pp. 591-628 and 785-814.}
His undaunted courage was eulogized and
his boast re-echoed throughout Germany: “No one, please God, shall
awe me so long as I live.”\footnote{Cf. Grisar, \textit{Luther}, Vol. VI, pp. 396 sqq., where other similar passages are reproduced.}
Just as though the moral value of the
ends pursued, as well as the morality of the means employed, are not
a necessary element in evaluating courage and perseverance! This
very defiance which would assail heaven, the very quality of
“superman,” induced many admirers to refer to Luther as a great
historical phenomenon. Most of them, however, base their admiration upon
something quite different. Luther’s courage has begun to
gain that unrestrained spiritual liberty which they desire to enjoy.
Luther destroyed for his adherents the authority of the old Church.
It is this destructive phase of his activity which makes him so important
to our modern age. The freedom of the intellect which he won
by his struggles, we are told, must be extended. Men must advance
beyond the beginnings which he inaugurated, and strive for a more
natural Christianity. In the attainment of this end Luther must be
our guide. That is the slogan of the great majority of Protestant
scholars.\footnote
{According to Friedrich Loofs, \textit{Wer war Jesus?} (Halle, 1916, p. 216), “all learned
[Protestant] theologians of Germany--even those who do not express themselves openly--
are agreed that the ancient orthodox theology of the two natures in Christ cannot be maintained
in its traditional form.” Belief in the divinity of Christ is relinquished. “All systematic theologians are seeking new ways in Christology.” (Cf. p. 180).}
Luther’s responsibility for such a fate, which is tantamount
to a disavowal of his life’s work, cannot be denied. And yet, with
the aid of Lutheran propositions, \textit{i.e.}, a selection of his doctrines on
the religion of Christ, reproduced in his own forceful language, it is
possible to deliver a scathing indictment against the ever-increasing
ranks of his admirers.
