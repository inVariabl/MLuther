\section{At Death’s Door}

The last will and testament which Luther made during his severe
illness in 1537 at Gotha upon his return from the Schmalkaldian conference,
boldly declared that he had done right in “storming” the
papacy “with the aid of God’s Word.”\footnote
{Text \textit{ibid.}, Vol. XI, pp. 209 sq., of February 28, 1537. Cfr. Grisar, \textit{Luther}, Vol. III,
pp. 436 sq.}
A second and final testament,
dated January 6, 1542, the original of which is preserved in the
Archives of the Augsburg Confession at Budapest, refers with equal
solemnity to his claim that “the Father of all mercies has entrusted
to him the Gospel of His beloved Son.” He styles himself “God’s notary
and witness in His Gospel.”\footnote{Text in \textit{Briefwechsel}, Vol. XIV, pp. 149 sq.}

This testament, as well as the first, lacks the necessary legal formalities.
His controversy with the lawyers, who refused to regard his
marriage as valid and his five children as his legitimate heirs, induced
him to disregard the notarial form and accounts for the fact that he
styles himself a notary of God, whereby he sought to justify himself,
and also for the high-sounding words he used in that document:
``Let it be admitted that I am the kind of person which I truly am,
namely, public, one who is known in heaven and on earth, as well
as in hell, and who possesses sufficient reputation and authority to
be trusted and believed in preference to any notary.” At the same
time he styles himself a “condemned, poor, undeserving, miserable
sinner.”

He testifies in this document that his “dear and faithful housewife,
Catherine,” “ever treated him with love, appreciation, and affection
as an upright and faithful consort.” He demands that his children
should “look into her hands,” not she into theirs. He bequeathes to
her his estate at Zulsdorf and a house he had purchased from Bruno
Bauer; also his “goblets and jewels, such as rings, chains, and show
coins, of gold and silver,” valued at about 1,000 gulden. These bequests,
however, were subject to the obligation of liquidating his
debts, which amounted to approximately 450 gulden. Ready cash,
he says, he had none, owing to the expense of keeping their house
in repair and maintaining the household. The monastery in which he
resided automatically reverted to the Elector, whom he requests “graciously
to guard and administer” his legacy to Catherine. The
authenticity of his signature is confirmed at the bottom of the document
by Melanchthon, Cruciger, and Bugenhagen. An imperial decree required
that a testament which was not drawn up by a notary,
should bear signatures of seven witnesses, together with their seals.
But the elector supplied these deficiencies and officially confirmed
Luther’s last will and testament on April 11, 1546.

At the age of sixty, Luther wrote to his elector, on March 30, 1544,
that he was “old and cold and ungainly, weak and sickly.” He feared
that evil times were coming, but the consolation of the “dear Word of
God,” and of prayer, would ever remain to his territorial lord. “The
devil, the Turk, the Pope and his followers, cannot enjoy” these
two unspeakable treasures.\footnote{Grisar, \textit{Luther}, Vol. VI, p. 341.}
This, in a certain sense, was his farewell
letter to his protector.

The sufferings of which Luther complained were constriction of
the chest and heart, a result of Lardening of the arteries, and renewed
“phobias,” in addition to the extraordinary nervousness which
accompanied him throughout life; finally, dizzy spells and pains from
gallstone.

Notwithstanding his afflictions, he continued to take part in literary
controversies almost until his last breath.

Immediately after the publication of his treatise “Wider das Papsttum
zu Rom vom Teufel gestiftet,” he issued a description of the
attitude of Popes Hadrian IV and Alexander III towards
the Emperor Frederick Barbarossa.\footnote{Erl. ed., Vol. XXXII, pp. 358 sqq.}
 It was translated from the Latin history
of the popes by Robert Barnes—a thoroughly unreliable and hostile
book, which Luther furnished with a preface and annotations. This
pamphlet was intended to incite the temporal rulers of his time
against the papacy, which was charged with contempt for, and abuse
of, princes. The prefaces to the various sections of the second volume
of his collected German writings, and the still more important prefaces
to those of the first volume of his Latin works in the
Wittenberg edition, date from 1545. In the general preface to the latter
volume he essays a historical presentation of the origin and development
of his agitated life.\footnote{\textit{Opp. Lat. Var.}, 1, pp. 15 sqq.}
 This narrative is a strongly colored and
deliberate recasting of his early career. “The picture of his youth
is made to tally more and more with the convictions of his later
years.”\footnote
{Thus Hausrath; see Grisar, \textit{Luther}, Vol. VI, p. 191. For the following cf. ch. XXXVII
of the same volume: “Umdichtung des jungen Luther durch den alternden.”}
It may be well to make a résumé of this artificial account
in its main outlines, since Protestant biographers accept it as the
truth. According to this fiction, Luther was a blameless, penitent
monk, who was swept into the controversy with the Church only
by his unavoidable opposition to the abuses connected with the sale
of indulgences. At first he was unaware of the stupendous theological
abyss which separated him from the teaching of the Church and quite
unconscious of his ardent desire to obtain recognition for the dogmatic
system which he had excogitated for the sake of quieting his
neurotic fears. Until he got into the controversy with Tetzel, Luther
was a simple monk who had died to the world and was given to heroic
mortifications, seeking nothing else but a merciful God. But he was
unable to discover this merciful God in the monastery and, as a result
of grueling experiences, became convinced that God was not
in the papacy. While he was engaged in public controversy about indulgences,
in 1518, he was suddenly enlightened on the truth that man
is justified by faith alone.

No further word is necessary on the perverted form in which he
desired to hand down to posterity the development of his theoretical
ideas by means of the preface to his collected Latin works. The whole
incident is characteristic of the controversial spirit that moved him
to the last. The fiction thus concocted was intended to be a blow to
the Catholic Church and a vindication of his agitated life.

In 1545 he issued a pamphlet against the theologians of Louvain
University, who had taken a stand against him at the outset of his
carcer. In a certain sense, this pamphlet was a return to the beginning
of his religious innovation. The Louvain theologians had published thirty-two
articles against him in the previous year. He replied in seventy-six antitheses,
“Against the Louvain Theologasters,”
of which his Protestant biographer Köstlin says that they are “abusive
and derisive rather than convincing.”\footnote
{\textit{Opp. Lat. Var.}, IV, pp. 486 sqq.; ‘German in the Erl. ed. of his works, Vol. LXV, pp.
169 sqq. Cfr. Köstlin-Kawerau, \textit{Martin Luther}, Vol. II, p. 609.}
It had been his intention to
expand this pamphlet into a treatise entitled, “Against the Asses in
Paris and Louvain,” but death snatched the pen from his hand before
he was able to complete it.”\footnote
{G. Buchwald, \textit{Luthers letzte Streitschrift} (Leipzig, 1893). In the fragment published
by Buchwald, Luther declares that the theologians of Louvain and Paris were doomed to
hell (\textit{absque dubio peribunt}) and that the same fate would overtake the respective rulers,
unless they opposed them.}
As a definite determination of his doctrinal
position, the two last-mentioned productions, inspired by an
unbroken opposition to the ancient religion, are significant, in so far
as they categorically repeat his three principal dogmas, the articles
of the “standing and falling Church,” as he termed them; \textit{i.e.}, that
of justification and grace, that of the law, and that of sin continuously
inherent in man. To study theology without these articles he said,
as his opponents in those learned seats of harlotry did, was like trying
to teach an ass to play the lyre. Among his past publications was
his reply to a “mendacious pamphlet”\footnote{Erl. ed., Vol. XXXII, pp. 426 sqq.}
 describing his alleged frightful
death; also various hymns and prefaces.

Archbishop Albrecht of Mayence, who did not embrace Lutheranism, figures
in Luther’s letters as the “pestilence of all Germany”
down to the very end of his correspondence. Albrecht departed this
life on September 24, 1545, at peace with the Church. During the
years 1542 and 1543, as mentioned above, this ecclesiastical prince
had associated intimately with Morone and Blessed Peter Faber, whom
he kept near him. He issued commendable regulations for the protection
and prosperity of the Church during his declining years.\footnote
{Cfr. L. Cardauns, \textit{Zur Geschichte der Kirchl, Unions- und Reformbestrebungen}, 1910,
pp. 210--276.}

As late as 1542, Luther had ridiculed the cardinal-elector because
of the latter’s collection of relics and distributed among the masses a
fabricated list of these. This list mentions “a piece of the left horn
of Moses, three flames from the burning bush on Mt. Sinai, two wings
and one egg of the Holy Ghost,” etc. The pamphlet was characterized
by lawyers as a public libel (\textit{libellus famosus}) against a prince of the
empire, which was punishable at law. Luther wrote to Jonas that
even if his pamphlet were a libel in the legal acceptation of the term,
which was impossible, he nevertheless claimed the right to write thus
“against the cardinal, the pope, the devil and all their crew.” If he
lived long enough, he hoped to tread yet another measure with the
bride of Mayence, despite asses and jurists.\footnote{Grisar, \textit{Luther}, Vol. IV, p. 293.}
The reforms which the
cardinal, in his zeal for religion, endeavored to introduce, had excited
the wrath of Luther, who, after the dignitary’s death, unhesitatingly
consigned him to hell.

Luther’s rudeness enabled him to gain the victory in his contest
with the lawyers of the imperial consistory in the matter of the
validity of “clandestine marriages” contracted without parental consent.
In January, 1544, he delivered a sermon which contained brutal
attacks against the jurists who opposed his attitude in this matter.
Never before in his life, not even in his controversy with the pope,
he says, had he been so much agitated as in these contentious days.\footnote{\textit{Op. cit.}, VI, 358 sq.}

In this connection, he was also agitated by an affair of his own house.
Caspar Beier, a student, endeavored to dissolve a clandestine marriage
which he had contracted, in order to take another wife. Only with
the special aid of the elector, Luther succeeded in having all such
marriages declared invalid until the consent of the parents had been
obtained, or until a decree had been delivered by the consistory which
pronounced parental resistance as groundless. The “divine precept”
of preaching the fourth commandment of the Decalogue, which,
he said, had been entrusted to him, finally won the day and he triumphantly
conducted Beier to his new bride. No one was able to
resist the all-powerful dictator. Catherine Bora had assisted Beier, who
was a relative of hers, in winning the intervention of Luther on his
behalf. Cruciger on this occasion called her “the domestic torch”
(\textit{fax domestica}).\footnote{\textit{Ibid.}, p. 359.}
