\section{Luther as a Student at Erfurt}

Young Luther was next concerned with his reception into one of
the existing students’ inns of the university town. “Burses” or “contubernia”
were terms used to designate the homes to which all
students of the Alma Mater were obliged to belong, in virtue of
a time-honored prescription. Here they usually lived under the
supervision of one of the masters of the university. Aided by the
master, or by someone else, the students were directed in their scientific pursuits. By the payment of a small tuition fee they had board
and lodging in common. At first Luther resided in the “burse” known
as Porta Coeli, which he afterwards exchanged for that of St. George
in the parish of that name. Thus St. George, the patron saint of his
Mansfeld parish, accompanied him also on his road to knowledge.
In a verbose letter, the earliest preserved from his pen, Luther announced his residence at the Porta Coeli to his paternal friend Braun
at Eisenach. He signs “Martinus viropolitanus” (Martin of Mansfeld),
and refers to himself as the ``bothersome hair-splitter,''--an allusion,
no doubt, to his inclination to criticize.\footnote
{Letter of September 5, 1501, with excuse for delay. First appearance of text in
H. Degering, \textit{Aus Luthers Frübzeit}, in the \textit{Zentralblatt für Bibliothekswesen}, Vol. XXXIII,
(1916), p. 78. Afterwards in Luther’s \textit{Briefwechsel}, Vol. XVII, ed. by P. Flemming (Leipzig, 1920), p. 82, The letter is undoubtedly authentic.}

Erfurt at that time enjoyed an eminent reputation for learning.
Since the university of Prague, in consequence of the Hussite controversy and
the consequent emigration of the Germans, had lost its
exalted position in the academic world, Erfurt was called the New
Prague. Its faculty of law in particular became celebrated. No less
distinguished were the members of the philosophical faculty and its
preparatory courses in the quadrivium of the liberal arts. In respect of
the study of philosophy, strictly speaking, Luther came under the influence of two excellent teachers, Jodocus Trutvetter and Bartholomew Arnoldi of Usingen. Prior to his admission to the higher
branches of learning, custom constrained him to complete the study
of grammar, rhetoric, and poetics. This preparatory discipline was
indispensable for those students whose knowledge of these branches of
learning was incomplete. Martin, though he was well versed in these
branches, availed himself of this course in order to perfect his style,
as evidenced by the fluent Latinity of his later writings as well as his
early letters. He endeavored to catch the spirit of the classical authors.
Among the poets, he first familiarized himself with the writings of
an excellent neo-Latin poet, who was highly celebrated in his day,
Baptista Mantuanus, whose real name was Giovanni Spagnolo. At one
time general of the Carmelite Order, he was a pious and austere man,
who was later beatified by Leo XIII. In due course of time Luther
read Ovid’s Heroics or Love Epistles and the poems of Vergil. The
study of scholastic theology, he asserts, prevented him from reading
more of these authors. This statement, however, is not to be accepted
literally, for he found time to read the comedies of Plautus, and,
either at the same time or somewhat later, Horace, Juvenal, and Terence, from
whom he was able to quote in his later years. Jerome Emser, his subsequent
adversary, explained to him Reuchlin’s comedy of “Sergius.”

The so-called “minor logic” constituted a part of the subject matter of the
first lectures in the faculty of liberal arts. The Old
and the New Logic, interpreted by the aid of Aristotle, was the next
step in his studies. With the Stagirite as a guide, the student then
learned “natural philosophy, \textit{i. e.}, the physics of Aristotle, and read his
treatises on the soul and on spherical astronomy. Although the subjects of the
ancient trivium still pertained to the faculty of arts, the
subjects of the quadrivium were freely treated. The lectures were delivered in
the “burses” or in the auditorium of the so-called Old College. They were
accompanied by constant exercises and disputations.
These “exercises and disputations” often commenced as early as six
in the morning; they stimulated industry, sharpened the wits, and
cultivated the faculty of oral expression.

In this respect, too, the faculty of arts of the University of Erfurt
revealed itself as a medium of a general academic training, well
adapted to the requirements of the age, preparing the student for
the higher branches of learning, theology, jurisprudence, or medicine.
Aristotle, who was celebrated almost universally as the Philosopher,
was the supreme pathfinder to scholarly independence.

The bachelor’s degree, the lowest academic honor, was the proximate
aim of Luther. He obtained this degree as soon as possible, in the
autumn of 1502, by means of a severe test in the presence of five
examiners. As “baccalaureate of the liberal arts” he now wore the gown
of his office according to the constitutions of the faculty, since he was
now a member of the teaching staff. This first academic distinction
prepared the way for the master’s degree. But, before he could achieve
this honor, it was necessary for him to devote himself to protracted
study. In the meantime he was obliged to assist beginners in the
studies he had completed. This was the rule of the university. A strict
law regulated the curriculum of studies and the scientific occupation
of the teachers at the universities of that day.

Nevertheless, a certain freedom of life prevailed, and the students
found frequent opportunities for pleasantries and merriment.

Young Luther had experienced this at the very threshold of his
academic career, when he was obliged, like all newcomers, to take the
so-called “deposition” in order to become a full-fledged student in the
eyes of his associates. On this occasion the newcomer (\textit{Bean}) was compelled
to masquerade with horns and elongated ears, and swine’s teeth
were attached to the corners of his mouth. Then, by means of a plane,
he was fashioned into the proper academic form. The merry procedure
was crowned by a baptism with water or wine. Originally a certain
spiritual significance had been attached to this ceremony: it was
intended to hold up to the freshman the moral aims which he was to
pursue through the renunciation of his shortcomings. When Luther
was professor at Wittenberg in later years, he alluded to the symbolic
meaning of this “deposition.”

The life of the beneficiaries of the various “burses” was invested
with manifold privileges. Although the constitutions of some, \textit{e. g.},
the Porta Coeli, were strict and decisive in their religious and educational import, they were not always conscientiously observed. In comparison with the beginnings of the “burses,” there was possibly a lack
of zeal and vigilance in Luther’s student days. In some instances, the
frequentation of taverns was expressly permitted. The life of the
students at Erfurt must have been rather unrestrained. The relatively
large number of students (at that time an average of over 300 were
enrolled annually) leads us to expect excesses. In 1530 Luther says
that the University of Erfurt was “a place of dissolute living and an
ale-house,” in which subjects the students were interested in preference
to everything else.\footnote{\textit{Tischreden}, Weimar ed., II, Nr. 2719b.}
His added remark that “there were neither lecturers
nor preachers” at this institution, is evidently to be characterized as
a polemical exaggeration directed against a Catholic university. For
there were deserving preachers at Erfurt, though they did not preach
the new Gospel; and there were also respectable professors, as he
himself attests in his reference to Trutvetter and Usingen. Nevertheless, the acrid judgment just cited reveals a modicum of evil experiences made at Erfurt. The city council in a measure harbored immorality by allowing prostitutes to live together upon the payment
of a tax, a custom which prevailed also in other cities. Even among
the clergy of Erfurt there were open infractions of chastity. It is
questionable, however, whether Luther meant Erfurt, when in his
\textit{Table Talks} he said he knew a city where the mistresses of clergymen
were honored as Madame Deaconess, Madame Provost, Madame Cantor, etc.\footnote{Erlangen ed., LX, p. 280 (Chap. 27, Nr. 1301).}
“Gossip may have seized upon many a story and embellished it with its own peculiar frivolity,” says Otto Scheel.\footnote{Scheel, I, p. 136.}
When, at a later period, Luther speaks of his own youth he states that the
clergy were not suspected of adultery and immorality, whereas, since
then, dissoluteness had constantly increased.\footnote{\textit{Opp. Exeg.}, IX, p. 260.}
At all events, Erfurt,
during the student days of Luther, was a city whose inhabitants observed
the moral law and were imbued with a profound religious
faith--a city replete with well-attended churches and numerous busy
monasteries whose bells never ceased ringing by day and by night.

The splendid church of Our Lady excelled all other churches.
Erected on a hill, it overlooked the entire city. The church of St.
Severus, which was situated next to it, still presents a magnificent
sight. A few decades before Luther used to visit this church, St. John
Capistran preached from the lower steps of the mighty terrace which
leads up to the hill, and stories were told of numerous conversions
and miraculous cures. In the year 1502, while Luther was present, the
city and the plaza witnessed the solemn procession of the jubilee
which was arranged under the auspices of the papal legate, Raymond
Peraudi of Gurk, accompanied by the general reception of the
Sacraments and a plenary indulgence. Like the jubilee indulgence
granted on a previous occasion, in 1488, that of 1502 was a source of
spiritual renovation for citizens and students. For a number of years
the voice of an Augustinian monk, John Genser (Jenser) von Paltz,
- the scholarly promoter of papal indulgences, resounded through the
spacious edifice. He was a profoundly pious man, endowed with
a powerful gift of speech. His associate, the Augustinian John von
Dorsten, rivaled him as a preacher of great popularity and impressive
forcefulness. Indeed, the Augustinian monastery excelled all others
in learning and practical activity. By means of their confraternities,
monasteries, and churches, these monks educated the faithful to a
more active religious life. Perhaps young Luther, the son of a miner,
may have experienced a sort of predilection for the confraternity of
St. Ann, the patroness of miners, which was directed by the Augustinians.
We shall see that he supplicated St. Ann, when, during a
storm, he made his vow to enter a monastery.

But, until that time, he devoted two years to preparing himself for
the master’s degree. During this time he studied the logic of Aristotle,
certain questions of natural philosophy, mathematics in general, and
several minor branches of the old quadrivium; and, finally, moral
philosophy, politics, and metaphysics. All these subjects were learned
in the light of Aristotle and with the assistance of preceptors who
explained the writings of the Stagirite to the student.\footnote{
Cf, Scheel, I, pp. 157 sqq.; 170 sqq., where the branches of study are set forth in
detail for the period before and after the baccalaureate.
}

In its application of the Aristotelian philosophy, and in the study
of philosophy in general, the University of Erfurt claimed to pursue
a purified course, which at that time was designated as “modern.”
Trutvetter and Usingen, the principal teachers of Luther, were outspoken
representatives of the \textit{via moderna}, as were also their fellow
instructors. At other universities there were advocates of the new as
well as adherents of the old method. The old system favored the so-called
philosophy of realism, whereas the new system betrayed
a nominalistic tendency. The latter originally derived its name
from the fact that it held that all universal concepts were mere
names (nomina.) The better class of Nominalists were strictly Catholic
and avoided the dangers ordinarily attached to Nominalism. William
of Ockham, the daring and skeptical “\textit{Doctor invincibilis}” (died
at Munich in 1347), was the accepted leader of the Nominalist school.
But there were men like the learned and influential Gabriel Biel, as
well as the Erfurt professors mentioned above, who knew how to
avoid the reefs of Ockhamism.

Nominalism was a stage in the decline of Scholasticism. The heyday
of Scholasticism was past; logical investigations and useless hairsplitting
were in vogue. The barren subtleties of Terminism, so-called,
were an evil, though a very large and living stream of learning was
still flowing. The Scholastics of the thirteenth century, especially
Thomas of Aquin, who followed a moderate realism, had offered a
better foundation and, according to content and form, a better
school. St. Thomas had brought the Aristotelian philosophy into a
better organic union with Catholic truth. But his writings were not
properly utilized. We shall see how certain deficiencies of his Erfurt
training avenged themselves upon Luther. In many respects he
claimed to be a consistent Ockhamite; in some points he went beyond
him, in others he abandoned him. It is important to bear in
mind, however, that he imbibed no reformatory ideas from his
teachers at Erfurt, but learned to combat the Nominalism of Ockham,
wherever it deviated from true philosophy to the detriment of
dogma.

The same attitude was maintained toward the authority of Aristotle.
Although he reigned supreme in natural philosophy, the Catholic
school contradicted his theses as far as they were in conflict with
the Christian faith. Divine revelation as a source of knowledge was
rightly regarded as inviolable against the objections of pagan philosophy.

As yet Luther had not occupied himself with the study of the
Sacred Scriptures, with which he was acquainted only through the
widely spread excerpts of the \textit{postillae} or \textit{plenaria}. These books, which
contained principally the pericopes of the Sunday gospels and epistles, as well as the very sententious so-called history-bibles, all composed in German, were in the hands of the clergy and laity.

Luther relates in his \textit{Table Talks} how, while yet a student at Erfurt,
he accidentally happened upon a complete Bible, and how the
Old Testament story of Anna, the mother of Samuel, made him
realize what inspiring narratives were contained in the historical
books of the Old Testament.\footnote{\textit{Tischreden}, Weimar ed., I, Nr. xxs;. ITI, Nr. 3767; 5, Nr. 5346. I see no reason to doubt this statement.}
The book with which he became acquainted on that occasion, was a volume of the university library.
He was unable to read the whole Bible at the time, but he felt a lively
desire to make himself more familiar with the Sacred Scriptures.
But the great expense of a complete Bible in those days, when the
art of printing was in its infancy, permitted him to buy only a
\textit{postilla} for his own use. He was amazed to discover many passages
which did not appear in the liturgy of the Church. According to
his own testimony, Luther was twenty years of age when he made
his first acquaintance with the complete Bible. This late discovery,
however, is not surprising. How many persons twenty years of age
may there have been, even among the highly educated, who never
had a complete Bible in their hands, and how many such individuals
might not be found to-day? There was no reason for Protestant
writers to adorn the story of this event with romantic embellishments
and to assert that Luther then and there became the great
discoverer of the Bible, which, in the days that preceded his birth,
lay hidden under a bench and fastened to a chain. At most it was
fastened to a chain in this sense that it was secured for the sake
of safety and general utility. Even at the present day, all manuscripts
in the Laurentian Library of Florence are secured by a chain.

Luther also informs us that once, when he and a companion
journeyed to Mansfeld, he injured his leg with the sword, which he,
‘as a bachelor, was wont to carry as a badge of his academic
status. The accident happened half a mile from Erfurt, and blood
flowed so profusely from the wound that he was in danger of death.
He lay helpless on the ground, pressing the swelling wound with his
hand until his companion had summoned a surgeon from the city,
who bandaged the wound. At death’s door, he tells us, he imploded
the aid of the Mother of God. “Then would I have died, placing
full confidence in the help of Mary.” In his room at Erfurt, the following
night, the wound broke open. Feeling faint, he once more
invoked the Bl. Virgin Mary. While lying convalescent on his bed,
he sought distraction by learning to play the lute, an accomplishment which he ever afterwards cherished.\footnote{\textit{Tischreden}, Weimar ed., V, Nr. 6428. Cf. I, Nr. 119.}

The playing of the lute proved a great advantage to him. It helped
him to dispel a natural tendency to melancholia, and also enabled
him to amuse and entertain his fellow-students. Crotus Rubeanus,
who in later years became a celebrated free-thinker, was a room-mate
of Luther’s. In a letter to the latter, dated April 28, 1520, Crotus
writes: “You were at one time the musician, you were the learned
philosopher in my contubernium.”\footnote{\textit{Briefwechsel}, II, p. 391.}
This statement reveals not only
Luther’s musical accomplishment, but also his zeal in the pursuit
of his studies. His success in philosophy was especially esteemed by
his fellow-students. Mathesius ascribed to him “a great earnestness
and special diligence” in his studies. “Though naturally an alert and
jovial young fellow,” he says, “yet he commenced his studies every
morning with prayer and a visit to the church.” Mathesius writes
this in his sermons on Luther in order to represent the piety of his
young companion as exemplary.\footnote{Mathesius, p. 19. The laudatory expression of Crotus also serves a partisan purpose.}

Luther at this time had deeply-rooted Christian convictions as
well as loyalty to the clergy and the head of the Church. The anti-ecclesiastical
ideas of Ockham were completely alien to him. According
to his subsequent declarations, he dismissed, by means of
prayer and acts of faith, the objections to the teaching of the Church
which he occasionally met with in Ockham’s works. He heard bitter
complaints about the condition of the Church, but they did not
produce any impression upon him at the time. An old man from
Meiningen, with whose son, who was also a student at Erfurt, Luther
was well acquainted, visited him during his illness and spoke to him
about a great change which was bound to come, since the present
state of affairs could not continue. When Luther complained of his
indisposition, the man consoled him, saying: “Do not feel aggrieved,
you will be a great man some day.” Luther mentions this conversation
in his \textit{Table Talks} for the year 1532, and comments upon it as follows: “There I heard a prophet.” “I mean,” he adds, “that
the change of which the old man spoke, has taken place.”\footnote{
    \textit{Tischreden}, Weimar ed., I, Nr. 223; II, Nr. 1368, 2520.
    The visitor designated in these passages probably was one and the same person.}
Such fanciful prophecies as those of the man of Meiningen and of Hilten
of Eisenach, captivated him, not in his student days, but in later life.

During 1503 and 1504 his mind was absorbed by his efforts to
qualify himself for the attainment of the degree of master of arts.
At the beginning of 1505, he took the examination, which was
held around the feast of Epiphany. Among the seventeen students
who were examined, Luther took second honors. The examination
was followed by the solemn presentation of the insignia of the
master’s office, a brown biretta and a ring. Luther informs us that
on such occasions the new masters were publicly honored amid great
pomp. At eventide, accompanied by torch-bearers, they rode horseback through the city amid flourish of trumpets, the noisy acclamation of the populace, and the pleasures of Bacchus. They were
obliged to give a supper to the faculty, for which, characteristically
enough, the minimum and not the maximum number of participants
was designated.

In compliance with his father’s wishes, Luther, now a master of
arts, entered the faculty of law. He commenced to study in preparation
for the secular career which his parents had chosen for him.
His father was even planning an advantageous marriage for him.
Formerly he addressed his son, while yet a bachelor, with the familiar
personal pronoun ``du'' (you); but henceforth he addressed him with
the honorary title ``Ihr'' (you). With a great outlay he purchased
a \textit{Corpus Juris} for the young law student.
The faculty of jurisprudence of the University of Erfurt in
those days had several professors of renown. Henning Gode in
particular was an excellent representative of Canon Law. We do
not know what lectures Luther attended during the first few
months, that is, the period which preceded his sudden entrance into
the monastery. He appears not to have been satisfied. It is fair to
assume that he had no inclination for the study of law. His subsequent
strong condemnation of lawyers and their science justify the
inference of an early opposition. Then, too, his mind was so much
taken up with spiritual matters that he regarded the lectures on
law and the private studies which it entailed as too arid, of small
profit for heart and head. He did not suffer from a lack of emotional
capacity, as the near future soon revealed, but rather possessed a
superabundance of emotion, which was destined to become dangerous
for him.

Did he, perhaps, permit his philosophy of life to be influenced
at that time by the humanistic tendencies, whose history in Germany
is so intimately connected with that of the University of
Erfurt? Some writers have assumed that a very strong influence
was exerted upon the impressionable youth by the neo-humanistic
movement which commenced at that time, and which was fostered
to excess by certain circles in Erfurt. It has been alleged that the
seeds of the future reformer were sown in Luther’s soul by the poets
and the untrammeled critics who had been alienated from the
Church. This view needs rectification. To a certain extent, young
Luther became surcharged with the ideas of his age, which sought
to imitate the classical form of the ancients without becoming enthusiastic
over their pagan philosophy. The school of the trivium,
especially under Trebonius at Eisenach, formally directed his mind
into that channel. His first Latin essays manifest an affected, humanistically
inspired style. Usingen and Trutvetter, his Erfurt
teachers of the “liberal arts,” although Scholastics and Nominalistic
philosophers, were neither unfamiliar with nor hostile to humanism.
They cultivated the sound old German humanism which, corresponding
with that of the Middle Ages, was favored by the statutes
of the University. This was the kind of humanism of which they
endeavored to learn the forms and which fructified their ideas. It
was devoid of that corrosive and repulsive element which characterized
the new humanism about to commence at Erfurt. The
movement with which Luther had affiliated himself, was marked
by a reverential attitude towards the Church. As yet he did not
study Greek, but he read the Latin classics with such enthusiasm that
it redounded to his great advantage in acquiring versatility of form
in Latin as well as in his German mother-tongue. When he entered
the monastery, he took with him the comedies of Plautus and the
poems of Vergil and disposed of all his other books to a bookseller.
It was only in proportion as his devotion to the Bible increased
that his predilection for humanism declined.

The neo-humanistic school did not assume shape at Erfurt until
after the doors of the monastery had closed behind Luther.\footnote{Proof offered by Scheel I, pp. 223 sqq.}
He was not in contact with its spirit, least of all with the anti-ecclesiastical
endeavors of the canon Mutianus Rufus (Conrad) of Gotha,
the chief of the rising neo-humanism, who, since 1507, gradually
formed more intimate connections with the Erfurt humanists,
Eobanus Hessus, Crotus Rubeanus, John Lang, Peter and Henry
Eberbach, and others. From 1515 to 1520 the neo-humanists prevailed
mightily at the University. In collaboration with the talented
and frivolous Crotus Rubeanus, they launched, in 1515, a hostile
publication against the Scholastics and the monks, which appeared
under the title of \textit{Epistolae Obscurorum Virorum}. Luther’s contact
with Crotus, of which the latter afterwards boasted, was restricted
to their association in the burse of St. George. Both were members of
that small band of young friends (\textit{consortium}) who celebrated
Luther’s departure from the University and accompanied him to the
gates of the monastery.

Another humanist, George Spalatin, a native of Spalt, afterwards
Luther’s friend and most effective helper in the religious innovations,
was a student of Erfurt from 1498 to 1502, and returned thither
for a brief period in 1505 as a private tutor, when through the
mediation of Mutianus he obtained a position as teacher at the
monastery of Georgental. We have no authority for saying that this
man exerted a strong influence upon Luther during the days
he spent at Erfurt while yet a layman. But a strange tradition among
Catholics in Spalt, which originated there several decades later, attached
itself to the name of Spalatin and other natives of Spalt who
associated with Luther in Erfurt.\footnote{Grisar, \textit{Luther}, III, pp. 284 sqq.}
It is to the effect that Luther,
while a young monk, became enamored of the daughter of a widow
whose acquaintance he made in Spalatin’s home. This affair, so it was
alleged, was the beginning of his renunciation of the monastic life
and of the Catholic faith. It is not necessary to refute this fable.
The fall of the Augustinian friar is not to be attributed to a desire
for matrimony, but to quite other causes. The story of Luther’s love
affair in Erfurt was probably circulated by George Ferber, a native
of Spalt.\footnote{\textit{Ibid.}, p. 287.}

In addition to the men mentioned above, Luther in his academic
years was associated with Caspar Schalbe, who was happy to procure
Luther’s intercession with the Elector of Saxony when accused of
a crime against morality before that sovereign in later years.\footnote{\textit{Ibid.}, 1, p. 7.}

Certain declarations of his contemporaries are not favorable to
Luther’s student life. We refer to Jerome Emser of Dresden, who
came in contact with him at Erfurt, and to Jerome Dungersheim, a
professor at the university of Leipsic. Emser, who was secretary of the
Duke of Saxony, was engaged in open warfare with Luther in 1520
because of the latter’s writings against the Catholic Church. At that
time he wrote a letter to him containing personal reproaches which
is no longer preserved. Thereupon Luther upbraided him because
of his past life. Emser, indeed, was not innocent, according to his own
admission. In a counter-reply he says, among other things: “What
need was there for you to reproach me publicly with past mistakes,
most of which are inventions, because of a letter which tells the truth
about you? What do you, think is known to me of great derelictions
(\textit{flagitia}) on your part?” He did not desire to mention these derelictions
because it was not his purpose to repay evil for evil. “That
you also fell,” he continues, “I believe to be attributable to the same
cause which effected my fall, namely, the cessation in our day of that
public morality which permits young men to live as they please,
unpunished, and to take all sorts of liberties.”\footnote{\textit{Ibid.}, p. 27.}
Luther never replied to these reproaches.

The other witness is Dungersheim of Leipsic. He was a scholarly
man and zealous in behalf of the Church. In a printed pamphlet
directed against Luther he appeals directly to communications which,
as he avers, originated with one of the companions who escorted
him to the portals of the monastery and charges him with gross shortcomings
in his student years. Besides these reproaches, he mentions,
in another polemical writing, “bad habits” which were attributed
to Luther and says it must have been due to the latter and the neglect
of prayer that Luther now maintained the impossibility for a monk
to observe his vow of chastity.\footnote{\textit{Ibid.}, pp. 26 sq. The two documents of Dungersheim of Ochsenfurt, quoted in this passage, were combined by him with others in 1531 in a volume which bears the title: \textit{Aliqua Opuscula Magistri Hieronymi Dungersheim \dots contra Lutherum edita}.}
It must be remembered, however,
that at that time a strong sentiment prevailed against Luther at
Leipsic and many an unfavorable rumor was launched against him
which had no foundation in fact, as, for instance, his alleged petition
to the Holy See for permission to marry on the occasion of his
pilgrimage to Rome, in 1510. But Dungersheim reechoes the report of
a companion of Luther. It is possible that both witnesses relied upon
the communications of a comrade of Luther. As in the previous instance,
so now Luther preferred to ignore the allegations made against

Luther afterwards denied that he suffered strong temptations
against chastity while he was a monk.\footnote{Grisar, \textit{Luther} (original German ed.), 11, p. 1003.}
In his bizarre manner he
repeatedly asserts that the devil reproached him not with moral
transgressions, but with his monastic virtues and the celebration of
Mass: these were the sins of his youth. On one occasion, however,
at the conclusion of his solemn profession of the Last Supper, he
concedes that he “spent his youth in a damnable manner and lost
it”; but he adds that the greatest sin he committed was the celebration
of the Mass.\footnote{Erlangen ed., XXX, p. 371; Grisar, I, p. 27.}
At another period of his later life, he once
more alludes to the sins of his youth, namely, the Mass “and this
or that youthful act”; that he often enjoyed internal “rest and good
days” until he was shaken with fright “of despair and the fear of
God’s wrath.”\footnote{\textit{Tischreden}, Weimar ed., I, Nr. 141: \textit{``quod sacrificavi in missa, quod hoc aut illud feci adolescens.''}}
Spiritually he often vacillated between extremes.
All depended, as the last quotation shows, on his ability to form
a strong conception of God’s mercy. Then he imagined--we speak of
his later years--that “it is only a temptation of Satan, the greatest
temptation, in fact, when he says: God hates sinners, and you are
a sinner.” “It is simply false that God hates sinners.” “If He hated
sinners, He would not have sacrificed His Son for them.” “We have
the forgiveness of sins,” etc. “However, many whom we do not
know must thus struggle in the world.”\footnote{\textit{Ibid.}}
These violent and sudden internal transitions are a sample of his temperament in later years.

It is possible that his soul was afflicted with a similar tension, albeit
in a milder degree, already in his youth.

While a student at Erfurt, Luther maintained friendly and stimulating
relations with his benefactors and friends at Eisenach.
Trebonius, his former teacher, inquires sympathetically of Luther’s
friend, the student Louis Han at Erfurt, concerning the welfare
and progress of “my Martin.” He sends him a message encouraging
him to strive after “wisdom and discipline.” The letter is one
of a recently published (1916) collection pertaining to the Eisenach
circle, with whom Trebonius and Luther’s benefactor at that place,
John Braun, were associated.\footnote{Cf. note 44, \textit{supra}; Degering, p. 90, letter of February 5, 1505.}

What is still more important, some of the letters of this collection
seem to be written by Luther himself. The Latin inscription of the
old collector over the Latin letters, which had been copied in many
instances without the address or name of the sender, reads very
definitely: “Twenty-four letters of Luther and his teachers and
friends of Eisenach and Erfurt.”\footnote{Degering, p. 71.}
The letters were collected shortly
after 1507, supposedly at Eisenach, for the purpose of furnishing
models of style in epistolary correspondence for the use of schools.
Luther is the author of the verbose letter of the 27th or 28th of
April, 1507, in which he invites Trebonius to attend his first Mass,
and jocosely subscribes himself as “Martinus Lutherus Augustiniaster,”
an attempted witticism meaning “an inferior Augustinian.”\footnote{\textit{Briefwechsel}, XVII, p. 84; Nr. 16 of Degering’s collection.}
Luther is also to be regarded as the author of the letter
to Braun, dated September 5, 1501, in which he speaks of the commencement
of his studies at Erfurt. It is probable that a third letter,
dated February 23, 1503, is likewise the product of his pen.\footnote{Cf. text and note in Degering, p. 85.}
The editor of the collection refers to the spirit and mood of this letter
as a proof of Luther’s authorship. The unsigned document was addressed
to a spiritual “benefactor and dearest friend.” Others have
rejected this assumption on the ground that the contents do not
quite agree with what we know of Luther the student (or rather with
what some pretend to know!). An eminent Protestant theologian,
however, has taken exception to this criticism by remarking that
the supposition that Luther must have been entirely free from the
mistakes of his fellow-students has no “historical support.”\footnote
{Hermann Jordan in \textit{Theologie der Gegenwart}, 1917, pp. 158 sq.: “I regard Degering’s
assumption as not absolutely impossible; but it is as much a pure hypothesis lacking historical support as the assumption that Luther as a student must have been entirely free
from the intemperance of his comrades.”}
The recipient of the letter had praised the writer, who declines the compliment
in stilted humanistic phraseology and says that he “is surfeited
with human weakness, dereliction, and negligence in every
respect, interiorly and exteriorly,” “that habit rules like a second
nature” and “the times are evil, men are worse, their works replete
with wickedness. Prevented by gluttony and drunkenness, I have
hitherto [since my last letter?] neither written nor read anything
good; for, being placed in the midst of men, I have lived with men.
But as soon as I had torn myself away, I at once seized my pen,
in order to reply to you, most cherished father.”\footnote
{“\textit{Crapulis et ebrietatibus impeditus hucusque minime quid boni scripserim aut legerim,
quis constitutus cum hominibus conversabar cum hominibus.}”}
In the course of the letter, which does not offer anything very remarkable, the writer
requests the loan of a work by Nicholas of Lyra, which he had seen
in the library of his correspondent. Reference is probably made to
one of the religious tracts of Lyra, who was famous as a Biblical
commentator. The style of the letter, both in its humanistic composition
and in various peculiarities, corresponds entirely with the two
other letters of the collection ascribed to Luther no less than with his
oldest previously known letters. The letter, moreover, harmonizes
with the above-mentioned accusations against the young student. It
must be noted, however, that a certain exaggeration attaching to
the form of his admissions is attributable to the style and character
of the writer. In a letter written to Staupitz in 1519, and in
another written to Melanchthon in 1521, Luther also complains of his
gluttony, though the expression is not to be taken in its literal sense.\footnote{Grisar, \textit{Luther}, 11, p. 87.}
The vigorous words, in which the young student emphasizes
the decline of morality,are quite as energetic as those contained in
his first lectures, sermons and letters at the commencement of his
public career.

Taking all things into consideration, it cannot be denied that
young Luther was very probably the author of the letter with which
we are concerned. In a publication on Luther which appeared in
1917, at the time of the Luther jubilee, a certain Lutheran churchman
writes: “Must we not believe that the man who was seized with fear
of sin in the monastery, who was personally acquainted with sin,
possessed a weak heart of profound depth and obscure corners, at
the sight of which he, at times, closed his eyes with trembling, but
into which he had descended with wide-open and burning eyes when
his evil hour was upon him? Why, else, should he have experienced
such biting qualms of conscience and such indescribable fear?”\footnote
{G. Tolzien, Landessuperintendent, \textit{Martin Luther} (Schwerin, 1917), p. 4. Tolzien
seems to be ignorant of the testimonies of Emser and Dungersheim. It is remarkable how
consistently they arc passed over by all, even scholarly, Protestant biographers of Luther.}

For the rest Luther already in his youth suffered from a natural inclination
towards melancholia. Nature, the severe discipline of his parental
home, and the first school which he attended, produced a certain
depressed atmosphere in his soul. This never quite left him, although
it was frequently interrupted by intermittent periods of great mental
uplift and inspiration. Of his inclination to religious melancholia,
not to say despair, he wrote in 1528, in a letter to one who suffered
from the same malady, that he “was not unacquainted with it since
his youth.”\footnote{\textit{Briefwechsel}, VI, p. 173.}
He regarded melancholia and despondency as the
inseparable portion of man. “Melancholia is born in us, the devil
fosters the \textit{spiritus tristitiae}.”\footnote{Weimar ed., VIII, p. 574.}
“From the days of my boyhood (\textit{a pueritia mea}),”
we read in the \textit{Table Talks}, “Satan foresaw in me something
of that which he must now suffer [in virtue of my Gospel].
Therefore he sought with incredible frenzy to injure and obstruct
me, so much so that I often asked myself in amazement: `Am I the
only one of all mortals whom he pursues?' ''\footnote{\textit{Tischreden}, Weimar ed., II, Nr. 12795 cf. Nr. 2342ab.}

There is another indication that a very singular temperament
resided in the highly gifted soul of Luther. This must be taken into
consideration in judging the catastrophe which drove him precipitately into a monastery.

“Despairing of myself” (\textit{desperans de me ipso}), he tells us, he
entered the monastery.\footnote
{Text from Rörers \textit{Handschriften}, published by E. Kroker in the \textit{Archiv für Reformationsgeschichte}, Vol. V, (1908), p. 346.}

“Internal anxiety,” says a Protestant student of Luther, “or despair
of himself led him into the monastery,”\footnote{Scheel, \textit{Martin Luther}, II, 2nd ed., p. 7.}
at the same time emphasizing the “godliness” in which Luther learned to exercise himself
already as a layman. However, godliness was not the sole driving
power which impelled young Luther. In view of his early religious
training it is hardly possible to dispute that the most pious impulses
often influenced him. Yet, despite his sentiment of despair, he was
subject to other factors, namely, in addition to his natural temperament,
the well-founded premonition that the life of a layman was
accompanied by moral dangers with which he felt himself unable
to cope. In his subsequent attacks upon the monastic vow of chastity,
he applies a proverb which says: “Monks and clerics are mostly the
product of despair.” In the case of Luther, “despondency” connotes
chiefly a depressing sense of moral incompetence and the experienced inability to preserve chastity.\footnote
{Erlangen ed., XXI, p. 359. Cf. \textit{ibid.}, X, p. 400, and XIII, p. 130, where he adduces
the “proverb”: “Despair makes the monk,” applying it to the necessity of gaining a livelihood.}

Another Protestant biographer speaks of the “psychical abnormality”
of Luther in his youth, though, of course, his “especially tender
and impressionable conscience” is again emphasized.\footnote{Böhmer, p. 50.}
A third author
exaggerates when he says that the young man’s “nervous system was
unsettled from early youth,” and attributes this trouble to his excessively
severe training.\footnote{A, Hausrath, \textit{Luthers Leben}, Vol. 1, p. 4.}
These statements indicate that we are confronted with a complicated phenomenon.
At all events, the existence of a somewhat disordered constitution, grounded in his very
soul, must be assumed. It was united to the depressing idea of guilt,
to which the ambitious youth appears often to have given free rein.
Both together, the disordered condition of his nerves and the strong
sense of guilt, may possibly explain the fear which, as we shall see,
pursued him in a most terrifying manner in the monastery, and
which reappeared ever and anon throughout his life. During his
monastic years, according to his own statements, he ever sought the
aid of a merciful God in his struggles amid great despair, without
being able to discover Him in the Catholic faith, and most Protestant
biographers hold that he was taken up with this quest even in
his student days, until he finally sailed into the haven of rest upon
the discovery of the new Gospel. This theory, as we shall show
later, is not correct. There is no proof whatever that Luther, prior to
the time when he entered the monastery, had a sense of insufficiency
of the Catholic faith and consequently struggled to find a merciful God.
Let us quote here the words of another Protestant biographer,
who, in this instance, judges correctly. Such a construction,
he says, “is neither supported by the facts known to us, nor has it
any probability.” “He [Luther] does not doubt the sufficiency of ways
and means. All things as yet co-exist ‘naively’ or succeed one another
in the normal Catholic rotation.”\footnote
{Scheel, I, p. 26. Scheel properly excludes from the juvenile period of Luther a struggle
for a merciful God in the sense of most biographers of Luther and as a motive for his
entrance into the monastery (pp. 242 sqq.). According to him, Luther’s resolution to become
a monk was not “the natural result of a long struggle for God’s mercy” (p. 243).
The hastily composed Latin and German transcript of a sermon delivered by Luther on
February 1, 1534, furnished by Rörer, supplies us with the following reading: “\textit{Ego fui XV
annis monachus et tamem munquam potui baptismo me comsolari. Ach quando vis semel
fromm werden? donec fierem monachus. Non edebam, non vestiebar, friere, papa, antichristus
treib mich da bin, qui abstulit baptismum}” (Weimar ed., XXXVII, p. 661). In this entire
passage Luther speaks of his monastic period; hence the words “\textit{donec fierem monachus}”
are to be attributed to a misconception of the copyist, as Scheel observes in his notes
(Weimar ed., XXXVII, p. 519). While it is not permissible to translate these words by: “as
long as I was a monk,” especially on account of the \textit{fierem}, it is probable nevertheless that
Luther intended to say something similar. The German text of the whole passage (Weimar
ed., XXXVII, p. 661), which appears in the form of a sermon originally reported by
Cruciger, concludes with the following words: “and by such thoughts I have been driven
to embrace monasticism.” The text is a corruption without foundation in fact. Luther apparently
had made his polemical assertion, that he arrived at his new doctrine in consequence
of his struggle for a merciful God, so popular that his followers unconsciously extended
his struggles and doubts to the time prior to his entrance into the monastery, though
he himself does not supply us with one word to that effect.}

The same author rightly rejects the favorite assumption that the
generality of mankind at that time was agitated by a knowledge
or feeling of the insufficiency of the Catholic way of life and
yearned for better things--which yearning was bound to captivate the frank soul of young Luther. “It is an erroneous
impression to hold that the human race, prior to the rise of Luther,
was moved by a passionate spiritual commotion and an impatient
quest after a more profound religious philosophy of life.”\footnote{Holl, \textit{Luther}, Tübingen, 1923, p. 13.}
Many authors endeavored to discover in that age such a highly significant
preparation, a great prelude, as it were, to the Reformation. Leading authors have discussed “the lively consciousness of guilt which
inspired the people of that age, for the relief of which many thousands made great exertions,” but in vain, until Luther appeared as the liberator and prophet. Up to his advent “the mighty desire
for certitude in matters of salvation exhausted itself in a gradation
and excess of ecclesiastical performances without obtaining
any repose.”\footnote{F. von Bezold, \textit{Gesch. d. deutschen Reformation}, Berlin, 1890, pp. 248, 242.}
This is not the place to present a characterization
of the true ecclesiastical state of affairs. We are at a point in the
development of Luther, when he had had no opportunity to survey
the surrounding world. The student is still traveling his own road,
immersed in his studies and occupied with the affairs of his soul.

While he traveled this road, did the thought of embracing the
monastic state occasionally arise in his mind? Although he himself
is reticent about this matter, we may well believe that this thought
did engage his attention. For it is not likely that when his sudden
resolve was forced from him, the idea of the monastic life entered
his mind all at once. It is to be assumed psychologically that the
question of the monastic state had agitated him for some time,
even though he had not made a decision.\footnote{Thus H. Preuss in \textit{Theol. Literaturblatt} (Leipsic, 1916), p. 94.}
May not the wish to secure the salvation of his soul by becoming a monk have dawned on
him while he was in a melancholy frame of mind, overwhelmed
with a sense of weakness, vacillation and fear of going completely
astray on account of his youthful frivolity? His assertion that he
became a monk out of despair, points to such a strong impulse. His
early religious education, then the active religious life at Erfurt,
the example and activity of the religious Orders at the latter
place, \textit{e.g.}, the Carthusians, whose very strict life he had observed;
furthermore, his intercourse with pious preceptors, such as Usingen,
who eventually became an Augustinian; finally, reminiscences of
Magdeburg, such as that of the princely discalced monk and mendicant: all these things might have induced him to contemplate the
monastic life.

An event, however, was destined to happen which impelled him to
form a precipitous and premature resolution which was fraught
with momentous consequences.
