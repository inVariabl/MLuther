\section{Eisleben and Mansfeld}

In the night of November 10-11, 1483, there was born to the miner
John Luther at Eisleben a son who was destined to achieve distinction.
The house in which Martin Luther was born was situated on the socalled Lange Strasse, in the southwestern part of the little city, which
was encircled by walls and towers. Even at the present day the somewhat deteriorated building is partially preserved, the upper story only
having been reconstructed because of a fire in 1689.

The house in which Martin first saw the light of day lay within the
limits of St. Peter’s parish. On the day succeeding his birth, the infant
was brought to the parish church. As it was the feast of St. Martin of
Tours, a bishop and monastic founder, the name Martin was given
to the child in baptism; and thus the saint who was commemorated on that day became his patron. In accordance with this custom,
the name of Martin was presumably likewise selected for Bucer, who
will be frequently mentioned as a subsequent helper of Luther. The
Sacrament which ushered the son of John Luther into the halls of Catholic Christianity was administered by the pastor, Bartholomew Rennebecher.
The sacred rites were performed in the Gothic tower chapel
which is still preserved in its pristine condition. Upon entering the
chapel, one is vividly carried back to that hour. Profound thoughts are
aroused by the impression of the hallowed semi-darkness of the venerable walls and the sight of the precious side-altar, ornamented with
ancient statues of saints that look down solemnly upon the worshippers.
The local memorials of men frequently mentioned in history, which are
so sedulously sought to-day, possess an undoubted historic value. The
traces related to Luther’s person, were preserved with an uncommon
love by his friends and adherents. Tradition, however, has interwoven
errors with the truth. There is no historical warrant either for the baptismal font, which is pointed out in the chapel as Luther’s, or for the older superimposed part, despite its inscription which proclaims that
Martin was baptized here.\footnote{Beschreibende Darstellung der alteren Bau- und Kunstdenkmaler der Provinz Sachsen, Heft XVIII: Mansfelder Gebirgskreis, von H. Grossler und A. Brinkmann (Halle, 1893), p. 145.}

As one steps out of the chapel, which is surmounted by a tower, into
the church of St. Peter and contemplates its stately exterior, the Gothic
forms still recall the period of Luther’s youth. The church had been
gradually completed by 1513; but the tower, without its present
crown and the baptistery upon which its heavy pile is erected, dates
from 1474.

The friendly little city of Eisleben participated in the general zeal
for building churches which at that time proclaimed the religious devotion and Christian charity of the faithful in many parts of Germany. The large church which was dedicated to St. Andrew and in which Luther delivered his last sermons,
was rebuilt in the fifteenth century in conformity with an older plan and adorned with two tall,
pointed spires. In the year 1462, the church of St. Nicholas was completed. In that part of the town called the “new city,” built by Count
Albrecht von Mansfeld in 1511, a church was erected in honor of St.
Ann, the patroness of the mining industry, for the benefit of the
resident miners. The mining of copper ore, extracted from the near-by
hills, constituted even then, as it had for a long time previously thereto,
the chief industry of the city. The industry was very much promoted
by the counts of Mansfeld, who ruled Eisleben as the capital of their
county.

Martin’s father at the time of his son’s birth resided only temporarily in the beautifully situated and ambitious city.
He had settled with his family in this city some time before, in the hope of acquiring a better income.
Previously he had lived in Mohra, a village near Eisenach, whence the Luthers originated.
There he lived, a descendant of a race of peasants, engaged in husbandry.
In his Table Talks, Martin Luther says: “My great-grandfather, my grandfather,
and my father were real peasants.”\footnote{\textit{Tischreden, Weimar} ed., V, Nr. 5574.}
In another passage, after stating:
“I am the son of a peasant,” he adds, to indicate that he is not ashamed
of his descent: “Peasants have become kings and emperors.”\footnote{\textit{Ibid.,} Nr. 6250: \textit{“Ego sum rustici filius,”} etc.}
Indeed, he ever remained conscious of the fact that something of the sturdiness of the Saxon country-folk inhered in him.

The hardy population of Mohra counted among its number a
younger brother of John Luther, who also bore the name of John,
which had been given to him in honor of his god-father, regardless of
its being a repetition of the same name in the family. The first John
was referred to as the elder, or big John; his younger brother as the
younger, or little John. In the extant fragmentary court records of
Mansfeld, whither he had gone, the name of the younger brother occurs repeatedly.
According to these documents little John achieved
notoriety on account of certain brutal acts for which he was sentenced.
During one of his frequent visits to the tavern, he struck his neighbor’s
hand with a knife or inflicted bleeding gashes on his head. Once in the
course of a brawl he beat his opponent on the head with a tankard until
the blood flowed.\footnote{
\textit{Zeitschrift des Harzvereins für Geschichte und Altertumskunde}, XXXIX (1906), art.
by W. Möllenberg, pp. 169 sqq. In adverting to the acts of violence the name “Hans
Luder” is frequently mentioned without distinction of person; so in Nrs. 7 and 8, where
the aggressor inflicts bloody blows upon two persons with a knife.
}
Perhaps it was this same irascible uncle of whom
James, an elder brother of Martin, states in the Table Talks of 1540,
that he trapped two Franciscan monks in a hole that was infested by
wolves.\footnote{Tischreden, Weimar ed., IV, Nr. 4891.}

It is related of Luther’s father that he was a serious-minded, severe,
and industrious man. At times, however, he drank to excess, so the
Table Talks assure us; and then, contrary to his habitual nature, he
was in high spirits and jovial. In this respect he differed, as Luther tells
us, from a nephew named Polner, who became vicious and dangerous
and compromised the Gospel in his frequent lapses from sobriety.\footnote{
    \textit{Ibid.}, Nr. 5050 of the year 1540: \textit{“Reliqui ebrü sunt laeti et suaves, ut pater meus,”} etc.}
The father, according to Luther’s expression, was endowed with
“a robust, solid body.” As he practised hard manual labor, so too,
Margaret his wife, was habituated to unremitting toil. From the scanty
notices which history has preserved, she appears to have been an industrious housewife. Besides Martin, Margarethad a number of other children who caused her sufficient cares and worry. Luther recalls that she
was wont to carry home on her back the wood that was needed for the
household. Margaret Luther, née Ziegler, was a native of Franconia.
There was no alien blood in her family, and certain early opponents
of Luther were unjustified in claiming to have discovered that
Luther’s ancestors originated in Slavic Bohemia, in order to connect
him with the country of the heretic Hus. Luther is a good Old-German name.
It is identical with Lothar (Luothar) and signifies: the
pure one. This fact is stressed occasionally by Luther in his Table
Talks. However, in the first years of his public appearance he spelled
his name Ludher or Luder. The form Luder or Lueder likewise appears in the family record. He complained that his enemies spelled the
name Lotter (Lotterbube), which signifies a vagrant scamp. In the
beginning of his revolt he used, for a time, the more euphonious
Greek term Eleutheros or Eleutherius (freeman, liberator). Later on
he jocosely derived his Christian name Martin from Mars, the valiant
god of war.\footnote{\textit{Ibid.,} Nr. 4378; II, Nr. 1829. \textit{Briefwechsel} I, p. 19 of the year 1514: “F[rater]
M[artinus] Luder; cf. pp. 44, 47, 53, 65 of the year 1516.
}

The Luthers had left Möhra and removed to Eisleben, because John
believed that the mining industry of the latter city would afford
him a better subsistence. At Eisleben, according to his son, he
became a “metallicus,” or poor miner. The poor miner, however,
must soon have become conscious of the meagerness of his prospects
for advancement, for, in the year succeeding the birth of Martin,
towards the middle of 1484, he repaired with his family to Mansfeld,
the center of a flourishing mining industry.\footnote{
\textit{Tischreden, Weimar} ed., V, Nr. 6250: \textit{“Darnach ist mein Vater nach Mansfelt gezogen und daselbs ein Berghheuer worden. Daber bin ich.”}
}
The traveller of to-day,
who follows the road from Eisleben, passes friendly villages of frame
dwellings, conspicuous for their protruding upper story, a style of
building customary in that country. Continuing his journey, he soon
comes upon hills of dross and stones and smoking furnaces, which
even to-day form a characteristic picture of the city of Mansfeld,
lying between rising hills and green fields and woods. Upon his arrival,
he is greeted by the ancient castle of the counts of Mansfeld,
majestically enthroned on the hill at his right. To his left and in the
immediate vicinity of the small hollow of the valley, lies the moderately
sized city itself, situated on a rising plain. A broad street,
originally somewhat steep and at Luther’s time the only thoroughfare of the city,
wends its way upward between the houses. At the
right there is an inconspicuous dwelling which is described as the
former residence of Luther’s parents. Only a small part of the old
homestead of the Luthers is preserved. The letters I. L. and the date
1530 are inscribed upon a semi-circular arch above the door. It is
reminiscent of James Luther, Martin’s elder brother, who, in the
year of the death of his father, thus perpetuated his property right.
For a long time the parents of Martin lived in poverty and anxiety
during their occupancy of this house.

Only gradually did the miner succeeed in improving his condition.
In speaking of his early youth, Martin Luther tells us that hard toil
was the lot of his parents. At a later date he narrates how he and two
other lads once collected sausages. It happened during a procession of
poor schoolboys singing in the streets. A burgher approached them,
offering to give them sausages, but as he addressed the children,
albeit in fun, in a rough tone of voice, they fled, not understanding
the well-meant joke. Thus, says Luther, men in their shortsightedness
and ignorance often fear God when He wishes to bestow benefits
upon them.\footnote
{\textit{Ibid., I,} Nr. 137: \textit{“cum caneremus ad colligenda farcimina”} etc. Cf. \textit{Ibid.}, III, Nr. 2936;
V, Nrs. s804, 5989aa.}

He first attended school in Mansfeld. The school was situated somewhat higher up the street than his parental home, to the right of an
extended place similar to a plaza. It is partially preserved even to this
day. It was one of the elementary schools, known as Latin schools,
in which, according to an extensive custom, the students were introduced to the rudiments of Latin immediately after their first lessons
in reading and writing. Reading was learned by means of the catechism and the primer; the elements of Latin were acquired by the
recitation of the usual Latin prayers, such as the Pater Noster and
the Credo. These were arduous and toilsome years for young Martin.
The severity of the teachers and the vexatious declensions and conjugations lingered in his memory for many years. According to the
customs that obtained in the schools of those days, there was an
“asinus” (ass) which was wrapped around the lazy or ignorant
pupil. There was also among the pupils a “lupus” (wolf) appointed
to this office by the preceptor; it was his duty to record for punishment the omissions of his schoolmates. Punishment was inflicted
summarily at the end of the week. Luther states in his Table Talks
that he had once been disciplined with the rod fifteen times on a
certain forenoon. If true, this was due either to great lack of
diligence, or to stubbornness; or it was a penalty for misconduct of
which he had been guilty for a whole week.\footnote{\textit{Ibid.}, V, Nr. ss71.}
His later complaints concerning the abuse of the rod in the schools of Mansfeld and in
the schools in general, are too specific to exculpate his teachers and
many of their colleagues in other places from the charge of excessive
severity. The age was strongly biased in favor of the rod. There
were laws against such excesses, but we have no guarantee that they
were observed.

Undoubtedly the stern discipline of the school contributed to
intimidate the character and depress the spirits of young Luther.

He retains one pleasant reminiscence of his school days, when he
gratefully mentions in his writings how an elder pupil, Nicholas
Semler, often carried him to school in his arms; for the ascent of the
street, especially when covered with ice and snow, was assuredly a
hardship.

There were not many pleasant memories of his paternal home
which accompanied him in life. He did not experience the joyousness
of youth.

He avers, it is true, that his parents meant it well with him. His
father, ordinarily not communicative, with thoughts engrossed in
his labors, and his mother, who was similarly inclined, undoubtedly
told him about religion and its consolations; for they were loyal
Catholics. Thus, Martin heard from his father the edifying narrative
of the happy death of Count Günther of Mansfeld (1475), made
beautiful by the great trust he placed in the redemptive merits of
the death of Christ.\footnote{
M. Ratzeberger, \textit{Chronik}, ed. by Neudecker (Jena, 1850), p. 42, who, however, disregarding other matters, distorts the affair in conformity with the idea of the new Gospel.
Thus Luther in 1537, undoubtedly because of his father’s opposition to his entering the
monastic state. In this connection it is also necessary to take into consideration Luther’s
disposition relative to his former vocation. The same is true of his expression concerning his
father in the \textit{Tischreden}, Weimar ed., I, Nr. 881: \textit{“Semper fruit iniquus monasticae vitae.”}
Cf. In \textit{Genesim}, Weimar ed., XLIV, p. 411, and Ratzeberger, \textit{Chronik}, p. 49.
}
When, in 1530, the dying John Luther, who
at the time espoused the religious party of his son, was asked
whether he confidently accepted the traditional teachings of salvation,
he replied with simplicity and bluntness: “I would in truth be a
knave did I not believe in them.” He had never been qualified
by his educational attainments to pass judgment on the orthodoxy
of the new doctrines. For the rest he respected the sacerdotal state
during Martin’s youth, though he freely indulged in criticisms of
it after his own blunt manner. On occasion, too, he expressed his
indignation at monasticism, perhaps on account of its obvious faults
or because it was the fashion of the age, when monasteries, their
possessions or mendicant practices were disliked by many.\footnote{
\textit{Tischreden}, Weimar ed., III, Nr. 3556a: ``Er bhat der Monche Schalkheit wobl erkannt?''
}

Despite the high veneration in which the Church was held in his
parental home, young Luther did not enjoy an excessive amount of
loving and solicitous religious care. The school and the Church had
to supply the deficit; and they did.

In general, the external discipline to which his parents subjected
him was too rigorous. This caused an aversion to the father on the
part of the son, which lasted for a long time, so that, as the latter
says, the father found it necessary to regain the lost affection of
his son. Unfortunately both father and mother, Luther subsequently
complains, could not distinguish between the disposition and the
spirit of their children and drove the son to despondency (\textit{usque ad
pusillanimitatem}).\footnote
{\textit{Tischreden}, Weimar ed., III, Nr. 3566.}
His own sad experience is reflected in his
admonition to all parents not to indulge in excessive severity,
but to “associate the apple with the rod” in the training of their
offspring. “My mother,” he says, “flogged me until I bled on account of a single nut.”\footnote{\textit{Ibid.}}
It is surprising that Luther never in
later life mentions his mother with a friendly and warm feeling,
despite the frequency with which he recalls the days of his childhood and boyhood.
The consoling picture of a mother’s love, which
accompanies most men on their journey through life, was apparently
denied him. Mother and father, it appears, often acted in anger.
The latter, for example, became “thoroughly enraged,” as Luther
himself says, when his son entered the monastery. Hence, there
is no exaggeration in the statement of Albert Freitag that there is
discernible in the boy Luther “a substratum of the melancholy which
pervaded his parental home”;\footnote{\textit{Historische Zeitschrift,} 1918, Heft 2, p. 264.}
nor in that of Friedrich von Bezold
who says that his “disposition was intimidated and wrapped in gloom”
early in life.'\footnote{\textit{Geschichte der deutschen Reformation} (Berlin, 1890), p. 248.}

The pupils of the Mansfeld school were obliged to attend divine
service diligently. They worshipped in the parish church which
was dedicated to St. George and lay almost directly opposite
the school. This church originated in the thirteenth century. Rebuilt in the year 1497, when Luther was a boy, it still exhibits
the same form in which his eyes beheld it. A large statue of St.
George, the gallant knight, in the act of slaying the dragon, graces
the beautiful Gothic portal. The windows are exquisitely decorated.
The interior is adorned with richly carved Gothic altars ornamented
with figures of the saints. An altar of St. Ann, patroness of miners, was
erectedin 1502.
The well-regulated divine services were bound to send more cheerful thoughts into the soul of the child and to touch his heart
with the supernatural destiny of man.
The singing of the congregation and of the boys’ choir was
especially treasured in the memory of Luther. They produced a
beneficial influence upon him. The boy was endowed with a high
voice. Docile and highly talented, he mastered the melody both according
to form and spirit. He remained a life-long friend of sacred
music. We may picture him in imagination as he and the other
youthful choristers, vested in their white gowns, are led by the
cantor or the sub-cantor from school to church to participate in
the liturgical solemnities. The psalms, responsories, antiphons, the
Magnificat and the litanies still lived in him when he had attained
to a man’s estate. They were venerable melodies, handed down from
former ages, in the Latin text of which the singers were carefully
exercised by the cantor. Instruction in ecclesiastical chant constituted
an important part of the educational methods of the school; it was
adapted to their spiritual content as well as to their external
rendition. Besides the liturgical hymns there were the religious songs
of the people, such as the “fine hymn” at Pentecost, as Luther styles
it: “\textit{Nu bitten wir den Heiligen Geist},” and the Easter hymn,
“\textit{Christ ist erstanden},” etc. “In the days of the papacy,” says Luther
at a later period of his career, “there were excellent songs.” That
Luther was the first to introduce congregational singing is a claim
not founded on fact, but on prejudice. He himself adequately refutes this prejudice. “Congregational singing flourished before the
Reformation.”\footnote{Hans Preuss, \textit{Luthers Frömmighkeit} (Leipzig, 1917), p. 52.}

The boy learned and loved every phase of the regular ecclesiastical
life of Catholicism without meeting with any “reformatory” tendencies.
The veneration of the Saints, reminiscent of the august communion of all the servants
of God here on earth and in the realms of Heaven, the reception of
the Sacraments and attendance at the
holy Sacrifice of the Mass, the processions in the cemetery and to
the chapel of the departed, as well as those through the decorated
streets of the little city, the blessings and ceremonies of the Church
no less than the sermons of the clergy—everything made him feel
at home in the great spiritual house of God founded by Jesus Christ.
There is prejudice in the frequent assertion of non-Catholies that
there was “over-emphasis of good works in Roman piety and divine
service” which thus “suppressed pious impulses.” The congregation
and the boy Luther, on the contrary, knew “that men assembled
to praise and worship God”; “‘the heart was truly raised aloft and
joyousness replete with life was imparted to it.”\footnote{Otto Scheel, \textit{Luthers Stellung zur bl. Schrift}, I, pp. 17, 19.}
Luther heard of the Vicar of Christ, Innocent VIII (1484-1492), who governed
all Christendom with spiritual power from his see at Rome. He
sympathetically sensed the danger with which the advancing forces
of Islam threatened the faithful, and which urged the latter to have
recourse to prayer, and perhaps his lips recited the prayer for deliverance
from the Turkish menace at the sound of the “Turkish bell,”
a practice introduced about the year 1456. He always treasured
the memory of St. George, the patron saint of his church, to whose
legendary story it was his delight to give a beautiful spiritual
significance. Of the public customs of his boyhood he boasted
that all games, such as dice and cards and even dances, had been
prohibited, whereas, in his old age, he saw them gain ground.\footnote{\textit{Tischreden}, Weimar ed., IV, Nr. 126.}
It
is known that the most widely read book of those days, which was
probably also perused by the boy Martin, who had a passion for
reading, was the excellent life of Christ written by the Carthusian,
Ludolph of Saxony (died 1377), a work firmly grounded upon
Christian doctrine and following in the attractive footsteps of St.
Bernard. Rivaling Ludolph’s book as a guide to the interior life
was Henry Seuse’s “Of Eternal Wisdom” and the “Imitation of
Christ” by the immortal Thomas 3 Kempis. All these writings were
permeated with genuine religiosity and the most solid piety, undiminished by even a taint of “reformatory” thought. The principal
spiritual companion of youth, however, was the catechism, concerning which Luther’s pupil, Mathesius, admits that God has “wonderfully retained it in the parish churches, for which we must thank
our God and the ancient schools.”\footnote{Mathesius, p. 16.}

According to the latter authority, young Luther learned his catechism “diligently and rapidly.”\footnote{Ibid.}

On the other hand, his boyhood days were unfortunately contaminated by the superstitions which prevailed in his parental home
and, in general, among his Saxon and German contemporaries. Incredible notions of the devil, expressed in the crassest of anecdotes,
haunted the minds of the people. Mysterious and credulous meanings were attached to ordinary phenomena of nature. This evil of
superstition was deeply rooted especially among the miners. Their
dangerous occupation in the mysterious bowels of the earth awakened
dismal phantasies. They feared that preternatural powers spied on
them with evil intent; then, too, they expected their assistance in
discovering rich ores; or, finally, as Luther’s speeches attest, they were
familiar with queer deceptions which they imputed to them. The
death of a younger brother in the parental home was attributed to
witchcraft.\footnote{Weimar ed., XL, I, p. 315: \textit{“Frater mibi occisus per veneficia.”}}
Many stories were told about Dame Hulda and her
strange apparitions. Luther learned that devils inhabit forests, lakes,
and streams, and use water-sprites to deceive men. He was told of
a lake in the neighborhood of Mansfeld which was filled with
captive devils who caused a storm whenever a stone was cast
into the water. In conformity with his traditions, Luther steadfastly maintained that the devil had carnal intercourse with women.
Changelings as well as goiters were caused by him. In his childhood,
he says, there were many witches in Saxony. One of these lived in
the vicinity of his father’s house, and his mother had to make every
possible effort to protect the family against her. It often happened
that the children screamed frantically when subjected to incantations.
The same witch, by means of her diabolical craft, brought on the
death of a clergyman who had opposed her activity, by means of
the ground over which he passed. In his youth Luther had seen many
persons whose sickness or death had been caused by witches. “When
I was a boy,” he says, “I was told of an old woman who had brought
misfortune upon a peace-loving husband and wife” against whom
the devil had attempted his insidious wiles in vain. The witch induced them to lay sharp knives under their pillows and influenced
them to believe that one party intended to murder the other; this

being so true, she averred, as the fact that they would discover
knives under their respective pillows. As a consequence, the husband
cut his wife’s throat. Thereupon the evil one appeared to the old
woman and, without approaching her, presented her with a pair
of shoes on a long pole. Why do you not approach more closely?
asked the witch, and the devil replied: Because you are more wicked
than I, you have succeeded wherein I failed.\textit{Tischreden}, Weimar ed., II, Nr. 1429.

In his early sermons, delivered in 1516 and 1518, Luther mentions
the most superstitious things; indeed, strange ideas of this character
permeate his entire life. The excrescences of popular credulity, into
which he was introduced as a boy, were combated by the Church,
but without avail. Books that were held in high esteem even furthered
the abuse, as for instance the “Hexenhammer,” which was published
in 1487 by two Inquisitors. Ofttimes superstitious powers were
ignorantly ascribed to ecclesiastical practices, such as the sacramentals
and the veneration of the Saints. It is scarcely necessary to emphasize
that the rational use of the blessings of the Church and the veneration of the Saints were as little favorable to these errors as the
doctrinal teachings of the Church and of Sacred Scripture concerning
the devil. The popular imagination created its own worlds. What
was invisible was distorted and rendered visible; what was sacred
was subjected to profanation. Thus statues of St. Christopher that
were erected on the gates of the cities, and on the churches, even
in the churches, were made immensely large, because numerous prayer
books declared that whoever saw Christopher once a day, was protected against the evil spirits and sudden death. In his youth, these
giant statues were round about him. The legend of St. Christopher
made a lasting impression upon his mind, but, like that of St.
George and other legends, he interpreted it critically and in a spiritual sense. Nevertheless, his criticism of the legends did not dispel
his other superstitions.
