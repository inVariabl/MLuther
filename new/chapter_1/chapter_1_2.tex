\section{Magdeburg and Eisenach}

About Easter, 1497, when Martin was fourteen years of age, his
parents sent him to Magdeburg to continue his studies. The eminent
talents which he displayed were: deserving of a better education than
that which Mansfeld offered. In the city on the Elbe there lived a
citizen of Mansfeld, Paul Mosshauer by name, who was an official at the
archiepiscopal curia. It appears that the poor boy had been recommended to him, for Martin ate his meals with him and, perhaps,
also lodged in his house. The range of his ideas was enlarged in that
ancient city, which even at that time was famed for its magnificent
cathedral and other splendid buildings. Martin had reason to be satisfied with the teaching and the treatment he received at school. He
was instructed by the Brethren of the Common Life,\footnote{Undoubtedly their popular name (Loll Brothers) was derived from \textit{lulben}, to sing softly (in choir).}
who undoubtedly furnished the major portion of his higher training.\footnote{\textit{Briefwechsel}, II, p. 402, 15 June, 1522. Cf. E. Barnikol, \textit{Luther in Magdeburg} (see note 26),
pp. 3 sqq., 54. Probably Luther attended the elementary instructions of the pensioners
of the Brethren.
}

In respect of educational and ecclesiastical matters, the road he
now trod was decidedly a fortunate and very fruitful one; for the
Brethren belonged to a community that was permeated by a profoundly religious spirit, whose efforts of reform within the Church
blossomed forth splendid results at that time. Originating in the
Netherlands through the efforts of Geert Groote, who distinguished
himself by his labors for the Church, the society of the Brethren
spread to Germany. They were representatives of the so-called
\textit{“devotio moderna,”} a new conception of monastic piety more in
conformity with the requirements of the age, according to which,
in addition to prayers and begging, an active and timely efficacy
was to be cultivated in behalf of mankind. The Brethren particularly
devoted themselves to the education of the youth who attended the
existing schools, but they also erected schools of their own. Everywhere they laid special emphasis on devotional lectures, by means of
which they desired to foster the religious life of their pupils.\footnote{Cf. Barnikol, \textit{``Die Brüder vom gemeinsamen Leben,''} in the \textit{Zeitschrift für Theologie
und Kirche}, Supplementary number, 1917; \textit{ib., “Das Magdeburger Brüderbaus,”} in the
\textit{Theolog. Arbeiten aus dem rhein.-wiss. Predigerverein,} Series of 1922, pp. 8 sqq., and
\textit{“Luther in Magdeburg und das dortige Brüderbaus,” ib.,} series of 1917, Pp. I sqq.
}
Among
their students who became celebrated men were the pious Thomas à
Kempis, the scholarly Gabriel Biel, the astronomer Nicholas Copernicus, and Cardinal Nicholas of Cusa. Thomas à Kempis, having
completed his studies under the guidance of the Brethren, became
a canon regular of the Order of St. Augustine in the monastery of
Agnetenberg near Zwolle, where he died in 1471, in the ninety-second year of his life.
His writings, above all his \textit{Imitation of Christ},
faithfully reflect the spirit which he imbibed in his youth from his
teachers, the Brethren of the Common Life. The Fraternity lived and
labored in complete harmony with the Church. It was absolutely
free from that aversion to the existing hierarchy which was formerly
ascribed to it by uninformed Protestant writers. Those so-called
“reformatory tendencies” which would qualify the Fraternity as a
kind of precursor of the Protestant Reformation cannot be attributed
to it in any sense. The constitutions, which were composed by Henry
of Ahaus, the founder of the German Brethren, are animated throughout by the spirit of ecclesiastical submission; and the documents which
we possess descriptive of the activities of the Brethren, confirm their
loyalty to the Church.\footnote
{Paul Mestwerdt, \textit{Die Anfänge des Erasmus, Humanismus und der devotio moderna}
(Vol. I of \textit{Studien zur Kultur und Geschichte der Reformation}, ed. by the Verein für
Reformationsgeschichte), Leipzig, 1917. The author, on pp. 83 sqq., establishes proof of
the decidedly Catholic attitude of the Brethren. He calls (p. 86) the writings of Thomas
à Kempis “a classical memorial of the piety of the adherents of the \textit{‘devotio moderna’}”--
O. Scheel I, p. 85: “The Brethren never: abandoned Catholicism. Their ‘modern’ devotion
was erected entirely on the foundation of Catholic piety. This must, alas, be emphasized
to-day; for the Brethren have been made precursors of the Reformation.”
}

Luther’s sojourn at Magdeburg lasted but a year, and few details
of it have been handed down to us. Mathesius, a pupil of Luther,
says: “The boy, like many a child of honest and wealthy parents,
begged for bread and cried out his \textit{panem propter Deum}.”\footnote{\textit{Ibid.}, p. 17.}
Luther
also confesses that he was “Partekenhengst” during his stay at Magdeburg
and subsequently at Eisenach.\footnote{Weimar ed., XXX, II, p. 576.}
This was a term of contempt
applied to scholars who collected small donations contributed for their
sustenance, by singing before the residences of the burghers or in other
ways. Mathesius is correct in his assertion that even sons of well-to-do
parents were sent out to engage in this humiliating practice of begging
so that they might learn humility and sympathy for the poor. In
Luther’s case it can hardly have been the impoverished condition of
his paternal home that compelled him to go begging.
‘Once, while sick with a fever in Magdeburg, he dragged himself to
the kitchen and, without stopping, drank the contents of a vessel of
fresh water. Thereupon he was seized with a profound slumber. When
he awoke, the fever had left him. This event is narrated by Luther’s
friend, the physician Ratzeberger, who was impressed by it on account
of its peculiar circumstances. He perceived that the robust
frame of the boy was endowed with endurance.\footnote{Ratzeberger, pp. 41 sq.: \textit{“Er kreuchet auf Händen und Füssen abwärts in die Kuchen,”} etc.}

Luther tells us how deeply impressed he was by an edifying scene
he once witnessed in the streets of Magdeburg. A certain prince
William of Anhalt-Zerbst, who had become a Franciscan monk and
assumed the name of Lewis, passed him with a beggar's sack hanging
about him. The body of the prince had been reduced to a shadow
from fasting, night watches, and flagellations. A strapping brother
of the Order, a companion of his, more competent to carry the sack
and its burden, walked alongside of the staggering man, because the
latter, imbued with a spirit of penance, wished to bear the burden
alone.\footnote{Weimar ed., 38, p. 105: \textit{“Er frug den Sack wie ein Esel, das er sich zur Erde krümmen mausste,”} etc.}
For thirty years the noble brother Lewis, a disciple of the
Saint of Assisi, had borne the habit of his Order with honor until his
48th year, in 1504, when death relieved him of his burden. He performed
the functions of a guardian in the houses of his Order in
Magdeburg and Halle. He was known for his kindness and charity,
no less than for the fearlessness with which he criticized the sins of
the mighty, for his delicate art of mediation in their quarrels, and for
his solicitude in behalf of the poor and oppressed. The establishment
of new houses of his Order, especially in Prussia, was attributed to
his successful labors.\footnote{L. Lemmens, O.F.M., \textit{Aus ungedruckten Franziskanerbriefen des 16. Jahrbh. Reformationsgeschichtliche Studien, Heft 20 (1911),} pp. 8 sqq.}
The rare example of renunciation of the world
which he displayed, was highly esteemed, especially because he was the
eldest child of his parents. Four sons of these parents embraced the religious state.
The historian of the Reformation in the city of Zerbst
praises “the earnest and profoundly sincere piety” which characterized the princely family during the time prior to the religious schism.\footnote{H. Becker-Lindau, \textit{Reformationsgeschichte der Stadt Zerbst}, in \textit{Mitteilungen des Vereins für Anbaltische Geschichte}, Vol. XI, (Demsau, 1910), p. 250.}
In his Magdeburg days Luther was filled with admiration for the man
whose countenance he was often privileged to see. He was yet free
from the animus that impelled him in later days to condemn him and
his monastic state.

If the lad had been given an opportunity to extend the time of his
sojourn at Magdeburg, there is no doubt but that the instructions
imparted at that place, and his association with the Brethren of the
Common Life, would have affected his career most advantageously.
About Easter of the subsequent year (1498), however, he had to
transfer his residence to Eisenach, where he continued his studies at
the Latin school. It is possible that the internal crisis which overtook
the house of the Brethren at Magdeburg, contributed to this exchange.\footnote{Barnikol, \textit{Das Magdeburger Brüderbaus} (see note 26), p. 40.}
Relatives of his parents lived at Eisenach, with whom the
student was expected to establish connections.
A brother-in-law, Conrad Hutter, sexton of the church of St. Nicholas, extended a cordial
welcome to the young student. In grateful remembrance Luther invited him in after years to attend the celebration of his first Mass.

At Eisenach, too, the industrious scholar experienced the directions
of a benevolent Providence and was further inducted into the spiritual life. At school and in the families of his acquaintances the religious
life of the town furnished him with an adequate spiritual cultivation
and a healthful interior development. In order to sustain himself, the
young beggar, who was endowed with the gift of song, continued to
solicit alms at the doors of the burghers, at least in the beginning of
his student days at Eisenach. Shortly afterwards he was supported by
a burgher named Henry Schalbe, designated as Heinricianus in the
old report, whose son he escorted to school.\footnote{\textit{Tischreden}, Weimar ed., V, Nr. 55362.}
Later on a still better
home opened its doors to him. It was the residence of Kuntz Cotta.
The wife of this opulent native of Eisenach, who was descended from
an Italian family, was Ursula Schalbe, a charitable lady who is said
to have rescued the poor student from the streets, an act which many
later Protestant authors describe with touching sentimentality. Mathesius simply says that the pious matron invited him to partake of the
hospitality of her table, “because she cherished a strong affection for
the boy on account of his singing and fervent praying in church.”\footnote
{Mathesius, p. 17.}
Luther does not mention her name; only once does he refer to his
“hostess”” of Eisenach (a reference which might mean the wife of
Henry Schalbe), and avers that he learned from her the saying that
there is nothing on earth superior to the love of a woman for one so
fortunate as to win it.\footnote{Weimar ed., XLIII, p. 692.}
He entered this saying as a marginal note in
his translation of the Proverbs chap. 31, where Solomon sings the
praises of the virtuous and industrious housewife, the priceless treasure of her husband.
At Eisenach the house where Frau Cotta is supposed
to have lavished her benevolence upon Luther is still indicated
to the traveller. According to the better topographers of the town, the
residence of the Cottas is unknown.
Besides the Schalbes and the Cottas, who were devoted to the
Catholic faith, Luther became intimate with the vicar of the collegiate
church of St. Mary at Eisenach, John Braun, and his circle of friends.
Braun was versed in music and the humanities. With gratitude and
love Luther later on recalls his friendly and congenial intercourse with
him, in whose company music and song furnished many a happy and
inspiring hour.\footnote{\textit{Briefwechsel}, I, p. 1, letter to Braun, April 22, 1507, Cf. Degering (see following note), p. 88.}
In a letter to Braun, written during his student
days at Erfurt, Luther recalls his sister Catherine, whose versatility
in singing induced him to refer to her jocosely as “Cantharina”
(the singer).\footnote{H. Degering, \textit{Aus Luthers Frühzeit: Briefe aus dem Eisenacher und Erfurter Lutberkreise}, 1497--1510, in the \textit{Zentralblatt für Bibliothekswesen}, Vol. XXXIII (1916), Heft 3 and 4; Sonderdruck, p. 78.--\textit{Briefwechsel}, XVII, p. 82,}
He also desired the presence of Braun at his first Mass.

It is probable that Luther underwent his greatest spiritual experience in the small Franciscan monastery at the foot of the Wartburg,
which rises above Eisenach. The monastery was known as the Schalbian college, because it depended upon the Schalbe family for its endowment and juridical status. The Fathers at the college became his
dear friends, as is revealed by the correspondence concerning his first
Mass. He says that the excellent men of this monastery deserved well
of him.\footnote{\textit{Briefwechsel}, I, p. 3, to Braun, April 22, 1507.}
This monastery of discalced monks beneath the Wartburg
nurtured lively local remembrances of St. Elizabeth, the celebrated
landgravine of Thuringia.

As the talented boy ascended the towering castle which commanded
the surrounding country, it is small wonder that he was animated by
lofty sentiments of veneration for the princess, who was the counterpart on 2 larger scale to the Magdeburg monk of princely descent.
There his ideas of the service of God and the significance of the things
of this world became enlarged. The example of the sainted princess,
who ministered with her own hands to the sick and indigent in the
hospital she herself had founded at Eisenach, was for him an effective
and beautiful illustration of applied Christianity in the most attractive form of humility, charity, and heroism. The beautiful popular
legend of the miracle of the roses and the narrative of the cruel banishment of Elizabeth and her children moved him more than the living
monuments of the days of German knighthood and the contests of
troubadours which were enclosed by the walls of the castle. Thence
he would cast his eyes upon Eisenach itself, a city of churches and replete with pious traditions of Elizabeth. The city was girdled by a
large wall, surmounted by many round, square and hexagonal towers,
and a moat filled with water. In the fifteenth century, the city with
its vast belt of walls, was no longer the residence of counts. Like the
one-time lively Wartburg, it had become a quiet place. The churches,
however, were the scenes of an active religious life. This was especially
true of the three parish churches, each one of which boasted a school.
The Romanesque church of St. Nicholas was located in the north of
the town, the church of Our Lady in. the south. Another church,
dedicated to St. George, had been practically reconstructed in 1515.
In addition to the churches, several monasteries opened their chapels
to the faithful. There were about seventy priests in the town, of whom
some were not actively engaged in the ministrations of religion. All
too generous provision had been made for the religious requirements
of the population.

The school which Luther attended, the so-called trivium, was
situated in the immediate vicinity of the parish church of St. George.
John Trebonius--such was the humanist form of his name--was its
rector. “He was an imposing and a learned man and a poet,” in the
words of the physician Ratzeberger. Melanchthon reports that Luther
in later life praised his talents. He drilled his pupils thoroughly in
the Latin grammar of Donatus and in the syntax book of Alexander.
The “asinus” and the “lupus” were not absent. It was customary to
study ecclesiastical as well as classical literature. In order to master
the church calendar, the students of Eisenach, as in many other places,
memorized the so-called “cisio-janus,” a composition in Latin hexameters which rendered the sequence of the feasts of the ecclesiastical
year by abbreviation of the introductory words. “Cisio” signified circumcisio, the feast of the Circumcision of Christ, the first feast
in January. Prosody and the preliminaries of the “art of oratory,” and
poetry were taught. The cantor played an important part. Besides
the cantor, there were a master and probably other preceptors, who
functioned under the rector, Trebonius. The latter manifested great
respect towards his pupils. When he entered the class-room, he was
wont to remove his biretta until the pupils were seated, because, so he
said, there might be some among his scholars who were destined to
occupy stations of dignity, such as the office of burgomaster, chancellor, doctor or regent. Luther long maintained grateful and friendly
connections with the urbane Trebonius.

A prophet lived at that time in the larger Franciscan monastery of
Eisenach. The privilege of indulging in public prophecy was denied
to him, since the Brethren wisely confined the fantastic and dangerous
man to his cell. The name of this prophet was John Hilten. Luther
scarcely heard of him during his sojourn at Eisenach, but afterwards,
in 1529, he got Hilten to communicate his prophecies to him. He
then pretended to discover that his own opposition to the Pope had
been foretold by Hilten “at the time of his youth.”\footnote{\textit{Tischreden}, Weimar ed., III, Nr. 3795, of the year 1538.}
By means of
astrological calculations and a bold application of passages of Daniel
and the Apocalypse to the evils of the age, against which he agitated
with excessive zeal, Hilten had discovered that Rome was destined to
fall about the year 1514. He never mentioned Luther. According to
his prognostications, the year 1651 would witness the end of the
world. Prophecies which actually came true, such as the advance of
the Turks, invested him with a certain reputation. Towards the close
of his life, at the end of the fifteenth century, Hilten is said by witnesses to have recanted his various theological errors, especially his
erroneous and invidious propositions concerning the monastic life,
which he uttered in his exaggerated enthusiasm for the Franciscan
rule. He died at peace with his Order and the Church. It is not true
that he was immured alive, as asserted by Ernest L. Enders.\footnote{\textit{Briefwechsel}, I, p. 198, in note 1 to Friedrich Myconius’s letter to Luther concerning Hilten.}
In his
Apologia for the Augsburg Confession, Melanchthon erroneously
cites him as a papal witness to “evengelical truth.” It was even greater
perversion of the truth when Luther stated some years afterwards
that Hilten was murdered because he foretold that the execution of
John Hus “must be avenged; another shall come, whom the contemporaries will yet see.”\footnote{\textit{Tischreden}, Weimar ed., III, Nr. 3795, of the year 1538.}
After a three years’ stay Luther left Eisenach. ‘The memory of the
town always remained dear to him. His father ordered him to continue his studies at the university of Erfurt in Arthurian Saxony.
Erfurt, with its struggling university, was the property of the archbishop
of Mayence and subject to the protectorate of the Elector of
Saxony. John Luther was determined that Martin should become a
proficient jurist, one who could qualify for dignities and offices. In
order to achieve this object, Martin had first to complete what is now
called the curriculum of the liberal arts as a preparation for a professional course.

In the interim, John Luther had so improved his financial status
that he was able to support his son adequately during his academic
years. Whereas previously, according to Luther’s admission, his parents had been engaged in a bitter struggle for the necessities of life,
in later years the industrious father became the lessee of a number
of forges and pits. Mines at that time were operated under a leasehold, and often subleased. Moreover, as early as 1491, John had become a member of the “Council of Four” of the section of the city
in which he resided. He held the same office again in 1502. The members of the “Council of Four” were associated with the city council
in the government of their district. John’s repeated elections indicate
that he was regarded as an intelligent and a practical man who gradually improved his economic condition. In the spring of 1501, he sent
his promising son, now in his eighteenth year, to the university city on
the Gera.
