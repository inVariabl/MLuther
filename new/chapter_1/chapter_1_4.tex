\section{Luther's Precipitate Entrance into the Monastery}

On July 2, 1505, as the young man was returning from a visit
to his parents at Magdeburg, a violent storm overtook him not far
from Erfurt. As he was travelling alone near Stotternheim a bolt
of lightning struck in his immediate vicinity and prostrated him
to the ground. In consequence of his fall, as we know from a newly
discovered source,\footnote{\textit{Tischreden}, Weimar ed., V, Nr. 5373.}
his foot sustained a severe injury (\textit{fracto propemodum pede}).
The realization of the danger of death forced this vow
from his lips: “Do thou help, St. Ann; I will become a monk.”\footnote{\textit{Ibid.}, IV, Nr. 4707.}
His power of self-control deserted him almost completely. His friend
Jonas later would have it that, at the same time “a terrible manifestation
from heaven” appeared to him, which the frightened Luther interpreted
as a sign that he should become a monk.\footnote
{Report of 1538, published by P. Tschackert in the \textit{Theol. Studien und Kritiken}
(1897), 578. The report is not reliable in several points.}
The idea of some
kind of a vision actually formed itself, and Luther himself probably
conceived it. For, immediately after the event he expressed the belief
that the thunder-clap was for him a call from Heaven to dedicate
himself to God in the religious life. He says that “intimidations from
heaven had called him to that state.” He advised his father that he had
received a heavenly call to the monastery. John Oldecop, a pupil of
the young monk, following Luther’s representation of the event, said
that Luther “entered the monastery in consequence of strange fears
and specters.”

When his father, in opposition to Martin’s monastic vocation, used
the word “specter,” he merely meant, in the usage of those days, a
sudden excitement and imagination. Luther himself, after he had
deserted the monastery, said that he “made a forced vow while surrounded
by death”; that force made him a monk; that he made
his vow amid “consternation”; and he adds: “I became a monk
against my will and desire.”\footnote
{Weim. ed., VIII, p. 573: \textit{“neque libens et cupiens fiebam monachus”}. Cf. the words
to his father in the foreword to \textit{De votis} on his vow during the thunderstorm:
\textit{``spontaneum et voluntarium non erat.''} K. Zickendraht (\textit{``Was hat Luther im Juli 1505 bei
Stotternheim erlebt?''} in the \textit{Zeitschr. für Kirchengeschichte}, 1922, p. 142) correctly conceived
the so-called apparition as the “influence of an electric disturbance on the nervous
system” and adduces analogous examples.}
All the references in this passage pertain
to the vow he had made in the course of a raging storm, but
not to the voluntary profession made at the close of his novitiate upon
twelve months’ mature reflection, in virtue of which he became a monk.

After his return to Erfurt, having happily escaped the danger,
he decided to keep his vow at once and entered the Augustinian
monastery. His friends counseled against such a step. He himself
desisted for a time and even admits that he regretted his vow. Nevertheless, he finally insisted upon keeping it and determined to carry it
out without consulting his parents.

The vow, made under pressure of overwhelming fear, cannot be
regarded as valid. It lacked the most necessary preliminary condition,
namely, freedom of the mind and deliberation. Any well-informed
spiritual adviser could have told him that. In view of reasonable doubt,
a release from the vow could have been obtained.\footnote
{N. Paulus in \textit{Historisches Jahrbuch}, 1921, pp. 85 sq.}
Luther, however, was obstinately determined to follow the higher voice which he supposed had called him.

Considering the character of Luther, his qualities and natural
propensities, as they gradually developed after he had embraced the
religious life, we must conclude that this singular man was not made
for the monastery.

The monastic vocation presupposes qualifications entirely at variance with
those possessed by Luther’s undisciplined nature, dominated
by imagination, and especially self-will. No one who is conversant
with the religious state and its requirements, will attribute to him
a genuine vocation to that state, which is one of renunciation,
obedience, and peaceful cohabitation with spiritual brethren.
Nevertheless, he announced his immediate entrance to the prior, and
severed his connection with the burse. Fourteen days after the storm
he celebrated his departure in the company of his invited comrades
and other guests. He entertained them for the last time on his flute,
mingling plaintive strains with merry tunes. He said: “Today you
still see me, but tommorow you will see me no more.” On the following
morning—it was the seventeenth of July, the feast of St.
Alexius—he went forth with courageous steps to the portals of the
monastery, accompanied by a few sorrowing friends. The prior received
him with a joyful embrace. He was glad to be able to welcome
this promising young scholar into his community.

Luther was twenty-two years of age when the portals of the
monastery closed upon him, and a new future began for him within
its sacred precincts.

Prior to his entrance and the commencement of his novitiate, he
was obliged by custom to spend a short time of probation in a
segregated room. Then he was invested with the habit and commenced
his probationary year. The habit of his Order consisted of a white
gown with a white scapular, over which a black vestment was worn.

The scapular was furnished with a white cowl and a white shoulder-cape. The mantle, worn only on official occasions, was equipped with
a black cowl and a black shoulder-cape. The habit which Luther
wore about the house, was entirely white and had no cowl.

Internal peace was not to be the lot of the entrant: The very haste
with which he entered the monastery must have made its impression
upon the young monk; for the step he took signified a great and
lifelong sacrifice. The fright which he had experienced during the
storm still agitated him. Amid flashes of lightning he beheld the
tribunal of an angry God who demanded an account of him. Moreover,
the emotion of fear, which had lately been awakened in him,
tortured him. Melanchthon, supported by subsequent communications
from him, says that terror had attacked Luther, “at first, and
mostly in the course of that year,” \textit{i.e.}, about the time he began his monastic career.\footnote
{Grisar, \textit{Luther}, I, p. 17.}
The same authority, referring to the frequent attacks of terror (\textit{terrores}), in Luther’s
subsequent life, writes: “As he told us, and as many know, he was often
convulsed when he meditated on the wrath of God, or reflected upon striking
examples of punishment inflicted by His justice. He was subject to such
profound fear that he almost gave up the ghost.”\footnote{Ibid.}
This testimony is very valuable in explaining the condition of Luther’s soul at that time as well as afterwards.

When the lightning struck, he sustained a terrible nervous shock, which must have profoundly affected his future.
This fact constrains us to reconsider his previous condition.

As we have seen, Luther was inclined to nervousness. The melancholia
which always depressed him, was largely of a nervous kind. The thoughts
of despondency which accompanied him, arose principally from an unhealthy
psychological substratum. It appears that this constitutional evil was
in part the result of hereditary oneration. The excitable temperament of
his mother, who, on one occasion, chastised him until he bled on account of a nut, may have
been transmitted to the son. From his father he had inherited not
only tenacity of purpose, perseverance, an indefatigable zest for
work, and energy in the pursuit of his aims, but also a conspicuous
irritability concerning his own ideas and contradictions on the part
of others.

It is related of his father that he struck a peasant dead in a
quarrel at Möhra, in the period preceding his removal to Eisleben
and thence to Mansfeld. He beat his adversary so violently with
a harness--although he had not intended to kill him--that the
man succumbed to the blow. The earliest mention of this homicide
is made not two hundred years after the event, as has been maintained,
but during the very lifetime of Martin Luther; it is publicly averred
in print three times by George Witzel (Wicel), a well-informed
contemporary, who had formerly been a Lutheran, but afterwards,
from 1533 to 1538, functioned as a Catholic priest at Eisleben. The
charge of this polemical writer was never denied by Martin Luther.
Nor has one word in contradiction to it been uttered by any of his
contemporary friends and literary defenders. It was only at a later
date that objections were voiced by Luther’s friends. According to
the Protestant historian, Johann Karl Seideman (1859), “the contention,
which has ever and anon been revived, is decided by the testimony of Wicel.”\footnote
{Ibid., p. 16. F. Falk, \textit{Alte Zeugnisse über Luthers Vater und die Möhraer}, in \textit{Hist.- pol. Blätter}, Vol. CXX (1897), pp. 415--425.}

When, furthermore, his father became “thoroughly enraged,” as
Luther himself puts it, at his son’s unexpected and hasty entrance
into the monastery; when the father, somewhat reconciled, attended
the first Mass of his son at Erfurt and became infuriated at his
son on account of his violated paternal right; and when, finally,
Martin Luther, mindful of his treatment at home, indulged in the
exaggerated expression that his parents had driven him into the
monastery by the bad treatment which they accorded him;
\footnote{Erlangen, ed., LXI, p. 274 (=\textit{Tischreden}, Weimar); Mathesius, \textit{Aufzeichnungen}, p. 235.}
then, indeed, the character of John Luther appears sufficiently sanguine
to favor the assumption that the son inherited this characteristic
from his father. The most recent Protestant biographer of Luther
describes the father as possessing a sanguinary temperament.

Under these conditions the terrible shock which Martin sustained
when the thunderbolt felled him to the ground, was bound to produce
incurable results in one whose very constitution was neurotic.

Medical authorities tell us that, as a rule, neither time nor medical
skill can completely master the effects of such a nervous shock in
the case of neurotics. Recent experiences of those who sustained
shellshock in the World War confirm this conclusion. The former
malady of nervous fear usually attained a degree that was beyond
control. And even where there was no predisposition, incurable
results often supervened. We are constrained, therefore, to regard
Luther, after the thunderbolt had driven him into the monastery,
as a monk who was afflicted with an extreme case of “nerves” and
deserved commiseration; as one who, even in his subsequent career,
often was sorely tried by suffering. We are able to comprehend his
complaints about the states of fear with which he was seized both
during his monastic life and after he had abandoned it, and which
he compares with the genuine death agony.

The Augustinians of Erfurt did not perceive this state of affairs.
They were too happy at the reception of a new member whose
talents for preaching and teaching were so promising. They attached
great importance to the reports of the candidate concerning his call
from Heaven. A scholarly member of the Order, John Nathin,
professor of theology in the \textit{studium generale} of the Augustinians at
Erfurt, said at that time to the nuns at Mühlhausen that Luther
had entered the monastery like another Paul, miraculously converted by Christ.\footnote
{Grisar, \textit{Luther}, I, p. 4, quoted from Dungersheim.}
In later years he spoke of the antagonist of the
Church in a different language, saying, \textit{e.g.}, that “the spirit of
apostasy had descended upon Luther,” \textit{i.e.}, he had evolved his
doctrine under the influence of the devil.\footnote{\textit{Ibid.}, p. 17.}
There were other brethren of the Order who could not understand his peculiarities.
They afterwards said superstitiously to John Cochläus, his celebrated opponent,
that young Luther must have had intercourse with the devil. Others,
again, regarded him as an epileptic.\footnote{\textit{Ibid.}, IV, 353,}
His epileptic or seemingly epileptic attacks in the choir will receive more
detailed attention in the sequel.

It did not take long before strange stories began to be told of
Luther’s tragic resolve to enter the monastery. Thus Mathesius reports
that the sudden death of a dear companion who had been stabbed, had
frightened Luther quite as much as the thunderbolt, so that he
entered the monastery startled at the wrath of God and the Last
Judgment. In his brief reference to this matter, Melanchthon says
that Luther at that time lost a dear friend, who perished in an
accident. According to Oldecop, that friend was “suffocated” by
lightning at his side. Ludwig von Seckendorff asserts in his History
of Lutheranism (1692), on the authority of Bavarus’ manuscripts
(1548 sqq.), that the name of Luther’s friend who died a sudden death
was Alexius or Alexis. It is probable, however, that the name was
formed from that of St. Alexius, on whose feastday Luther entered
the monastery. Of the other circumstances mentioned, only one appears
to be based on certain evidence, namely, that a friend of Luther
died rather suddenly in 1505. The public records of Erfurt are silent
concerning any murder, although they note that one of the students
who was to be promoted to the master’s degree with Luther became
seized of a severe sickness during the days of the examinations and
soon after passed away. Probably this event had a terrifying
effect upon Luther. Yet it is striking that Luther himself never makes
mention of this comrade whom death carried off so suddenly.\footnote{Scheel, I, p. 246.}
He is conscious only of his own danger during the storm and of his sudden
vow. Somewhat later an epidemic in Thuringia and Erfurt claimed
for its victims two students of Luther’s acquaintance. This fact did
not, however, influence his resolve. The pestilence spread even after
he had taken up his abode with the Augustinians. The tradition of
Luther’s resolution to become a monk has been correctly preserved
by Crotus Rubeanus, who had befriended Luther in his youth. He
writes on October 16, 1519, basing his statement on Luther’s own
words: “While you returned from your parents, a heavenly stroke
of lightning dashed you to the ground before the city of Erfurt like
another Paul, and drove you into the monastery of the Augustinians.”\footnote
{\textit{Briefwechsel}, II, p. 208.}
