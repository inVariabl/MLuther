\section{Luther’s Marriage}

The tempestuous year, 1525, as we have seen, was a profoundly
stormy one also in Luther’s interior life. We have sketched the convulsions
into which he was thrown by the peasants’ revolt. In addition to this,
he was overwhelmed with labor in virtue of his leadership of a new religion.
Besides, there were the increasing worries
about the durability of the work he had undertaken, and the increase
in the so-called temptations which he habitually attributed to the
devil.

During that period, Melanchthon, too, was oppressed by worry
and lack of sleep; nay, “brought to the brink of the grave.” But
none of Luther’s friends suffered as he did in consequence of the
state of fear through which he passed. His correspondence of 1525,
especially with Spalatin, reflects these painful struggles. “How does
Satan rage,” he groans; “how he rages everywhere against the
Word.”\footnote{Grisar, \textit{Luther}, Vol. II, pp. 167 sq.}
Again he laments: “Satan is enraged against Christ, because
he has discovered Him to be the stronger.” “The rage of Satan,”
he says, “is not the least significant sign that the end of the world is
approaching.” “Of these there are greater signs than many believe.”\footnote{\textit{Ibid.}, p. 168.}
It is astounding that the condition of physical exhaustion which he
had experienced shortly before, was not more frequently repeated.
Mention was made of the fact that he had once been found lying
stretched out on the floor unconscious. He had been overcome by
“melancholy and sadness,” coupled with incapacity to digest his food.

In 1523 he suffered from a peculiar indisposition. Fever and insomnia
were among the symptoms. Johann Eberlin (Apriolus), a
renegade Franciscan, and John Magenbuch, a student of medicine at
Wittenberg at that time, attended the patient. When his condition
had improved, Eberlin sent a detailed account of the sickness, coupled
with a request for advice, to his friend Wolfgang Rychard at Ulm,
a physician who was personally acquainted with Luther. The account
is no longer extant, but the Latin reply addressed to Magenbuch was
published by the Protestant historian, Theodore Kolde. It first mentions
the necessity of curing Luther’s insomnia, calling him the new
Elias of whom Eberlin had written, “among other things,” that “it
is not to be marveled at that a man who, like master Elias, is overwhelmed
with so much intellectual labor, should experience dryness
of the brain and lose sleep as a consequence.” A poultice composed of
women’s milk and violet oil was to be applied to the patient’s head.
Then the letter continues: “If the pains of the French sickness disturb
his sleep, they may be alleviated by means of a poultice made of
the marrow of a stag, mixed with earth-worms, and boiled with some
saffron and quicksilver (\textit{vinum sublimatum}). This, if applied on retiring,
will induce sleep.”\footnote
{\textit{Ibid.}, p. 163: ``\textit{\dots et si cum hoc dolores mali Francie somno impedimento fuerint,
mitigandi sunt cum emplastro, quod fit ex medulla cervi}'' etc.}
These odd remedies are in conformity with
the state of medical science at that period. But attention has been attracted
by the passage referring to the French sickness. The \textit{malum
Franciae} is notoriously syphilis. That Luther was a victim of this disease
is not confirmed by any other source, whereas it is known, on the
other hand, that this disease spread over Europe from France and
Spain as a kind of pestilence at that time. It was also well known from
experience that sexual intercourse was not the only means by which
venereal diseases could be contracted.

Oppressed by interior afflictions, Luther in March, 1525, wrote a
letter to his intimate friend Amsdorf at Madgeburg, beseeching him
to come in haste to Wittenberg, in order to aid him “with consolation
and friendly services,” since he was “very melancholy and much
tempted.” The captain of the garrison, Hans von Metzsch, also desired
his help because of the troubled state of his mind.\footnote{\textit{Ibid.}, p. 169.}
No other
details have come down to us. In the case of Metzsch, who was a
bachelor, it is highly probable that there was question of interior
conditions, such as afflicted him four years later, when Luther urged
him to marry forthwith, and hence, it has been assumed by Protestants
that Luther’s temptations were of the same kind.\footnote
{\textit{Ibid.}, p. 169. Cfr. Köstlin-Kawerau, \textit{M. Luther}, Vol. I, pp. 796 and 729, note 2.}
It was, indeed, related
to the serious question of his marriage, which had not
as yet been solved at that time.

Although he vigorously recommended marriage to others, even to
priests and religious, Luther long resisted the idea of taking a wife
unto himself. He feared that such a step would injure his personal
reputation and that of his gospel. He foresaw that his opponents
would avail themselves of his marriage to launch a more vigorous
and successful attack upon him. “We are the spectacle of the world”
(\textit{speculum mundi sumus}), was the phrase he used in 1521.\footnote
{I Cor. IV, 9; cf. \textit{Briefwechsel}, V, p. 218.}
Later he
said: “The devil keeps a sharp eye on me, in order to render my
teaching of bad repute or to attach some shameful stain to it.”\footnote{\textit{Ibid.}, p. 134.}
He knew, moreover--and this helped to decide the issue--that the Elector
Frederick, solicitous about conserving the old conditions as much as
possible, did not favor the marriage of priests and monks. It is possible
also that he was restrained by those moral considerations and
qualms of conscience which still survived in him, and at times asserted
themselves with tempestuous vigor, since his monastic days,
when he had lived in conformity with his sacred vows.
Among others, Spalatin and Amsdorf actively promoted Luther’s
marriage. On November 24, 1524, he still wrote that he had no inclination
to matrimony, but it appears that only a few months afterwards he was
ruled by other sentiments.\footnote{\textit{Ibid.}, p. 139.}
It is quite characteristic
of him that these sentiments triumphed at the very height of the
Peasants’ War, in the days when he was subjected to the greatest
mental stress, when he feared that he was destined to die and his
work to perish. His titanic nature required a reaction against the
devil. Marriage was to furnish this reaction in spite of all the powers
of hell and the papacy. His announcement to the counselor of Mansfeld,
Rühel, under date of May 4, 1525, referring to “his Katy” and
the defiance of the devil, says enough. There are on record other vigorous
and defiant declarations relative to his marriage, such as the
announcement that he was in duty bound to proclaim by his example the
value of the married state in the eyes of the world. These
declarations were intended to conceal from himself and others the
fact that, in the last analysis, the force of nature, which he did not
restrain by prayer, impelled him to break the solemn vow he had made.

According to Melanchthon’s testimony, Luther entered into too
frequent and close relations with the fugitive nuns who had come
to Wittenberg. It should be remembered that some of them found
lodging with different families in the city, while others found a temporary
refuge in Luther’s monastery.

In his oral intercourse with people, Luther exercised an even greater
freedom of speech than in his writings. He indulged in unbecoming
jokes. On Easter, 1525, he jokingly wrote to Spalatin that he himself
was a “famous lover” and had set him an example; for he had had
three wives on his hands at one time, of whom he had lost two and
was scarcely able to keep hold of the third; that it was really astonishing
that he had not become a woman long ago, since he had written
so often about marriage, and had so much to do with women (\textit{misceor
feminis}).\footnote
{\textit{Briefwechsel}, V, p. 157, April 16, 1525. Cfr. Grisar, \textit{Luther}, Vol. II, p. 140.}
The three wives appear to be the three women whom
common report had designated as likely to wed Luther.
In the pious style of this letter he described his actual marriage as
entirely dependent on the will and direction of God. “Take care lest
I do not precede you in marriage, for God is wont to bring to pass
what we least expect.” In a similar vein he wrote to an inquisitive
lady, Argula von Staufen, who inquired about his intentions concerning
marriage as early as November, 1524. He told her that he was
“in the hands of God, as a creature whose heart He could fashion as
He would”; his feelings were yet foreign to matrimony; “but I shall
neither set bounds to God’s operation in my regard, nor listen to my
own heart.”\footnote{\textit{Ibid.}, p. 173.}
Mystical thoughts even in this slippery field. His enemies
speak without mysticism. He knows it: “Alas, poor monk, how
he must feel the weight of his cowl, how pleased he would be to have
a wife”--thus Luther, while he still sojourned at the Wartburg, heard
in spirit his Catholic opponents speak against him and his whole undertaking.\footnote
{\textit{Ibid.}, p. 87.}

These scruples were finally overcome by his peculiar mentality,
“through the operation of God.” In a letter dated June 2, 1525, he
frankly and freely requested Dr. Rühel to tell the Cardinal of Mayence:
“Should my marriage fortify His Grace, I am quite willing to
trot on ahead of him by way of example, since it is my intention anyway,
before I depart this life, to be in the state of matrimony, which
I regard as demanded by God, even if it be nothing more than an
espousal or Joseph’s marriage.” If the Elector desired to know more
about the reason why he had deferred his marriage (thus he writes
somewhat obscurely to Rühel), tell him “that I have always feared
that I was not fit for it.”\footnote
{Erl. ed., Vol. LIII, p. 308 (\textit{Briefwechsel}, V, p. 186). On the last passage Enders remarks:
``At this late date it is hardly possible to establish whether Albrecht ever entertained
the idea of following the example of his relative, the grand master.'' As late as
1525, after Luther's marriage, he sent a present of 20 gold gulden to Catherine Bora.
Köstlin-Kawerau, I, p. 738.--The decisive victory over the peasants at Königshofen was
gained on the same day on which Luther promised to ``trot on ahead.''}
Is it necessary to connect the Joseph’s
marriage with this unfitness? “Scandal for scandal, necessity breaks
even iron and gives no scandal,”--thus he had exclaimed in his published
invitation to the nuns to break their solemn vow of chastity.\footnote
{Grisar, \textit{Luther}, Vol. II, p. 243. }

“The desire to be pure”--thus runs another phrase taken from the
very heart of his dogmatic system--`‘may not be brought about by
good works, but the birth of Christ must be renewed within us by
means of faith \dots Sin is incapable of injuring me; the force of
sin has been spent. We adhere to Him who has vanquished sin.” “In
sum, despite good works, everything depends upon doctrine and
faith.”\footnote{\textit{Ibid.}, p. 148. }
The Catholic view of the matter is that the marriage which
he contemplated was not only a shameful sacrilege, but, in addition,
invalid because of the sacred vow of chastity made by Luther and
its solemn acceptance on the part of the Church.
The flames of the Peasants’ Uprising were still ablaze in the background
of his mind. The news of the bloody punishment inflicted by
the victorious princes harrowed the souls of thousands, but it did
not deter Luther from enacting the scene of his marriage before the
eyes of the world. On the contrary, the horrors of the age, as we saw,
helped to mature his resolution to wed.

His choice fell upon one of the ex-nuns who had left the convent,
a circumstance which, in the eyes of Catholics, invested the step he
took with the character of a grave scandal. Catherine von Bora herself
had been very active in the prosecution of her choice.\footnote
{Cf. Kroker, \textit{Lutherstudien}, Weimar, 1917, pp. 140 sqq. According to this Protestant
writer, Catherine’s interview with Amsdorf as mediator, contained in a Vienna manuscript,
is quite credible.}
She spurned
other alliances which were open to her. Her mind was set upon higher
aims. Either Luther or Amsdorf, she said, would be her husband. She
understood how to influence Luther with female artfulness. With
her chubby face she was not exactly a striking beauty, but she was
endowed with great prudence, energy, and a facile tongue. Luther
afterwards said that he had always observed signs of pride about her
and pretended that he had not married her for love.\footnote
{Grisar, \textit{Luther}, Vol. II, p. 185,}
She gave
promise of becoming an industrious and devoted housewife. In general
she satisfied this expectation without any particular display of
intellect. The rumors which had arisen to the effect that Luther had
had sexual relations with her prior to their marriage are unproved
and can be satisfactorily accounted for by the habitual freedom with
which he mingled in society and also, partly, by the haste with which
he married.

The wedding took place at his home, in the evening of June 13,
as the result of a sudden resolve on Luther’s part and without the
previous knowledge of most of his friends. Bugenhagen, Jonas, Lucas
Cranach and his wife, and the jurist Doctor Apel were the only witnesses.
Bugenhagen seems to have officiated in his capacity of minister. A public
wedding followed on June 27, in which a number of
invited guests participated, conformably with the custom that prevailed
in those parts.

In one of the first epistolary comments on his marriage, Luther
again voices the sentiment with which he had overcome his principal
scruple: “I have become so low and despicable by this marriage,” he
says jokingly, “that I hope the angels will laugh and all the devils
weep.” Thereby, he says, he had “condemned and challenged the judgment
of those who in their ignorance resist things divine.”\footnote{\textit{Ibid.}, p. 175.}
This remark was aimed at his lawyer friend, Jerome Schurf, who had said:
“If this monk takes a wife, all the world and the devil himself will
laugh, and Luther will undo the whole of his previous work.”\footnote{\textit{Ibid.}, p. 176.}
Schurf afterwards relented somewhat. The jurists generally, however,
supported by canon law, raised objections to the marriage.\footnote
{Erl, ed., Vol. LV, p. 157 (\textit{Briefwechsel}, XI, p. 90).}
Luther, on the contrary, repeatedly indulged in such assurances
as: “God willed it”; “the Lord plunged me suddenly into matrimony,
while I still clung to quite other views.”\footnote{Grisar, \textit{Luther}, Vol. II, p. 175.}
Jonas, one of the witnesses
present on June 13, forces himself to use similar language, although
his sentiments were divided. “I do not know,” he says, “what strong
emotion stirred my soul; now that it has taken place and is the will of
God, I wish [them] every happiness.”\footnote{\textit{Ibid.}, p. 174.}
The chief reason for the unprecedented
haste was the growth of slanderous rumors. Bugenhagen
testifies to this fact with unconcealed discomfiture, when he states that
evil tales constrained Luther to marry so unexpectedly. Luther himself
announces his marriage to his friend Spalatin in these significant
words: “I have shut the mouths of those who slandered me and
Catherine von Bora.”\footnote{\textit{Ibid.}, p. 175.}

For the rest, he is not deficient in adducing reasons for his marriage,
but on the contrary, quite resourceful. Beside the law of nature, he
mentions the will of God as revealed to him, and the malice of his
slanderers. In addition there is the motive that, prior to his imagined
assassination, which he believed to foresee in this period of storm and
stress, he was bound to “defy the devil” because the latter was causing
the world to apostatize from him.\footnote{\textit{Ibid.}, pp. 181 sq.}
Moreover, he was under obligation
to annoy and irritate his papistical opponents, “to make them
still madder and more foolish” before the end of the world. He likewise
felt bound to show obedience to his father, who at one time
wanted him to marry. Finally, and as a seventh reason, he was obliged
to “have pity” on poor, abandoned Catherine.

A peculiar impression is created by the pleasantries in which he
indulges and to which he gives utterance in proportion as the voices
of the critics grow louder, even in the ranks of his followers. Thus
he writes to his friend Leonard Koppe that he is now “entwined in the
meshes of his mistress’ tresses.” Elsewhere he speaks of the thoughts
which come to a man when he sees “two tresses of plaited hair” next
to himself upon awakening.\footnote
{Köstlin-Kawerau, I, pp. 738 sq.; cfr. Grisar, \textit{Luther}, Vol. II, p. 183.}
Writing to his friend Link, he attempts an
indelicate pun on the name of Bora, which sounds like
bier, thus: “I lie on the bier [Bore = modern German, \textit{Bahre}], \textit{i.e.},
I am dead to the world. My ‘\textit{Catena}’ [\textit{Kette} or chain] rattles her
greetings to you and to your Catena (\textit{catena} meaning chain, hence
Katie).”\footnote{\textit{Ibid.}, p. 184.}
A mannish demeanor, which he perforce observes in his
wife, frequently causes him to indulge in a jocose interchange of
“Kate” with “Kette” (chain). In the invitation to the public marriage
celebration he styles her his “Herr Caterin.” Afterwards he
frequently calls her his “Herr” (Lord), “Herr Moses,” or “Most
Holy Doctoress.”\footnote{\textit{Ibid.}, Vol. V, p. 309.}

His indelicate jests concerning his marriage and the reasons by
which he sought to justify it, were of no avail to him in face of
the numerous, unpleasant criticisms to which he was subjected. At
Wittenberg, those who thought unfavorably of the marriage did
their utmost to relay to him every damaging report. The Frankfurt
patrician Hamman von Holzhausen wrote as follows to his son Justinian,
a student at the University of Wittenberg: “I have read your
letter telling me that Martinus Lutherus has entered the conjugal
state; I fear he will be evil spoken of and that it may cause him a great
secession,”\footnote{\textit{Ibid.}, Vol. II, p. 184.}

Yet, despite his jocularity, Luther was “sad and uneasy,” as Melanchthon
says. The latter was filled with bitter indignation. Cautious as he was,
he did not express himself openly, but in a Greek
letter to his friend Joachim Camerarius he unreservedly revealed his
sentiments, or rather the chagrin which struggled within him against
his devoted, almost servile demeanor toward the person of Luther.
Without reserve he points to the occasion of the misfortune.\footnote
{The whole Greek text of the letter in Grisar, \textit{Luther}, Vol. II, pp. 176 sq., note 3.
Cfr. the quotation from Jerome Dungersheim, \textit{ibid.}, p. 145. Melanchthon was more
agreeable to Luther and the latter’s nearest friends in referring to the marriage.}

In this letter of June 16, 1525, Melanchthon complains, in the first place,
that Luther “had not consulted any of his friends beforehand.” “Perhaps
you will be surprised,” he continues, “that at this unhappy time, when
upright and right-thinking men everywhere are being oppressed, he is not
suffering, but, to all appearance, leads a more easy life
\selectlanguage{greek}
(μᾶλλον τουφᾶν)
\selectlanguage{english}
and endangers his reputation, notwithstanding the fact that the German
nation stands in need of all his wisdom and strength. It appears to me that
this is how it happened: The man is approachable in the highest degree, and
the nuns who waylaid him with all their snares, have attracted him to
themselves. Perhaps his frequent association with them, although he is noble and
high-minded, has rendered him effeminate or has inflamed him. In this way it
appears that he has come to grief in consequence of the untimely change in
his mode of life. It is clear, however, that there is no truth in the gossip that
he had previously had illicit intercourse with her [Bora]. Now the thing is
done, it is useless to find fault with it, for I believe that nature impels man
to matrimony. Even though his life is low, yet it is holy and more pleasing
to God than the unmarried state. And since I see that Luther is to some extent
sad and troubled about this change in his way of life, I seek very
earnestly to encourage him by representing to him that he has done nothing
which, in my opinion, can be made a subject of reproach to him.”

Concerning Luther’s “having come to grief,” the writer finds consolation,
first, in the fact that advancement and honor are dangerous to all men, even
to those who fear God, as Luther does, and, secondly, in the hope that Luther’s
new state of life may teach him greater dignity, so that he may lay
aside the buffoonery (or mania for making ribald jests,
\selectlanguage{greek}
βωμολοχία),
\selectlanguage{english}
for
which we have so often found fault with him. Camerarius, therefore, should
not allow himself to be disconcerted, even though he may feel painfully
aggrieved. “Through frequent mistakes of the saints of old God has shown
us that He wishes us to prove His Word, and not to rely upon the reputation
of any man, but only His Word. He would be a very godless man indeed,
who, on account of the mistake of the doctor, should judge slightingly of
his doctrine.”

These forced reflections rather reveal the timid, learned humanist
than the open-minded man, let alone the theologian. Melanchthon’s
displeasure, moreover, may have been increased by the domestic
strain that existed between his middle-class wife and Catherine von
Bora, who was of noble descent. When Camerarius later on edited
the collected letters of Melanchthon, he had the original of the
above letter before him, but did not dare to print it as it was, but
suppressed some passages and falsified others. The genuine text was
not made known until 1876, when it was published by A. Druffel
according to the original holograph in the Chigi Library at Rome.
The falsified text, however, was incorporated in the \textit{Corpus Reformatorum}
(1834), a work which has been frequently used up to
the present time.

The newly married couple made their home in the former Augustinian
monastery at Wittenberg. Elector John relinquished the onetime abode of
the monks, which Luther had fumed over to him,
to be used by him and his relatives as a residence. Catherine converted
the monastic cells into rooms for the students who became
her boarders and thereby helped to increase her modest income.
With little consideration the entire furnishings of the monastery,
including the plates, yea, even pious memorials such as the drinking
glass of St. Elizabeth, were turned to profane uses.\footnote
{Grisar, Luther, Vol. III, pp. 313 sqq.}
As master of
the house, Luther gradually forgot the sadness and malaise concerning
the change in his mode of life, which Melanchthon referred to.

The restless activity with which he continued his literary labors
also helped to divert his mind completely at times from his qualms
of conscience.
