\section{The New State Church}

Luther was absolutely right in assuring the Erfurt preachers that
the Elector John of Saxony was pro-Lutheran. In fact, he put the
case too mildly when he said that John did not favor the religious
services of the papists. The Elector had been completely won over
to the new religious system by Luther, who knew how to approach
and influence Frederick’s successor, who was inadequately instructed
in matters of religion.

John of Saxony became the patron of Lutheranism and the founder
of the State Church.

The state-controlled Church was greatly promoted by the status
of the congregations which had adopted the new religion, but were
too weak to stand on their own feet. Sprung from the contests
which the members waged against one another, and organized in
the main as a result of the violent procedure of the magistrate, these
congregations promised little durability, because of the want of internal
cohesion. Hence, as was indicated before, Luther considered
the territorial lords as the natural pillars of his Church. The secular
rulers alone were in a position to defend the preachers of the new
evangel, to remove undesirable persons from office, and to overcome
the external consequences of the existing dissension among the
citizens.

Thus, with Luther’s cooperation, the system of territorial churches
was born as the result of a certain necessity. It was nurtured and
strengthened by the prospects of secularization, held out to the territorial
rulers, of the rich ecclesiastical possessions, and also by the
prospect of an increase of authority. The prevailing tendency of
the age, which consisted in the self-exaltation of the powers of
the princes and their endeavor to make themselves independent of
the Empire and the authority of the Emperor, aided them very
effectively. Since their victory over the peasants, the politicians felt
that an approximation to absolutism was the only salvation in
these chaotic conditions. For parishes and schools were threatened
with extinction, and the rural population was sinking back into
barbarism. Hence, it appeared to be the principal social and spiritual
task of the government to take complete charge of church affairs,
not only for individual communities, but for the country as a whole.
After the social revolution had been crushed, the influential classes
benevolently and submissively cooperated with the princes to forestall
future revolutions.

As long as Spalatin occupied the position of court-preacher at
the court of the new Elector of Saxony, he was a helpful assistant
to Luther in the latter’s attempt to establish a compulsory national
church. Both men regarded it as their primary duty to fan the
flames of the anti-Catholic prejudice of their ruler.

Spalatin wrote to the Elector on October 1, 1525: “Dr. Martinus also
says that your Electoral Grace is on no account to permit anyone to continue
the un-Christian ceremonies any longer, or to start them again.''\footnote{Grisar, \textit{Luther}, Vol. II, p. 331.}

In a letter to Spalatin, dated November 11, which was intended for the
Elector, Luther expounded the thesis that by stamping out the Catholic
worship rulers would not be forcing the faith on anyone, but merely
prohibiting such open abominations as the Mass. Moreover, he demanded that
the right to emigrate should be extended to obstinate papists.\footnote{\textit{Ibid.}}

The Elector was unable to resist this powerful appeal.
On February 9, 1526, he received a letter directly from Luther, which
was intended to encourage him to attack the idolaters. Should he protect
them, “every abomination would burden his conscience before God.” In the
second place, he should reflect that “mutiny and factionism” would sweep
over the cities and the rural districts in consequence of the existence of diverse
kinds of worship. Again he declares: “Only one kind of doctrine may be
preached in any one place.” John replied in a friendly tone, assuring Luther
that he would “know how to conduct himself in a Christian and correct
manner.”

Soon the Elector intervened in the appointments to ecclesiastical
positions and in the government of the new religious society. The
principle of territorial sovereignty in ecclesiastical affairs was established
rather by practice than by open declarations. With astounding
dexterity Luther often acted as if he regarded the territorial lord
as a kind of patriarchal ruler, similar to the rulers of Israel in the
Old Testament. He gradually advanced to this position after 1520,
when, in his sermon “On Good Works,” in which he addressed the
secular authority for the first time, he demanded that “kings, princes,
and the nobility” should commence to reform ecclesiastical conditions
according to his ideas.\footnote{K. Holl, \textit{Luther}, 2nd ed., 1923, p. 327.}
As long as possible, he had upheld
his impractical ideal of a congregational religion, especially since
the Elector Frederick was not in favor of a more compact organization
of the new religious system. But now, under John, his policy,
openly favored by the court, was completely changed.

He considered three points in particular.\footnote{Cfr. for the following Holl, \textit{op. cit.}, pp. 361 sqq.}
First the disposition to be made
of the property belonging to the Catholic Church. Who was to get this
property when confiscated? With a highly characteristic conception of jurisprudence,
he answers this question thus: As a matter of course, it accrues to
the territorial lord, though it should be preserved as much as possible for
ecclesiastical uses. And it will be necessary that the ministers of the new
religion be adequately supported therefrom. None other than the prince can
look to this, since the nobility, as experience has demonstrated, endeavor to
enrich themselves by the confiscation of church property under all kinds
of pretexts; and also because the newly established congregations show themselves
unwilling or incapable of supplying the ordinary necessities of the
preachers.

The second point he stresses is this: Consistent with the utmost liberty
possible, doctrine and ritual ought to be made uniform throughout the land;
which is impossible without the help of government. As a result Luther now
(1525) begins to direct his efforts towards a visitation, to be ordered by the
princes.

The third point is the continuation of the Mass in many places. “The
unity of our Church,” says Luther, “suffers in consequence thereof.”\footnote{\textit{Ibid.}, p. 363, n. 1.}
The prince alone, he says, can suppress the Mass--an object which
Luther pursued with passionate zeal.
On the thirty-first of October he conferred with the Elector relative to
the disposition of the church property and the support of his ministers. As
John does not show himself averse, Luther takes up the second point, the
internal condition of the congregations. This he does at first by innuendo,
then by definitely indicating his wishes.\footnote{\textit{Ibid.}, p. 364.}
On November 30, 1525, he proposed
a visitation to be held under the auspices of the Elector. He suggests
“that Your Electoral Grace order the visitation of all parishes in the entire
principality,” so that evangelical preachers may be appointed and properly
supported for the congregations that desire them.

The desired visitation was realized by the electoral instruction of
1527, which definitely completed the regime of territorial churches.
True, it was not all done according to Luther’s notions. He grievously
complained of the undue restraints which the State imposed upon
the authority of the Church. Thus the contradictoriness of his attitude
avenged itself upon him.

Nothing is so little to the point as to say of Luther’s attitude
and view of things as Holl does: “Everything has been clearly and
harmoniously worked out.”\footnote{\textit{Ibid.}, p. 350.}
In reality, Luther does not conceive
of the Church as a true society and is as little able to appreciate a
Christian State “as a Christian cobbler shop,” to quote another
Protestant author, since he divides the kingdom of the world from
the kingdom of God by a deep chasm.\footnote{\textit{Ibid.}, p. 347.}
Because he repudiated
both the ancient Church and the traditional conception of the State,
he had no foundation for his church except the good pleasure of the
princes.
