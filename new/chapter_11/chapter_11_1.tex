For various reasons the year 1525 must be called a tempestuous
year. First of all the Peasants’ War stamped its character upon it.
Germany experienced tumults such as it had never seen before, which
shook Luther’s work and influenced his mind. Seldom was he so vehemently
perturbed as during these momentous months, in the
course of which, strange to say, occurred his marriage.

The year 1525 witnessed the formation, on the one hand, of a more
compact union of the Catholic princes opposed to the new movement,
and, on the other, the preparation of the fatal Protestant
alliance of Torgau and the emergence of the war-like figure of Philip
of Hessen. It marked the energetic assembly of Mayence, which
fanned the fury of Luther, but also the defection of extensive territories,
among them Prussia, and the violent introduction of the religious
change in the electorate of Saxony, brought about by the new
Elector. Luther was compelled to wrestle with the Anabaptists and
other fanatics, with powerful enemies in the Catholic camp, and with
the triumphal march of Erasmus’ polemical pamphlet on free-will, to
which, in 1525, he opposed his treatise “On the Enslaved Will,” a
work begotten literally in storm and stress.

Beyond the German frontier Emperor Charles was engaged in a
sanguinary contest with the king of France. Francis was captured at
Pavia, but the victory of the imperial arms led to serious conflicts with
the Italian States and with Pope Clement VII. At the same time,
the power of the Turks assumed menacing proportions along the
eastern border of the Empire and in the Mediterranean Sea, with no
signs of an energetic defense, which was made impossible by the unsettled
state of the Catholic forces and Luther’s hostility towards the
Christian undertakings against the Turks.

\section{Luther and the Peasants}

In 1524, the cities of Allstedt and Orlamünde in Thuringia had
been violently excited by Thomas Münzer and Karlstadt, respectively.
Urged by the electoral court of Saxony, Luther visited these unsettled
districts in August. In vain he negotiated with Karlstadt in
an inn called the Black Bear at Jena. (At the end of the conference,
Luther presented Karlstadt with a golden coin as a token that the latter
might write against him.) Thence he proceeded to Orlamünde,
where, as he writes, he was lucky that he was not “expelled with
stones and mud.”\footnote{Köstlin-Kawerau, \textit{M. Luther}, 1903, Vol. I, p. 681.}
Karlstadt, as we have seen, was thereupon officially banished from the
electorate. Soon afterwards, however, he
continued his agitation in the city of Rothenburg on the Tauber.
Münzer meanwhile went to Mühlhausen, a highly developed industrial center,
where evangelical preachers had already succeeded in
creating a great ferment. A riot ensued, but Münzer was compelled
to flee from Saxony and thereupon propagated his revolutionary ideas
in southern Germany and allied himself with the Anabaptists of
Zurich.

It was generally believed among citizens and peasants that a serious
revolt against the princes and the clergy was imminent in 1524.
The revolt actually commenced in southwestern Germany in support of the
movement which had been started by Münzer and his
associate, Pfeifer, at Mühlhausen. It spread over a large part of
Thuringia and the Harz mountain district. According to an exaggerated
expression of Luther, Münzer, after his return to Mühlhausen,
was \textit{rex et imperator} of that city.\footnote{De Wette, II, p. 644.}
The sanguinary insurrection
was supported by the discontented peasants, who had been misguided
in spiritual matters, and was preached by this fanatic, “the servant of
God against the godless, armed with the sword of Gideon.”

Peasant revolts against the rulers had not been infrequent towards
the close of the Middle Ages, caused partly by the oppressive conditions
under which the peasantry lived, and partly by the spread
of insurrectionary and revolutionary social ideas. Now, however,
these insurrections derived their impetus from the Lutheran ideas
and slogans which had permeated the masses. It would be unhistorical
to throw the entire responsibility for the gigantic movement upon
Luther. Nevertheless, it cannot be gainsaid that the ideas and preachers
of the new movement were intimately connected with it. The
doctrine of evangelical liberty played the principal role.

In most districts the rebellious peasants not only demanded absolute
liberty to change their religion, or at least the confiscation of
church property and the cessation of clerical privileges, but they also
increased their justifiable temporal demands in the name of the so-called
evangelical liberty by claiming the most unwarrantable liberties, privileges,
and tributes. This is illustrated by the Twelve
Articles, which became current at the beginning of the Peasants’
War. They were composed by Balthasar Hubmaier, a native of Waldshut, an
apostate priest who had formerly been stationed at the cathedral of Ratisbon,
but latterly had become a preacher of the new
religion. The very first article demands for every congregation the
right to elect and depose its pastor; the elected pastor is obliged to
preach the gospel without any admixture and in plain terms. According
to the last article, emancipation from most of the tithes,
from the status of serfdom, from the manifold feudal burdens and
obligations which is insisted on in the remaining articles, should be
demonstrated from Sacred Scripture.

Fundamentally, however, brute force governed the movement. For
the first time the masses, incited by the preachers of the new Evangel,
became conscious of the power that inheres in union.

How often had not Luther himself summoned his followers to
destroy the churches, monasteries, and dioceses of Antichrist,\footnote
{For examples see the above and later sections.}
True he desired this to be done by the authorities, but the peasants felt
that they were the authorities. Then, too, without mentioning the
authorities, he repeatedly pointed out, in his violent and inconsiderate
language, that an insurrection of the masses was inevitable. It appeared
to the peasants that their hour for acting had now arrived.

The conflagration began in August, 1524, in the southern region
of the Black Forest, near Waldshut, It burst into flames at the beginning
of 1525, in the territory of the prince-abbot of Kempten. In
consequence of the advantages which the peasants of that place had
gained for themselves, almost the entire peasantry of southwestern
Germany, to the Lake of Constance and the Upper Rhine, rose in
open rebellion. The resistance of the Swabian League in Württemberg
was paralyzed by the invasion of the exiled Duke Ulrich. The peasants
stormed and burned castles and monasteries and plundered
churches. Priests and noblemen were subjected to horrible maltreatment.
From Swabia the horror soon spread to the Odenwald and to
Franconia. In the latter region alone 200 monasteries and castles were
sacked and wholly or partly destroyed. The atrocities committed at
Weinsberg constituted a horrible climax.

In South Germany, however, the irregular peasant bands at the
beginning of May succumbed to the strategy of the leader of the
Swabian League, Count George Truchsess von Waldburg. They were
subjected to severe punishment. Duke Antony of Lorraine put down
the peasants in Alsace, Landgrave Philip of Hesse crushed the insurrection
in his territory with bloody arms.

Meanwhile, however, Thomas Münzer, operating from Mühlhausen, had incited
the peasants of that district and also plunged some
cities (Erfurt, Nordhausen, and Eisenach) into the maelstrom of the
revolution. His movement was outspokenly communistic and religiously
fanatical. It was more conspicuous for cruelty than the
other revolutionary movements in Germany. Münzer soon met with
his fate. Philip of Hesse, the Elector John, and Dukes George and
Henry of Saxony, engaged his forces in a decisive battle near Frankenhausen
on May 15, 1525, and defeated them. Three hundred captives,
including Münzer, were executed. According to Cochlaeus and Landgrave
Philip he died repentant and received the Viaticum in conformity
with the Catholic rite.

The great rebellion had been put down and Germany freed from
the danger of destruction. The condition of the peasantry became
more oppressive than before, while the power of the princes grew.

What attitude did Luther maintain towards the various phases of
the Peasants’ War?

Prior to the commencement of hostilities, he published his ``Exhortation
to Peace.'' He had been requested by certain representatives
of the South German peasantry to express himself on the Twelve
Articles in the light of Holy Writ,\footnote
{Weimar ed., Vol. XVIII, pp. 291 sqq.; Erlangen ed., Vol. XXIV, pp. 269 sqq.}
This he does by acknowledging
the justice of some of the demands and insisting on reconciliation and
peace. However, he employs such violent terms against the ``oppression
and extortion'' of the authorites and princes, ``on whose neck
the sword lies,'' and whos ``presumption will break their necks,''
that the desire for revolution could be only strengthened among the
masses.

In his “Exhortations” he is chiefly concened with the ``admonition'' that
the Gospel be sustained. But he charges the peasents with meddling in his
project; you desire to assist the Gospel, he says, yet suppress it by your violent
measures. For the rest, if the lords and princes “prohibit the preaching of
the Gospel and oppress the people so intolerably, they undoubtedly deserve
that God depose them from their thrones.” In imagination he already sees
the hands outstretched which are to execute the divine judgment. “Since,
therefore, it is certain,” he tells the princes, “that you govern tyrannically
and madly, forbid the Gospel and harass and oppress the poor, you have no
consolation nor hope except to be destroyed.

The tide of rebellion had already begun to rise in the district of
Mansfeld and in Thuringia when, at the end of April and the beginning
of May, Luther left Eisleben and traveled about in the affected
districts, unsuccessfully endeavoring to stem the course of the rebellion
by his sermons. His journey impressed him with the seriousness
of the situation and the danger to his Gospel, nay, even to his life.
“At the risk of body and life,” he writes, “I passed through them. In
the case of the Thuringian peasants, I personally ascertained that the more
one warns and teaches them, the more stubborn, proud, and furious they
wax.”\footnote{Köstlin-Kawerau, \textit{op. cit.}, Vol. I, p. 709.}

He described his journey to Dr. John Rühel, a counselor of Count Albert
of Mansfeld, in terms of great excitement, which dominated him for weeks
after the journey. He wrote that “the peasants, no matter how numerous,
are after all but robbers and murderers”; “that the devil had particularly
aimed at him and by all means wanted him to be dead.” He said that after
his return home he intended “to prepare himself for death,” but would not
approve of the deeds of these murderers. He is determined to defy them and
all his enemies to the utmost.

Towards the close of his letter, he mentions a particular act, which he is
prepared to perform in defiance of the devil: “If I can arrange it, I will, to
defy him, marry my Kate before I die, in case I hear that they continue.
I hope they shall not deprive me of my courage and joy.”\footnote
{Erlangen ed., Vol. LIII, p. 294 (\textit{Briefwechsel}, V, p. 164).}
This is the first
reference to a more intimate relationship existing between him and Catherine
of Bora, and to their contemplated marriage.

Luther returned to Wittenberg on May 6. Shortly after his arrival
he published a small pamphlet “Against the Murderous and Rapacious
Hordes of the Peasants,” which is a severe declaration of war against
the rebels. It is a demand permeated with the most ardent passion,
that the princes crush with inexorable might the rebels in their own
blood. They [the rebellious peasants] “rob and rave and act like infuriated
dogs \dots Therefore, whosoever is able, should dash them
to pieces, strangle them, and stab them, secretly or openly, just as
one is compelled to kill a mad dog.” Just now, he says, a prince can
merit Heaven more effectively by shedding blood than by prayer.
He will not forbid the princes to strike at the rebels even “without
a previous offer of justice and fairness,” although an evangelical government
should make use of this means. He advocates mercy only for
those who have been carried along by the revolutionary movement
involuntarily and under compulsion. Forthwith, however, he drowns
the plea of mercy by shouting: “Strangle them, whoso is able,” etc.

The existence of a strange tension in his mind is revealed in his
trembling reference to the proximate end of the world. “Perhaps,”
he says, “God intends to throw the world into a mass of confusion as
a preliminary to the day of Judgment.”

Shortly after his anxiety for the “mass of confusion” was relieved
by the victory of the princes, Luther composed a new pamphlet on
the death of Münzer. It was written after the middle of May, and
entitled, “A Horrible Story and Judgment of God on Thomas Münzer.”
It is a refutation of the latter’s prophetical claims and, in addition,
an apologia of Luther’s own Gospel, directed to his enemies in
the Anabaptist movement. In view of reports of shocking cruelties
perpetrated by the victors, he openly admonishes the princes “to be
merciful towards the prisoners and those who surrendered.” This
exhortation, however, was not sufficiently strong. Rühel, counselor to
the Count of Mansfeld, and many others, took offense at the excessive
punishment which Luther wished to inflict upon the guilty parties.
“Many of those who are friendly to you,” thus Rithel warns him on
May 26, “deem it strange that you permit the tyrants to strangle
their opponents without mercy \dots It is necessary that you apologize
for this.” Others reproachfully accused Luther of making himself a vassal
of princes, because he approved and furthered their
bloody measures. Some apostatized from him, forgetting what, as
Luther complained, God had done for the world through him. The
author of the dreadful pamphlet “Against the Murderous Peasants''
consisted with characteristic obstinacy upon his rights. The devil, he
contended, had possessed Münzer and his hordes. “When the peasants
are seized by such a spirit, it is high time that they be strangulated
like mad dogs,” he writes to Rühel in defense of his attitude.

Meantime, however, he was not so indifferent as he pretended
to be towards the hostility that had arisen against him. A little while
later he published his “Circular Letter on the Severe Booklet against
the Peasants,” wherein he proposes to render an account of himself to
all “wiseacres who would teach him how he should write.”

“What I teach and write remains true, even though the whole world
should fall to pieces over it.” “I will not listen to any talk of mercy, but
will give heed to what the Word of God demands.”

With his wonted propensity to claim the victory, he repeats his former
exhortation: “Let him who is able, in whatsoever manner he can, cut and
thrust, strangle and strike at random, as if he were in the midst of mad
dogs.” “The ass wants to be beaten, and the mob wants to be ruled by force.”
With the aid of the mob, “the devil intended thoroughly to devastate
Germany, since he was unable to prevent in any other way the spread of the
evangel.” His Gospel was the guiding star of his conduct.

For the rest, Luther says he wrote only for the benefit of the authorities
who wished to conduct themselves either as Christians or as honest folks.
He hopes to be able to tell the truth “to the ferocious, raging, senseless
tyrants” later on.\footnote{Grisar, \textit{Luther}, Vol. II, pp. 208 sqq.}

Apparently he was able to salve his conscience because of his participation
in the atrocities connected with the repulse of the Peasants’ revolt.
In later years he once said: “I, Martin Luther, have slain all the peasants
at the time of their rebellion; for, I commanded them to be killed; their
blood is upon me. But I cast it upon our Lord God; He commanded me to
speak as I did.”

Even in these horrible circumstances he relies upon his usual claim that he
is an instrument of God.

In addition to censuring his ferocity, the Catholics frankly reproved
him for complicity in the disastrous war, which they attributed to his
religious revolution and to his preachers, who had incited the people
to rebellion. Cochlaeus and Emser pointed this out in
published writings. Erasmus, that acute observer of his age, also told
him that he was to blame. Ulric Zasius, a jurist, who at one time had
favored him, harbored the same conviction.\footnote{\textit{Ibid.}, pp. 211 sq.}
The author of a polemical
work printed at Mayence accused Luther thus: “In your public writings,
you declared that they were to assail the pope and the
cardinals with every weapon available, and wash their hands in their
blood \dots You called those ‘dear children of God and true Christians,”
who make every effort for the destruction of the bishoprics and
the extermination of episcopal rule \dots You called the monasteries
dens of murderers, and incited the people to pull them down.”\footnote
{\textit{Ibid.}, p. 190; cfr. Janssen-Pastor, Vol. II, 18th ed., p. 491.}
Other
fair-minded contemporaries held up before his eyes the difference
between the rather favorable opinion of the demands of the peasants
which he had entertained at the beginning of their uprising, and the
violent language in which he assailed them when he believed that his
gospel and position were jeopardized by the raging hurricane. They
maintained that the characterization of him as the newly risen vassal
of the princes was not without foundation.

At the present time even Protestant writers who are unacquainted with the
results of historical research, generally lament the unfortunate, nay,
disastrous attitude which Luther maintained towards the origin and course
of the great social revolution. One of the most esteemed historians of this
phase of the Reformation, Fr. von Bezold, recalls Luther’s dangerous proclamation
of Christian liberty and his criticism of the Catholic clergy. “How
else but in a material sense was the plain man to interpret Luther’s proclamation
of Christian freedom and his extravagant strictures on the parsons and
nobles?” He reminds his readers of Luther’s mutinous assault upon the decree
of the diet of Nuremberg (1524) and of the impassioned invectives he
wrote against the “drunken and mad princes.” “Luther could not have
spoken thus,” he writes, “unless he was resolved to set himself up as the
leader of a revolution.” He wonders “how he could expect the German
nation at that time to hearken to such inflammatory language from the
mouth of its ‘evangelist’ and ‘Elias’ and, nevertheless, to refuse to permit
themselves to be swept beyond the bounds of legality and order.” However,
like other historians who are favorable to Luther, Von Bezold sees an excuse
in the latter’s “ignorance of the ways of the world and the grandiose onesidedness
,” which supposedly “attaches to an individual who is filled and
actuated exclusively by religious interests.”\footnote
{\textit{Ibid.}, pp. 189 sq. from Bezold, Geschichte der deutschen Reformation, Berlin, 1890,
p. 447.}

Genuine religious interests combined with political necessity resulted,
at the close of the German revolution, July 19, 1525, in the
formation of the League of Dessau, which was patterned after the
League of Ratisbon. Joachim of Brandenburg, Henry and Eric of
Brunswick, George of Saxony, and Albrecht of Mayence and Magdeburg
joined the new League. A report of the Duke of Saxony, who
was the moving spirit of the League, designated as its object the
“extirpation of the root of the rebellion, namely, the damned
Lutheran sect,” on the ground that the revolt inspired by the Lutheran
evangel “could hardly be quelled except by rooting out the
Lutherans.”\footnote{\textit{Ibid.}, p. 214.}
At a convention held at Leipsic on Christmas day,
1525, the above-mentioned princes resolved to induce the Emperor
to furnish assistance in conformity with the decrees of Worms.

The spiritual estates who assembled at Mayence on November 14
of the same year also adopted a measure with a view to call the Emperor
to Germany and to induce him to intervene. Twelve bishoprics
of the province of Mayence were represented at this meeting by
priests, while the bishops held aloof. When the resolutions taken at
Mayence became known to Luther, he attacked them in a pamphlet of
terrific vehemence, composed at the behest of the new Elector of
Saxony, but it was suppressed because of the intervention of Duke
George.\footnote
{\textit{Ibid.}, p. 215. The document is printed in the Weimar ed. of Luther’s writings, pp.
260 sqq.}

Now that the Catholics had taken a decisive position against the
new movement, certain princes who were sympathetic towards
Luther also formed an alliance. They had not accepted the invitation
extended to them after the suppression of the peasants’ uprising to
ally themselves with the other victorious princes for the sake of insuring
tranquillity for the future. Philip I, the young and proficient
landgrave of Hesse, acted as one of the leaders of the new government.
He formed an alliance at Gotha with the Elector John of
Saxony for the defense and advancement of the new movement and
later concluded the treaty of Torgau (May 2, 1526). The threats of
the Emperor were of no avail, but merely induced the dukes of
Brunswick-Lüneburg and Philip of Brunswick-Grubenhagen, Henry
of Mecklenburg, Wolfgang of Anhalt, and Albrecht of Mansfeld to
join the Protestant alliance. So little had the Peasants’ War taught
men that salvation was not to be found in the disruption of the
fatherland. Instead of uniting their forces, they were bent upon
division.

Above all others Luther himself gave an unhappy example of internal dissension
in his attitude towards the peasants. In his writings
he treats them as a class with contempt and hatred. The peasantry
repaid him for his attitude during the Peasants’ War by open animosity
or indifference towards himself and his “gospel.” Luther’s
popularity with the lower classes declined perceptibly. When ill humor
was upon him, he could scarcely refrain from heaping insults upon the
peasants.\footnote
{Grisar, Luther, Vol. II, pp. 216 sqq.; IV, pp. 210 sqq.; VI, pp. 70 sqq.}

In his estimation they are “swine”; they are “all going to the devil”; they
are “not worthy of the many benefits and fruits which the earth yields.”

They have not given adequate support to the princes. “You powerless,
coarse peasants and asses, would that you were blasted by lightning! You
have the best of it, you have the marrow and yet are so ungrateful as to
refuse to give anything to the princes?”\footnote{\textit{Ibid}., VI, 73}

He went so far as to declare that it were best if serfdom and slavery were
revived.\footnote{\textit{Ibid}., Vol. II, p. 217.}

According to a transcript of a sermon which he delivered in 1526, he
declared that the authorities are called by God “to drive, strike, suffocate,
hang, decapitate, break on the wheel the mob, so that they [\textit{i.e.}, the rulers]
may be feared.” As “swine and untamed beasts are driven and forced,” so
the rulers must insist upon obedience to their laws.\footnote{\textit{Ibid}., p. 216.}

Luther zealously endeavored to gain the support of the high and
mighty beyond the circle of the princes and lords who were already
attached to his cause. In his desperate boldness he appealed by letter
to King Henry VIII of England, who at that time was still loyal to
the Catholic Church, requesting that he join him in the interests of
his gospel, after he had aspersed him with the basest calumnies. The
king sent a very humiliating reply, which was published together
with Luther’s letter.\footnote
{\textit{Briefwechsel}, V, pp. 231 sqq. On the reply (1526), \textit{ibid.}, p. 412.}

In a paroxysm of overwrought expectancy he even applied to Duke
George of Saxony, the most active of his opponents, “exhorting” him
“to accept the Word of God.” On December 22, 1525, he requested
the Duke in the humblest terms not to believe the flatterers and hypocrites
who surrounded him and to desist from his ungracious resolve
of persecuting Luther’s teaching, which was certainly “the work of
God.” It was not “the same thing to fight against Münzer and
against Luther.” If it came to a test, this could be demonstrated by
the effects of his prayer against the Duke. “I regard my prayer and
that of my followers as more powerful than the devil himself, and if
that were not true, things would long ago have gone differently with
Luther; even though people do not yet observe and notice the great
miracle which God has wrought in me.”\footnote{Erl. ed., Vol. LIII, pp. 338 (\textit{Briefwechsel}, V, p. 281).}

A few days later Luther received a reply from Duke George, which
showed him that his plea had failed to make an impression.\footnote
{\textit{Briefwechsel}, V, pp. 285 sqq. (December 28).}

In the very beginning of his reply, the Duke returns Luther’s compliment
concerning the flatterers and hypocrites by inviting him to look for them “in
those places where you are called a prophet, a Daniel, an apostle of the
Germans, an evangelist.” The prophets of old had “all been honest, truthful
and pious men,” whereas Luther was an apostate, surrounded by apostates. He
(the Duke) would ever remain loyal to the Church, the rock of truth.
There is no new Gospel. Luther pretends to have “pulled it forth from under
the bench,” but “it were better if it had remained there; for if you bring
forth another such gospel we shall not retain a peasant [in Christendom].”
“Your fruits cause us to entertain a great horror and aversion for your
doctrine and gospels.” “When have more sacrileges been committed by consecrated
persons than as a result of your gospel? When was there greater
spoliation of religious houses?” The severest censures are accumulated in the
Duke’s reply. Thus he reproaches Luther for having ``slanderously and
scandalously'' inveighed against the Roman emperor, “to whom we have
sworn allegiance”; for having revived all the errors of Hus and Wiclif and
despised the councils of the Church; and for having “produced by his doctrines
blasphemy of the Holy Eucharist, the most precious gift of
God.” Regarding the comparison between Luther and Münzer, he says, he
is well aware that Luther is not Münzer, but “that God punished Münzer
and his wickedness through us, should be a warning” to Luther. “We shall
gladly allow ourselves to be used for this purpose as an unworthy instrument
of God’s will.”

This unmistakable threat is supported by references to Luther’s activities
in Wittenberg. The Duke points out that Luther had established there an
asylum, a citadel for apostates, including such as belonged to his territory.
All monks and nuns “who despoil our churches and cloisters,” he says, “find
a refuge with you.” The wretchedness and misery of the fugitive nuns is
evident. “Were there ever more fugitive monks and nuns than are now at
Wittenberg? When were wives taken away from their husbands and given
to others, as is now the case under your gospel? When has adultery been
more frequent than since you have written: when a wife cannot become a
mother by her husband, she shall go to another and bear offspring, which the
husband is obliged to support?”\footnote{Cf. Erl. ed., Vol. XVI, 2nd ed., pp. 513 sq.}
Lastly, Duke George mentions Luther’s
marriage, which had but recently taken place. The devil, he declares, has
seduced him through the sting of the flesh. He should have recourse to prayer,
in order to free himself from the spell of Eve, who deceived him. If you
were able to get along without a wife for a time in order to please man,
why are you unable to do so for God’s sake? \dots Prostrate yourself at the
feet of Christ, “then, by the grace of God, the monk shall be relieved of the
nun.” The writer recalls the judgment of God, “to whom you both have
made a vow” to refrain from unchastity. On his part, he promises to pardon
all the injuries inflicted upon him and volunteers to intercede for Luther
“with our most gracious Lord, the Emperor,” if he will return to his duty
and to the Church.
