\section{Progressive Destruction of Catholicism}

1525 was a year of tempests in the history of the propagation of
Lutheranism, which brought great conquests. The extent to which the
religious revolution spread among the higher and lower strata of
society was unexpectedly great.

The want of rapidity in the progress of the new movement in the
Electorate of Saxony, attributable to the dissatisfaction of the rural
population with the leadership of Luther, was supplied by the activity
of the territorial government, which was counseled and directed
by Lutheran politicians who were zealous for the interests of the
crown, particularly by the electoral chancellor, Gregory Briick. It
was also supplied by the married preachers, who congregated in this
Electorate, and not in the least by the municipal magistrates who,
in virtue of their new réle, had suddenly been projected into important
cultural and civil positions.

In the adjacent Duchy of Saxony, on the other hand, the watchfulness and
energy of Duke George, the territorial ruler, prevented the
new religion from spreading to any large extent.
The gates of Hesse were opened to the new religion by the young
Landgrave, Philip II, who has been undeservedly surnamed ``the
Magnanimous.'' As a result of the frivolous and immoral life of
his mother (commonly called Madame Venus), this ruler lacked
a strict religious and moral training. He was lively and talented,
but devoid of religious sentiments or wants. A regent when but
twenty years old, Philip II met Melanchthon and became interested
in Lutheranism. On July 18, 1524, he issued 2 mandate in which he
granted free scope to the propagation of Luther’s teachings within his
jurisdiction. He embraced the new religion himself and obstinately
defended this step against the well-intentioned and forceful remonstrations
of his father-in-law, Duke George. In March, 1525, on the
occasion of a meeting with the Elector John of Saxony and the latter’s
son, Philip assured them that he intended to dedicate his country and
his people, nay, his very life to the new gospel. He spurned the invitation
of Duke George of Saxony to join the League of Dessau for the
pacification of Germany after the Peasants’ War, and for the suppression
of the ferment of unrest, namely, the Lutheran religion.

In opposition to that League, Landgrave Philip conceived the idea
of a military league of evangelicals, which was formed at Torgau and
Gotha between him and the Elector of Saxony. None of the German
princes were more determined than the young ruler of Hesse to gain
recognition for this League and to extend it far and wide. In the
background of his mind, however, were other soaring ambitions. It
was his intention to resist the power of the Hapsburgs and to frustrate
the contemplated elevation of the Archduke Ferdinand to the imperial throne.
“With Philip’s espousal of the evangelical cause,” says
Theodore Kolde, “a political element [rectius, a new political element]
entered into nascent Protestantism.”\footnote
{\textit{Realenzyklopädie für prot. Theologie und Kirche}, Vol. XV, 3rd ed., p. 299.}
It redounded to its extreme
advantage, but also brought with it many disadvantages. For his bold
policy soon induced Philip to form a close alliance with the South
German and Swiss Zwinglians, with whose aid he expected to oppose
the Emperor. His policy, however, was an explosive that created internal
dissension.

Philip’s dictatorial conduct manifested itself when, on July 12,
1526, accompanied by 200 cavaliers, he entered Spires to attend
the diet, where, despite the objections of the presiding officer, Archduke
Ferdinand, he permitted the preaching of “evangelical” sermons
from the open gallery of his headquarters, which was accessible to
the public. His reply to the Archduke’s remonstrances was that
he would tolerate no interference, even were he to forfeit his life.
The insignia which his retinue bore on their sleeves to indicate the
new religion they professed, were composed of the initials V.D.M.I.A.
(\textit{Verbum Dei manet in aeternum}). The Elector of Saxony had
adopted the same insignia for his followers. Both the Landgrave and
the Elector gave expression to their military alliance by vesting their
retinue with uniforms of the same color.

In the beginning of April, 1525, Luther wrote exultingly to
George Polentz, Bishop of Samland: “Behold the miracle! With a
rapid stride and with full sails the Gospel hastens to Prussial”\footnote
{\textit{Briefwechsel}, V, p. 159.}
In the same year the Grand Master of the Teutonic Order, Albrecht of
Brandenburg-Ansbach, illegally proclaimed himself first Duke of
Prussia, one of the territories of that Order, thus becoming the founder
of the Lutheran State Church in the ecclesiastical district entrusted to
his Order.

Albrecht, one of the fifteen children of the Margrave Frederick,
had received a defective training, since his father, owing to his limited
income and his numerous progeny, aimed at having his son make
a living by obtaining a situation with ecclesiastical or secular courts,
rather than by means of a thorough education. He procured two canonical
benefices at the archiepiscopal court of the Elector of Cologne.
Thereafter he plied the soldier’s trade and for some time, having been
taken ill, stayed at the residence of the Hungarian court. The
Knights of the Teutonic Order elected him Grand Master in 1511, at
the recommendation of Duke George of Saxony. He took the customary
vow of perpetual chastity, as prescribed by the statutes of the
Order, and promised, under oath before the altar, to preserve and
defend, as a possession of the Holy See, the territory of the Order
that belonged to the Church. Allured by material ambitions, however
, he formed a secret alliance with Luther, beginning with June,
1521, through his confidential adviser, Oeden. This alliance purposed
to effect an arbitrary reorganization of the Order. It was in
contravention of the papal directions, to which Albrecht was in duty
bound to adhere, and which were designed to effect the amelioration of
the condition of the unmarried knights and the clergy of the
Order--a condition which was very much in need of reform. Afterwards he
paid a personal visit to Luther at Wittenberg. Incompetent
to pass judgment on the latter’s teachings, he was, nevertheless,
familiar with the decision of the Church.

The ardent demands which Luther made upon Albrecht, such as
the secularization of the Prussian territory of the Teutonic Order,
and that he himself should marry, infatuated his mind. They were
invitations which Luther, in his desire to gain a mighty position in
the east, confirmed and generalized in his “Admonition to the Teutonic
Order to avoid false chastity,” published in 1523.

The Grand Master permitted evangelical preachers, such as Briessmann,
Speratus, and, later on, Poliander, to enjoy untrammeled liberty
of action in Konigsberg, his residential city. Thence they extended their
activity into the country. The apostasy of two bishops
who belonged to the Prussian territory of the order, Georg von
Polentz, bishop of Samland, and Eberhard von Queiss, bishop of
Pomesania, opened the gates wide to the Reformation. The Grand
Master permitted both apostates to continue in office, while he himself
almost continuously sojourned abroad and succeeded in concealing
his intentions of secularization and marriage. When the report of his
intended marriage was noised about, his brother John warned him
in touching words and pleaded with him not to disgrace his name
and family by breaking his vow. However, he merely received an
evasive reply.

In spring, 1525, Albrecht of Brandenburg believed that the time
had arrived for carrying out his plan.

On April 9, he concluded a dishonorable and humiliating peace
with King Sigismund of Poland, who had warred upon the territory
of the Order. In return, Albrecht accepted as a fief from Poland the
entire territory of the ecclesiastical State of Prussia. At the same
time he declared himself secular “Duke of Prussia.” Six days after
this event went forth a ducal mandate ordering a change of religion
for all the inhabitants of the territory. It imposed a penalty upon
the clergy who were disobedient or rebellious to the new evangel.
On July 1, 1526, the castle of Konigsberg witnessed Albrecht’s solemn
marriage to Dorothy, a young daughter of the Danish King.
His example was imitated by the two bishops who had become Lutherans.
They were the first apostate bishops of the age of the Reformation.
In a new ordinance for the government of the territory, the
first territorial diet, convened at Königsberg on December 6, 1525,
had formulated laws to correspond to the new and altered religious
conditions. The banner of ducal Prussia, which Albrecht was forced
to accept from King Sigismund, waved over the assembly hall. In
place of the former black cross of the Order on a white background,
appeared the black eagle, which has remained the Prussian coat-ofarms
up to the present day. The protests of the Knights of the Teutonic
Order outside of Prussia, the declaration of the ban and the
executory mandates of the Empire were alike futile against the accomplished
violation. The solemn protest of the Pope, whose right
to the territory of the Order had been grossly outraged, was equally
futile. Naturally, the opposition of those inhabitants of the territory
who remained loyal to the ancient religion and were determined not
to adapt themselves to the religious innovation which had come upon
them like a raging storm, was likewise ineffectual.

“Thus at an early date,” says a Protestant historian of the Reformation,
“Lutheranism took its place among the political factors, and
its development was to a certain extent dependent upon the tendencies
and inclinations of the [civil] authorities and ruling sovereigns
of that day.”\footnote{W, Friedensburg, quoted by Grisar, \textit{Luther}, Vol. I, p. 333.}

The forcible intervention of the secular governments furnishes the
key to the solution of the mystery why the Reformation made such
rapid progress.

As early as 1523, a fanatical furrier named Melchior Hoffmann,
a native of Swabian Hall, made his appearance in Livonia as a lay
preacher of Luther’s doctrines. An attack was made upon the residences
of the cathedral canons and upon churches and cloisters at
Dorpat, in January, 1525. Owing to dissensions that had arisen between
the preachers of the new religion, Hoffmann obtained a favorable testimonial
for his person from Luther at Wittenberg. In conjunction with Bugenhagen,
Luther wrote his admonition “To the
Christians in Livonia.” In this letter, which was forthwith published,
he exhorts his followers not to cause any trouble on account of differences
due to external customs.\footnote{Erl. ed., Vol. LIII, pp. 315 sqq. (\textit{Briefwechsel}, V. p. 198).}
Following Luther’s trail, Hoffmann
became absorbed in eschatological chimeras. Thus he prophesied that
the year 1533 would witness the end of the world. He became one
of the leaders of the Anabaptists. Indeed, it was due chiefly to the
snfluence which he exercised in his ceaseless journeys, that the Anabaptist
sect was transplanted from Upper to Lower Germany. After
a stormy career at Reval, Stockholm, Holstein (disputation at Flensburg,
1529), in East Frisia, and elsewhere, Hoffmann finally made
his appearance in Strasburg, which had been thoroughly upset by the
reformers. Owing to the “Gospel of the Covenant” which he preached
enthusiastically, he and his followers (“Melchiorites”) became accomplices
in the atrocities which were perpetrated by the Anabaptists
at Miinster. Nowhere is the spiritual affinity between the Anabaptist
system and Lutheranism so clearly manifested as in the internal experiences
and inspirations which moved this “apostolic herald,” as he
described himself, despite the fact that Luther combated him after
1527 and wished to see him return to his former craft of furrier. For
this reason Hoffmann in his writings stigmatized Luther as a “Judas”
who persecuted the faithful. This remarkable prophet of the Anabaptist
movement died as a prisoner of the Zwinglians about 1543 at Strasburg,
after extensive wanderings, in which he believed himself accompanied by
heavenly voices.

The Anabaptists of Upper Germany possessed a type of overexcited preacher
and leader in the weaver Augustine Bader. He was a
friend of Denk, Hetzer, and Hut, Anabaptist leaders of Southern
Germany, and not only passed himself off for a prophet, but also for
a future “king,” which rank he intended to obtain with the aid of
the Turks. Secretly his adherents had supplied him with the insignia
of royalty, made of gold-plated silver, such as a crown, scepter,
poniard and chains, together with a sumptuous costume. Destiny,
however, overtook him in a nightly assembly at Blaubeuren; he was
apprehended as an insurrectionist, tortured with glowing tongs, and
burnt in the market-place of Stuttgart on March 30, 1530.\footnote
{G. Bossert in \textit{Archiv für Reformationsgeschichte}, Vols. X and XI (1912-1914).}

The adherents of the new religion, who proceeded against Melchior
Hoffmann, at Strasburg, obtained control of that city in 1529. In
that year, the magistrate, being under the dominant influence of
Zwinglian-minded preachers, completely abolished the Mass. Even as
far back as 1524, the authorities of the city had authorized the destruction
of images in churches. The defection from the Church
was especially promoted by Matthew Zell, an apostate priest, who
had married the daughter of a Strasburg artisan in 1523; also by
Caspar Hedio, until 1523 preacher at the court of Albrecht of Mayence;
but above all by Martin Bucer, a native of Schlettstadt, at one
time a Dominican and afterwards pastor at Landstuhl. In 1523, Bucer entered
upon an epistolary correspondence with Zwingli and
soon after embraced many of the latter’s rationalistic teachings, especially
the denial of the Real Presence of Christ in the Holy Eucharist. Owing
to the violent procedure of himself and his friends, Strasburg,
after 1524, experienced the progressive destruction of sacred
images, as demanded by Zwingli. The most severe measures of repression
were adopted against the Catholics. The Zwinglian gospel, however,
produced so little fruit that Bucer was forced to write after
some years: “Among us in Strasburg there is scarcely any church, no
recognition of the Word of God, no frequentation of the Sacraments.”\footnote
{Janssen-Pastor, \textit{Gesch. d. deutschen Volkes}, Vol. III, 20th ed., p. 106.}

As matters fared in the free imperial city of Strasburg, so they
developed in other imperial cities and in cities subject to episcopal
rule. Insurrection, iconoclasm, and sacrilegious violation of churches
accompanied the introduction of the new gospel in Basle by Oecolampadius,
in 1529, in St. Gall by Vadian in 1529, and in Constance
by Blaurer--to mention only those cities which were Protestantized
according to the Zwinglian idea.

The year 1525 also marked a decisive change in the free imperial
city of Nuremberg. Here one could observe how another motive fatally cooperated
in the religious upheaval, namely, the activity of
renegade priests and religious. A number of Augustinians at that
place, who were friendly to Luther, commenced by deserting their
cloister. Shortly afterwards, apostate members of the secular and
regular clergy began to preach the reformed religion. At first the
magistrate of the town prohibited only the discussion of controversial
questions from the pulpit. Two provosts and the prior of the
Augustinians abolished the Mass. John Walther, an Augustinian
preacher at the church of St. Sebaldus, the abbot of St. Aegidius, and
the provost Pressler embraced the state of matrimony. One of the
prime movers was Andrew Osiander, a renegade priest and preacher
who later became famous as a Protestant controversialist. He, too,
married. At the diet of Nuremberg, in 1524, the Catholic prelates
were mocked by the excited mob. The condition of the many loyal
or doubting Catholics became even worse after the impetuous Wenceslaus
Link, a companion of Luther’s in the monastery of Wittenberg,
came to Nuremberg from Altenburg in the company of his
wife and, in August, 1525, commenced to function there as custodian
and preacher in the New Hospital. In this latter year the towncouncil
formally decreed the adoption of the Lutheran religion. Lazarus
Spengler, clerk of the town-council, was mainly instrumental
in bringing about this decision.

During the period of the religious upheaval, Spengler and other
members of the town-council, like Jerome Ebner and Caspar Nützel,
succeeded in preserving from destruction at least the images, altars,
and other objects of religious art for which the imperial city
was famous. The ornaments of the churches also survived, to a great
extent, the subsequent iconoclastic assaults of Zwinglianism; and even
at this late day they evidence the profoundly religious life and artistic
fervor of Nuremberg’s Catholic period.

By the adoption of tyrannical decrees the magistrates shackled the
old religion. The exercise of pastoral functions was denied to the religious
orders, the clergy were classified as civilians, and those who
complied willingly were assured the life-long enjoyment of their
benefices. The monastery of St. Aegidius, with a community numbering twenty-five
persons, surrendered to the town-council in 1525.
The Augustinian convent, of which no less than twenty-five members, allured
by liberty and matrimony, had embraced Lutheranism,
likewise surrendered. The Carmelite and Carthusian monasteries
eventually also surrendered, although many of their inmates remained
loyal to the ancient religion, among them being the courageous prior
of the Carmelite monastery, Andrew Stoss, a son of Vitus Stoss, the
celebrated sculptor. He determinedly resisted the town-council for
a long time. Thus, in the course of one single year, 1525, Nuremberg
experienced a complete transformation. The Dominicans remained
loyal till 1543, when five of the last remaining members surrendered
their monastery to the city authorities.

The most notable resistance was offered by the Order of Friars
Minor, whose members suffered every kind of persecution and the most
bitter poverty until the last survivor passed away, in 1562. The Poor
Clares, pious daughters of the Saint of Assisi, remained loyal under the
rule of their highly cultured abbess, Charitas Pirkheimer, a sister of the
famous humanist, until their gallant community became extinct.
Deprived of their preacher and confessor, these nuns, eighty in all,
most of whom were of patrician descent, were forced to listen to the
sermons of Osiander and other Protestant dominies behind the latticework
of their cloister. The Gothic choir of their church, preserved to
this day, solemnly towers aloft amidst the modern buildings that surround
it, a monument to the heroic fortitude of these nuns. The unpretentious
old cloister, once the residence of the venerable, prudent,
and matronly Charitas Pirkheimer, was demolished only a few years
ago.
