\section{The Storms at Erfurt}

The city of Erfurt, which was subject to the archbishop of Mayence, affords
a typical illustration of the storms which accompanied
the progress of the religious revolution in 1525. It will repay us, therefore,
to review the events which occurred there in the course of this
year, and Luther’s attitude towards them.

After the first “anticlerical upheaval” (\textit{Pfaffensturm}), Luther,
who had just returned from the Wartburg, issued a printed ``Epistle
to the Church of Erhurt,'' warning the members against disturbances
and setting forth his own attitude: “As yet I have not raised a finger
against them [the papists]; Christ has slain them with the might
of His tongue.”\footnote
{July 10, 1522; Erl. ed., Vol. LIII, p. 139 (\textit{Briefwechsel}, III, P. 431).}
At Erfurt, he delivered a sermon which was
calculated to arouse strong animosity against “the fat and lazy
priests and monks,” as he characterized them, “who only feed their
big bellies.” “We must crush the seed of this Satanic head.” He
boldly maintained that the secular rule of the archbishop of Mayence
had no right to exist, and “in virtue of the orthodox faith you are
spiritual and should judge all things.” “Faith does everything, and
good works, too, result from it.”\footnote{Grisar, \textit{Luther}, Vol. II, p. 347.}

The religious upheaval at Erfurt, however, failed to produce these
good fruits, but engendered bad moral conditions, so that even
Eobanus Hessus, a friend of the Reformation, known as “King
of poets,” wrote in 1524: “Immorality, corruption of youth, contempt of
learning, and dissensions, such are the fruits of your
Evangel.” “Oh, unhappy Erfurt!”--he cries in a letter of this same
year, in which he stigmatizes “the outrageous behavior of these godless
men of God,” namely, the apostate priests and new preachers.
He sees the battle-field of the passions tinged with “blood.”\footnote
{\textit{Ibid.}, pp. 349 sq.}

The scholarly Augustinian, Bartholomew Usingen, who had once
been Luther’s professor at the University of Erfurt, also predicted
a bloody revolution for the same year (1524), which was to break
out at Erfurt and reach the most remote districts. In his gloomy forebodings
he prophesied that the religious storm would bring about
the decline of the empire and the loss of Germany’s ancient greatness.
“Why,” he remonstrates with one of the revolutionary preachers
of Erfurt, “why have you ordered out the pickax, the mattock, and
the spade in your sermons, if the Word of God is sufficient for the
Church? Why have you called out to the mob that the peasant must
abandon the soil in order to come to the aid of the gospel?”\footnote
{\textit{Ibid.}, pp. 336 sqq., where proof is adduced for the following statements.}
Oblivious of the debt of gratitude which he owed him, Luther, who
was familiar with the energy with which Usingen championed the
cause of the Church, denounced the venerable man as a fool. In a tone
of frivolity he jeered at the teachers of the university to which he was
indebted for his education, and who still remained Catholic, decrying
them as the “sophists of the biretta and the pointed hats.”

The Reformation received every imaginable assistance from the
majority of the town-councilors at Erfurt. Thus encouraged, Johann
Lang, a renegade Augustinian monk, proposed the slogan, “The
gospel must be sustained by the aid of the sword.”\footnote{\textit{Ibid.}, p. 350.}
The Catholic
canons of St. Mary’s and St. Severin’s were repeatedly compelled
to protest against acts of violence. By confiscating their possessions
the town-council intended to force them and the remaining clergy
to yield. On April 27, 1525, when the revolutionary spirit already
stalked through wide stretches of Germany, and the war-cries of
the peasants resounded, the treasuries of the two afore-mentioned
churches were subjected to a close search by the magistrate. Splendid
works of art, which had been given by the forefathers, and faithfully
preserved by the Church, were squandered and destroyed.

When the day of the peasants’ revolt dawned in the vicinity of
Erfurt, they impetuously demanded the new liberties. Their object
was equality with the citizenry, who were hostile to the magistrates.
They made their peremptory demands in the name of the new
gospel. “God has enlightened us to march upon Erfurt with arms,”
one of their leaders exclaimed.\footnote
{\textit{Ibid.}, pp. 357 sqq., there further proofs are adduced, particularly from the excellent
work of Eitner, \textit{Erfurt und die Bauernaufstände im 16. Jahrhundert}, Halle, 1903.}
The representatives of fourteen
villages in the district of Erfurt, having met in a tavern, swore
“with upraised hands” “to fortify the Word of God, and to form
an alliance for life, with the object of abolishing the ancient tribute
of which they had freed themselves.” On April 25 or 26, they appeared
with scythes, hoes, and carbines beneath the walls of the
city. The magistrate at once conceived the clever idea of diverting
the malcontents upon the clerical estate and the Electorate of Mayence.
Having completed their negotiations with Hermann von Hoff, president
of the Erfurt town-council and an opponent of the clergy, they opened
the gates of the City to the threatening mob, on condition that it would
spare the property of the citizens. The palace of the archbishop and the
custom house, however, were turned over to them. The salt-kilns and almost
all residences of the clergy were eventually stormed and plundered by
the mob, who with unspeakable barbarity disposed of the sacred vessels,
images, and relics belonging to the churches, assisted by many of the
lower citizenry. ``Lutheran preachers like Eberlin von Günzburg, Aegidius
Mecheer, and Johann Lang, mingled with the citizens and peasants
before the palace at Mayence and harangued them.” Every convent
occupied by nuns was confiscated, and the inmates were expelled
without mercy in order to furnish quarters for the peasants. The
Augustinian monastery, where Luther had sojourned as a monk, was
desecrated by the invaders.

Sentence of death was pronounced upon the ancient cult by the
adoption of a resolution which decreed that pastors were to be installed
and deposed only by the congregations and that “the pure
Word of God” alone was to be preached in the pulpit, “clearly and
without any addition of human commandments, ordinances, and
doctrines.” Johann Lang, “the apostate, fugitive, and married monk,”
as a contemporary Catholic writer calls him, was appointed preacher
at the cathedral. Most of the clergy left the city.

With the consent of the magistrate the archiepiscopal rule was
declared terminated.

The magistrate was soon deposed and replaced by two committees,
one constituted of the lower citizenry, the other of the peasants, who
jointly assumed the government of the city. The former members of
the magistracy were threatened with decapitation. The preachers,
however, succeeded in restoring their authority.

As a norm for the future guidance of the community, which
was deeply divided, twenty-eight very accommodating articles were
proposed by the town-council and received the “new seal” of the
municipality and the peasantry. The preaching of the pure Word
of God and the free election of pastors again headed the list. For
the rest, the citizens and peasants hedged the town-council about
with many limitations, On account of certain debatable points, it
was agreed to leave the regulation of the town to Luther, who,
however, wisely refused. His intervention could only have made
matters worse. He was not qualified for such pacific labors and
had no talent for public organization. Moreover, at that time,
when the defeat and punishment of the peasants had already begun, he
was agitated by that unhappy frenzy against the mob which crops
out in his writings. On this account, the twenty-eight articles were
placed before him only for his written opinion.

The town-council knew how little the demands which had been
extorted from it and which were so favorable to the peasants, could
be expected to meet with the approval of Luther. In matter of
fact he pronounced the articles absolutely “inept” and wrote\footnote
{On September 19; Erl. ed., Vol. LVI, p. XII (\textit{Briefwechsel}, V, p. 243).}
that they clearly revealed that they were being proposed by “men
who are too prosperous,” whose demands were being made at the
expense of the magistracy and to the detriment of the public welfare.
He claimed that “they wanted to subvert the existing order
of things with unheard-of presumption and malice”; that these
articles reduced the council to the level of servants of the community;
that they caused the “city of Erfurt to be lost” to the
princes; and that it had even been resolved to withhold from the
Saxon Elector, the legal protector of Erfurt, the tax to be paid by his
subjects for protection. The “mob” should not be allowed to govern
all things, to bind the magistrate hand and foot and set it up as
an “idol” and let it see how “the horses drive the coachman.” It
is worthy of note that, whilst he was in this restraining mood, Luther
found it quite inadmissible that congregations should appoint their
own pastors and demanded that the town-council should at least
exercise a certain “supervision” in this matter.

However, he was pleased with that article which provided that
persons who plied an indecent trade should no longer be tolerated,
and that the “house of common women” should be closed--a measure
which he advocated also for other cities.\footnote{Grisar, \textit{Luther}, Vol. II, p. 359.}
For the rest, we may remark
that, during the archiepiscopal regime, a house of correction
for the punishment of loose women had existed at Erfurt, but had
been razed to the ground when the peasants entered the town.

Luther is silent about the abolition of the sovereignty of the
archbishop of Mayence. In a sharp letter issued May 26 through
his vice-regent, Archbishop Albrecht had refused to relinquish his
rights over the city and demanded the expulsion of the Lutheran
sects and an expiatory tribute. But the Elector John of Saxony
promised the town-council that he would support them on the religious
issue and act as their “liege lord, territorial and protecting
prince,” since he, too, was devoted to the Word of God. Relative
to the secular government of Mayence, John, no less than the other
protector, Duke George of Saxony, insisted upon an amicable understanding.
This demand became effective only after the Elector of
Mayence threatened to appeal for armed intervention to the dreaded
Swabian League. The remnant of the peasants withdrew from the
occupied city, Albrecht’s sovereignty was reestablished, and the expelled
clerics were reinstated. They had yielded only to superior
force. The Catholic element at Erfurt was still numerous, wealthy,
and influential. The city council was compelled to promise the
Archbishop to reconstruct the demolished buildings and to make
restitution, as far as possible, for the spoliation which he and the
churches had sustained. In addition, the town-council was obliged
to pay him 2,500 gulden and to indemnify the two collegiate churches,
namely, the cathedral and the church of St. Sever, to the extent of
1,200 silver marks. These two stately churches have remained in
the possession of Catholics up to the present time.

On the other hand, however, Lutheranism obtained complete liberty to propagate
itself. At the time of the restoration of the two
churches, Cardinal Albrecht, who was formally reinstated in 1530,
declared with a striking and far-reaching indulgence, “as regards the
other churches, and matters of faith and ritual, we hereby and on this
occasion neither give nor take, sanction nor forbid anything to any
party.”\footnote{\textit{Ibid.}, p. 361.}

The Augustinian monastery at Erfurt, that submerged seat of
Catholic piety, did not survive the revolution of 1525. On July
31 of that year, Adam Horn, the last prior, received permission
from John von Spangenberg, the vicar-general of the dispersed
Congregation, to abandon the monastery, since he was no longer
safe in it. Usingen joined the brethren of his Order at Würzburg.
The last trace of Nathin is found in 1523. Of an aged monk, who
remained loyal to the Church and was compelled to live outside
the cloister, Flacius Illyricus relates that he used to recall the religious
zeal which Luther had displayed in the monastery and his dutiful observance
of the rule.\footnote{\textit{Ibid.}, p. 361, note 2.}

A courageous and eloquent Franciscan, Dr. Conrad Kling, rallied
the Catholics of Erfurt about himself. When he preached in the
capacious hospital, the audience was so numerous as to overflow
into the churchyard outside.

The Catholic members of the town-council, encouraged by Kling,
to the annoyance of Luther championed the cause of Catholicity
with such success that the Lutheran preachers saw their mission in
the city rendered precarious. They complained to Luther that their
revenues were restricted and they were reduced to “hunger, misery,
and destitution.” The people demanded to know who had sent
them. Suffering from public contempt, they thought of abandoning the town,
when, in 1533, Luther encouraged them to remain and
sustained them by means of a letter which he had composed jointly
with Melanchthon and Jonas.\footnote{Erl. ed., Vol. LV, p. 25 (\textit{Briefwechsel}, IX, p. 341).}
Their mission, he said, was not to
be contested, since they had been “openly and unreservedly” called
by the town-council; they should “be patient for a year or a short
time.” Referring to the end of the world, he said that the treatment
to which they were subjected was but an “unsightly, horrible
aspect of the end of the world, and of the last fury and wrath of
Satan, equally terrible to behold.” He promised to appeal to the
Elector of Saxony, “who does not favor the religious services of the
papists,” against the reprobate monk Kling.

The acts of the soldiery against the peasants, and Luther’s state of mind
against the McJob, inspired him at that time to compose a work which he
curiously entitled: “Whether Soldiers can be in the State of Grace.”\footnote
{Weimar ed., Vol. XIX, pp. 628 sqq.; Erl. ed., Vol. XXII, pp. 244 sqq. this: pamphlet
was written towards the close of 1526.}
He says previous ages did not raise this question, and pretends it was the first
time that it was found necessary to solve a case of conscience for
the soldiers. He tries to show that it is “divine” to subdue the mob and unjust
enemies with violence. The plea sounds like a justification of his pronouncements
against the murderous peasants, especially when he asserts that
it is God, not man, who destroys unjust, hostile force. “It is God who hangs,
quarters, decapitates, slaughters, and makes war.”\footnote{Erl, ed., Vol. XXII, p. 250.}
Subjects--the mob--may in no instance constitute themselves judges of authority. It is foolish to
yield too much to the mob, lest it become frenzied; the mob is rather to bear
and suffer the utmost, as a Christian duty, even if the authorities do
not observe the oaths they have taken. No one may rise against tyrants; but if the
masses, nevertheless, expel or slaughter them, it is to be regarded as a divine
fate; the sword of Damocles hangs suspended over their heads all the time.
