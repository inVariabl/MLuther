\section{Luther’s Principal Work: On The Enslaved Will}

Mention has already been made of Luther’s impassioned work,
“Against the Heavenly Prophets,” which appeared in the beginning
of 1525. In a pamphlet boiling with indignation he attacked, about
the same time, the two Bulls of Clement VII on the ecclesiastical
Jubilee of 1525. Now comes Antichrist again, he writes, with his
putrid, reeking, mendacious indulgence-wares, which have long ago
been derided by mankind. Germany in the end will fare worse
than Jerusalem.\footnote{Weimar ed., Vol. XVIII, pp. 255 sqq.; Erl. ed., Vol. XXIX, p. 297.}
He had two “Sermons” printed in order to take the
field against the framing of new ecclesiastical ordinances: marriage,
the laying aside of the religious habit, the eating of flesh meat and
similar matters must not be subject to the tyranny of the pope.\footnote
{Köstlin-Kawerau, II, p. 141; Weimar ed., Vol. XV, pp. 571 sqq.; 609 sqq.; Erl. ed.,
Vol. XVII, 2nd ed., pp. 223 sqq.}
An illustrated satire, published by him in the beginning of 1526,
was entitled: “The Papacy Described and Depicted in its Members.”
In this work the secular and regular clergy appear in their habits
and are ridiculed in verse. In the introduction Luther says that there
is by far not enough of such derision; that kings and princes had
flirted with the papal harlot and still indulged in this practice. It
is necessary, according to the Apocalypse of St. John (XVII, 1 sqq.),
to fill her cup until she lies crushed like dirt in the street, and until
there is nothing more despicable than this Jezabel.\footnote
{Weimar ed., Vol. XIX, pp. 7 sqq.; Erl. ed., Vol. XXIX, pp. 359 sqq.; cfr. Grisar-Heege,
\textit{Luthers Kampfbilder} 3 (\textit{Lutherstudien}, n. 5), pp. 24--37, with illustrations.}
This he endeavored to do to the best of his ability in many passages of the
second part of the \textit{Kirchenpostille}, which he had printed at the
end of 1525.

Besides the works mentioned and those which Luther wrote during 1525,
especially against the peasants, he now published his exhaustive
treatise “On the Enslaved Will” (\textit{De Servo Arbitrio}).\footnote
{Weimar ed., Vol. XVIII, pp. 600 sqq.; \textit{Opp. Lat. Var.}, VII, pp. 113 sqq. For detailed
information an this work, cfr. Grisar, \textit{Luther}, Vol. II, pp. 223 sqq.}
According to his own statement, it excels all his other works in
importance and is devoted to the principal doctrine and cornerstone
of his system.

This Latin work was intended to convey to all countries his
defense against Erasmus’s attack in the matter of free-will and
grace and to demonstrate man’s absolute inability to do good. For
a long time he had hesitated to engage in an encounter with the
great humanist. The pleas of his wife, as he himself confessed at a
later period, finally determined him to tackle the task. True, theology
was a rather indifferent matter to Catherine, indeed, it was beyond
her ken; but she could not tolerate the reproach that Luther was
unable to reply.\footnote{Grisar, \textit{op. cit.}, Vol. II, p. 264. }
With bitter chagrin she had heard of the triumphs
of his adversaries, and that the many humanistic admirers of Erasmus
would apostatize from the new “gospel.” When Luther began to
write his reply, in the second half of 1525, he felt, as he himself
expressed it, as if a knife had been placed at his throat by Erasmus.\footnote
{Köstlin-Kawerau, \textit{M. Luther}, Vol. I, p. 659. }
As far as he himself was concerned, he was now resolved to strangle
the doctrine of free-will and all its representatives, and to demonstrate
that man can do no good without the co-operation of God.
He produced a work full of contradictions, marked by passion and
extremes. He not only divests man, by numerous misinterpreted Biblical
passages, of his capacity for discerning and choosing what is good, including
even the purely naturally good, if God does not substitute his omnipotent
efficacy for the human intellect and will; but in Luther's eager desire
for combat everything is absolutely subordinated to a blind fate, subject
to the sole activity of God. If God foreknows all things, which is beyond
controversy, then all things must happen by necessity, and He must be
the inevitable cause
of all. God brings everything to pass even where there is no question
of the influence of grace for the salvation of man (\textit{citra gratiam}).

“Whatever God has made,” he says, “He moves, impels, and urges forward
(\textit{movet, agit, rapit}) with the force of His omnipotence, which none can
escape or alter; all must yield compliance and obedience according to the
power conferred on them by God. God in particular moves the will “by means
of His omnipotence, in consequence of which man necessarily entertains this
or that desire, as God gives it to him, and as He forcibly impels it with His
movement (\textit{rapit}) \dots Whether good or bad, every volition is driven by
force to wish and to act.”\footnote{Grisar, \textit{Luther}, Vol. II, p. 265.}
Luther in other passages conceded the existence
of free-will “in inferior matters,” but not in respect of the good, which is a
contradiction. He himself shows that he “in reality does not wish to be
exactly understood in the sense of this restriction.”\footnote{\textit{Ibid.}}

With the same intensity he assumes the domination of the devil in the
realm of morals, but he was not sufficiently concerned with the compatibility
of the sovereign authority of God on the one hand and the activity of the
devil on the other. “If we believe”--these words of his can be read only with
anguish of heart--``if we believe that Satan is the prince of this world, who
constantly attacks the Kingdom of Christ with all his might and never
releases the human beings he has enslaved without being forced to do so by the
power of the Spirit of God, then it is clear that there can be no free-will.”\footnote{\textit{Ibid.}, p. 273.}
Either God or Satan rules mankind. This is his favorite idea, which destroys
free-will, the noblest gift of our nature. “The case is simply thus,”
he resolutely writes; “if God is within us, the devil is not there and we can only
desire what is good. But if God is absent, the devil is present, and then we
can desire only what is evil.”

“The human will,” he continues with a figure of speech which has become
famous, “stands like a saddle-horse between the two. If God mounts into
the saddle, man wills and goes forward as God wills \dots But if the devil
is the horseman, then man wills and acts as the devil wills. He has no power
to run to one or the other of the two riders and offer himself to him, but
the riders fight to obtain possession of the animal.”\footnote{\textit{Ibid.}, p. 374}

With a horrible temerity Luther declares this viewpoint to be
the essence and kernel of religion. It is his opinion that, without
it, the dogma of the Redemption falls, since with free-will Christ
would lose His unique and eminent significance, human works would
prevail, and self-righteousness, Pharisaism, and hypocrisy would occupy
the place of self-effacing humility.

Thus he leads his readers back to the pseudo-mystical errors whence
his entire system of theology sprang.

Protestant investigators, who generally display annoyance at these
propositions, incidentally and briefly touch upon the question whether
this “form of piety is not to be judged pathologically.”\footnote
{Julius Köstlin; cfr. Grisar, \textit{Luther}, Vol. II, p. 274.}
The
pseudo-mystical traces and many of the details concerning the mental
constitution of the youthful Luther, which have been heretofore
adduced, furnish an affirmative reply to this question. Thus the
Protestant theologian Kattenbusch describes Luther’s frame of mind
when he composed the latter work, as “not normal” nor “religiously
healthy.''\footnote{\textit{Ibid.}, p. 284.}
And Otto Scheel speaks of the ``fundamental idea'' of
the ``De Servo Arbitrio'' as the product of a morbid fram of mind.\footnote
{\textit{Ibid.}, p. 284.}

As a matter of fact, the morbid state of Luther's soul repeatedly
breaks forth in this book. He realizes that the predestination of the
damned is an inference from his denial of free-will. He states that
he often took grave offense at it and ``arrived at the verge of despair,''
so much so that he ``wished he had never been born.'' But a marvelous
change had come over his ideas. He recognized ``how salutary and
how near to grace this despondency was''; for whoever shares the
conviction that all things are dependent upon the will of God, choses
nothing for himself in despairing of himself, but only expects God
to act. He is next to salvation, although he be dead and strangled in
consequence of his consciousness of guilt, and spiritually immersed in
hell. Such a one is succored by the belief that the merits of Christ
cover his sins, the \textit{sola fides}, \textit{i.e.}, the conviction that man is justified
by faith alone. ``This,'' he says, ``is familiar to everyone who has read
our works.''\footnote{\textit{Ibid.}, p. 279. }

This doctrine of determinism, like his whole system, grew out
of personal motives and was patterned after his own abnormal mental
states.

In his acrobatic exposition he even goes so far as to idolize the consolation
which he derives from his denial of free-will: ``Without this doctrine I
believe I would be constantly tortured by uncertainty and compelled to
expunge all my work. My conscience would never enjoy certain ease\dots .
If free-will were offered to me, I would not accept it at all. I would not
want anything to be placed within my power, so as to give a practical proof
of my salvation, because I would nevertheless fear that I could not withstand
the spiritual dangers and the attacks of so many devils.”\footnote{\textit{Ibid.}, pp. 268 sq.}

He arbitrarily conceals from himself predestination to hell with its horrors,
but firmly insists upon the monstrosity of the absolute predestination to
eternal punishment of human beings who could not act otherwise than they
did. He suggests that we simply should not think of it! He has recourse to a
mysterious \textit{hidden} God, who, in His unlimited majesty, may have other
norms that our human sense of justice can devise. The essence of God is truly
inscrutable. The statement in the Apocalypse that God wills the salvation of
all men, applies to the Deus revelatus in the Gospel of Christ; but there
also exists a hidden God, a \textit{Deus absconditus}, whose decrees may be quite
different.

Relative to these doctrines, the Protestant theologian Kattenbusch, whom
we have already quoted, says: “Luther expressly advances it as a theory that
God has two contradictory wills, the secret will of which no one knows anything,
and another which He causes to be proclaimed; \dots in other words,
that He is free to lie.”\footnote
{Kattenbusch, “\textit{Deus absconditus bei Luther},” in the \textit{Kaftanfestschrift}, pp. 170 sqq.
Grisar, \textit{Luther}, Vol. II, p. 169, note 1.--Isaias (45, 1 5) praises the \textit{Deus absconditus}, but as
God of mercy who wills to save all men. Thus in verse 19, according to Luther’s own
translation: “I have not said in vain to the seed of Jacob: Seek me.” Cf. verses 22 and 24.
--R. Otto (Das Heilige, 7th ed., Breslau, 1922) says (p. 118): Luther flees from the \textit{Deus
absconditus} “like a badger into the fissures of a rock,” and (p. 120), owing to his personal
states of fear he reduces the whole of Christianity to fiduciary faith. According to Scheel,
Luther with his \textit{iustitia passiva}, introduces a “completely new theory of God.” (Article,
“Iustitia Passiva” in the Briegerfestschrift. Cf. Grisar in the \textit{Zeitschrift für katholische
Theologie}, Vol. XLII, 1918, p. 599.}
No less frank are the words of another Protestant
theologian, A. Taube: From Luther’s statements we must “conclude that God,
as He is preached [in Sacred Scripture], is not in every instance the same
God as He who actually works, and that in some cases in His revelation He
says what is quite untrue.”\footnote{Grisar, \textit{Luther}, Vol. II, p. 269. Cfr. \textit{ibid.}, p. 263.}
It cannot be denied that Luther, led astray by
Ockham’s theory of an arbitrary God, introduced a new concept of God,
which, however, is forthwith disproved by the inference just described.

Now, while he upholds, by means of his \textit{Deus absconditus}, the absolute
predestination to hell of every man as a possibility, and while he represents it
as an actuality in the case of such as are already damned, he does not wish
this subject to be made a topic of reflection and discussion. It is a point which
he emphasized innumerable times in his books and letters. As a means
of preventing despair he recommends, in an almost importunate manner, that no
thought be given to predestination; God, the Incomprehensible, must be
adored in silent submission. In his practical work, on the other hand, he frequently writes as though man’s salvation lay solely within his own power, by
co-operating with divine grace. Thus involuntarily he returns to the Catholic
doctrine.

There is no fundamental distinction in the dismal doctrine of
predestination as taught by Luther, Calvin, and Zwingli, except
that the latter two, particularly Calvin, are more systematic in their
exposition of it. Köstlin, the biographer of Luther, is constrained
to concede this when he says: “In the resoluteness with which Luther
accepts the most rigorous consequences of the doctrine of predestination
he is essentially one with Zwingli and Calvin, the other leaders
of the Reformation.”\footnote{Köstlin-Kawerau, M. Luther, Vol. 1, p. 664.}

Luther appeals to the authority of St. Augustine, that famous
Doctor of the Church, in confirmation of his doctrine. But he
woefully distorts Augustine’s utterances and merely asserts without
proof: “He is on my side.”\footnote{\textit{Ibid.}}

Luther never abandoned his position relative to determinism and
predestination, though he modified his expressions. He characterized
his book “De Servo Arbitrio,” while still in its formative stage, as
a “thunderbolt” against the Erasmic and popish heresy of free-will,\footnote
{Grisar, \textit{Luther}, Vol. II, p. 284.}
and always regarded it as a work which his opponents “shall
not be able to refute in all eternity.”\footnote{\textit{Ibid.}, p. 291.}
“I do not recognize any
of my writings as genuine,'' he writes as late as 1537 to Capito,
``except those on the Enslaved Will and the Catechism.'' He says
he would not shed any tears if the others should be lost.\footnote{\textit{Ibid.}, p. 292.}

It is incomprehensible that some Protestant theologians extol the
deeply religious spirit which is supposed to prevail in the “De Servo
Arbitrio.”\footnote
{\textit{Ibid.}, pp. 292 sq.; cfr. Vol. VI, pp. 452 sq.}
They admire its profound humility in the presence
of God’s omnipotence and the self-annihilation that pulsates
throughout the book. But they do not reflect that the motto of the
unfortunate treatise is not true humility, but the suicide of human
nature. In his preface to the new critical edition, the Weimar editor
styles the “De Servo Arbitrio” “the most splendid Latin and perhaps
the most splendid polemical work of Luther,”\footnote
{\textit{Ibid.}, p. 284. Shortly after its appearance, the work was translated into German by
Jonas under the title, “\textit{Dass der freie Wille nichts sei}.” Recently a new edition of
this translation was published by Gogarten, with an introduction which strongly assails
the appreciation of Luther as a hero of civilization. Albert Ritschl styled the treatise ``\textit{De Servo
Arbitrio}'' “an unfortunate piece of bungling” (Joh. v. Walter, \textit{Das Wesen der Religion
nach Erasmus und Luther}, 1906, p. 124). In 1559, Melanchthon, referring to the fantastic
ideas of Luther contained in this work, speaks of them as “\textit{stoica et manichaea
deliria}.” Cfr. Grisar, \textit{Luther}, Vol. VI, p. 153.}

but adds: “It must
not be concealed that the whole conception has a strongly pantheistic
and mechanistic appearance.”\footnote{\textit{Ibid.}, p. 284.}

Luther’s attitude towards the Commandments of God also aroused
strong opposition. If man is not free to observe the Commandments,
why should there be any at all, and why should punishment be
threatened for those who despise them? In consequence of this and
other writings of Luther, many placed themselves beyond the Commandments.
“Let us do as we please.”\footnote{\textit{Ibid.}, p. 288.}
Luther strongly opposes
this tendency. But his defense of the Commandments consists in
this: God gives His commandments with the wise intention of teaching us
how little we can do of our own accord. The law and its
threats should arouse within us a sense of our incompetence, enkindle
a desire for redemption by grace, and thus lead us to salvation through self-annihilation.

The assertion of God’s relation to sin was equally unintelligible
to many readers of Luther’s treatise.

If man lacks free-will, who is it that causes sin? Luther feels that it will
not do to hold God directly responsible for sin. He does not assert that there
is an immediate impulse to evil originating with God. But, quite consistently
with his system, he speaks of the treachery of Judas thus: “His [Judas’]
will was the work of God; God by His almighty power moved his will as He
does all that is in this world.”\footnote{\textit{Ibid.}, p. 282.}
He holds that Adam, at least in spirit, was
abandoned by God in Paradise and placed in a situation in which he could
not but fall.

“He is God,” says Luther, “and therefore there is no reason or cause of
His willing,” because no creature is above Him, and He Himself is “the
rule of all things.” Whatever He does in His arbitrariness is good, “not
because He must or ought to will thus.” “His [man’s] will must have reason
and cause, not so, however, the will of the Creator.”\footnote{\textit{Ibid.}, pp. 282 sq.}
These are Ockhamistic
subtleties and aberrations.

Relative to the Fall of Adam the essential point is that his sin, as the
Protestant Kattenbusch puts it, “is caused by God,” whereas “fundamentally
nothing is gained” by the other reflections of Luther.”\footnote{\textit{Ibid.}, p. 283.}
And that is the
sin of our first parents, through which, according to Luther, the whole
human race was plunged into original sin, a misfortune which--again following
Luther--radically tainted the entire race.

Since Luther held such views of God and sin already at an
earlier period of his career, it is no wonder that a controversy arose
at Erfurt among the preachers of the new religion, which could
not be terminated by the treatise “On the Enslaved Will.” In
reference to this controversy Luther’s friend Lang, the leader of
the Reformation at Erfurt, wrote to him for information. “I perceive,”
Luther replied, “how indolent you are whilst Satan is on
the offensive everywhere.” “Why do you quarrel among yourselves
about the evil which God does \dots we do evil because God ceases
to work in us,” etc.\footnote{April 12, 1522 (\textit{Briefwechsel}, III, p. 331.)}
This advice did not restore peace at Erfurt, since
the preachers there were a quarrelsome lot. Luther refused to send
them an official letter of instruction as he had been requested to do by
Lang, but declared: “Let them practice faith and love; everything
else is well known.”

Erfurt, the city which had harbored the whilom peaceful cells
of Luther and his fellow-monks, was on the verge of a profound
agitation.
