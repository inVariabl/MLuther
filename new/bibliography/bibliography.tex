\textit{Analecta Lutherana}. Letters and documents pertaining to the history of
Luther. Containing at the same time a supplement to his hitherto collected
correspondence. Edited by Th. Kolde. Gotha, 1883.

\textit{Analecta Lutherana et Malanchthoniana}; cf. Mathesius, \textit{Aufzeichnungen}.

\textit{Archiv für Reformationsgeschichte. Texte und Untersuchungen}. Edited by
W. Friedensburg. Berlin (afterwards Leipsic), 1903--1904 sqq.

\textsc{Balan P.} \textit{Monumenta Reformationis Lutheranae ex Tabulariis S. Sedis
Secretis. 1521--1525}. Ratisbonae, 1884.

\textsc{Baranowski S.} \textit{Luthers Lehre von der Ehe}. Münster, 1913.

\textsc{Barge H.} \textit{Andreas Bodenstein von Karlstadt.} 2 vols., Leipsic, 1905.

\textsc{Below G.} v. \textit{Die Ursachen der Reformation. (Histor. Bibliothek.}, Vol. 38),
3rd ed., Berlin, 1917. (Cfr. Theol. Revue, 1919, p. 49.)
---\textit{Die Bedeutung der Reformation in der politischen Entwicklung}. Leipsic,
1918. (Cfr. Theol. Revue, 1919, p. 112.)

\textsc{Berger A. E.} \textit{Martin Luther in kulturgeschichtlicher Darstellung.} (Collection
“\textit{Geisteshelden.}”) 3 Parts. Berlin, 1895--1921.

\textsc{Bezold F.} v. \textit{Geschichte der deutschen Reformation.} Berlin, 1890.

\textsc{Bichler Fr.} \textit{Luther in Vergangenheit und Gegenwart.} Ratisbon, 1918. (Cfr.
Theol. Revue, 1921, p. 59; Hist.-polit. Blätter, 163 [1919], p. 66.)

\textsc{Biereye Joh.} \textit{Die Erfurter Lutherstätten.} With 13 Plates. Erfurt, 1917.

\textsc{Böhmer H.} \textit{Luther im Lichte der neueren Forschung. (Aus Natur und
Geisteswelt}, No. 113.) Leipsic, 1906; 5th ed., 1918. \\
---\textit{Luthers Romfahrt.} Leipsic, 1914. \\
---\textit{Der junge Luther.} Gotha, 1925. 363 pages with 39 Portraits.

\textsc{Brandenburg E.} \textit{Luthers Anschauung von Staat und Gesellschaft. Schriften
des Vereins für Reformationsgeschichte}, No. 70, Halle, 1901.

\textsc{Braun W.} \textit{Die Bedeutung der Concupiscenz in Luthers Leben und Lehre.}
Berlin, 1908. \\
---\textit{Biographisches und theologisches Verständnis der Entwicklung Luthers.}
Berlin, 1917.

\textsc{Briefwechsel}; see \textsc{Correspondence.},

\textsc{Brieger Th.} \textit{Aleander und Luther. Die vervollständigten Aleander-Depeschen
nebst Untersuchungen über den Wormser Reichstag}. Part I. Gotha, 1884. \\
---\textit{Die Reformation}. Berlin, 1914. \\
---\textit{Martin Luther und wir. Das reformatorische Christentum Luthers}. 2nd
ed., Gotha, 1918.

\textsc{Buchwald Georg.} \textit{Dr. M. Luther}. 3rd, revised ed. Leipsig, 1917.

\textsc{Burkhardt C. A.} \textit{Geschichte der sächsischen Kirchen- und Schulvisitationen
von 1524 bis 1545}. Leipsic, 1879.

\textsc{Calvin John.} \textit{Opera quae Supersunt Omnia}. Edited by G. Baum, E. Cunitz,
	E. Reuss. 59 vols. (29--87 in Corpus Reformatorum.) Brunsvigae, 1863--1900.

\textsc{Cochlaeus Joun.} \textit{Commentaria de Actis et Scriptis M. Lutheri \dots ab a.
1517 usque ad a. 1536 conscripta}. Moguntiae, 1549.

\textsc{Cohrs Ferd.} \textit{Phil. Melanchthon. (Schriften des Vereins für Reformationsgeschichte},
No. 55.) Halle, 1897.

(\textsc{Collogquia}, ed. Bindseil), \textsc{Bindseil H. E.}, \textit{D. Martini Lutheri Colloquia,
Meditationes, Consolationes, Iudicia, Sententiae, Narrationes, Responsa,
Facetiae e Codice ms. Bibliothecae Orphanotrophei Halensis cum perpetua
collatione editionis Rebenstockianae Edita et Prolegomenis Indicibusque
Instructa}. 3 vols. Lemgoviae et Detmoldae, 1863--1866.

(\textit{Commentarius in Epist. ad Galat.) M. Lutheri Commentarius in Epistolam
ad Galatas.} Ed. \textsc{I. A. Irmischer}, 3 vols. Erlangae, 1843 sq.

(\textsc{Cordatus}, \textit{Tagebuch}) \textsc{Wrampelmeyer H.}, \textit{Tagebuch über Dr. Martin
Luther, geführt von Dr. Konrad Cordatus}, 1537. First edited in Halle, 1885.

\textit{Corpus Catholicorum}. Works of Catholic Authors in the Age of the Reformation;
founded by \textsc{J. Greving.} Vol. I (containing plan of the undertaking).
Münster, 1919; Vol. II to VII (containing Contarini’s Documents
of the Counter-Reformation, edited by \textsc{Fried. Hünermann}), Münster
1919 to 1923.

\textit{Corpus Reformatorum}. \textsc{Ed. Bretschneider} et \textsc{Bindseil}. Halis Saxoniae, 1834
sqq. Vols. I-XXVII: \textit{Melanchthonis Opera}; Vols. XXIX-LXXXVII: \textit{Calvini
Opera}; Vols. LXXXVII-XCVI: \textit{Zwingli Opera. Supplementa Melanchthoniana.}
1 vol.: The Dogmatic Writings, edited by \textsc{O. Clemen.}
Leipsic, 1909.

(\textit{Correspondence.}) \textit{M. Luther’s Briefe, Sendschreiben und Bedenken vollständig
gesammelt von M. de Wette.} Sect. I-V. Berlin, 1825--1828; Sect.,
VI, ed. by \textsc{J. K. Seidemann}, Berlin, 1856.

(\textit{Correspondence.}) \textit{Dr. Martin Luthers Briefwechsel. Bearbeitet und mit Erläuterungen
versehen von L. Enders}. Vols. I-XI. Frankfort on the Main,
thereafter Stuttgart, 1884--1907; Vols. XII-XVI: revised by \textsc{G. Kawerau,}
Leipsic, 1910--1915; Vol. XVII, revised by \textsc{G. Kawerau} and \textsc{P. Flemming}, \textit{ibid.}, 1920; Vol. XVIII, revised by \textsc{P. Flemming} and \textsc{O. Albrecht}
\textit{ibid.}, 1923.

\textit{(Correspondence.)} \textit{Luthers Briefwechsel, mit vielen unbekannten Briefen und
unter Berücksichtigung der De Wetteschen Ausgabe}, ed. by \textsc{C. A. Burkhardt}.
Leipsic, 1866. \\
---\textit{Briefwechsel des Beatus Rhenanus, gesammelt und herausgegeben von A.
Horawitz und K. Hartfelder}. Leipsic, 1886. \\
---\textit{Briefwechsel der Brüder Ambrosius und Thomas Blaurer}, 1509--1548, ed.
by \textsc{Tr. Schies.} Vol. I. Freiburg i. B., 1908. \\
---\textit{Briefwechsel des Justus Jomas, gesammelt und bearbeitet von G. Kawerau.}
2 vols. Halle, 1884. \\
---\textit{Briefwechsel Landgrafs Philipps des Grossmütigen von Hessen mit
Bucer}, ed. by \textsc{M. Lenz.} (\textit{Publikationen aus dem Kgl. Preuss. Staastsarchiv.})
3 vols. Leipsic, 1880--1891. \\
---\textit{Akten und Briefe zur Kirchenpolitik Herzog Georgs von Sachsen}, ed.
by \textsc{F. Gess.} Vol. I-II, Leipsic, 1905--1917. \\
---\textit{Erasmi Opus Epistolarum}, ed. \textsc{P. S. Allen.} Vol. IV: 1519--1521. London
(Oxford), 1922. \\

\textsc{Creutzberg}, H. A. \textit{Karl von Miltitz. Sein Leben und seine geschichtliche
Bedeutung. (Studien und Dartellungen aus dem Gebiete der Geschichte,}
edited under the auspices of the Goerres-Gesellschaft, Vol. VI, no. 1).
Freiburg i. Br., 1907.

\textsc{Cristiani, L.} \textit{Du Luthéranisme au Protestantisme. Thèse de Clermont}. Paris,
1911.

\textsc{Denifle, H.} (O.P.) \textit{Luther und Luthertum in der ersten Entwicklung,
quellenmässig dargestellt}, Vol. I, 1st ed., Mayence, 1904; 2nd ed., Part I,
\textit{ibid.}, 1904; Part II, supplemented and edited by \textsc{A. W. Weiss, O. P.},
\textit{ibid.}, 1906.\\
---\textit{Quellenbelege zu Bd. I, 2te Aufl., Abt. 1--2: Die abendländische
Schriftauslegung bis Luther über Justitia Dei (Rom. 1, 17) und
Justificatio. Beitrag zur Geschichte der Exegese, der Literatur und des Dogmas
im Mittelalter. Ibid.}, 1905. Vol. II of the main work, edited by \textsc{A. M.
Weiss, O. P.}, \textit{ibid.}, 1909.\\
---\textit{Luther in rationalistischer und christlicher Beleuchtung. Prinzipielle
Auseinandersetzung mit A. Harnack und R. Seeberg}. Mayence, 1904.\\
--Works provoked by Denifle; cf. Wolf, Quellenkunde, Vol. 11, Sect. 1, p.
230, note 4.\\

\textit{Deutsch-evangelische Blätter. Zeitschrift für den gesamten Bereich des deutschen
Protestantismus.} Halle, 1891 sqq.

\textsc{Dilthey W.} \textit{Glaubenslehre der Reformatoren. (Preussische Jahrbücher}, Vol.
75). \\
---\textit{Gesammelte Schriften}. Vols. 1, 2, 4. Leipsic, 1914 sqq.

(\textit{Disputations.}) \textsc{Drews P.}, \textit{Disputationen Dr. Martin Luthers, in den
Jahren 1535--1545 an der Universität Wittenberg gehalten. Zum ersten
Mal hg. Göttingen}, 1895.

(\textit{Disputations.}) \textsc{Stange C.}, \textit{Die ältesten ethischen Disputationen Dr. Martin
Luthers. (Quellenschriften zur Geschichte des Protestantismus}, Vol. I).
Leipsic, 1904.

\textsc{Döllinger} J. v. \textit{Luther. Eine Skizze.} Freiburg i. Br., 1890. (Also in Wetzer
and Welte’s \textit{Kirchenlexikon}, 1st and 2nd ed., article, “Luther.”) \\
---\textit{Die Reformation, ihre innere Entwicklung und ihre Wirkungen im
Umfange des lutherischen Bekenntnisses}. 3 vols. Ratisbon, 1846--1848 (Vol.
I, 2nd ed., 1851).

\textsc{Doumergue E.} \textit{Jean Calvin.} Vol. I-V. Lausanne, 1899--1917.

\textsc{Eck Joh}. \textit{Opera contra Ludderum} Pars I-V. Ingolstadii, 1530--1535. \\
---\textit{Epistola de Ratione Studiorum suorum 1538}, ed. \textsc{I. Metzler} in \textit{Corpus
Cath.}, no. 2; cfr. \textit{Theol. Revue}, 1922, p. 147. \\
---\textit{Pfarrbuch U. L. Frau in Ingolstadt}, ed. by \textsc{J. Greving.} Leipsic, 1908.

\textsc{Eckart R.} \textit{Luther und die Reformation in Urteilen bedeutender Männer.}
2nd ed. Halle, 1917.

\textsc{Ehses} St. \textit{Geschichte der Packschen Händel. Ein Beitrag zur Geschichte der
deutschen Reformation.} Freiburg 1. Br., 1881.

\textsc{Ellinger G.} \textit{Philipp Melanchthon. Ein Lebensbild.} Berlin, 1902.

\textsc{Erasmus D.} \textit{Opera Omnia Emendatiora et Auctiora}. Ed. \textsc{Clericus}. 10
vols. Lugd. Batavorum, 1702--1706.

\textit{Erläuterungen und Ergänzungen zu Janssens Geschichte des deutschen Volkes},
ed. by L. v. \textsc{Pastor.} Freiburg i. Br., 1898--1925.

\textsc{Evers G.} \textit{Martin Luther. Lebens- und Charakterbild, von ihm selbst gezeichnet
in seinen eigenen Schriften und Korrespondenzen.} 6 vols. Mayence,
1883--1891.

\textsc{Falk F.} \textit{Die Bibel am Ausgang des Mittelalters}. Mayence, 1905. \\
---\textit{Die Ehe am Ausgang des Mittelalters. (Erläuterungen und Ergänzungen
zu Janssens Geschichte des deutschen Volkes}, Vol. VI, no. 4).
Freiburg i. Br., 1908.

\textit{Flugschriften aus den ersten Jahren der Reformation}, ed. by \textsc{O. Clemen.}
Leipsic and New York, 1907 sqq.

\textit{Flugschriften aus der Reformationszeit in Faksimiledrucken}, ed. by \textsc{O.
Clemen.} No. 1--4. Leipsic, 1921. See \textit{Zeitschrift für kath. Theologie}, Vol.
46 (1922), pp. 137--140.

\textsc{Förstemann C. E.} \textit{Neues Urkundenbuch zur Geschichte der evangelischen
Kirchenreform}. 1 vol. only. Hamburg, 1842.

\textsc{Friedensburg A.} \textit{Geschichte der Universität Wittenberg.} Halle, 1917.

\textsc{Göller E.} \textit{Der Ausbruch der Reformation und dis spätmittelalterliche
Ablasspraxis.} Freiburg i. Br., 1917.

\textsc{Gottschick J.} \textit{Luthers Lehre.} Tübingen, 1914.

\textsc{Grisar H.} (S. J.) \textit{Luther.} Vol I: \textit{Luthers Werden. Grundlegung der Spaltung
bis 1530.} 3rd ed. with supplementary notes, Freiburg i. Br., 1924. Vol II:
\textit{Auf der Höhe des Lebens,} 3rd ed., with supplementary notes, \textit{ibid.}, 1924.
Vol. III: \textit{Am Ende der Bahn. Rückblicke.} 3rd ed., with supplementary
notes, \textit{ibid.}, 1925 \\
---(The \textit{Nachträge} to the 3rd ed. of this work can be
had separately.) \\
---\textit{Luther.} Authorized translation from the German by E. M. Lamond,
edited by \textsc{Luigi Cappadelta}. London and St. Louis, 1913--1917. 6 vols. \\
---\textit{Luther-Studien. Luthers Kampfbilder. In Verbindung mit Franz Heege
S. J. hg.} \\
Heft 1: \textit{Passional Christi und Antichristi. Eröffnung des Bilderkampfes.}
Heft 2: \textit{Der Bilderkampf in der deutschen Bibel.}
Heft 3: \textit{Der Bilderkampf in den Schriften von 1523 bis 1545.}
Heft 4: \textit{Die “Abbildung des Papsttums” und andere Kampfbilder in Flugblättern 1538--1545. \\
---Luther zu Worms und die jüngsten drei Jahrhundertfeste der Reformation. \\
---Luthers Trutzlied “Ein’ feste Burg” in Vergangenheit und Gegenwart.}
Freiburg i. Br., 1921 sqq. \\
---\textit{Der deutsche Luther im Weltkrieg und in der Gegenwarst}. Augsburg,
1924. \\
--Articles on Luther and Melanchthon in relation to education, in the
\textit{Lexikon der Pädagogik} by Roloff. Freiburg i. Br., 1921. \\
---\textit{Luther-Analekten}, I-VII. (\textit{Historisches Jahrbuch der Görres-Gesellschaft},
Vol. XXXIX-XLII. Munich, 1919--1922. \\
---\textit{Prinzipienfragen moderner Lutherforschung. (Sonderabdruck aus den
Stimmen aus Maria-Laach}, Bd. 83, Heft 10). Freiburg i. Br., 1912. \\
---\textit{W. Köhler über Luther und die Lüge. (Sonderabdruck aus dem Historischen
Jahrbuch der Görres-Gesellschaft}, Bd. 23, Heft 1). Munich,
1913. \\
---\textit{Literatur des Lutherjubiläums von 1917, ein Bild des heutigen Protestanismus.
(Sonderabdruck aus der Zeitschrift für katholische Theologie,}
Bd. 42 [1918], S. 391, 628, 785--814.) Innsbruck, 1918. \\
---\textit{Einige Bemerkungen zur protestantischen Kritik meines Lutherwerkes.}
(\textit{Theol. Revue, 1919, pp. 1 sqq.}) \\

\textsc{Grützmacher R. H.}; see \textit{Reformationsschriften.}

\textsc{Haller Joh.} \textit{Die Ursachen der Reformation.} Tübingen, 1917.

\textsc{Hansen H.} \textit{Stimuli et Clavi, Spiesse und Nägel (zur Reformationsfeier} 1917).
Altona, 1917. (Cfr. Grisar, \textit{Literatur des Lutherjubiläums}, pp. 596--598).

\textsc{Harnack A.} \textit{Lehrbuch der Dogmengeschichte}. Vol. III: \textit{Die Entwicklung
des kirchlichen Dogmas}. II and III, 4th revised ed., Tübingen 1910. \\
---\textit{Luther und die Grundlegung der Reformation}. Berlin, 1917. \\
---\textit{Die Reformation}. (\textit{Internationale Monatschrift für Wissenschaft} usw.,
1917, Heft 11, pp. 1281--1364.) Reprinted in \textit{Reden und Aufsätze}, Vol.
VI, Giessen, 1923, pp. 71--136. \\

\textsc{Hartfelder K.} \textit{Phil. Melanchthon als Praeceptor Germaniae. (Mon. Germ.
Paedagogica}, tom. VII.) Berlin, 1889.

\textsc{Hauck A.} \textit{Die Reformation in ihren Wirkungen auf das Leben.} Leipsic,
1918. Cfr. \textit{Hist.-politische Blätter}, Vol. 163 (1919), pp. 34--42.

\textsc{Hausrath A.} \textit{Luthers Leben.} 2 vols. Berlin, 1904. (2nd and 3rd imprint
with supplementary notes by H. v. Schubert.)

\textsc{Heege F.} See \textsc{Grisar}, \textit{Lutherstudien.}

\textsc{Hermelink H.} \textit{Reformation und Gegenreformation in Handbuch der
Kirchengeschichte} by G. Krüger, Vol. III.) Tübingen, 1911.

\textsc{Holl K.} \textit{Gesammelte Aufsätze zur Kirchengeschichte.} Vol. I: \textit{Luther}. 2nd
and 3rd ed., Tübingen, 1923.

\textsc{Humbertclaude H.} \textit{Erasme et Luther}. Fribourg i. Sw., 1909.

\textsc{Hunzinger A. W.} \textit{Luther-Studien.} Heft 1--2. Leipzig, 1905. Cfr. Wolf,
\textit{Quellenkunde}, Vol. II, Sect. 1, p. 236.

\textsc{Hutten Ulr.} \textit{Opera.} 5 vols., ed. \textsc{E. Böcking.} Lipsiae, 1859--1862.

\textsc{Imbart de la Tour P.} \textit{Les Origines de la Réforme.} 3 Vols. Paris, 1905--1914.

(\textsc{Janssen-Pastor.}) \textsc{Janssen J.}, \textit{Geschichte des deutschen Volkes seit dem
Ausgang des Mittelalters}. 19th and 20th ed. by \textsc{L. von Pastor}. Vol. I: \textit{Die
allgemeinen Zustände des deutschen Volkes beim Ausgang des Mittelalters,}
Freiburg i. Br., 1913; Vol. II: \textit{Vom Beginn der politisch-kirchlichen Revolution
bis zum Ausgang der sozialen Revolution von 1525, ibid.}, 1915;
Vol. III: \textit{Bis zum sogenannten Augsburger Religionsfrieden von 1555, ibid.},
1917. Cf. also \textit{Erläuterungen und Ergänzungen.} \\
---\textit{An meine Kritiker. Nebst Ergänzungen und Erläuterungen zu den
drei ersten Bänden meiner Geschichte des deutschen Volkes}. Freiburg i.
Br., 1882. \\
---\textit{Ein zweites Wort an meine Kritiker. Nebst Ergänzungen und Erläuterungen
	zu den drei ersten Bänden meiner Geschichte des deutschen
Volkes.} Freiburg i. Br., 1883. \\

\textsc{Jordan} Herm. \textit{Luthers Staatsauffassung.} Munich, 1917.

\textsc{Kalkoff P.} \textit{Forschungen zu Luthers römischem Prozess. (Bibliothek des
Kgl. Preuss. Histor. Instituts in Rom}, Vol. II), Rome, 1905. \\
---\textit{Luther und die Entscheidungsjahre der Reformation. Von den Ablassthesen
bis zum Wormser Edikt.} Leipsic, 1917. \\
---\textit{Der Wormser Reichstag von 1521.} Munich and Berlin, 1922. \\
---\textit{Das Wormser Edikt.} Munich, 1917. \\
---\textit{Erasmus, Luther und Friedrich der Weise.} Leipsic, 1920. \\
---\textit{Ulrich von Hutten und die Reformation.} Leipsic, 1920. \\
---\textit{Zu Luthers römischem Prozess.} Gotha, 1912 (with a list of Kalkoffs
previous writings, pp. v ff.)--Cfr. \textit{G. Krüger}, \textit{Kalkoffs Studien zur
Frühgeschichte der Reformation} in \textit{Theol. Studien und Kritiken,} 1918,
pp. 144 sqq. \\

\textsc{Kampschulte F. W.} \textit{Johannes Calvin, seine Kirche und sein Staat in Genf.}
2 vols. Leipsic, 1869 and 1899. \\
---\textit{Die Universität Erfurt und ihr Verhältniss zum Humanismus und
zur Reformation}. 2 vols. Treves, 1858 and 1860. \\

\textsc{Kaulfuss-Diesch} Karl. \textit{Das Buch der Reformation, geschrieben von Mitlebenden.}
Leipsic, 1917.

\textsc{Kaser Kurt.} \textit{Reformation und Gegenreformation} (in \textit{Weltgeschichte,} ed.
by Ludo \textsc{M. Hartmann}, Vol. VI, Sect. 1.) Stuttgart, 1922.

\textsc{Kawerau Gust.} \textit{Reformation und Gegenreformation.} 3rd ed., Freiburg. 1907
(in \textsc{Möllers} \textit{Lehrbuch der Kirchengeschichte}, Vol. III.) \\
---\textit{Die Versuche, Melanchthon zur katholischen Kirche zurückzuführen.}
(\textit{Schriften des Vereins für Reformationsgeschichte}, Heft 73.) Leipsic,
1903. \\
---\textit{Luthers Schriften nach der Reihenfolge der Jahre verzeichnet. (Zweite
Publikation des Vereins für Reformationsgeschichte.}) Leipsic, 1917. \\
---\textit{Glossen zu Grisars “Luther” (Schriften des Vereins für Reformationsgechichte,}
Heft 105.) \textit{Leipsic}, 1911. Cfr. \textsc{Grisar}, \textit{Luther}, (Engl. tr.), Vol.
VI. and \textit{Stimmen aus Maria-Laach}, 1913, Heft 1, pp. 9 sqq. \\
---\textit{Luthers Randglosses zum Marienpsalter des Markus von Weida (Theol.
Studien und Kritiken,} 1917. pp. 81 ff.) \\

\textsc{Kiefl F. X.} \textit{Katholische Weltanschauung und modernes Denken.} 2nd and
3rd ed., Ratisbon, 1922.

\textit{Kirchenordnungen, Die evangelischen des 16. Jahrhunderts}, ed. by \textsc{E. Sehling.}
I: \textit{Die Ordnungen für die ernestinischen und albertinischen Gebiete}, Leipsic, 1902;
II: \textit{Die vier geistlichen Gebiete} usw., \textit{ibid.}, 1904;
III: \textit{Die Mark Brandenburg} usw., \textit{ibid.}, 1909.

\textsc{Kirn P.} \textit{Friedrich der Weise und die Kirche. Seine Kirchenpolitik vor und
nach Luthers Auftreten im Jahre 1517}. Leipsic, 1926.

\textsc{Klingner Erich.} \textit{Luther und der deutsche Volksaberglaube.} Berlin, 1912.

\textsc{Köhler Walter}. \textit{Katholizismus und Reformation. Kritisches Referat über
die wissenschaftlichen Leistungen der neueren katholischen Theologie auf
dem Gebiete der Reformationsgeschichte.} Giessen, 1905. \\
---\textit{Luther und die Kirchengeschichte.} Vol. 1, Sect. 1. Erlangen, 1900. \\
---\textit{Luther und die deutsche Reformation.} Leipsic, 1916. \\
---\textit{Luther der deutsche Reformator.} 2nd ed., Contance, 1917. \\
---\textit{Luther und die Lüge.} Leipsic, 1912. (\textit{Schriften des Vereins für Reformationsgeschichte.}) \\
---\textit{Dokumente zum Ablaszstreit von 1517.} Tübingen, 1902. \\
---\textit{Zwingli und Luther. Ihr Streit über das Abendmahl}. Vol. 1. Leipsic,
1924. \\

\textsc{Köstlin Jul.} \textit{Luthers Leben.} 9th ed., Leipsic, 1891. \\
---\textit{Luthers Theologie in ihrer geschichtlichen Entwicklung und in ihrem
Zusammenhang dargestellt}. 2 vols., 2nd ed. completely recast. Stuttgart,
1901. \\

(\textsc{Köstlin-Kawerau}). \textsc{Köstlin J.}, \textit{Martin Luther. Sein Leben und seine
Schriften.} 2 vols., 5th revised ed., continued by \textsc{G. Kawerau}. Berlin,
1903.

\textsc{Kolde Th.} Cfr. \textit{Analecta Lutherana.} \\
---\textit{Die deutsche Augustinerkongregation und Johann v. Staupitz. Ein
Beitrag zur Ordens- und Reformationsgeschichte nach meistens ungedruckten
Quellen.} Gotha, 1879. \\
---\textit{Martin Luther, Eine Biographie}. 2 vols. Gotha, 1884--1893. \\

\textsc{Kroker Ernst.} \textit{Katharina von Bora}. 2nd ed., Leipsic, 1925.

\textsc{Kurrelmeyer W.} \textit{Die erste deutsche Bibel.} Vols. I-X. Tübingen, 1903--1915.

\textsc{Lauchert Friedr.}, \textit{Die italienischen literarischen Gegner Luthers. (Erläuterungen
und Ergänzungen zu Janssens Geschichte}, Vol. VIII.) Freiburg i.
Br., 1912.

(\textsc{Lauterbach’s} \textit{Tagebuch}). \textsc{Seidemann J. K.}, \textit{A. Lauterbachs Tagebuch
auf das Jahr 1538. Die Hauptquelle der Tischreden Luthers}. Dresden, 1872.

\textsc{Lewin Reinh.} \textit{Luthers Stellung zu den Juden}. Berlin, 1911.

\textsc{Loesche G.} Cfr. Mathesius, \textit{Aufzeichnungen, Historien.}

\textsc{Loofs F.} \textit{Leitfaden zum Studium der Dogmengeschichte.} 4th, completely
recast ed., Halle a. d. S., 1906.

\textsc{Löscher V. E.} \textit{Vollständige Reformationsacta und Documenta.} 3 vols.
Leipsic, 1720--1729.

\textsc{Luthardt C. E.} \textit{Die Ethik Luthers in ihren Grundzügen.} 2nd revised ed.,
Leipsic, 1875.

\textsc{Luther Martin.} \\
(1) collected editions of his writings, see \textit{Werke, Opera
Lat. Var., Opera Lat. Exeg., Commentarius in Epist. ad Galatas, Römerbriefkommentar.} \\
(2) Correspondence, see \textit{Correspondence, Analecta.} \\
(3) Table Talk, see \textit{Tischreden} ed. by Aurifaber and Förstemann; also
\textit{Werke}, Weimar ed.; Erl. ed., Vols. LVII-LXII; \textit{Werke}, Halle ed., Vol.
XXII; \textit{Colloquia, Cordatus, Lauterbach, Mathesius, Schlaginhaufen.} \\
(4) Various writings, see \textit{Analecta, Disputations.}

\textsc{Luther-Gesellschaft} (Wittenberg): \textit{Jahrbuch,} 1919 sqq.; \textit{Flugschriften,}
1919 sqq.; \textit{Mitteilungen} (now \textit{Luther}). \textit{Vietrteljahrsschrift.}

\textit{Luther-Studien von den Mitarbeitern der Weimarer Lutherausgabe.} Weimar,
1917.

\textsc{Mathesius J.}, \textit{Aufzeichnungen:} Loesche G., \textit{Analecta Lutherana et Melanchthoniana.
Tischreden Luthers und Aussprüche Melanchthons hauptsächlich
nach den Aufzeichnungen des Johannes Mathesius, aus der Nürnberger
Handschrift im Germanischen Museum mit Benützung von Seidemanns
Vorarbeiten hg. und erläutert.} Gotha, 1892.

\textsc{Mathesius J.} \textit{Historien von des ehrwirdigen in Gott seligen thewren Manns
Gottes Doctoris Martini Luthers Anfang, Lehr, Leben und Sterben.}
Nuremberg, 1566; ed. by \textsc{G. Loesche} in the \textit{Bibliothek deutscher Schriftsteller
aus Böhmen}, Vol. IX, Prague, 1898; 2nd ed., 1906. (Our quotations
are from the Nuremberg edition.)

\textsc{Mathesius J.} \textit{Tischreden}: \textsc{Kroker E.}, \textit{Luthers Tischreden in der Mathesischen
Sammlung. Aus einer Handschrift der Leipziger Stadtbibliothek.} Leipsic,
1903.

\textsc{Maurenbrecher W.} \textit{Studien und Skizzen zur Geschichte der Reformationszeit,}
Leipsic, 1874. \\
---\textit{Geschichte der katholischen Reformation,} Vol. I, Nördlingen, 1880.

\textsc{Melanchthon}, see \textsc{Mathesius}; also \textit{Vita Lutheri}.

\textsc{Melanchthon}, \textit{Opera Omnia.} Ed. \textsc{Bretschneider.} (\textit{Corpus Reformatorum,}
vols. I to XXVIII.) Halis Saxoniae, 1834--1863. \\
---\textit{Supplementa Melanchthoniana. Werke Philipp Melanchthons, die im
Corpus Reformatorum vermisst werden.} Ed. by the Verein für Reformationsgeschichte, Leipzig. \\
---\textsc{Mentz Georg.} \textit{Deutsche Geschichte im Zeitalter der Reformation, 1493--1648.}
Tübingen, 1913.

\textsc{Möhler J. A.} \textit{Symbolik oder Darstellung der dogmatischen Gegensitze der
Katholiken und Protestanten nach ihren öffentlichen Bekenntnisschriften.}
1st ed., Ratisbon, 1832. roth ed., with supplementary notes by \textsc{J. M.
Raich.} Mayence, 1889.

\textsc{Möller W.} \textit{Lehrbuch der Kirchengeschichte.} Vol. III: \textit{Reformation und
Gegenreformation,} by \textsc{G. Kawerau}, 3rd revised and enlarged ed.,
Tübingen, 1907.

\textsc{Müller Alfons Vikt.} \textit{Luthers theologische Quellen.} Giessen, 1912. Cfr.
M. Grabmann in the \textit{Katholik}, Vol. XCIII, pp. 151 sqq. \\
---\textit{Luther und Tauler}. Bonn, 1918. \\
---\textit{Luthers Werden bis zum Turmerlebnis.} Gotha, 1920. See \textit{Theol. Revue,}
1920, pp. 297 sq. \\

\textsc{Müller H. T.}, \textit{Die symbolischen Bücher der evangelisch-lutherischen Kirche,
deutsch und lateinisch. Mit einer neuen historischen Einleitung von Th.
Kolde.} 10th ed., Gütersloh, 1907.

\textsc{Müller Karl.} \textit{Luther und Karlstadt. Stücke aus ihrem gegenseitigen Verhältnis
untersucht.} Tübingen, 1909. \\
---\textit{Kirche, Gemeinde und Obrigkeit nach Luther.} Tübingen, 1910. \\
---\textit{Kirchengeschichte.} Vols. I and II. Tübingen, 1902--1919. \\
---\textit{Luthers Äusserungen über das Recht des bewaffneten Widerstandes
gegen den Kaiser.} Munich, 1915. \\

\textsc{Müller Nik.} \textit{Die Wittenberger Bewegung von 1521 bis 1522.} 2nd ed.,
Leipsic, 1911.

\textsc{Münzer Th.} \textit{Hochverursachte Schutzrede und Antwort wider das geistlose
sanftlebende Fleisch zu Wittenberg}, ed. by \textsc{Enders}, \textit{Neudrucke deutscher
Literaturwerke,} No. 118, Halle, 1893.

\textsc{Murner Thomas, O. S. Fr.} \textit{Von dem Lutherischen Narren,} ed. by \textsc{P. Merker}
(see \textit{Theol. Lit.-Zeitung}, 1919, 224).--\textit{Die Mühle}, ed. by the Elsäss. Verein
(\textit{Murners Schriften}, Vol. IV) 1923.--\textit{Ausgewählte Dichtungen}, ed. by
\textsc{Georg Schumann}, Ratisbon, 1915.

\textsc{Myconius Friedr.} \textit{Reformationsgeschichte}, ed. by \textsc{O. Clemen.} (\textit{Voigtländers
Quellenbücher}, Heft 68), 1915.

\textsc{Neubauer Th.} \textit{Luthers Frühzeit. (Jahrbücher der Akademie zu Erfurt}, Vol.
XLIII), 1917.

\textit{Nuntiaturberichte aus Deutschland nebst ergänzenden Aktenstücken.}
Part I: 1533--1559, ed. by the Kgl. Preuss. Institut in Rom and the Kgl. Preuss.
Archivverwaltung;
Vol. V: \textit{Nuntiaturen Morones und Poggios; Legationen Farneses und Cervinis} 1539--1540, ed. by \textsc{L. Cardauns};
Vol. VI: \textit{Gesandtschaft Campeggios; Nuntiaturen Morones und Poggios 1540--1541}, ed. by
\textsc{L. Cardauns}. Berlin, 1909.

\textit{Opera Latina Exegetica}.--\textit{M. Lutheri Exegetica Opera Latina.} Cura C. Elsperger.
28 vols. Erlangae, 1829 sqq.--Printed separately: \textit{D. M. Lutheri
Commentarius in Epistolam ad Galatas}, ed. \textsc{I. A. Irmischer.} 3 vols. Erlangae,
1843 sq.

\textit{Opera Latina Varia}.--\textit{M. Lutheri Opera Latina Varii Argumenti al Reformationis
Historiam Imprimis Pertinentia.} Cura \textsc{H. Schmidt.} Vols. I-VII.
Francofurti, 1865 sqq. (Forms part of the Erlangen ed. of Luther’s
\textit{Werke.})

\textsc{Oergel G.} \textit{Vom jungen Luther. Beiträge zur Lutherforschung.} Erfurt, 1899.

\textsc{Pastor L.} v. \textit{Geschichte der Päpste seit dem Ausgang des Mittelalters. Mit
Benutzung des Päpstlichen Geheimarchivs und vieler anderer Archive
bearbeitet.} Vols. I and III, 8th and 9th revised ed.; Vol. III, 5th to 7th ed.;
Vol. IV, 8th to 9th ed.; Vol. V, 8th to 9th ed.

\textsc{Pauls Theodor}, \textit{Luthers Auffassung von Staat und Volk. (Bonner Staatswissenschaftliche
Untersuchungen,} Heft 12). Bonn and Leipsic, 1925.

\textsc{Paulsen F.} \textit{Geschichte des gelehrten Unterrichtes auf den deutschen Schulen
und Universitäten vom Ausgang des Mittelalters bis zur Gegenwart. Mit
besonderer Rücksicht auf den klassischen Unterricht.} Leipsic, 1885, 2
vols.; 2nd ed., 1896--1897; 3rd ed., Berlin, 1921.

\textsc{Paulus N.} \textit{Die deutschen Dominikaner im Kampfe gegen Luther, 1518--1563.
(Erläuterungen und Ergänzungen zu Janssens Geschichte des deutschen
Volkes,} Vol. IV, Heft 1 and 2.) Freiburg i. Br., 1903. \\
---\textit{Hexenwahn und Hexenprozess vornehmlich im 16. Jahrhundert.}
Freiburg i. Br., 1910. \\
---\textit{Luther und die Gewissensfreiheit (Glaube und Wissen}, Heft 4),
Munich, 1905. \\
---\textit{Luthers Lebensende. Eine kritische Untersuchung. (Erläuterungen und
Ergänzungen zu Janssens Geschichte des deutschen Volkes,} Vol. I, Heft
1.) Freiburg i. Br., 1898. \\
---\textit{Kaspar Schatzgeyer, ein Vorkämpfer der katholischen Kirche gegen
Luther in Süddeutschland. (Strassburger Theologische Studien}, Vol. III,
Heft 1.) Freiburg i. Br., 1898. \\
---\textit{Johann Tetzel, der Ablassprediger}. Mayence, 1899. \\
---\textit{Bartholomäus Arnoldi von Usingen. (Strassburger Theologische Studien},
Vol. I, Heft 3.) Freiburg i. Br., 1893. \\
---\textit{Zu Luthers Romreise} (in the \textit{Histor. Jahrbuch}, 1901, Vol. XXVI, pp.
79 sqq. and in the \textit{Hist.-polit. Blätter,} 1912, I, pp. 126 sqq.) Cfr. Grisar,
\textit{Luther-Analekten,} I. \\
---\textit{Protestantismus und Toleranz im 16. Jahrhundert.} Freiburg i. Br., 1911. \\
---\textit{Geschichte des Ablasses im Mittelalter. 3 vols.} Paderborn, 1922--1923. \\

\textsc{Pflugk-Harttung J.} v. \textit{Im Morgenrot der Reformation.} Hersfeld, 1912
(\textit{Sammlung von Abhandlungen verschiedener Verfasser und von Abbildungen.})

\textsc{Preuss Hans,} \textit{Unser Luther. Jubiläumsgabe.} Leipsic, 1917.
---\textit{Luthers Frömmigkeit,} Leipsic, 1917. (Cfr. \textit{Zeitschrift für Kath. Theologie,}
Vol. XLIII [1919], pp. 150--155.) \\
---\textit{Lutherbildnisse.} 2nd ed., Leipsic, 1922. \\

\textit{Quellen und Forschungen aus dem Gebiet der Geschichte in Verbindung
mit ihrem historischen Institut zu Rom. bg. von der Görres-Gesellschaft.}
Paderborn, 1892 sqq. \\
---\textit{aus den italienischen Archiven und Bibliotheken, hg. vom Kgl. Preuss.
Histor. Institut in Rom.} Rome, 1897 sqq. \\

\textsc{Oldecop Joh.} ed. by K. Euling. (\textit{Bibliothek des literarischen Vereins von
Stuttgart,} Vol. CXC), Tübingen, 1891.

\textsc{Rade.} \textit{Martin Luther in Worten aus seinen Werken.} Berlin, 1917. \\
---\textit{Luther und die Communio Sanctorum.} Berlin, 1917. (Cfr. \textit{Stimmen der
Zeit,} Vol. XCV, 1918), pp. 501 sq.) \\
---\textit{Das königliche Priestertum der Gläubigen.} Tübingen, 1918. \\
---\textit{Luthers Rechtfertigungsglaube.} Tübingen, 1917. \\

\textsc{Ratzeberger M.} \textit{Handschriftliche Geschichte über Luther und seine Zeit,}
ed. by Ch. \textsc{G. Neudecker.} Jena, 1850.

\textsc{Raynaldi} \textit{Annales Ecclesiastici. Accedunt Notae Chronologicae} etc., \textit{auctore}
\textsc{I. D. Mansi.} Tom. 12--14. Lucae, 1755.

\textit{Reformationsgeschichtliche Studien und Texte,} ed. by \textsc{J. Greving.} Münster i.
W., 1908 sqq.

\textit{Reformationsschriften,} ed. by \textsc{R. G. Grützmacher.} Leipzig, 1917, sqq.

\textit{Reichstagsakten, Deutsche.} First Series, Vol II: \textit{Deutsche Reichstagsakten
unter Karl V.,} ed. by \textsc{Adolf Wrede}. Gotha, 1896.

\textsc{Riffel K.}, \textit{Christliche Kirchengeschichte der neuesten Zeit, vor dem Anfange
der grossen Glaubens- und Kirchenspaltung des 16. Jahrhunderts.}
3 vols. Mayence, 1842--1846.

\textsc{Risch Adolf.} \textit{Luthers Bibelverdeutschung (Schriften des Vereins für Reformationsgeschichte,}
No. 135). Leipsic, 1922.

\textsc{Ritschl A.} \textit{Rechtfertigung und Versöhnung.} 3 vols. 2nd ed. Bonn, 1882-
1883.

\textsc{Ritschl O.} \textit{Dogmengeschichte des Protestantismus.} Vols. I and II. Leipsic,
1908 sq. \\
---\textsc{Luthers religiöses Vermächtnis.} Bonn 1918. (See \textit{Theol. Revue,} 1919, p-
111.)

\textsc{Rockwell W. W.} \textit{Die Doppelehe des Landgrafen Philipp von Hessen.} Marburg,
1904.

\textit{Römerbriefkommentar.} \textsc{Ficker J.}, \textit{Luthers Vorlesung über den Römerbrief
1515--16. 1 Teil: Glossen; 2 Teil: Scholien (Anfänge reformatorischer
Bibelauslegung, hg. von. J. Ficker).} Leipsic, 1908.

\textsc{Rost H.} \textit{Der Protestantismus nach protestantischen Zeugnissen.} Paderborn,
1920.

\textsc{Schäfer E.} \textit{Luther und die Juden.} Gütersloh, 1917. \\
---\textit{Luther als Kirchenbistoriker.} Gütersloh, 1897.

\textsc{Scheel O.} \textit{Luthers Stellung zur Heiligen Schrift. (Sammlung gemeinverständlicher Vorträge und Schriften aus dem Gebiete der Theologie,} No.
29). Tübingen, 1902. \\
---\textit{Martin Luther. Vom Katholizismus zur Reformation.} Vol. I, 3rd ed.
Tübingen, 1921; Vol. II, 1917. See \textit{Theol. Revue,} 1920, pp. 207 sqq; \\
\textit{Zeitschrift für kath. Theologie,} Vol. XLIV (1920), pp. 586--593; \textit{Stimmen
der Zeit,} Vol. CIII (1922), p. 382. \\
---\textit{Dokumente zu Luthers Entwicklung.} Tübingen, 1911.

\textsc{Schlaginhaufen.} \textit{Tischreden Luthers aus den Jahren 1531 und 1532 nach
den Aufzeichnungen von Johann Schlaginhaufen aus einer Münchener
Handschrift hg. von.} \textsc{W. Preger.} Leipsic, 1888,

\textsc{Schreckenbach P.}, and \textsc{Neubert F.} \textit{Martin Luther. Mit 384 Abbildungen
besonders aus alten Quellen} (text by Schreckenbach). 3rd ed., Leipsic,
1921.

\textit{Schriften des Vereins für Reformationsgeschichte.} Halle, 1883 sqq. No. 100:
\textit{Jubiläumsschrift,} Leipsic, 1910. (With contributions by Friedensburg,
Scheel, Bauer, Hermann, Benrat, and Kawerau).

\textsc{Schubart Chr.} \textit{Die Berichte über Luthers Tod und Begräbnis.} Weimar,
1917.

\textsc{Schubert Hans} v. \textit{Luther und seine lieben Deutschen.} Stuttgart, 1917. \\
---\textit{Luthers Frühentwicklung. (Schriften des Vereins für Reformationsgeschichte,} No. 124).

\textsc{Seckendorf V. L. de.} \textit{Commentarius Historicus et Apologeticus de Lutheranismo sive de Reformatione Religionis ductu D. Martini Lutheri \dots
recepta et stabilita.} Lipsiae, 1694

\textsc{Seeberg Reinh.} \textit{Die Lehre Luthers. (Lehrbuch der Dogmengeschichte,} Vol.
IV, Part I, 2nd and 3rd ed. Leipsic, 1917; Part II, 2nd and 3rd ed., 1920.
See \textit{Theol. Revue,} 1919, pp. 241--247 and 1921, pp. 386 sqq.) \\
---\textit{Luthers Anschauung von dem Geschlechtsleben und der Ehe und ihre
geschichtliche Stellung (Jahrbuch der Luthergesellschaft}, Vol. VII, pp.
77--122).

\textsc{Sehling E.} \textit{Geschichte der protestantischen Kirchenverfassung.} 2nd ed.
Berlin, 1914.

\textsc{Spahn M.} \textit{Johann Cochläus. Ein Lebensbild aus der Zeit der Kirchenspaltung.}
Berlin, 1898.

\textsc{Spangenberg Cyriacus.} \textit{Theander Lutherus,} 1589. (See Kawerau in the
\textit{Real-Enzyklopädie für prot. Theologie,} 3rd ed., Vol. XVIII, p. 567; \textsc{Döllinger},
\textit{Die Reformation,} Vol. II, pp. 270 sqq.)

\textsc{Stähelin R.} \textit{Huldreich Zwingli,} 2 vols. Bâsle, 1895--1897.

\textsc{Stange Carl.} \textit{Luther und das sittliche Ideal.} Gütersloh, 1919. (See \textit{Theol.
Revue,} 1921, p. 99.) \\
---\textit{Zur Einführung in die Gedanken Luthers.} Gütersloh, 1921.

\textsc{Strohl H.} \textit{L’Evolution Religieuse de Luther jusqu'à} 1515. Strasburg,
1922. \\
---\textit{L’ Épanouissement de la Pensée Religieuse de Luther de 1515 à 1520.}
Strasburg, 1924.

\textit{Studien zur Reformationsgeschichte, G. Kawerau dargebracht.} Leipsic, 1917.

\textit{Tischreden oder Colloquia M. Luthers,} ed. by \textsc{Aurifaber,} 2 vols. Eisleben,
1564--1565. Edited by \textsc{Förstemann.} \\
---\textit{Nach Aurifabers erster Ausgabe, mit sorgfältiger Vergleichung sowohl
der Stangwaldschen als der Selneccerschen Redaktion.} 4 Vols. (Vol. IV, ed.
with the co-operation of H. E. Bindseil.) Leipsic, 1844--1848.

\textsc{Troeltsch E.} \textit{Protestantisches Christentum.} (\textit{Kultur der Gegenwart,} Vol.
I, Heft 4), 2nd ed., 1909. \\
---\textit{Soziallehren der christlichen Kirchen. (Gesammelte Schriften,} Vol. I),
1911. \\
---\textit{Die Bedeutung des Protestantismus fir die Entstehung der modernen
Welt. (Histor. Bibliothek}, No. 24. 3rd ed., 1924, Munich.) (See \textsc{Wolf,}
\textit{Quellenkunde,} Vol. I, p. 46.

\textsc{Ulenberg C.} \textit{Historia de Vita \dots Lutheri, Melanchthonis, Matth. Flacii
Illyrici, G. Maioris et Andr. Osiandri.} 2 vols. Coloniae, 1622.

\textit{Vita Lutheri.} In \textit{Vitae Quatuor Reformatorum} by Melanchthon, Berolini,
1841. Also in \textit{Corp. Reform.}, Vol. VI, pp. 155 sqq., and previously as preface
to Vol. II of the Wittenberg edition of Luthers Latin writings.

\textsc{Walther W.} \textit{Für Luther, Wider Rom. Handbuch der Apologetik Luthers
und der Reformation den römischen Anklagen gegenüber.} Halle a. S., 1906. \\
---\textit{Luthers deutsche Bibel.} Berlin, 1917. \\
---\textit{Die deutsche Bibelübersetzung des Mittelalters.} Braunschweig, 1889--1892.
See \textsc{Wolf}, \textit{Quellenkunde}, Vol. I, p. 186. \\
---\textit{Das Erbe der Reformation.} Vol. I-IV. Leipsic, 1909--1917. \\
---\textit{Luthers Charakter.} 2nd ed., Leipsic, 1917. \\
---\textit{Luther und die Juden.} Leipsic, 1921. \\
---\textit{Deutschlands Schwert durch Luther geweiht.} 2nd ed., Leipsic, 1919.

\textsc{Ward F. G.} \textit{Darstellung der Ansichten Luthers vom Staat.} Halle, 1898.

\textsc{Weiss A. M. (O. P.)}, \textit{Lutherpsychologie als Schlüssel zur Lutherlegende.
Denifles Untersuchungen kritisch nachgeprüft.} 2nd ed., Mayence, 1906. \\
---\textit{Luther und Luthertum.} Vol. II; see \textsc{Denifle.}

\textsc{Works,} Erl. ed., \textit{M. Luthers sämtliche Werke.} 67 vols., ed. by J. G. Plochmann
and J. A. Irmischer. Erlangen, 1826--1868. Vols. I-XX und XXIV-
XXVI, 2nd ed. by L. Enders, Frankfort a. M., 1862 sqq.--The Erl. ed.
comprises the \textit{Opp. Lat. Exeg.}, the \textit{Comment. in Epist. ad Galat.}, die \textit{Opp.
Lat. Var.} and \textit{Briefwechsel,} ed. by L. Enders (cf. these four titles above). \\
--Weimar ed., \textit{Dr. Martin Luthers Werke. Kritische Gesamtausgabe.} Weimar,
1883 sqq. Under the editorship of J. Knaake, G. Kawerau, P. Pietsch,
N. Müller, K. Drescher, and W. Walther. \\
--(Weimar ed.) \textit{Tischreden} in 6 vols. 1912--1921; ed. by \textsc{Ernst Kroker.} \\
--Altenburg ed., 1661--1664. 10 vols. (German); reimpression, Leipsic,
1729--1740. 22 vols. \\
--Eisleben ed. (Supplement to the Wittenberg und Jena editions) by \textsc{J.
Aurifaber.} 2 vols. 1564--1565. \\
--Halle ed., by J. G. Walch. 24 vols. 1740--1753 (German). New ed.
under the auspices of the German Lutheran Synod of Missouri, Ohio, and
other States. St. Louis, Mo. 22 vols. 1880---1904. Vol. 23 (Index vol.),
1910. \\
--Jena ed., 8 vols. German and 4 vols. Latin. 1555--1558. \\
--Wittenberg ed., A. 12 vols. German (1539--1559) and 7 vols. Latin
(1545--1558). \\
---\textit{Luthers Werke in Auswahl.} Ed. by \textsc{Otto Clemen} and \textsc{Alb. Lietzmann.}
Vols. I to IV. Bonn 1913 sq. \\
---\textit{Luthers ausgewählte Werke.} Ed. by \textsc{H. H. Borcherdt.} 6 vols. Munich,
1913--1923. \\
---\textit{Martin Luther. Auswahl,} by \textsc{Neubauer}, in \textit{Denkmäler der älteren
deutschen Literatur}, Vol. III, Heft 2 and 3. 3rd ed., 1903--1907. \\
---\textit{Ausgewählte Schriften Luthers.} Ed. by \textsc{Arnold Berger.} 3 vols. Leipsic,
1917. \\
---\textit{Auswahl,} ed. by \textsc{Buchwald, Kawerau, Köstlin, Rade,} etc., (Known
as “Braunschweiger Ausgabe”). 8 vols. 3rd ed. Braunschweig and later
Berlin, 1905 sqq. Also 2 supplementary vols. \\
---\textsc{Wernle Paul.} \textit{Der evangelische Glaube nach den Hauptschriften der Reformation.}
Vol. I: \textit{Luther.} Tübingen, 1912. (Vol. II: \textit{Zwingli}; Vol. III:
\textit{Calvin}).

\textsc{Wiedemann Th.} \textit{Johann Eck, Professor der Theologie an der Universität
Ingolstadt.} Ratisbon, 1865.

\textsc{Wolf Gustav.} \textit{Quellenkunde der deutschen Reformationsgeschichte.}
Vol. I: \textit{Vorreformation und allgemeine Reformationsgeschichte,} Gotha, 1915;
Vol. II, Part 1: \textit{Kirchliche Reformationsgeschichte,} 1916;
Vol. II, Part 2: \textit{Kirchliche Reformationsgeschichte,} Fortsetzung, 1922.

\textsc{Zickendraht K.} \textit{Der Streit zwischen Erasmus und Luther über die Willensfreiheit.}
Leipsic, 1909.

\textsc{Zwingli H.} \textit{Opera.} 5 vols., ed. by \textsc{Emil Egli} und \textsc{Georg Finsler,} since
1910 by \textsc{Finsler} and \textsc{Walter Köhler}. 1905--1925. (\textit{Corpus Reformatorum,}
Vols, LXXXVIII sqq.)
