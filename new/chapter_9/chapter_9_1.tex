\section{Storms Within and Without}

At the Wartburg, Luther vested himself in the costume of a
cavalier and allowed his beard and hair to grow. In order to conceal
his identity, he assumed the name of Squire George (Ritter Görg).
He wrote to Melanchthon at Wittenberg that he would not recognize
him if he were to meet him. On the tenth day of his stay, he says
in a letter to Spalatin: “I sit idle here the livelong day and eat more
than enough.” But soon he found an occupation; for in the same
letter he says: “I read the Bible in Greek and Hebrew. I shall write
a German sermon on the freedom of auricular confession; I shall
continue my commentaries on the Psalms and my sermons as soon
as the necessary materials arrive from Wittenberg. From there I also
expect to obtain the Magnificat which I have begun.”\footnote{\textit{Briefwechsel}, III, p. 154; May 14, 1521.}
He did
not carry out all these plans, however. But the period of quiet retirement
became for him a time of gigantic labors, chiefly devoted to
a direct attack upon the teachings and the power of Antichrist,
whose determined opposition to the Gospel he believed to have experienced
at Worms. In one of his first letters we hear: “While I
quietly sojourn here, I contemplate the face of the Church all day,
and hear the voice of the Psalmist raised to God: Why hast Thou
made all the children of men in vain? (Ps. 88:48 [Vulgate]).
% "Ut quid in vanum creasti omnes filios hominum?"
% ("Why have You created all the children of men in vain?")
O God, what a horrible monster of divine wrath is this execrable empire of
the Roman Antichrist. I curse the hardness of my heart, since I
do not completely dissolve in tears because of the murdered sons of
my people.”\footnote{\textit{Ibid.}, p. 148; letter to Melanchthon, May 12.}

The frame of mind expressed in these rhetorical words to Melanchthon,
continued during the ten months of his sojourn at the
Wartburg. It was there he was to receive, as it were, his spiritual
baptism in preparation for his future work.

His room at the Wartburg was an unpretentious cell, situated not
within the castle itself, but in the outbuildings set apart for ordinary
guests. He had no elevating outlook upon the green mountains,
which some biographers picture him contemplating from his room.
Nor was his cell equipped with the oft-admired bay-window, which
was added to the corner room only about a century ago. The room,
which is much frequented today, is reached by ascending a steep,
narrow flight of stairs. It is stocked with numerous souvenirs of
doubtful authenticity. There “Squire George,” secluded from intercourse
with others, was attended by servants of the castle. The castle
itself made a rather unfavorable impression, its upkeep having
been neglected for a long time. The bare walls harmonized with the
disposition of the new inmate. A priest conducted divine services for
the castellan and a few domestics and rare guests. Luther avoided the
society of the “mass-priest.” At times he participated in the chase,
more for the sake of appearances than for diversion.

So secretly kept was his seclusion, that even for months afterwards
but few knew of his whereabouts. Many supposed that he had been
abducted by his enemies, others that he had been assassinated. Among
those who anxiously desired information about him was Albrecht
Dürer, the famous painter, who was on intimate terms with Pirkheimer,
Luther’s patron at Nuremberg, and favorably disposed
towards the reforms which Luther had promised, without as yet
realizing the full import of his religious changes.\footnote{Grisar, \textit{Luther}, Vol. II, pp. 39, 41 sq.}

During the first few days of his sojourn at the Wartburg, Luther
was very much concerned over reports of violence at Erfurt. In
his opinion, the demonstrations in his behalf were going too far. On
the day following his departure from Erfurt, which he visited on
his journey to Worms, he heard that the students, assisted by a mob,
had risen against the canons of the church of St. Severin, who
were loyal to their faith, and had stormed the dwellings of the
canons, committing all kinds of excesses. These scenes were reenacted
with the permission of the academic senate and the authorities on
several days in June. Fundamentally Luther himself had furnished
the cause of these demonstrations by the hatred which he had
aroused against the Catholics at Erfurt. This same hatred burst
forth anew when the clergy, who had participated in the festive
reception in honor of Luther, were threatened with exclusion from
choral service and from their benefices because they had also become
subject to the edict of outlawry. Luther deplored the whole violent
movement in a letter to Wittenberg: “This kind of procedure will
bring our Gospel into disrepute and justly bring about its repudiation \dots Satan
attempts to mock our endeavors.”\footnote
{\textit{Briefwechsel}, III, p. 158; letter to Melanchthon, about the middle of May.}
His Gospel,
however, was exposed to other and still greater annoyances from
without. In the meantime he sought relief from his own mental
storms by resorting to controversy.

During the first two months of his sojourn at the Wartburg, his
lengthy reply to Ambrosius Catharinus was printed.\footnote{\textit{Werke}, Weimar ed., VII, pp. 705 sqq.; Erl. ed., \textit{Opp. Lat. Var.}, V, pp. 289 sqq.}
In comparison
with his previous writings, this reply was especially striking on account
of the visionary application of the real or apparent Biblical
passages concerning Antichrist. In the application of these texts
to his idea of the papacy, he indulges in a kind of dreamy fanaticism.
In his opinion the prophet Daniel (7:28) had definitely foretold
the different characteristics of Antichrist which were realized in
the pope. The enemy of God, according to Luther’s (false) translation of
Daniel, has different “faces,” which are all discoverable in
minute detail in anti-Christian Rome. According to Daniel, he held,
the spirit out of the mouth of God, not force and human fury, will
kill Antichrist and that within a brief space; for the Lord and
His day are nigh. This work was intended to be that production of
which his tract on the Babylonian Captivity was the prelude. Undoubtedly
it is representative of the mental excitement with which
Luther was seized at the Wartburg.

During his first week at the castle, he composed his treatise \textit{Von
der Beicht} (“On Confession--Whether the Pope has Power to Impose
it”). It was dedicated on June 1 to Franz von Sickingen as
his “special lord and patron.”\footnote{\textit{Werke}, Weimar ed., VIII, pp. 138 sqq.; Erl. ed., XXVII, pp. 318 sqq.}
Confession, as imposed by the papacy,
he asserts therein, is an unauthorized and insidious institution, whereas
private confession made to anyone, even to a layman, if entirely
voluntary, is a “precious and wholesome thing,” because of the
humiliation it involves and the comfort produced by the consolatory
words of one’s fellowman. Absolution received in this manner alone
corresponds with the liberty of a Christian. In general, no one may
be compelled to receive the Sacraments, just as no one may be or can
be compelled to accept the faith.

It is not likely that Sickingen went to excess either in, receiving
the Sacrament of Penance or free lay-confession. His two castles
in the Palatinate, Ebernburg and Landstuhl, were asylums for the
friends of the new religious movement and the revolution. Elected
captain of the “Fraternal Union of Knighthood,” in. August, 1522,
he declared war against the Archbishop of Treves, Richard von
Greiffenklau, and, after an unsuccessful attack upon his episcopal
city, devastated the district about Treves and portions of the Palatinate.
In May of the succeeding year, Sickingen succumbed to the
wounds he had sustained at the siege and capture of his fortress
Landstuhl by the princes who had allied themselves against him.

In his reply to Catharinus, Luther had interpreted one of the
“faces” of the supposed Antichrist (Daniel 7:7–8, 24–25)
% (Daniel, ch. ix) -> (Daniel 7:7–8, 24–25)
as referring to
the papal universities. These high schools of Satan, he alleged, are
the waters of the bottomless pit described in the Apocalypse, whence
Jocusts with the power of scorpions issue as in a thick smoke. In
1518, a book had been issued against him by the University of
Louvain, which Luther declared to be the most thorough and the most
dangerous of all the works written against him. Its author was the
erudite theologian James Latomus (Masson). In the twelve days
intervening between June 8 and 20, Luther composed a reply to
Latomus, entitled, \textit{Rationis Latomianae Confutatio},\footnote{Weim. ed., VIII, pp. 43 sqq.; \textit{Opp. Lat. Var.}, V, pp. 395 sqq.}
wherein he attempts to refute the Catholic doctrine of sin and grace by
citations from the Bible, no other aids being available to him at the time.
A pronouncement made against his heresies by the theological faculty
of Paris he tried to dispose of by publishing a translation of this
document, accompanied by a preface and an epilogue.\footnote{Weimar ed., VIII, pp. 267 sqq.; Erl. ed,, XXVII, pp. 379 sqq.}
He denounces
the faculty, which had been the glory of the Middle Ages, as “the
greatest spiritual harlot under the sun and the back-door to hell.”\footnote{At the end of the publication.}
Again he indulges his mania concerning the Antichrist and the end
of the world. The faculty is the sinful chamber “of the pope, the
true Antichrist.” “When the belly of these gentlemen of Paris
rumbles,” etc., they exclaim: “It is an article of faith.” Their actions
are to him an additional proof that the “pope has not regarded us
otherwise than as unworthy to be \dots his privy, etc \dots So many
noble-minded persons have been obliged to harbor the stench, dung,
and filth,” etc. His crudities are not deserving of full quotation.
It is worth while, however, to mention here the works which
Luther composed during his brief sojourn at the Wartburg; particular
reference will be made to some of them in the sequel. They are: the
\textit{Verbandlungen zu Worms} (“Transactions at Worms”) ; two treatises
against monastic vows; two against the Mass; the interpretation of
the Magnificat; a “Warning” against rebellion; a discourse on the
“Bull Caena Domini”; an illustrated “Passional of Christ and Antichrist,”
for which he was responsible at least in part; a Christmas
postil, and other explanations of Biblical texts, besides smaller polemical
tracts, and, finally, his translation of the New Testament. Surely
no small amount of labor. Aided by Spalatin, Melanchthon, and other
friends, he entrusted the publication of his works to the Wittenberg
press.

What a contrast between the tender and charitable activity of the
sainted princess whose memory the Wartburg preserved and Luther’s
agitated labors, sustained mainly by strong hatred, passion, and a
slanderous disposition. St. Elizabeth with her loving heart for the
poor, with her loyal devotion to the Church, and her soul aglow with
prayer, everywhere confronted the man of the violent pen within
the castle-walls. There was the Kemenate where she had her quarters,
still in a state of beautiful preservation; there was the richly adorned
chapel, her favorite retreat; there, rising heavenward above the court,
was the tower whence she so often contemplated the splendor of her
celestial home in the mirror of nature.

The letters he wrote to his friends cast a lurid light upon Luther’s
frame of mind in the intervals between his oppressive labors. Anyone
who reads them would be greatly disillusioned had he expected that
the solitude which came to him as an extraordinary grace from above
would have induced Luther to reflect seriously upon himself or to examine
quietly his activities which were fraught with so much responsibility.
Prayers, indeed, there are, brief and ardent prayers, for
himself and against his adversaries; but we miss the principal prayer,
the petition for complete submission to the divine will and the expression
of willingness to be led anywhere, even to the abandonment
of his struggle, if it were God’s will.\footnote
{Grisar, \textit{Luther}, Vol. VI, pp. 511 sq.; more fully in the original German edition, Vol.
III, pp. 995 sqq.}
Instead of “Thy will be
done” one hears everywhere “My will be done”; so that the resignation of
the soul to God, which Luther had so strongly emphasized in
the days of his so-called mysticism, now seems to be forgotten in a
cause so decisive for himself and for thousands of others. God should,
nay He must, so Luther thinks, place the seal of His approval upon
his revolt from the entire past of the Church.

Luther had to suffer much from temptations. ‘These were always
combated by devout Catholics by prayer accompanied by penance,
but no mention is made of penitential practices in the case of Luther.
He himself acknowledged his deficiency in the matter of prayer.\footnote
{Grisar, \textit{Luther}, Vol. II, pp. 82 sq.; cfr. V, pp. 225 sqq., letter to Melanchthon, July
13, 1521.}

“Alas, I pray too little instead of sighing over the Church of God \dots For
a whole week I have neither written, prayed nor studied, plagued
partly by temptations of the flesh, partly by the other trouble [constipation].
Pray for me, for in this solitude I am sinking into sin.”

A previous passage in this same letter says: “I burn with the flames of
my untamed flesh; in short, I ought to be glowing in the spirit, and instead
I glow in the flesh, in lust, laziness, idleness and drowsiness, and know not
whether God has not turned away His face from me, because you have
ceased to pray for me.”

A little later he writes: “I am healthy in body and am well cared for, but
I am also severely tried by sin and temptations. Pray for me, and fare you
well!”\footnote{\textit{Op, cit.}, II, p. 83; letter to Lang, December 18, 1521.}

Here, at all events, powerful sexual temptations (\textit{ferveo carne},
\textit{libidine}, etc.) are openly acknowledged. As early as 1519, he had
written to his superior Staupitz concerning such visitations (\textit{titillationes}).\footnote
{\textit{Op. cit.}, Vol. V, pp. 319 sqq.}
These assaults at the Wartburg, however are disagreeable
to him. His self-revelations are somewhat inflated by his habitually
superlative style of writing; and he may have referred the “sins”
which he mentions to sensuality, on the one hand, and, on the other,
to the frailty of his fiduciary faith in God, which he made the center
of his Gospel. The devil, so he believed, ever and anon sought to deprive
him of this faith.

Luther saw the Wartburg filled with devils. This, in part, was the
result of the fear of demons which he had imbibed in his youth; while
in part it was a consequence of the inquietude caused by his internal
doubts and self-reproaches. The voices of self-reproach he imagined
to be voices from the Satanic empire.

“Believe me,” he wrote on November 1, “that I am cast before a thousand
devils in this idle solitude. It is much easier to struggle with men, even if
they be incarnate devils, than with the spirits of iniquity that infest the air.
I fall, but the right hand of God sustains me.”\footnote
{On November 1, 1521, in a letter to Nicholas Gerbel; \textit{Briefwechsel}, III, p. 240: ``\textit{Mille
credas me Satanibus obiectum in bac otiosa solitudine \dots Sacpius ego cado, sed sustentat
me rursus dextra Excelsi}.”}

According to another utterance of his, he wishes to praise God in the
name of his Gospel,--God who has not only given us this combat with the
spirits of iniquity, but has also revealed to us [\textit{revelavit nobis}] that in this
matter it is not flesh and blood that take the field against us \dots It is
Satan, who rages against us according to his way and within his limits.”
Thus convinced of his great struggle against the evil spirits, he discovers,
in his own imagination, that they become visible and audible to him, as
will be shown in the following pages.

Meanwhile we must mention the internal struggle which he sustained when
he had persuaded himself of the invalidity, nay, absolute
reprehensibility of the monastic vows and the vow of celibacy. It
was a violent struggle. Hitherto he had adhered to his monastic vow
of chastity as a matter of principle, but now his false idea of Christian
liberty began to seduce him to break his vow.

Bartholomew Feldkirch (Bernhardi), provost of Kronberg, was
the first, or one of the first adherents of Luther among the clergy
who married during the time of Luther’s sojourn at the Wartburg.\footnote
{Bernhardi is known to us through his part in the Wittenberg disputation of 1516.}
Karlstadt, Luther’s tumultuous theological colleague at Wittenberg,
had published a tract against the vows. Not long afterwards, he,
too, took unto himself a wife. When Luther first heard of the movement
to permit the inmates of monasteries to marry, he was
somewhat taken aback and wrote to Spalatin: “O God, shall the Wittenbergers
give wives to the monks! But they shall not force a wife
upon \textit{me}.”\footnote{On August 6, 1521; \textit{Briefwechsel}, III, p. 215.}
He found very many grounds for criticizing the treatise of Karlstadt, especially
its method of demonstration. Melanchthon, too, who opposed the monastic vows, had not hit upon the
right solution in Luther’s opinion. His own fermenting soul now
embraced the question, with no intention of dismissing it until the
right solution, or rather the least disquieting disposition of the duty of
the vows had been found. On August 3, he revealed the state of his
mind in a painful discussion with his friend Melanchthon: “You see
with what heat I burn [\textit{quantis urgeor æstibus}], and yet I cannot
confirm any satisfactory conclusion except that I greatly desire to
support your efforts.”\footnote{\textit{Briefwechsel}, III, p. 213.}

As was to be foreseen, however, he soon discovered what he sought.
The solution was destined to bring him what he desired, an influx of
male and female members of monastic Orders who had grown tired of
their vows. He resolved to liberate these “unfortunates” from the
“impure and damnable state of celibacy,” as he styles it, and to induct
them into the “paradise of matrimony.”\footnote{Grisar, \textit{Luther}, Vol. II, pp. 83 sqq.}
This solution also afforded
the advantage to make him independent of his rival Karlstadt, and furthermore
, to enable him to watch over the “firstfruits of the spirit”
also in this matter. The point of departure was furnished by his idea
of evangelical liberty. “Whoever has taken a vow in a spirit opposed
to evangelical freedom”--thus he sets forth his saving idea--“must
be set free, and his vow be anathema. Such, however, are all those
who have taken the vow in the search for salvation or justification.”\footnote{Letter to Melanchthon, \textit{op. cit.} II, 84.}
In this spirit all religious, including himself, had taken their
vows. This spirit was inseparable from the vow as long as good works
were regarded as efficacious; for the voluntary relinquishment of
freedom, offered at the throne of the Most High, is always connected
with the certain expectation, guaranteed by the Word of God, that
the sacrifice will assist in the attainment of salvation and justification,
through the merits of Jesus Christ whom the person who takes the
vow promises to follow in humility.

For the sake of advancing his new discovery, Luther first wrote
“theses” intended “for the bishops and deacons of the church of
Wittenberg.”\footnote{Weimar ed., VIII, pp. 323 sqq.; \textit{Opp. Lat. Var.}, IV, pp. 344 sqq.}
These were followed by his momentous work \textit{Ueber die
Ordensgelübde, ein Urteil Martin Luthers} (“On the Monastic
Vows: an Opinion of Martin Luther.”)\footnote
{Weimar ed., VIII, pp. 573--669; \textit{Opp. Lat. Var.}, VI, pp. 238--376; Denifle, in \textit{Luther
und Luthertum}, has subjected the Latin text to a searching and extensive criticism. Cf.
Grisar, \textit{Luther}, Vol. II, p. 85. sqq. On the question of Christian perfection and the alleged
dual ideal of life fostered by Catholicism, see \textit{op. cit.}, II, 85, Note 3.}
It abounds in misrepresentations
of the monastic life and the Catholic teaching concerning
perfection, good works, and penance as well as in frivolous indecencies
and vulgar calumnies. The author prefaced the book with a dedicatory
letter to his father, in order to invest it with an attractive foil of
personal experience. His father, he writes, should not be angry at him
because, by entering a monastery, he had violated the grave commandment
of obedience to parents. He now realized that his vow was
worthless, for God had released him from his fetters.

Desertion of the monasteries amid strife and tumult such as, for
instance, his friend Lang at Erfurt had proposed, Luther at that time
censured. Concerning himself, he announced his intention of adhering to
his present mode of life. He wore the habit of his Order for
several years after his sojourn at the Wartburg.
