\section{Confusion at Wittenberg. Other Writings Composed by Luther During His Sojourn at the Wartburg}

Luther’s attacks on the Mass were attended with greater success at
Wittenberg than he had desired. Gabriel Zwilling (Didymus), a
young Augustinian, achieved prominence there as an enthusiast for
the purified celebration of the Lord’s Supper without the Mass. Some
called him the new prophet and a second Luther. Melanchthon took
his part and missed none of Zwilling’s sermons. Karlstadt was no
less enthusiastic; he and a committee constituted of like-minded
zealots petitioned the Elector for “the speedy abolition of the abuse
of Masses in his principality.” It was the first application of the principle
that ecclesiastical reforms were the affair of territorial rulers.
But the circumspect Elector on October 25 issued an order inhibiting
any acts which might result in dissension and uprisings.

When discontent nevertheless increased, Luther became alarmed
and formed a startling resolution. On December 3, he suddenly left
the Wartburg for a secret visit to Wittenberg, clad in his squire’s
costume. He remained there from December 3 to 11, in the company of Lis
friends, for the sake of obtaining information on the state
of affairs, but eight days later he was back at the Wartburg.
He used the information thus acquired, which was partly unsatisfactory,
in the composition of a tract, entitled \textit{A Sincere Exhortation
to all Christians to Guard against Rebellion}.\footnote
{\textit{Werke}, Weimar ed., VIII, pp. 676 sqq.; Erl. ed., XXII, pp. 44 sq.; written at the
beginning of 1522.}

It appears, he says, as if an insurrection is threatening, in which priests,
monks, bishops, and the whole clerical estate might be slain. By hearkening
to his words, however, such a general attack might yet be avoided. Christ
had reserved to Himself the punishment of papism, in order to slay Antichrist
by the breath of His mouth. People should wait two years, then it
would be accomplished as a result of the gigantic and irresistible progress
of his Gospel, which was very evidently the work of God. “It is not our
work \dots ’tis someone else who propels the wheel.”

In the sequel, he requests his followers to call themselves Christians, not
Lutherans. “Who is Luther? The doctrine is not mine; nor have I been
crucified for anyone.”

But above all: The weak and inadequately instructed must not be taken by
surprise or violence. The authorities only, and not “Herr Omnes” (\textit{i.e.}, the
masses) have the right and the duty to intervene against whatever is contrary
to the Gospel. The devil purposes to injure the evangelical doctrine by fomenting rebellion.

The extent of his illusion regarding the speedy collapse of the
kingdom of Antichrist may be inferred from the postil he at that
time composed for the second Sunday of Advent. According to the
astronomers, he says, a great constellation of planets was imminent
for the year 1524. The powers of heaven, according to the Sunday
Gospel, would be convulsed. God grant that it be the “Day of Judgment to
which the signs certainly point.”\footnote{Köstlin-Kawerau, \textit{Martin Luther}, Vol. I, p. 478.}

During his sojourn at the Wartburg Luther also completed his
remarkable interpretation of the Magnificat, which he dedicated to
the heir to the Saxon throne.\footnote{Weimar ed., VII, pp. 544 sq.; Erl. ed., XLV, pp. 212 sq.}
It is a remarkable production on account
of the note of sincere piety which the author, in the very
midst of a hostile and agitated campaign, sounds in these pages; remarkable
, too, because of the author’s eulogy of the “Blessed Virgin,”
“the tender Mother of the Lord.” The commentary is largely
pervaded by the traditional devotion with which the Augustinian Order revered
Mary, though it fails to do justice to her exalted virtues
or to her position as the advocate of Christendom before the throne
of God. Luther mainly extolled the faith of the Blessed Virgin,
as conceived in accordance with his new doctrine of justification.
He was aware of, and carefully took into consideration, the sentiments
of the prince to whom the work was dedicated, and who had a lively
attachment to many practices of the ancient religion which were still
appreciated by Luther. Later on, when the prince warmly supported
his person and his work, he learned.that the successor to the throne
regarded him as a profoundly pious man and a peaceful religious reformer.
Besides its religious note, Luther’s treatise on the Magnificat is
pacific in so far as the polemical ideas which it contains are veiled and
not clothed in his customary harsh language. Fundamentally, however,
it, too, is a controversial treatise, as is indicated by such superfluous admonitions
as that Mary is “no helping goddess,” even though Luther
does not as yet condemn the practice of praying for her intercession.\footnote
{Grisar, \textit{Luther}, Vol. IV, p. 237.}
While the book is not, as has been asserted, a monument of the
author’s profound piety, nevertheless, one must marvel at the ideas in
which it abounds, the dexterity with which the style is varied, and the
adroitness with which the author adapts himself to his readers. For
the rest, Luther always believed in the virginity of Mary, even post
partum, as affirmed in the Apostles’ Creed, though afterwards he
denied her power of intercession, as well as that of the saints in general,
resorting to many misinterpretations and combated, as extreme
and pagan, the extraordinary veneration which the Catholic Church
showed towards Mary. His prayer-book, which appeared in 1522, retained
the Ave Maria side by side with the Pater Noster and the
Creed. As late as 1527 he even acknowledged the doctrine of the Immaculate
Conception of Mary, in conformity with the theological
traditions of the Augustinian Order.\footnote{\textit{Op. cit.}, IV, 238 and 500 sqq.}
At the beginning of 1522, Luther dedicated to the Pope his scornful tract
\textit{Vom Abendfressen des allerheiligsten Herrn des Papstes}.\footnote
{\textit{Werke}, Weimar ed., VIII, pp. 601 sqq.; Erl. ed., XXIV², pp. 166 sqq. “\textit{Fressen}” is a
contemptuous term used only of animals.}
This crude production is a reference to the Bull Iz Caena Domini,
which was published annually on Maundy Thursday at Rome. This
document was a comprehensive condemnation of heresy, and it now
listed Luther among the condemned heretics. Luther derides the
pope as a drunkard who in his frenzy curses and swears and uses the
Latin of a “kitchen scullion.” He translates the solemn juridical document
and accompanies it with coarse annotations. In the introduction
he declares that “the Rhine is scarcely large enough to drown all
the scoundrels”--such as the “retailers of bulls, cardinals, legates,”
etc., besides archbishops, bishops, abbots, etc. Did such language serve
his previously mentioned purpose to quell violence and sedition?

It is a puzzle how Luther, during his short sojourn at the Wartburg, in
addition to his other work, was able to translate the New
Testament from Greek into a German of undeniable excellence. He
practically completed this important task within the incredibly short
space of three months. We postpone an appreciation of his New Testament
to a later page, where we shall deal with his rendition of the
Bible as a whole. As a literary document, it is truly monumental.\footnote{Grisar, \textit{Luther}, Vol. V, pp. 494 sqq.}
For the present we will consider only its polemical purpose. Luther
intended his Bible to be read by the masses, so that it might win followers
to his new gospel. To accomplish this purpose he did not
scruple to alter the text in numerous places. The aggressive tendency
of the translator is emphasized by the wood-cuts which illustrate
the Apocalypse. There the woman of Babylon is repeatedly shown
crowned with the papal tiara; Catholic dignitaries, and even the Emperor,
are depicted as rendering homage to the bearer of the cup of
sin and blasphemy; like Babylon, papal Rome collapses and is consumed by
fire; the defenders of the papacy are depicted as dragons with
seven heads, and so forth. These illustrations escaped observation until
quite recently. They are in complete harmony with the utterly abnormal
apocalyptical frame of mind in which Luther was at that
time. In this respect, Luther’s celebrated German version of the New
Testament ranks with the polemical illustrations of the “Passionale
of Christ and Antichrist,” by which at the beginning of the Wartburg period
he enlisted the aid of the graphic arts in the campaign
against religion.\footnote
{Grisar-Heege, \textit{Luthers Kampfbilder: I. Das Passionale, II. Der Bilderkampf in der
deutschen Bibel}, with 14 plates.}

\section{The Return. Victory at Wittenberg}

Whilst Luther was occupied with the translation of the Bible in
the solitude of the Wartburg, events happened at Wittenberg which
induced him to move back to the university town, in spite of many
and great dangers.

On December 3, students and citizens attacked certain priests who
intended to celebrate Mass in the parish church there. The monastery
of the Discalced Friars as well as a number of cathedral canons were
menaced. Worst of all, the tumultuous Karlstadt, who cut such a sorry
figure at the disputation of Leipsic, set himself up as the leader of the
new movement with the intention of becoming not a dilatory, but a
consistent Luther. He announced that matrimony must be obligatory
for the clergy and that the reception of the Eucharist without the
chalice was not permissible. On Christmas he for the first time commemorated
the Lord’s Supper without celebrating Mass, by administering
bread and wine to all who so desired, without previous confession. On January
19, he, as deacon of All Saints, solemnly took unto
himself a wife. The Augustinians who still remained at Wittenberg
removed every altar but one from their church and burned the images
of the saints as well as the holy oils. Karlstadt asserted that images
were prohibited by Holy Scripture (Ex. 20, 4) and denounced them
as idols. As a consequence, the sacred images and statues were cast
out of the various churches. The monk Zwilling conducted a similar
iconoclastic campaign in the towns and villages about Wittenberg.
He proscribed ecclesiastical vestments and preached his revolutionary
innovations clad in the characteristic attire of the students of that
time. Confessions were discontinued; the law of fasting was disregarded;
there was a movement to abolish all feast-days except Sundays; the religious
consolations for the sick, prisoners and those about
to be executed fell into desuetude. In lieu of these it was proposed to
devote greater attention to the alleviation of temporal needs by the institution
of a so-called community chest which was enriched by mass
stipends and other ecclesiastical funds. Karlstadt introduced a thoroughly
shallow lay Christianity. It was his intention to promote a natural but
spiritualistic religiosity in opposition to Luther’s massive
doctrine of the imputation of the merits of Christ.\footnote
{H. Barge, \textit{Karlstadt}, Vol. I, p. 404.}
Like Luther he believed that he was inspired from on high.

A new and dangerous element was injected into this movement
when, on December 27, the so-called prophets of Zwickau visited
Melanchthon at Wittenberg. They were: Nicholas Storch and Mark
Stübner, two cloth-weavers, who claimed to have received a direct
call from God to preach. They had been inducted by Thomas Münzer into
the prophetic life and into direct converse with God and
maintained that man is obliged to learn everything through the Spirit,
and to aspire to the most complete self-renunciation and apathy. Not
only was the Church to be reformed by one greater than Luther, but
the civil order, too, was to be altered, all priests were to be slain, and
all godless people extirpated. They attacked especially Luther’s doctrine
of infant baptism and hence were called Anabaptists. If faith
alone makes the Sacrament efficacious, they contended against Luther,
then adults only can be baptized, since children are incapable of faith.
The inference cannot be denied.

Like many others, Melanchthon, who was deficient in intellectual
acumen, allowed himself to be taken in by these “prophets.” When
Luther heard of this he was greatly worried and tried to inspire
Melanchthon with distrust toward these fanatics.\footnote
{Briefwechsel III, p. 273, 13th of January, 1522: “\textit{Neque enim Deus unquam aliquem
misit, nisi vel per hominem vocatum vel per signa declaratum \dots Quaeras, num experti
sint spirituales angustias et nativitates divinas, mortes infernosque.}” He says God Himself
speaks in the verse: “\textit{Contrivit omnia ossa mea}” (Is. XXXVIII, 13.)}
His troubles were
increased in virtue of the tumultuous innovations which were introduced
in Wittenberg and environs. Luckily, Karlstadt did not make
common cause with the “prophets,” but the feeble efforts of the
Elector were impotent to arrest the innovations of Karlstadt.
These reforms were favored by a semblance of truth--if one
took the Lutheran position. Luther himself had set a precedent in
altering the divine services. The others were simply following his example.
The enthusiasts of Zwickau were not only justified in their
opposition to his doctrine concerning baptism, but their private revelations
and theories of interior experience were essentially not far
removed from Luther’s contention that the Spirit alone must
guide man. Nor did they deviate very much from his view of supernatural
revelation, although he himself was naturally unwilling to
grant that their strange and arbitrary conduct was the result of
visions. Strangely enough, he asked Melanchthon to tell the “prophets”
of Zwickau that even raptures capable of transporting men
into the third heaven furnished no proof of their claims, but they
would have to work miracles and also to show that they had been
made partakers of the spiritual travails and the divine re-birth which
penetrated death and hell--that terrible regeneration which he believed
to have experienced and which he maintained to be the standard by which
to judge whether one were really and truly justified.

Writing of affairs at Wittenberg, the Elector Frederick said:
“Everything went wrong, everybody became perplexed, and no one
knew who was cook or cellarer.”\footnote{Köstlin-Kawerau, \textit{M. Luther}, I, p. 495.}
It was then that the imperial government
finally issued a strict warning (June 20, 1522) to all the
bishops and to the above-mentioned Elector to ferret out and punish
those who disturbed the religious peace and violated ecclesiastical discipline.
The bishops of Meissen and Cheeseburger indicated their intention to obey
and Frederick caused representations to be made to
Luther.

What was decisive, however, was that Luther himself now resolved
to terminate the disorderly state of affairs by resuming his residence
at Wittenberg. Naturally he realized that his territorial lord might
be seriously embarrassed by the imperial authorities if he openly
permitted the outlaw to return to the university and sheltered him
in his territory. However, cognizant of the high esteem in which that
prince held him, and relying upon Frederick’s favorable attitude
towards the new religion, Luther ventured upon a course of action
which would otherwise have been foolhardy. Moreover, at the conclusion
of his last letter, Frederick had practically put the decision
up to Luther by saying that he would leave everything to his discretion,
since he was experienced in such important matters.\footnote
{\textit{Briefwechsel}, III, p. 295. The importance of these words is enhanced by the fact
that shortly before, at the end of February, 1522, Luther had announced to the Elector his
early return to Wittenberg. Erl. ed., LIII, p. 183 (\textit{Briefwechsel}, III, p. 291).}
His return was energetically requested by the authorities of the city of
Wittenberg as well as by the University, which had been induced to
take this step at the behest of Melanchthon, who was completely bewildered.
Thus Luther started on the hazardous journey without
notifying the Elector.

He left the Wartburg on the morning of March 1, attired in his
squire’s costume, and rode towards Borna, south of Leipsic, by way of
Jena, where he met the Wittenberg student Kessler, a native of
Switzerland, with several companions, in. the Inn of the Bear. At
Borna he wrote a letter to the Elector, which Protestant historians
are wont to represent as a marvelous product of devout heroism.

Since he had received his Gospel not from man but from Heaven--thus he
writes in a vein of beguiling confidence--he would now return, in order
not to yield to the devil, who had caused the consternation at Wittenberg.
He realized that he had powerful opponents, such as Duke George of Saxony
(who had issued the most dire threats against Luther in a letter to the
Elector, his cousin); yet he was not afraid. Even if the state of affairs at
Leipsic, where George resided, were as bad as at Wittenberg, he would,
nevertheless, proceed to that place, even if the menaces of vain George became
nine times more dire and if every inhabitant were nine times worse than he.
He was going to Wittenberg under a far higher protection than that afforded
him by his Elector. God was with him, and therefore he did not need the
protection of the ruler; indeed, he intended rather to protect Frederick than
be protected by him; for the Elector’s faith was still weak, whereas his
own was strong. He requests the prince, in the event that he be apprehended
or murdered, not to resist the authorities of the empire, for that would be
contrary to the will of God who had instituted them. The empire, he hopes,
would not expect the prince to become his [Luther’s] jailer. In conclusion
he rises to the height of a spiritualistic style: “If your grace would have the
faith, you would vision the majesty of God; but as you have not yet the
faith, you have seen nothing.”

This is Luther’s so-called “heroic letter” from Borna. It is undoubtedly
the utterance of an incomparably courageous man; but it
is also the product of a mind which cannot be comprehended except
on the assumption that he was impelled by the fixed idea of being
divinely inspired. No man can write thus unless he pretends to be an
ambassador of God and who does not realize that his entire future is
at stake. The situation in which Luther found himself compelled him
to impress his territorial lord as forcefully as possible in order to assure
the success of his move. And he succeeded in doing this because
Frederick was a diplomat, sickly, and a Bible reader who favored
the Lutheran innovations. Luther, on the other hand, was simply
compelled to resume his abode at Wittenberg and to sever the chain
which bound him to the Wartburg, if he did not wish to see his lifework
shattered. In no other way was he able to keep control of the
reins which, as he believed, had been entrusted to him by God.
Thus considered, his resolution as well as the letter lose somewhat
of their heroic character and enter into the sphere of tangible and practical
calculation. At all events, considering the edict of Worms and its
threats, his enterprise was a venture which might just as well
have led to the fall of Luther’s work, as it actually did to its progress.
This remains true, even if two other natural motives are taken into consideration.
Luther’s long confinement was bound eventually to become
a stifling burden to a man of his temperament--a burden to be shaken
at any price, even if ice effort involved great danger. The
second factor is the order issued by the imperial government to off
the Elector and the bishops. It was bound to move Luther to form a strong
resolution to settle the disturbances at Wittenberg. If he succeeded in
this, he would be the man of the hour, and would prove himself useful, if not
rehabilitated in the eyes of the empire. In the opinion of many,
therefore, he would no longer be accounted a noxious revolutionist,
as his opponents represented him to be. This is offered in explanation
of the “heroic act” of Borna.

“Squire George” made his entry into Wittenberg on March 6,
escorted by a number of knights who had joined him on the way.
Kessler, the Swiss student, describes his exterior appearance as “quite
corpulent”; his bearing was erect, but as he walked, he bent backward
rather than forward; his head was lifted high. Kessler particularly
observes his “deep-set black eyes and twinkling like stars, so that it was not
comfortable to behold them.”
In the following year, this striking eyebrows, blinking and
expression of Luther’s eyes is thus
described by the bishop of Kulm and Ermland, John Dantiscus, who
had seen and conversed with Luther at Wittenberg: “His eyes are
penetrating,” he says; “they have a certain sinister sparkle, such as
one finds from time to time in persons who are possessed by the
devil.” His features, he added, reminded one of the character of his
books, his conversation was excited and replete with risque allusions
and ridicule.\footnote
{Grisar, \textit{Luther}, Vol. IV, p. 357; cfr. Vol. III, p. 429; Vol. I, pp. 86, 279; Vol. II,
pp. 158 sq.}

In addition to the letter which Luther had dispatched from Borna,
the Elector was pleased to receive another communication from him
upon his arrival at Wittenberg. In this letter, which had been expressly
requested by the Elector, Luther spoke of his return to Wittenberg, and
the Elector intended to exhibit it in public, to enable
Luther to justify his step himself. The letter was actually read by the
Elector’s representative at the imperial court and by other important
persons.

On the following Sunday Luther, elated at his achievement, confidently
ascended the pulpit of the town-church, whence for eight
successive days he delivered the most vehement sermons imaginable
against the prevalent frenzy. Owing to his powerful action, not to
say his hypnotic gifts, he was able to repel all opposition and to restore
order. His chief thought was that the weak, \textit{i.e.}, those who had
not as yet attained to a full conviction of the value and freedom of
the Gospel, should not be perturbed by precipitous innovations. The
objects of a general amelioration had better be sought by gentle and
circumspect methods. By his prudence and the force of his eloquence,
he succeeded in gaining control of the situation and forcing Karlstadt
with his associates into the background. The text of these famous sermons
was subsequently revised by Aurifaber, who published
them. Luther made them the basis of his treatise “On the Reception
of the Sacrament under Both Forms,” which appeared soon after.\footnote
{Weimar ed., Vol. X, II, pp. 11 sq.; Erlangen ed., Vol. XXVIII, pp. 285 sqq.}

In this work he declares that charity was violated towards those who
were not of his party, “who belong to us and must be made to join us.”
A man is not bound to do what he has a right to do. Then, too, he (Luther)
should have been consulted, since the town council had conferred the office
of preaching in Wittenberg upon him. The Mass was to be abolished, but
not by violence. Marriage of the clergy, the monastic life, fasting, images in
the churches, etc., were matters left free by God. Things would right themselves
without ordinances and coercion, provided the Word of God entered
the hearts of men.

“While I slept and drank beer with my Philippo [Melanchthon] and with
Amsdorf in Wittenberg, the Word did so much to weaken the papacy as no
prince or emperor ever did before. I have done nothing; the Word has done
and achieved everything.”

Everyone who is convinced he is right should make a determined use
of his liberty. If he is forbidden to eat meat on fast-days, he should
straightway eat it; if the pope tries to compel him to remain in the monastic state,
he should discard the cowl. Then, contradicting himself, he states that the
use of the chalice by the laity is an ordinance of Christ, as it pertains to
the essence of the Eucharist, yet it is free and must not be made compulsory.
The chalice is to be given to those who desire it, but it is not to be forced
upon the congregation. Consequently, here too we have a twofold form of
administering the Sacrament with the inevitable danger of provoking a
cleavage among the faithful.

Luther’s attitude respecting the use of images was equally indefinite. The
images of saints, crucifixes, etc., are not to be abolished on their own
account, but only to serve God. Because of the harmful abuse to which
they have been put, they do not deserve to be retained. If people once find
out that images are worthless, they will disappear.

The old form of confession should be abolished and a voluntary confession
of sins substituted for it. He says that he himself had experienced the consolation
and strength to be derived from this latter kind of confession,
without which the devil would have choked him to death long ago. “I am
well acquainted with him [the devil]; and he knows me. If you knew him,
you would not thus repudiate confession.” Luther, therefore, proposes to
retain confession on the strength of his own experience and by virtue of
his own authority, whereas the Catholic Church appeals to the authority
of Christ. He orders the penitent to confess to anyone whom he may select,
whereas the Church directs the penitent to an ordained priest. He directs
the penitent to confess for the sake of obtaining consolation, whereas
the Church directs him for the sake of obtaining remission of his sins
through the Sacrament of Penance. He does not acknowledge the sacramental seal
of confession which the Church imposes upon the priest.

But this is not the place to emphasize all the defects and contradictions
of the ecclesiastical system advocated by Luther as contrasted with the laws
that prevailed in the ancient Church. They will suggest themselves spontaneously
to the thoughtful reader.

In one particular Luther openly proclaims the necessity, and at the same
time the impracticability of a measure proposed by himself. For the practice
of auricular confession he would substitute a kind of ecclesiastical ban as
an indispensable disciplinary measure. After exhortations have proved unavailing,
“adulterers, usurers, robbers, and drunkards” should be excommunicated
in the name of the congregation. “However,” he says, “I do not
venture to carry this out alone.” Owing to the internal defects of his
religious system, his well-intentioned endeavors in this respect were never fully
successful, though stricter ecclesiastical discipline would have been precisely
the most necessary requirement after the tumultuous scenes witnessed at
Wittenberg.

On the whole, as even the most eminent among the Protestant biographers
of Luther acknowledges, one is “compelled to question whether the way
which Luther in these sermons prescribed for ecclesiastical reforms, was
actually fit to attain the object he had in view, namely, a Christian regimen
with purified regulations.”\footnote{Köstlin-Kawerau, \textit{Martin Luther}, 5th ed., Vol. I, p. 507.}
Indeed, the unlucky star of dissension between
his basic demand of a false liberty and the need of unity and order
hovered visibly over the beginnings of his practical attempts at reform.

Karlstadt adapted himself grudgingly to the altered state of affairs,
but pursued his own course. He declared that learned studies were
useless and on one occasion, when a young theologian was being promoted
to the doctorate, publicly attacked the title of “Master” as
repugnant to the Word of Christ. He said he preferred to live in the
country as a peasant and to learn from peasants the interpretation of
Holy Writ as it had been imparted to them from on high. He chose to
call himself “Neighbor Andrew” and appeared in the grey garb of a
peasant. Having a propensity for false mysticism, he established relations
with the seer and reformer, Thomas Münzer.

The “prophets of Zwickau” had abandoned Wittenberg. In the
course of an interview with Mark Stübner, Luther had challenged
him to prove his mission by a miraculous sign. It was a suggestion
which Stübner could have cast back at Luther. Stübner, however,
boldly declared that he would comply with the demand. Thereupon
Luther asserted that his God would prevent the gods of such false
prophets from performing any miracles. This ended the conference,
and Luther saw nothing more of the “prophets of Zwickau.” The
restless and fanatical Münzer, on the other hand, endeavored to re-approach
him and explained to him in a letter how, amid inward
fears, he had become assured that his was a true divine revelation. In
spreading his false ideas, therefore, he applied the same standard which
Luther claimed for himself, namely, the way of interior agony. His
claim, however, did not impress Luther, whose practical sagacity
would not permit him to place the least trust in this over-excited
Communist.

One who entered the parish church of Wittenberg after Luther’s
victory, discovered that the same vestments were used for divine service
as of yore, and heard the same old Latin hymns. The Host was
elevated and exhibited at the Consecration. In the eyes of the people
it was the same Mass as before, except that Luther omitted all prayers
which represented the sacred function as a sacrifice. The people were
intentionally kept in the dark on this point. “We cannot draw the
common people away from the Sacrament, and it will probably be
thus until the gospel is well understood.”\footnote{\textit{Von beider Gestalt, das ander Teil}.}
The rite of the celebration of the Mass he explained as “a purely external thing” and said
that the damnable words referring to the sacrifice could be omitted
all the more readily, since the ordinary Christian would not notice the
omission and hence there was no danger of scandal. The words in question,
especially those of the canon, are pronounced almost inaudibly
in the popish Church.\footnote{Köstlin-Kawerau, op. cit, Vol. I, p. 511.}
