\section{“Pecca Fortiter.” The Mass}

The famous expression, “Sin boldly, but believe more boldly
still,”\footnote
{\textit{Briefwechsel}, III, p. 208; Grisar, \textit{op. cit.}, Vol. II, pp. 194 sqq.; M. Pribilla, article
``\textit{Pecca fortiter}'' in the \textit{Stimmen der Zeit}, 1924, Vol. CVII, pp. 391 sqq.}
which Luther embodied in a letter written from the Wartburg to his friend
Melanchthon, under date of August 1, 1521, is to
be ascribed to his strong prepossession in favor of his theory of salvation
by “faith alone.” This paradoxical aphorism was not, as has
frequently been assumed, 2 command to commit sin, against which
Luther always wrote and preached, but a very offensive hyperbolical
expression of the certitude, inculcated by him, that faith in a merciful
God suffices to obtain pardon for all sins, provided that faith in
God is “boldly” asserted.

It seems that Melanchthon, who was spiritually weaker than Luther, was
afflicted by the fear of sin. Luther, in his robust way, wished
to rid him of this fear and hence reminds him that he is “a preacher
of grace” who should not occupy his mind with imaginary sins, but
rejoice in Christ, the conqueror of sin; for sins are inevitable in this
life. Sin, he says in one of his strongest expressions, will “not tear us
away” from the pardoning mercy of the Lamb of God, “even if we
should have committed fornication and murder a thousand times in
one day.” Then he arrives at the notorious expression: “God does not
save those who merely fancy themselves sinners. Be a sinner, and sin
boldly, but believe more boldly still” (\textit{Esto peccator et pecca fortiter,
sed fortius fide}). In the context the phrase “sin boldly” conveys
a sinister impression, involving as it does, fundamentally, a strong self-condemnation
of the Lutheran theory of fiduciary faith and justification. In lieu of
interior contrition, self-humiliation and the penitential spirit, justification
is made dependent upon the presumptuous
apprehension of the merits of Christ, and sin loses its terrifying
character for the believer. Möhler rightly detects in these offensive
words “an evident mental derangement.” It is to be noted, however,
that Luther used similar language also on other occasions, for example,
abstracting from other instances at the beginning of his reformatory
career, in his letter to Spenlein (1518) and, shortly before his death in
his extraordinary letter of 1544 to Spalatin, who had become melancholy.
The offensive “\textit{pecca fortiter}” flows naturally from his whole
system of doctrine.\footnote{Cfr. Grisar, \textit{Luther}, Vol. 11, pp. 194 sqq.}

After Melanchthon had concluded his lectures on the Epistle to
Titus, the Gospel of St. Matthew, and the Epistle to the Romans, he
was transferred to the theological faculty of Wittenberg and there
not only continued his philological labors, but also undertook a theological
work, entitled \textit{Loci Communes}, which was destined to be a
pillar for the support of the Lutheran doctrine. The work was completed
in December, 1521, and appeared in sixteen editions before
1525. The Latinity of the book is classical, but its theology clearly
betrays the lay-theologian completely enchanted by the new doctrine.
The author pretends that, following in the path of Luther, he
must re-create the system of theology, after the greatest minds of
the centuries had supposedly labored in vain on it. Among other
things he teaches that all things happen of necessity (\textit{necessario eveniunt})
in accordance with divine predestination, and that the human
will is not free. He gives a decidedly affirmative answer to the question
whether God is also the author of moral evil.\footnote{Grisar, \textit{Luther}, Vol. II, pp. 239, 282 sq.}
Luther was delighted with
the precious work, which mirrored forth himself. He
termed it an “unconquered work,” which, in his opinion, deserved not
only immortality, but also to be received into the canon of the Bible.\footnote{\textit{Op. cit.}, 11, 282.}
Only gradually did Melanchthon recede from his harsh attitude in
subsequent editions of his work, and did not conceal his disapproval
of Luther’s denial of free will. Luther repeatedly invited him from
the Wartburg to take up preaching at Wittenberg. This the learned
layman, however, would never consent to do. He preferred the lecture
room and the serenity of home life. In August, 1520, he had married Catherine
Krapp, the daughter of the burgomaster of Wittenberg,
and thus formed close social connections with the inhabitants
of that city, which lasted to the end of his life.

The same spirit which impelled Luther to launch his attack upon
monasticism, also led him to attack the Sacrifice of the Mass. The
monastic state and the Mass he regarded as the most important pillars
of the papacy. In the letter in which he informs Melanchthon of his
passionate struggle with his vows, he also announces: “I shall never
again celebrate a private Mass.” Thomas Murner’s defense of the sacrificial
character of the Mass (1520) did not convince Luther. In 1521,
under the influence of his “spiritual baptism” at the Wartburg,
Luther composed a Latin booklet \textit{On the Abolition of the Private
Mass}, of which he published also a German translation under the title,
\textit{On the Misuse of the Mass}.\footnote
{Weimar ed., VIII, pp. 411 sqq.; German version, pp. 482 sqq.; Erl. ed., \textit{Opp. Lat. Var.},
VI, pp. 115 sqq. and (German version) XXVIII, pp. 28 sqq.}

This was the alleged justification of the fight upon the holy sacrifice
which at that time commenced at Wittenberg.

Luther did not favor any overhasty discontinuation of the traditional liturgical
celebrations. He knew that any attempt in this direction would meet with
resistance on the part of the Electoral court.
Nevertheless, he did whatever he could to realize his designs against
the Mass in the university town. In the German treatise just quoted
he appealed to his “dear brethren, the Augustinians of Wittenberg,”
who had already discontinued saying Mass in their church and limited
themselves to preaching. He tells them that he rejoices in their work
and begs them to take up their position on the rock of firm conviction
and at the same time to spare the feelings of the weak. He established
his thesis by an appeal to the qualms of conscience which he experienced.

“Daily I feel how very difficult it is to lay aside scruples of long standing,
controlled by human laws. Oh, with what great pains and labors, and reliance
upon Holy Writ, I have scarcely been able to justify my own conscience,
that I, one individual, have dared to oppose the pope and regard him as
Antichrist, the bishops as his apostles, and the universities as his brothels! How
often has my heart been tantalized, how often has it punished me and reproached
me with their only strongest argument: Are you alone wise? Can it
be supposed that all others have erred, and erred so long a time? What if
you should be mistaken and should lead many into error, who would be
eternally damned? Thus I felt until Christ fortified and confirmed me with
His only certain Word, so that now my heart is uneasy no longer, but resists
this argument of the papists, as a stony shore resists the waves, and
ridicules their threats and fury.”\footnote{At the beginning of the German text, \textit{Vom Missbrauch der Messe}.}

This strong faith, which the mysterious “only certain Word of
Christ” supposedly conferred upon him, he wished to impart to his
brethren and all his readers in the course of his arguments against the
Mass.
