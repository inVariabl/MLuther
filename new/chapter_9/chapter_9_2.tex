\section{Ghosts and Illusions}

Whilst Luther was engaged in the composition of his last mentioned book,
he announced to Spalatin: “I am suffering from
temptation, and out of temper; so don’t be offended. There is more
than one Satan contending with me. I am alone, and yet at times not
alone.”\footnote{\textit{Op. cit.}, Vol. II, p. 82.}
He believed that he was visibly pursued by infernal powers
because of his praiseworthy discoveries. His intimate friend, the
physician Matthew Ratzeberger, quotes Luther as saying that, “Because he
was so lonely, he was beset with ghosts and noisy spirits
which gave him much concern, and he drove them all away by prayer;
but did not wish to talk about it.”\footnote{\textit{Op. cit.}, Vol. VI, p. 123; Köstlin-Kawerau, \textit{M. Luther}, Vol. I, p. 440.}
Nevertheless Luther in his later Table Talks expressed his firm
conviction that he had encountered the visible Satan. Both Ratzeberger
and Luther make mention of the devil’s assuming the form of
a dog.\footnote{Grisar, \textit{Luther}, II, pp. 83}
A big black bull-dog resisted him one night as he was about
to go to bed, and departed only after Luther had recited a verse of
the Psalms. According to Luther’s own story, he seized the dog
and threw him out of the window. He states that there was no such
dog at the Wartburg. On another occasion he was disturbed at night
by a sack of hazel-nuts, locked in a chest, which by the power of the
evil one were thrown one after another against the rafters of his room,
while his bed was violently jolted. At the same time a rumbling noise
was heard on the staircase, as if barrels were being rolled down. And
yet, the staircase, as Luther convinced himself, was locked below with
bolts and chains.\footnote
{\textit{Op. cit.}, Vol. VI, p. 124. This and the next pages of my larger work also furnish the
evidence for the following incidents.}
Thus, as Mathesius assures us, ``he often heard the
rumbling noises of the evil spirit at night in his Patmos [\textit{i.e.}, at the
Wartburg].'' It was frequently a struggle, he continues, like Christ’s,
when He was tempted in the desert. When he changed his quarters to
accommodate the wife of Hans Berlips, she heard such an ado in the
room that she fancied a thousand devils were in it.” Luther himself
reports that “at Eisenach” (\textit{i.e.}, probably at the Wartburg) he had
contemptuously called to the devil on such an occasion: “If you are
Christ’s master, so be it!” “I have learned by experience,” he says
elsewhere, “that ghosts go about affrightening people, preventing
them from sleeping and so making them ill.”

But he experienced not only tribulations, but also consolation and
encouragement from the world beyond.

“Ten years ago,” he said (1532) to his pupil Schlaginhaufen, “God
strengthened me in my struggles and writings through His angels.”
The period thus indicated probably refers to the months he spent
at the Wartburg. Perhaps the vision with which his pupils were acquainted
also happened at this time. While he, engaged in the service
of the Word--thus the story runneth--was praying in his chamber,
the image of Christ bearing the five wounds, appeared to him in shining
splendor. However, as he was in doubt, thinking it might be the
evil spirit, he said: “Begone, thou infamous devil,” whereat the image
forthwith disappeared. For some definite reason, Luther disliked to
indulge in such narratives, because the fanatics, his enemies, piqued
themselves on their enlightenment and revelations, instead of abiding
by the Biblical texts as propounded by Luther. He did not look
favorably upon communications from the other world. If, nevertheless, he
exploited his own experiences in terms such as the following,
their significance is all the more enhanced. “Ah, bah, spirits!”
he exclaims, “I too have seen spirits!”

He had other similar experiences both before and after his sojourn
at the Wartburg.

His mind was naturally receptive to such experiences. Even as a
monk, when he studied at night in the refectory preparing to become
a professor, he heard the devil rustling about in the wood-bin, and
again later on in his room at the monastery. “The devil,” he says,
“often had me by the hair of my head, yet was ever forced to let
me go.” He claims to have seen “gruesome ghosts and visions” from
time to time in the monastery, and ``no one was able to comfort'' him.
More important is the following report. In the course of official business
with Gregory Casel, a delegate of the reformed theologians of
Strasburg, in 1525, Luther assured him that, while in the monastery,
he “frequently had inward experience that the body of Christ is indeed
in the Sacrament” (a dogma which the Zwinglians did not believe); that
he had seen dreadful visions, also angels (\textit{se angelos vidisse}),
so that he had been obliged to stop saying Mass.” “What do
the Strasburgers mean with their alleged ghost?” he asks. “Are they
alone in possession of it? But particularly, have they experienced the
terrors of death which I have been through (\textit{mortis horrorem expertus})?”

Luther’s visionary experiences cannot be doubted. They were gross
imaginings of preternatural annoyances and corroborations, misinterpretations
of internal and external experiences which are well established, particularly
for the period he spent at the Wartburg. He
named the castle his Patmos, evidently because it was there that he,
like the Apostle John when in exile on the isle of Patmos, had
preternatural experiences. His extremely active imagination rendered him
very susceptible to hallucinations and illusions, especially
when accompanied by precordialgia, a physical ailment from which
he frequently suffered, or by severe constipation, to which he was
also subject at times, and of which he complains at the Wartburg, or
when his nerves were overwrought in consequence of excessive literary labors.
The enlightenment which he imagined to have received, naturally
revolved about his divine vocation as herald of the new gospel. Thus
Luther, for all future time, received his spiritual baptism at his Patmos.
The most precious “first-fruits of the spirit” (\textit{primitie spiritus},
as he calls them), were allotted to him there.\footnote{\textit{Op. cit.}, Vol. III, p. 116.}
He says, apparently
in allusion to a mysterious event relative to his doctrine: “Under
threat of the curse of eternal wrath, I have been found worthy in no
manner to doubt these things” (\textit{fui dignus, cui sub wterne ire maledictione
interminaretur, ne ullo modo de iis dubitarem}).\footnote
{The passage is now to be found in Tischreden, Weimar ed., IV, No. 4852, among the
Table Talks of the Khumer collection, July, 1543, and is reproduced as a copy from
Luther’s Psalter. Luther wrote in almost identical terms to his friend Jonas, when the latter
was ill in 1540 and “in the greatest temptation,” introducing his letter as follows: “\textit{Contra
tentationem indignitatis mostrae sic respondendum est diabolo}” (Weimar ed., \textit{l.c.}, note). He
wished to indicate to his frightened friend, how he quieted himself. The possible relation
to a single experience of Luther is made clear by the connection of the longer passage:
“\textit{Martinus Lutterus indignus sum, sed dignus fui creari a creatore meo, dignus fui redimi a
Filio Dei, dignus fui doceri a Filio Dei et Spiritu sancto, dignus fui, cui ministerium verbi
crederetur, dignus fui, qui pro eo tents paterer, dignus fui, qui in toto malis servarer,
dignus fui, cui praeciperetur ista credere, dignus fui, cui sub aeternae},” etc. Aurifaber thus
reproduced the conclusion: “[Yet I am worthy] that I ought by no means to doubt it, I
who have been severely threatened and enjoined by the wrath of God, His displeasure and
execration.” (Weim. ed., \textit{l.c.}) Luther’s idea that God was leading him into hell, in order to
assure him of his salvation, seems to have arisen from an “experience” made by him at a
certain juncture of his life.}
How then,
in Luther’s imagination, was it possible that the devil should not have
opposed his election?

Legend has expanded this struggle with the devil. There is no certain warrant
for the report of the apparition which in time has come
to be the most popular of the Wartburg tales. Luther nowhere says
that he hurled the ink-well at the devil, nor do his pupils mention
the incident. The famous spot on the wall is unattested, and its historicity
is not confirmed by the fact that it has constantly been retouched,
whenever the devotion of relic-hunters had gradually scraped
it off. Such spots, all originating from an ink-well which Luther
hurled at his satanic majesty, were formerly to be found also in other
places, \textit{e.g.}, in the rooms which Luther occupied in Wittenberg and
at the Koburg.\footnote{Grisar, \textit{Luther}, Vol. II, p. 96.}
